\section{ОСНОВЫ ВСЕХ БУЛЬОНОВ, ФАРШЕЙ И СОУСОВ, НЕОБХОДИМЫЕ В КАЖДОЙ ПОРЯДОЧНОЙ КУХНЕ}

\z{Ампотаж (Empotage).}\index{Ампотаж}

(Ампотаж~--- темный бульон, употребляемый для подкраски супов и всяких соусов, при чем он дает и приятный вкус).

Внутренность кастрюли вымазать 1/4 ф. чухонского масла. Тогда в эту вымазанную таким образом кастрюлю кладется 4 фун. говядины и 1 ф ветчины, у которой снят весь жир и которую нарезать ломтиками толщиною в палец, равно как лопатку телятины и курицу. Воды налить столько чтобы все эти мяса были покрыты ею в половину. Тогда кастрюля ставится на сильный огонь, который уменьшается постепенно, по мере того, как выходит сок из этих мяс. Таким образом, вы получите крепкий, густой навар, называемый на поваренном языке гласиром. Но при этом приготовлении должно быть очень внимательным, чтоб не допустить бульон слишком поджечься. Советую стараться делать так, чтобы этот густой бульон становился постепенно темно-красного цвета, а чтобы отнюдь не вдруг, моментально совершалось это окрашивание. Когда бульон сделается уже достаточно темен, вы вольете в кастрюлю 4 стакана холодной воды, чрез что бульонная гуща отделится со дна кастрюли на ее поверхность. Затем вы снова наполняете кастрюлю водою, кладете в нее тех же кореньев, какие клали прежде, даете всему этому варится часа 2.5 и тогда вы его пропускаете чрез салфетку. И вот,~--- ампотаж готов.

\z{Пассир.}\index{Пассир}

Есть два сорта пассира: пассир белый и пассир красноватый. Нельзя не советовать хозяйкам иметь всегда в запасе пассир, который служить превосходным подспорьем при приготовлении всех блюд с соусами и подливами. Вот рецепт пассира. В кастрюлю кладется фунт чухонского масла и растапливается, туда подбавляют 3/4 ф. просеянной муки; все это тщательно размешивается и ставится на плиту, в умеренный жар, накрывается кастрюльною крышкою и, по истечении каждых 10 минут, размешивается все это лопаткой. Чтобы варка была ровная, надо давать беспрестанно кипеть, но только исподволь, слегка, отнюдь не допуская до вскипания. Когда прошло 3/4 часа варки, снимите кастрюлю с плиты и слейте то, что в ней содержится, в какую-нибудь посудину с крышкою, чтобы употреблять по мере надобности. Так поступают с белым пассиром. Что касается до красноватого, то поступать точно также, с тою лишь разницею, что надо давать ему вариться несколько долее белого (часа 2), чтобы он получил золотистый цвет.

\z{Белый кулис.}\index{Кулис ! белый}

Нарезать тоненькими ломтиками 1.5--2 фунта телятины, и также нарезать ломтиками 0.5--1 фунта ветчины. Накрошить кружечками моркови, петрушки и луку по одной штуке. Уложить дно кастрюли ломтиками телятины и ветчины, а на них положить коренья; закрыть кастрюльку, поставить на плиту, на легкий огонь, чтоб мясо и коренья дали из себя сок. Кастрюлю по временам потряхивать, чтоб мясо не пригорело. Когда сок, который дало из себя мясо, закипит, влить 1 или 2 стакана бульона, дать хорошо увариться. Потом взять 3 яичных желтка, круто сваренные, белое мясо от 1 курицы,~--- истолочь мягко, подливая понемногу сливок, чтоб сделалось жидкое тесто, положить в кастрюлю, в которой варится кулис, вымешать хорошенько, дать 1/4 часа покипеть на самом малом огне, процедить сквозь сито. Кулис этот надобно употреблять в тот же день.

\z{Красный кулис.}\index{Кулис ! красный}

Взять 2 фунта телятины, 0.5 ф. ветчины нежирной, нарезать тоненькими ломтиками, положить в кастрюлю, накрошить кружками кореньев: петрушки, пастернака и моркови, по 1 штуке, накрыть кастрюлю, поставить в печь или в духовой шкаф, чтоб мясо хорошо упарилось и зарумянилось, но не пригорело. Потом положить шпику, нарезанного тоненькими ломтиками, посыпать немного муки, вымешать, дать еще попреть; когда все хорошо уварится, влить по равной части бульону и жюсу. Также можно класть в кулис дичину, рябчиков и куропаток, особливо если кулис нужен для приправления кушанья из дичины. При окончательном приготовлении кулиса, положить трюфелей, шампиньонов и разных пряностей\footnote{Обыкновенно кладут от 3–-4 зерен английского перца, 2-–3 гвоздики, 1 шт. лаврового листу.}, уварить, процедить сквозь салфетку. Приправы и пряности всякий может класть по своему вкусу, какие пожелают или какие случатся.

\z{Раковый кулис.}\index{Кулис ! раковый}

Положить в кастрюлю 1/4 фунта свежего чухонского масла; сварить 50 шт. раков, снять с них верхние черепки, обломать клешни, шейки и ножки, и все вместе истолочь в ступке до-мягка. Потом положить в масло, и жарить, мешая, чтоб не пригорело; когда хорошо ужарится, влить 3 стакана рыбного бульону, уварить хорошенько и процедить сквозь сито.

\z{Жюс из говядины.}\index{Жюс ! из говядины}

Положить в кастрюлю 0.5 фунта шпику и 0.5 фунта говядины, нарезанной ломтиками; прибавить луку, накрошенного кружками; накрыв кастрюлю, дать преть на легком огне, пока мясо даст сок; тогда подлить немного бульону; уварить, потом процедить и, дав остыть, снять жир.

\z{Жюс на скорую руку.}\index{Жюс ! на скорую руку}

Если нужно подцветить кушанье, положить на сковороду или в кастрюльку столовую ложку мелкого сахару, смочить водою, и держать на жару, пока сахар покраснеет. Тогда влить чайную чашку бульону, кипятить, покуда сахар отстанет от кастрюльки и распустится в бульоне.

\z{Фарш для супов.}\index{Фарш ! для супов}

Взять 1 фунт филею телятины, вырезать жилки и перепонки, изрезать в мелкие кусочки, прибавить в половину против телятины говяжьего почечного сала, в мелкие куски изрезанного; а если случатся мозговые кости, то вместо сала можно положить мозгу. Потом изрубить мягко, как тесто; на 1 фунт рубленного мяса положить 2 ломтика французского хлеба.~--- хлеб для этого должно брать черствый, 1 ломтик истереть на терке, а другой, размочив в сливках или цельном молоке, положить в фарш,~--- и еще порубив, прибавить в фарш 3 яйца, соли и мускатного орешка; перемешать хорошенько. Из этого фарша наделать маленьких шариков, или, как называют их, фрикаделей. Можно также сделать из фарша род сосисок, или скатать его на подобие толстой колбасы, завернуть в чистую холстину, завязать нитками и сварить в бульоне. Когда фарш будет готов, нарезать ломтиками, или, вырезав разными фигурными штучками, опустить в суп. Фарш этот можно употреблять для паштетов. Также приготовляют его из говядины и баранины.

\z{Фарш для пирогов.}\index{Фарш ! для пирогов}

Взять 1 фунт жареной телятины, от задней ноги, 0.5 ф. белого мяса от жареных кур или индеек; изрезав мелко, прибавить 2 мелко накрошенный луковицы; положить 2 ложки коровьего масла, обжарить на сковороде, беспрерывно мешая; а потом изрубить мелко. Пять круто сваренных яиц изрубить мелко, положить в фарш, посыпать понемногу перцу и мускатного орешка, посолить, порубить еще, чтоб все хорошо перемешалось.

\z{Фарш из рыбы.}\index{Фарш ! из рыбы}

Приготовлять рыбный фарш всего лучше из щук, судаков и сигов. Снявши кожу, выбрать кости, истолочь мясо до-мягка, прибавить белого хлеба, размоченного в воде и выжатого досуха, масла прованского или макового, 2 или 3 ложки, смотря по количеству фарша. Для скоромного фарша, приготовляемого из рыбы, белый хлеб размочить в молоке или бульоне, и также выжать досуха, а масла положить чухонского. Из приготовленного таким образом фарша можно делать котлеты, фрикадели для рыбного или ракового супа. Если фарш нужен будет для начинки пирогов и паштетов, то, сварив его в бульоне, или поджарив на сковороде, изрубить, смешать с круто сваренными и мелко изрубленными яйцами. посолить, приправить чуть-чуть перцем. В постный фарш, вместо яиц, положить вязиги или свежих белых грибов, изрубленных и поджаренных в масле.

\z{Клер № 1.}\index{Клер ! №1}

Взять 0.5 ф. муки, замесить на пиве тесто; положить в тесто, одно за другим, три яйца, размешивая ложкою, прибавить ложку розовой воды и немного кардамону, подбить ложкою дрожжей. Тесто должно быть пожиже теста, приготовляемого на оладьи. Когда тесто будет совсем готово, поставить в теплом месте и дать подняться. Обмакивая в это тесто, жарят разные мяса и некоторые пирожные.

\z{Клер № 2.}\index{Клер ! №2}

Взяв 2 горсти хорошей муки, 2 столовые ложки толченого сахару, 3 яичные белка, взбить и замесить на них тесто, довольно густое; потом развести в надлежащую пропорцию виноградным белым вином.

\z{Раковое масло.}\index{Масло ! раковое}

Сварить полсотни раков, обломать клешни, шейки и ножки, также снять верхние черепки, и все вместе истолочь домягка в ступке; потом взять 0.5 ф. коровьего масла, а если нужно приготовить постного ракового масла то, вместо коровьего, положить макового или орехового масла. Ужарить хорошенько, чтоб масло покраснело, процедить сквозь чистое сито, и гущу выжать ложкой. Раковое масло должно сберегать в холодном месте.

\z{Брез.}\index{Брез}

Нарезать тоненькими ломтиками 1/4 ф. шпику и 3/4 ф. телятины, накрошить кореньев: моркови, петрушки и луку по одной штуке; уложив дно кастрюли ломтиками телятины, на нее положить кореньев, а на них птицу, часть говядины или язык, то есть, что хотят приготовить в брезе, сначала сварив до половины в бульоне. Сверху накрыть ломтиками шпику, засыпать кореньями, положить разных пряностей и, закрыв кастрюлю, поставить на плиту. Когда все это даст из себя сок, влить рюмку виноградного вина, несколько ложек хорошего бульону, и, закрыв опять кастрюлю крышкою, поставить в духовой шкаф или в печь и уварить до готовности.

\z{Консоме.}\index{Консоме}

Если случатся остатки или обрезки мяса, жил, перепонок, мозговые кости от птиц и жаркого, то, перемыв все хорошенько, кроме жареного, положить в кастрюлю, налить 1.5--3 стак. бульону; когда соберется на верху пена, снять дочиста, положить перцу, лаврового листу (гвоздики и имбирю). Прибавить, смотря по количеству приготовляемого консоме, фунт или два ветчины без жиру; варить часа четыре, процедить сквозь салфетку и снять жир.

\z{Фритюр.}\index{Фритюр}

Фритюр большею частью собирается из жиру, который снимается с бульона и обрезков жира телятины, баранины и свинины; все вместе топится и сливается в один горшок. Когда соберется около 3 ф., употребляют для жарения в особой для фритюра железной кастрюле. Если бы фритюру не оказалось, то взять 3 ф. воловьего жиру, изрезать мелко, подлить немного воды и поставить на огонь, а когда стопится, процедить сквозь полотенце и выжать остатки до сухости. За 15 минут до отпуска, поставить фритюр на плиту и, когда разогреется, отпускать назначенное и жарить, как сказано будет в своем месте.

Нужно заметить, что с фритюром должно обходиться весьма осторожно: во-первых, чтобы фритюр излишне не разогревался; но если бы это и произошло от недосмотра, то опустить в него сырой картофель, кусок репы или капусты, чтобы фритюр не почернел, и беречься, чтобы не попало воды в горячий фритюр: иначе выбежит из кастрюли и может наделать вреда за вашу неосторожность.

\z{Глас.}\index{Глас}

Когда не хотят употреблять для гласу особенного мяса, а есть остатки консоме и бульону, смешать их вместе, прибавить для цвета жюсу, варить, пока глас сделается густ, как сироп, и можно будет гласировать им. Для цвета прибавляют иногда ракового масла.

\z{Лезон.}\index{Лезон}

Выпустить 2 яйца, сбить их ложкою, прибавить 2 столовые ложки сливок, размешать. Лезон употребляют для смазывания паштетов, пирогов, булок и разного пирожного, когда садят их в печь.

\z{Бешамель.}\index{Бешамель}

Бешамель приготовляется таким образом: вскипятить бутылку не очень густых сливок, положить в кастрюлю столовую ложку свежего чухонского масла, 2 ложки муки, стереть масло с мукою, развести кипящими сливками и дать раз вскипеть.

\z{Красная подпалка.}\index{Подпалка ! красная}

Взять 1/4 ф. самого свежего перетопленного масла, распустить на сковороде, положить столько муки, чтоб было довольно густо. Потом жарить на легком огне, пока мука покраснеет. Когда будет готово, сложить подпалку в горшок или в муравленную кастрюльку; держать в холодном месте.

\z{Белая подпалка.}\index{Подпалка ! белая}

Растопить 1/4 ф. свежего коровьего масла, положить в него муки, чтоб было густо; потом жарить на малом огне, не давая, однако ж муке покраснеть, но только бы она хорошо соединилась с маслом. Когда будет готово, сложить в муравленный горшок; сохранять, для употребления, в холодном месте.

Соусы-приправы по методе бывшего придворного метр-д'отеля Эдмона Эмбера.

(Всего восемь рецептов).

Главные соусы-приправы. По мнению бывшего придворного метр-д'отеля г. Эдмона Эмбера, в поваренном деле только, два главных соуса, составлявшее основу всего, и при помощи которых приготовляются все остальные малые соусы, посредством различных в них прибавок и некоторых изменений. Эти соусы: велуте (vélouté, т. е. бархатистый) и эспаньоль (espagnole, т. е. испанский).

\begin{enumerate}
	\item \z{Велуте.}\index{Соус ! белый ! велуте}

Для белого соуса, велуте, надобно взять 4 ф. телятины без костей, от задней четверти, курицу, 0.5 ф. окорока без жиру. Все это изрезать жеребейками. Затем взять 0.5 ф. свежего сливочного масла и изрубить мелко-на-мелко 2 луковки, 2 моркови, 2 корзиночки шампиньонов, и все это сложить в кастрюлю, которую поставить на огонь, и лопаткой взбалтывать и размешивать всю эту смесь, чтобы все это приняло светло-желтый цвет. Тогда прибавить ко всему этому 1/4 ф. пшеничной муки, поставить на несколько минут на плиту, помешивая и взбалтывая беспрестанно, влив туда немного чистого белого телячьего бульону. Как только закипит, снять с огня и поставить кастрюлю на край плиты. Тут соус ваш должен тихонько кипеть. Наблюдать, чтобы соус не был чересчур густ: тогда он будет излишне жирен. Положить крошечку свежей петрушки с одним лавровым листом. Когда телятина окажется достаточно сварена, пропустить этот соус чрез сито и сложить в особую посудину, для употребления по мере надобности.

	\item \z{Красный соус (эспаньоль).}\index{Соус ! красный ! эспаньоль}

Для приготовление же красного соуса (эспаньоль), берутся те же материалы и в таком же количестве, с тою лишь разницею, что мяса и мука должны быть слегка поджарены. Во время поджаривания, надобно часто помешивать, чтобы не пригорело. Вместо белого бульону, подливается темный бульон, или ампотаж. Окончательное приготовление эспаньоля то же самое, как и окончательное приготовление велуте.

	\item \z{Пуаврадный соус.}\index{Соус ! пуаврадный}

Положить в кастрюлю несколько веточек зеленой петрушки, 10 шт. шарлоток, хорошо вычищенных, пол листика лаврового, большую щепоть крупно смолотого перцу, маленько сырой ветчины, изрезанной в жеребейки (за неимением ветчинного жира, употребляется сливочное масло) и стакан хорошего ренскового уксусу. Тогда в кастрюлю влить 2 больших стакана соуса эспаньоль (см. опис.) и такой же стакан обыкновенного говяжьего бульону. Все это хорошенько перемешать и дать вскипеть, поставив на край плиты, чтобы смесь эта порядком проварилась минут с 10. Тогда сняв пену и жир, поставить на большой огонь. Пусть соус варится до тех пор, пока он не примет густоты хороших сливок. Эту гущу пропустить чрез салфетку и употреблять с говядиной, в каком бы виде она ни была приготовлена, или ежели говядина подогрета, оставшись от вчерашнего дня, то она режется на куски и обливается просто этим пуаврадным соусом.

	\item \z{Энергический соус.}\index{Соус ! энергический}

Изрубите мелко-намелко 10 шарлоток, вымойте их и выложите в салфетку. Потом положите их в кастрюлю, влив туда стакан уксусу и положив пучочек чесноку, листок лавровый, прибавив стакана 2 самого крепкого говяжьего бульону. Все это распустите вместе и поставьте вариться; когда вся эта смесь дойдет до степени густого сиропа, положите с орешек величиною масла сливочного с примесью на кончике ножа маленького количества самого крепкого английского красного перцу в ложке прованского масла. Соус этот подается к холодной живности, оставшейся к завтраку или к ужину от обеда.

	\item \z{Помидорный или томатный соус буржуазный.}\index{Соус ! томатный}

10 или 12 томатов хорошенько перемываются, разрезываются пополам в ширину, слегка нажимаются, чтобы выдавить семечки, и кладутся в кастрюлю с 4 шт. репчатого луку, изрезанного в тоненькие ломтики, с 2 или 3 веточками зеленой петрушки, 1 или 2 шт. гвоздики и 1/4 ф. сливочного масла. Все это хорошенько сварить, наблюдая, чтобы оно не приставало к краям кастрюли. Когда все это поварится 3/4 часа, прибавьте к этому ложку темного соусу, снимите жир и пропустите сквозь волосяное сито. Перед подачей на стол, прибавить кусочек сахару.

	\item \z{Ярко-зеленый соус. (Ravigotte verte).}\index{Соус ! зеленый}

Ощипать добрую горсть эстрагону, 2 горсти кервелю, бедренцу и пол горсти зеленых цибулек. Всю эту зелень перемыть хорошенько в свежей воде. Тогда приготовить большой котелок красной меди не луженый, какие обыкновенно употребляются кондитерами, налить в него на 2/3 воды, всыпать 1/8 ф. соли, поставить на огонь и дать посоленной воде кипеть сильно. Когда вода хорошо прокипела, сложите в нее всю траву и дайте ей вариться минуть 20. Тогда откиньте на сито, чтоб дать стечь воде, и снова положите в свежую воду. Повторите это раза 2, т. е. делайте то, что на поварском жаргоне называется бланшированием, способствуя зелени сохранить тот ярко-зеленый цвет, который так пленяет глаза и возбуждает аппетит, вовсе не будучи результатом медного пятака (т. е. ужасного яда), как хотят рассказывать об этой яркой окраске иные псевдо-знатоки, ничего не разумеющие. Благодаря такому <<бланшированию>>, вы всегда будете иметь ярко-зеленые бобы, шпинат, щавель и пр. По окончании бланширование, зелень вынимается из воды, вода с нее спускается хорошенько, зелень рубится, кладется в иготь или в ступку, где толчется с прибавкою 1/4 ф. сливочного масла. Когда зелень приняла вид теста, вы пропускаете ее чрез частое сито, прижимая ложкою. Эта густая зеленая масса идет в разные блюда.

	\item \z{Финзербы для соусов.}\index{Финзербы для соусов}
    
Небольшую корзинку шампиньонов вычистить и вымыть хорошенько, выжав их на полотенце. Потом изрубить их мелко-намелко, положить в кастрюлю 1/8 ф. сливочного масла, сложить туда шампиньоны, взять 12 шарлоток, изрубить их также мелко, набрать зеленой петрушки, отделить ветки от корней и, тщательно их вымыв, также изрубить. При этом наблюдать, чтобы количество петрушки было одною четвертою частью менее количества шампиньонов. Все это смешать в кастрюле и поставить на огонь плиты, пусть оно варится; но в это время надо помешивать с осторожностью минут с 10. Прибавьте ко всему этому 0.5 рюмки сотерну, дайте всему этому повариться еще с 1/4 часа,~--— и ваши финзербы готовы. Чтоб сделать хороший финзербный соус, стоит только часть, взятую от приготовления сейчас нами описанного, соединить с такою же частью эспаньоля (см. описание), например, стакан того и стакан другого. Не худо прибавить немного бульону, чтоб соус не был чересчур густ. Не должно забыть и о потребном количестве соли. С финзербами, как с превосходною приправою, можно делать много хорошего.

	\item \z{Дичинный экстракт для придания вкуса дичины всякому другому мясному соусу.}\index{Экстракт ! дичинный}
    
Для этого вы употребляйте куропаток, рябчиков, зайца, дикую козу (смотря по потребности количества). Их, разумеется, надобно выпотрошить, вычистить, изрезать на довольно мелкие куски и в сыром состоянии сложить в кастрюлю. Прибавьте парочку или две моркови, 12 шарлоток, 2--3 листка лавровых, 2--3 шт. гвоздики, несколько веток зеленой петрушки. Все это надо смочить полбутылкою сотерна или вендеграва, поставить на огонь и затем дать кипеть до тех пор, пока то вино, которое тут варится с дичиною, осядет почти до степени гласира. Тогда влейте туда стакана 3 хорошего говяжьего домашнего бульону и дайте всему этому слегка кипеть на краю плиты. Когда дичина порядком уварится, вс. эту эссенцию пропустить чрез салфетку; затем вы сложите ее в другую кастрюлю, снимайте с поверхности жир и снова ставьте на огонь, варя до тех пор, пока она не сделается столько же густа, как должен быть густ хороший сироп~--- и вот~--- наш дичинный экстракт готов.
\end{enumerate}
