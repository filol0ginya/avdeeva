\section{ХОЛОДНЫЕ БЛЮДА} % отдел 7

\z{Глассированная баранья нога с огурцами}\index{Баранина ! холодная ! с огурцами}
Взять заднюю ногу баранины, приготовить в брезе\footnote{См. Брез}, загласировать. Потом взять свежих огурцов, очистить, вырезать семена, разрезать каждый огурец на четыре части, смотря по величине огурцов, обделать на подобие маленьких огурцов, посолить, облить уксусом и, дав полежать часа два, откинуть на сито, чтобы стек уксус. Наконец взять кулис, стереть на сахар с лимона цедру; положив огурцы и цедру в кулис, варить до спелости; когда будут готовы, выжать в огурцы сок из одного лимона. Выложив соус на блюдо, положить на него баранину.

\z{Холодная баранья голова}\index{Баранина ! холодная !голова}

К этому блюду, кроме головы, еще присоединяются ноги и рубцы бараньи. Эти ноги и голову опалить как должно, очистить, а рубцы, вычистивши, вымочить, перемыть в нескольких водах, разнять на части и по краям надрезать, или искрошить на мелкие полоски; потом все эти части сварить мягко в воде, после того уши с мясистыми их частями обрезать и концы с хрящей надрезать и положить в блюдо, так, чтобы стояли вверх и притом рядом; наконец взломить череп, а мозг вынуть осторожно, очистить от перепонки и, разделив надвое, положить близ ушей, а язык распластать надвое и положить рядом по средине и около него мягкие части от щек; из глаз вынуть черное, а потом сделать то же вокруг ноги;. все это накрыть рубцами; ежели рубцы не искрошены полосками, в таком случае не обливать их хреном, a хрен подавать со сметаною в особенной чашечке; а как только рубцы искрошены, то хрен растереть со сметаною и облить все перед отпуском на стол. (6–8)

\z{Поросенок под галантиром}\index{Поросенок ! холодный ! под галантиром}

Взять не очень большего поросенка, разнять на четверо, отрезать голову, уложить в кастрюлю, налить водой, посолить, поставить вариться, прибавить понемногу зернами перцу, корицы, гвоздики и лаврового листа. В другой кастрюле поставить вариться, смотря по величине поросенка, 4 или 6 телячьих ножек. Когда поросенок поспеет, вынуть из бульона, положить в холодную воду, а бульон вылить в кастрюлю к ножкам; дать ножкам хорошо увариться, чтоб мясо отстало от костей. Тогда бульон процедить, прибавить пряностей, лимон, нарезанный кружками, и стакан белого виноградного вина, поставить опять на огонь. Отняв от четырех яиц белки, взбить, положить вместе со скорлупами в бульон, мешать беспрерывно ложкою, не давая сильно кипеть, подцветить пережженным сахаром. Потом, сняв кастрюлю с огня, накрыть, положить на крышку угольев и дать стоять, пока уголья потухнут; перевернув вверх ножками стул, у которого нет поперечины, прикрепить к ножкам стула салфетку, процедить галантир; если с одного раза не будет чист и светел, процедить в другой, а когда будет нужно, то и в третий. Налив приготовленного галантира в форму или кастрюлю, слой пальца в два, дать застыть. Между тем вынуть поросенка из воды, отереть салфеткой, выбрать кости, а из головы вынуть мозг. Сварить круто несколько яиц, разрезать каждое на четверо. Один лимон нарезать тоненькими ломтиками. Взять горсть петрушки, ощипать листики со стебельков. Когда налитый в форму галантир застынет, положить несколько кусков поросенка, перекладывая яйцами, лимоном и листьями петрушки, залить галантиром, застудить. Наконец положить остальные куски поросенка, мозг из головы, яйца, лимон и петрушку, залить оставшимся галантиром, поставить на лед, чтоб галантир окреп. Когда нужно подавать на стол, опустить форму на минуту в горячую воду, выложить галантир на блюдо. (6)

\z{Огородная пулярдка}\index{Пулярдка ! холодная}

Два свежих огурца, две штуки сельдерея, две луковицы, пучок эстрагону, пучок укропу, пучок кервелю вымойте, очистите, изрубите все вместе, посолите и наполните этою смесью, через хлуп, заблаговременно освежеванную пулярдку. Затем обверните птицу цельным эстрагоном, обложите цельными половинками молодых огурцов, сельдерею, луковками, всего по равной части, посолите; положите все это в кастрюлю с холодной водой и поставьте кастрюлю на самый легкий огонь, часов на пять, подливая время от времени холодной воды, чтобы она всегда покрывала пулярдку. Это блюдо бывает и холодное, и горячее\footnote{Из лекций доктора Пуффа: <<О кухонном искусстве>>, 1844 г.}. (6)

\z{Индейка, залитая галантиром}\index{Индейка ! холодная ! под галантиром}

Индейка приготовляется точно также, как поросенок, но разница только в том, что ее начиняют иногда за кожу яйцами или фаршем. Разняв индейку на части, из белого мяса приготовить фарш, скатать на подобие колбасы, завернуть в холстину, обвязать ниткой, сварить в бульоне. Прочие части индейки сварить, выбрать кости. Потом взять один лимон, нарезать тоненькими ломтиками, несколько круто сваренных яиц, разрезав каждое на четверо, зеленой рубленной петрушки горсть; фарш нарезать ломтиками или другими фигурами. Налив в форму немного галантиру, застудить, положить на него несколько кусков индейки, фаршу, яиц, лимону, посыпать петрушкой, залить галантиром, застудить; а на этот слой уложить все оставшееся, залить галантиром, поставить в лед. Также заливают галантиром дичину и телячью головку. Вместо телячьих ножек, можно класть в галантир осетровый клей. Для приготовление большего блюда галантира надобно взять 1/4 ф. клею, размочить в холодной воде, разварить и влить в бульон. (6–8)

\z{Винегрет}\index{Винегрет}

Из остатков всякого жаркого можно приготовлять винегрет, прибавив свежих или соленых огурцов, вареного картофелю, свеклы, круто сваренных яиц, обваренных грибов или груздей. Потом облить винегрет следующим соусом: смотря по количеству винегрета, ложку прованского масла, две ложки уксусу, один круто сваренный желток, немного соли, чайную ложку горчицы, стереть все вместе хорошенько и облить винегрет. Другая обливка для винегрета приготовляется таким манером: берут молоки от двух селедок, вымочив их предварительно в уксусе и выполоскав в воде, протирают сквозь сито, прибавляют прованского масла, уксусу и горчицы, стирают все вместе и обливают винегрет. (6–8)

\z{Окрошка}\index{Окрошка}

В летнее время окрошка заменяет суп. Взяв остатки жаркого, какое случится, накрошить мелкими кусочками, также накрошить свежих или соленых огурцов, круто сваренных яиц, прибавить мелко изрубленного зеленого укропу и луку, раковых шеек, если случатся, налить квасом или кислыми щами, посолить, заправить сметаной, дать постоять час.

\z{Холодная белорыбица}\index{Белорыбица ! холодная}

Взяв звено белорыбицы, положить в кастрюлю, налить столько воды, чтоб она покрыла рыбу, посолить, прибавить понемногу английского перцу, лаврового листу, перцу горошинами, две луковицы, натыканные гвоздикой. Приставить кастрюлю на огонь, дать кипеть исподволь, наблюдая, чтоб рыба не переварилась, потому что она очень нежна. Когда поспеет, слить воду, выложить рыбу на блюдо, посыпать рубленной зеленой петрушкой и укропом. К белорыбице подают соус ремуладь; он приготовляется следующим образом: взять два или три анчоуса, выбрать кости, изрубить мелко, прибавить столовую ложку каперсов, также изрубленных; стереть на сахар цедру с 1 лимона, истолочь сахар вместе с 4 яичными желтками, круто сваренными; влить в желтки столовую ложку прованского масла, стереть хорошенько, положить туда же анчоусы и каперсы, развести уксусом. взбивать ложкою полчаса. Отпуская рыбу на стол, соус подавать в особом судке. (Пост.) (6–8)

\z{Студень с белым вином}\index{Студень ! с белым вином}

Взять свиную голову, обваренную в кипятке телячью голову, 4 телячьи ноги и кусок солонины; разварить хорошенько, чтоб с костей отделялось мясо, разрезать на четвероугольные куски, а уши на продолговатые пластинки, процедить бульон сквозь сито, поставить на сильный огонь и варить до тех пор, пока все загустеет. Налить туда 1/4 бутылки белого вина и столько винного уксусу, чтоб масса получила кислый вкус, облупить лимон, разрезать его на куски, сварить, подложить мелко-изрезанную луковицу, крупно-истолченной гвоздики, пряностей, перцу, и сварить. Положить туда мелко-изрезанное мясо, посолить, прибавить 4 яйца, мелко-изрезанные и сваренные в густую, вылить в круглую форму, дать охладеть и вынуть из формы. Если студень не выходит из формы, то ее обертывают тряпкой, намоченною горячей водой. Подают с холодным соусом или с уксусом и маслом. (10)

\z{Окрошка постная}\index{Окрошка ! постная}

Очистить и нарезать огурцов свежих или соленых, маринованных грибов, маринованных соленых груздей, волнушек, рыжиков, яблок свежих и моченых; можно класть сливы, вишни, персики и моченый виноград, сварить и очистить картофель, свеклу и зеленые бобы, сложить все это в суповую чашку. Перед самым обедом положить в каменную посуду немного готовой сарептской горчицы и соли, влить по несколько капель прованского масла, мешая, пока горчица не обратится в густой соус. Потом развести кислыми щами или квасом, положить соли, перцу, зеленого луку, петрушки, укропу, размешать; положить кусок льду.

\z{Холодник польский со сметаною}\index{Холодник ! польский со сметаной}

Горсть укропу и трибульки, т. е. зеленого луку-сеянца, растереть с солью. Взять молодого свекольнику и несколько штук самой мелкой молодой свеклы, вымыть, сварить в соленой воде, отлить на дуршлаг, мелко изрубить, сложить в суповую миску (мелко изрубленного свекольнику должно быть 1 полный стакан); влить 2 — 5 стаканов сметаны самой свежей, развести по пропорции хлебным квасом или кипяченою холодною водою, положить крутых, на несколько частей разрезанных яиц, мелкими четырехугольными кусочками изрезанных свежих огурцов, раковых шеек, ломтики лимону, соли, немного перцу и кусок льду. (6)

\z{Мхали с грецкими орехами (по-грузински)}\index{Мхали}

Шпинат отобрать и вымыть хорошенько, а затем сварить в кипятке. Когда шпинат готов, откинуть его на решето, отжать, изрубить мелко и положить в соусник.
Затем нарезать мелко укропу и петрушки, а грецкие орехи (шт. 5 — 6 в пропорции для 3 лиц) истолочь в ступке, чтоб сделались как тесто. Все это смешать со шпинатом в одну массу, посыпать солью и подавать холодными. (3)

\z{Свежая осетрина}\index{Осетрина ! холодная}

Положив звено осетрины в кастрюлю, налить холодной водой, чтоб вода покрыла осетрину, прибавить две луковицы и немного горошинами перцу, а когда поспеет, выложить на блюдо. Соус приготовить следующий: нарезать кружками луку, поджарить в масле, положить немного муки, развести бульоном, дать повариться полчаса, прибавить столовую ложку уксусу и чайную ложку горчицы, облить осетрину. Малосольную осетрину, сварив в воде, подают с горчицей и уксусом. Также подают к осетрине ботвинью.

\z{Шиганури (азиатское холодное)}\index{Шиганури}

Внутренность лососины, осетрины, белорыбицы сварить, воду сцедить, потом внутренность изрубить мелко-на-мелко с репчатым луком и перцем, посыпать соли, выложить на блюдо, посыпав рубленными укропом и петрушкою.

\z{Форель}\index{Форель ! холодная}

Вычистив и выпотрошив форели, облить уксусом, посыпать солью, дать лежать час, а потом вымыть. Налив кастрюлю водой, поставить на огонь; когда вода закипит, положить две луковицы, лаврового листу, горошинами перцу, зеленой петрушки и укропу, опустить в кастрюлю форель, посолить. Когда форель будет готова, дать остыть в отваре, выложить на блюдо, посыпать рубленной зеленой петрушкой, обложить свежими лимонами, разрезанными пополам или на четверо; подавать с уксусом и горчицею.

\z{Холодная щука}\index{Щука ! холодная}

Выпотрошив щуку, вымыть, но чешую не счищать, сварить в воде с солью. Кто пожелает, можно положить, когда варится щука, луку, кореньев и пряностей. Дав поспеть, оставить в отваре, пока щука немного остынет; тогда, вынув из отвара, снять кожу вместе с чешуей, — она очень скоро и легко снимается, — а щуку положить на блюдо, свернув кольцом, посыпать рубленным укропом и петрушкой; подавать с уксусом и горчицей. Если захотите приготовить щуку с голубым пером, то, очистив чешую, выпотрошить, вымыть и сварить в воде с солью. Когда поспеет, выложить горячую на блюдо, полить уксусом и закрыть другим блюдом.

\z{Холодная щука под соусом}\index{Щука ! холодная ! под соусом}

Сварить в чешуе щуку, потом снять чешую вместе с кожей, и уложить на блюдо. Влить в кастрюлю стакан виноградного вина, положить ложку чухонского масла, мускатного цвету, лимон, изрезав кружечками и выбрав семечки, кусок сахару; дать прокипеть, прибавить столовую ложку деланной горчицы, подбить двумя желтками, облить этим соусом щуку.

\z{Заливная рыба}\index{Рыба ! заливная}

Заливать галантиром можно всякую рыбу, но приличнее для этого осетрина, судаки, лососина, форель и другая крупная и мясистая рыба. Рыбу, которую хотят залить галантиром, приготовив как следует, разнять на части, вымыть, положить с разными пряностями в кастрюлю; сварив до спелости, выложить на блюдо, чтоб рыба остыла. В тот же бульон положить мелкой рыбы, маленьких щучек, окуней, а всего лучше ершей, смотря потому, сколько надобно галантиру. Рыбы всегда должно накладывать полкастрюли; налив водой, варить до тех пор, пока рыба совершенно разварится; тогда процедить сквозь салфетку и, дав постоять, слить в чистую кастрюлю, прибавить две столовые ложки уксусу и стакан виноградного вина, подцветить пережженным сахаром и, дав раз вскипеть, снова процедить сквозь салфетку. Между тем изрезать звездочками моркови, вскипятить один раз; изрезать кружочками один лимон, выбрать семечки; два круто сваренные яйца также нарезать кружочками; накрошить не очень мелко горсть петрушки и укропу. Налив в форму галантиру пальца на два или на три, застудить, а когда застынет, уложить яйцами, морковью, лимоном, посыпать зеленью и положить несколько кусков рыбы, а сверху положить опять слой яиц, лимону, моркови, зелени и залить галантиром. Наконец уложить все оставшееся, как-то: рыбу, яйца, лимон, морковь и зелень, облить остатками галантира, поставить в лед. Можно заливать рыбу просто следующим образом: сварить рыбу в соленой воде с пряностями и кореньями, а когда будет готова, уложить на глубокое блюдо, посыпать рубленной зеленой петрушкой и укропом. Бульон, процедив, уварить, чтоб осталось его только половина, потом облить рыбу и застудить. (8--10)

\z{Холодная рыба}\index{Рыба ! холодная}

Окуней, карасей, судаков, линей, карпов и другую рыбу, сварив в воде с солью, пряностями и кореньями, уложить на блюдо, облить сметаной, смешанной с тертым хреном, или, взяв стакан сметаны и чайную чашку тертого хрену, смешать вместе; подавать в особом судке. Подают также рыбу просто с уксусом и горчицей, посыпав зеленью петрушки и укропом.

\z{Свежая осетровая голова}\index{Осетрина ! холодная ! голова}

Свежую осетровую голову сварить мягко, обобрать кости, обложить соленым лимоном и солеными огурцами, нарезанными кружочками; подавать с горчицей и уксусом. Также можно обложить головизну кислой шинкованной капустой, облить прованским маслом, сбив его с уксусом. Соленую головизну сначала вымочить, а потом сварить, обобрать кости; подавать с уксусом и горчицею, или с тертым хреном. (6)

\z{Майонез из лососины}\index{Лососина ! майонез}

Возьмите такой кусок лососины, какой вы находите нужным для того числа гостей, которое будет за столом. Лососину вычистить и вымыть как можно лучше, и разрезать на куски шириною пальца в два каждый. Сварив рыбу в соленой воде, дайте ей простыть; снимите кожу с кусков и расположите их тщательно на блюде; середину уберите латуком, свежими огурцами с приличным соусом из прованского масла и горчицы, и обсыпьте куски лососины мелко толченным перцем\footnote{Из записок придв. метр-д'отеля Эдм. Эмбера, в 1844 г.}.

\z{Ботвинья}\index{Ботвинья}

Летом подают иногда вместо супа ботвинью с рыбой. К ботвинье лучше всего идет малосольная осетрина; также подают к ботвинье малосольную белугу, севрюгу и другую соленую рыбу, a где можно достать — свежую осетрину и лососину. Ботвинья приготовляется таким образом: взять какой угодно зелени, щавелю, шпинату или свекольнику, сварить в воде мягко, откинуть на сито и выжать воду; потом изрубить, прибавить свежих, мелко искрошенных огурцов, зеленого рубленного укропу и луку. Если есть раки, то, очистив потребное количество раковых шеек, изрезав, положить в ботвинью, налить квасом или кислыми щами. Приготовляют еще ботвинью из свеклы: сварив свеклу, вычистить, изрубить мелко, прибавить рубленного луку и укропу, накрошить свежих или малосольных огурцов.

\z{Раки}\index{Раки}

Вымыв раков, положить в кастрюлю или горшок, налить холодной водой, посолить, закрыть и сварить до спелости. Потом откинуть на сито, а когда стечет вода, уложить на блюдо. Если раков вымочить в молоке, то они делаются вкуснее. Положив в удобную посудину, налить молоком и дать стоять часа 2; сварить обыкновенным манером.

\z{Майонез из рыбы с салатом и зеленым соусом}\index{Рыба ! майонез ! с салатом}

Взять фунта 3 какой-нибудь большой рыбы, как-то: линя, щуку, лососину, больших окуней или сигов, срезать филеи ломтиками в палец толщиною; уложить на противень на 1.5 ложки растопленного масла, посолить, скропить 1 рюмкой вина и соком из 0.5 лимона; как только с одной стороны побелеет, перевернуть на другую: смотреть, чтобы рыба не была сыра, но и не поджарилась бы до темного цвета, потом переложить ее на круглое блюдо, остудить.

Из костей же рыбьих, чешуи, рыбьего клея и прочей мелкой рыбы сварить ланспик, положить в него кореньев и пряностей, уксусу или лимонного соку, 5–6 шампиньонов, очистить 2–3 яйцами или икрою, уварить до 3.5 стак., процедить сквозь салфетку, остудить, потом бить веничком, вливая 2 ложки прованского масла. Каждый кусок рыбы обмакнуть в этот мусс, уложить на блюдо в кружок, обложить салатом, а середину наполнить следующим соусом:

Из 10 оливок вынуть косточки, 10 анчоусов, 10 корнишонов, 1 ложку каперсов, 10 отваренных желтков изрубить, истолочь все вместе в ступке, протереть сквозь сито; один сырой желток мешать в каменной чашке, пока не побелеет, развести 0.5 стаканом прованского масла и 0.5 стаканом уксусу, положить сахару куска 2–3, ложку мелко изрубленной зелени, как то: укропу, кервелю, эстрагону, соли и эссенцию из шпината, смешать все вместе.

Салат же следующий: нарезать ровными ломтиками и сварить в соленом кипятке до мягкости зеленых бобов; когда будут готовы, отлить на дуршлаг, перелить холодною водою, поставить на лед; точно также сварить спаржи, картофелю, цветной капусты, перелив их холодною водою, смешать с бобами, положить туда же свежих огурцов, печеной свеклы (все это нарезать ломтиками и кусочками), зеленой петрушки, укропу, эстрагону, кервелю, заправить все это ложками 2 — 3 уксусу, прованским маслом, солью и перцем.

\z{Холодное и окуней по-оренбургски}\index{Окунь ! холодный ! по-оренбургски}

Взять крупных свежих окуней штук 10 или более, смотря по числу особ, которым придется кушать это холодное; выпотрошить каждую рыбу чрез жабры, не разрезывая брюшка и не повреждая головы, а после того, дав ей несколько полежать без воды, подварить ее немножко в приправленной пряностями и присоленной воде, стараясь, чтобы все окуни не доваривались. Покипевшую таким образом рыбу осторожно очистить от всей чешуи и от плавательных перышек. Между тем, сварить несколько яиц в густую и достаточное количество свеклы и моркови, изрубить белки и желтки особо, и точно также поступить и с сваренною свеклою, морковью и еще петрушкою, которую сперва выварить в горячем прованском масле, чтобы отчасти отнять у нее остроту, а потом дать ей стечь на сите. Всеми этими приправами (каждою особо) обложить недоваренных окуней, в виде косвенных полосок, в палец шириною, и, в заключение, разместить на них, изящно, несколько вареных раковых хвостиков и раковой икры. Окончивши это, осторожно облить украшенную рыбу простывшим окуневым или другим рыбным отваром, который сперва загустить немного раствором рыбьего клею, и приправить белым столовым вином и водяным настоем душистых специй (корицы, гвоздики, ямайского перцу и т. п.), и наконец, застудивши желе, подавать на стол.

\z{Ботвинья с огурцами и малосольной лососиной}\index{Ботвинья ! с огурцами и малосольной лососиной}

Нарезать мелко очищенных огурцов четырехугольными кусками, положить в кастрюлю и поставить на лед. Очистить, вымыть и сварить щавель в соленой воде, протереть сквозь частое сито и поставить в холодное место. Очистить немного молодого шпинату и столько же молодого свекольнику, вымыть, обланшировать до мягкости в кипятке и, когда будут готовы, отлить в холодную воду, отжать и изрубить мелко. За 15 минут до отпуска сложить все в одну кастрюлю, развести процеженными кислыми щами и квасом, снабдить по вкусу солью, перцем и положить мелко изрубленной зелени, т. е. эстрагону, кервелю, укропу и шарлоту, и опустить несколько кусков чистого льду. (4)

\z{Винегрет из рыбы}\index{Винегрет ! из рыбы}

Снять филеи с назначенной для винегрета рыбы, подрезать верхнюю кожицу и, сложив на растопленное в сотейнике масло, поставить на огонь; когда филеи поджарятся, т. е. побелеют, перевернуть, дожарить окончательно, снять горячими с сотейника на блюдо и остудить. Рыбьи головки, кости и обрезки сложить в кастрюльку, влить немного воды, посолить, положить кореньев и пряностей и сварить бульон. Когда будет он готов, процедить, положить ложку уксусу, 6 пластинок желатину, вбить один белок, поставить на огонь и мешать; а когда вскипит, отставить на легкий огонь и варить, пока ланспик очистится и будете достаточно крепок; тогда процедить половину, застудить на льду, а остальной оставить на столе не застывшим. Между тем изрезать очищенные огурцы, маринованные грибки, корнишоны, положить ложку каперсов, 10 шт. оливок без косточек, сложить все это на сотейник, влить 3 л. прованского масла, 1 л. уксусу, по вкусу соли и перцу, поставить на лед, влить несколько ложек ланспику и, когда начнет застывать, прибавить снова ланспику и рубленной зелени, застудить вторично, положить рыбные филеи, размешать осторожно, застудить окончательно, выложить на средину блюда горкой, без всякого фасона, и кругом обложить нарезанными из остального ланспика крутонами. (4)

\z{Свежая осетрина на холодное}\index{Осетрина ! холодная}

Сварить звено осетрины с солью, дать ей остынуть, нарезать кусками, положить в каменную чашку; первый ряд залить ланспигом и положить корицы, гвоздики, горошчатого перцу, лаврового листу, кружков свежего лимону; потом другой ряд осетрины, который также залить ланспигом и укласть пряностями, и таким образом продолжать это до тех пор, пока чашка будет полна; после того поставить в холодное место и дать остынуть. Ланспиг приготовляется следующим образом: выпотрошив сколько нужно окуней, вымыть их и варить в бульоне, в котором варилась осетрина; бульон процедить и прибавить к нему разваренного в воде рыбьего клею (по З.5 золотника на 10 окуней), стереть в чашке около 1/4 фунта икры (на 10 окуней), влить белого вина, ренского уксусу, положить разных духов, лаврового листу, развести бульоном, поставить на огонь и мешать лопаткою; когда все вскипит, накрыть крышкою и, спустя несколько времени, процедить сквозь салфетку. (8)

\z{Аспик из рыбы натурально}\index{Аспик ! из рыбы}

Очистить средней величины судачки, снять филеи так, чтобы мякоти при костях ничего не осталось; подрезать верхнюю кожицу и сложить филеи на тарелку. Кости, головки и все рыбные обрезки сложить в кастрюлю, прибавить 5 шт. ершей, влить холодной воды, не более 5-ти стаканов, опустить один белок и мешать ложкою на огне, чтобы ко дну кастрюльки не пристало. Когда закипит, сдвинуть на легкий огонь, положить кореньев и пряностей и варить до тех пор, пока бульон очистится и получит хороший вкус; тогда процедить сквозь салфетку, опустить в него рыбные филеи, поставить на огонь, и как только он вскипит и филеи всплывут на верх, снять с огня и оставить покрытым. Потом выбрать осторожно филеи цельными, уложить их в глубокое блюдо, залить собственным бульоном, вынести в холодное место и застудить, как быть должно ланспику. Подать с блюдом на стол.

\z{Щука по-жидовски}\index{Щука}

Взять щуки 4 фунта, разрезать вдоль хребтовой кости, вынуть из нее молоки жир и кровь (если будет икра, выкинуть вон), положить их в банку, со щуки же очистить кожу, разрезать ее поперек кусками в четыре пальца ширины, посолить в каменной чашке без воды и дать стоять 8 часов; после этого вымыть щуку, в трех водах, вырезать из всех кусков между костью и кожею мясо, изрубить, истолочь, положить половину французского белого хлеба, 1/2 фунта чухонского масла; 20 луковиц очистить изрубить как можно мельче, смешать с потолченным мясом молоки и кровь также потолочь, всыпать ложку толченного простого перцу, смешать все вместе и фаршировать порезанные куски в прежний вид сложить в рыбный котел кусок к куску, влить холодной воды, дать щуке в ней окрепнуть, потом поставить на плиту, подбавить кипяченой воды. Два золотника шафрану заварить одним стаканом кипятку поставив в теплое место часа на два; когда настоится, процедить в щуку, щуку же сварить до готовности, наблюдая, чтобы не пригорела ко дну, переложить на блюдо, облить тем же соусом. К такой щуке подается тертый хрен, разведенный сырым бураковым рассолом. (6) 


\z{Заливное из поросенка}\index{Заливное ! из поросенка}

Одного небольшого, но жирного поросенка зарезать, опустить в холодную воду потом тотчас в кипяток минуты на две, потом ощипать всю шерсть, оскоблить ножом, вымыть, осушить, опалить тотчас выпотрошить, вымыть как можно лучше, разрезать частей на семь, положить в кастрюлю, налить водою, положить кореньев, пряностей соли, варить на легком огне до готовности; вынуть на блюдо, выбрать осторожно все кости, положить под пресс, остудить, потом нарезать ровными кусками. Кости же положить в бульон обратно можно прибавить 1 зол. клею, влить уксусу, положить яйца, уварить стаканов до 4; кто хочет, поджечь сахаром; процедить все это сквозь салфетку, залить сложенного в форму поросенка, застудить. (6--8) 

\z{Заливной гусиный огузок}\index{Заливное ! из гуся}

Отрубить от огузка длинные кости, разварить его домягка, снять пену посолить и положить луковиц и пряностей; можно также прибавить изрубленные телячьи ножки. Процедить бульон сквозь сито в кастрюлю. Положить туда 1/4 ф. оленьего рога, поварить немного и налить туда крепкого винного уксусу и соку от одного лимона. Когда бульон прокипит, процедить его еще кипящий сквозь сито на огузок, лежащий в каменном горшке, и дать остынуть. По охлаждении, креп ко обвязать горшок свиным или телячьим пузырем и сберегать до употребления. Вместо воды, огузок можно варить в пивном уксусе. 

\z{Галантин из индейки}\index{Галантин ! из индейки}

Разрезав или распластав индейку, выбирают из нее кости и вырезывают мясо, так, чтобы не тронуть кожи; потом рубят мясо с свиным салом, телятиной и ветчиной, и варят в горшке; наконец, обвернув кожу белым полотенцем, кладут в нее слой начинки, другой слой ветчины, красных языков или земляных орехов, маленьких корнишонов; это продолжают делать до тех пор, пока не наполнят индейку; затем, завернув в полотенце, зашивают и обкладывают шпеком и разными мясами. Приготовленный таким образом галантин из индейки, приправляют студнем и подают на ст. 

\z{Майонез из кур}\index{Майонез ! из кур}

Очистить, вымыть и посолить молодых кур, отварить в брезе до мягкости, и отставить в холодное место; потом обрезать ножки, крылья и белое мясо, замариновать уксусом и прованским маслом и поставить снова в холодное место. За 0,5 часа до отпуска сложить правильно на блюдо, залить майонезом, обложить крутонами из ланспику, а в середину положить салат для холодного. 

\z{Заячий хлеб}\index{Заячий хлеб}

Очистить мясо зайца от костей, изрубить мелко и истолочь в каменной ступке, прибавив туда 2 фунта свиного сала, которое также должно сперва мелко искрошить, чайную чашку истертой булки, соли по пропорции, 2 чайные ложки английского мелкого перцу и русского перцу 1 ложечку, 1 истертый мускатный орех; вымешать эту массу хорошенько, вбить туда 8 свежих яиц, всыпать чайную чашку тертого голландского сыру или пармезана и, дав массе форму каравая, обложить тестом, из 3 стаканов муки с яйцами, и испечь в печи. Потом остудить и резать кусками холодное, а кушать с уксусом. (10--12) 

\z{Белорыбица под майонезом}\index{Белорыбица ! под майонезом}

Очистить часть белорыбицы, сложить в кастрюлю, обложить петрушкою, пореем, сельдереем и луком, налить холодною водою, прибавить пряностей и соли, и сварить на легком огне; когда будет готово, остудить бульон, вскипятить, как должно быть ланспику, процедить в кастрюлю, подлить по вкусу уксусу и прованского масла, сбивать на льду веничком, пока майонез не будет бел; за 1/4 часа до отпуска, очистить рыбу от кожи, нарезать ломтиками, уложить на блюдо, залить процеженным майонезом и обложить салатом для холодного. (6—8) (Блюдо постное и скоромное). 

\z{Филей из ростбифа на холодное}\index{Филей ! из ростбифа}

Изрезать пластами холодный ростбиф, посолить и, разложив на лист, поставить в холодное место; после того сделать соус-равигот и за 10 минут до отпуска обмакивать каждый пласт филея в горячий равигот; а когда простынет, сложить на блюдо, обложить крутонами и наполнить середину салатом для холодного. 

\z{Заливное из телячьих ножек}\index{Заливное ! из телячьих ножек}

Очистив телячью голову и ножки, положить в кастрюлю, налить водой и варить 5 часов; потом процедить бульон через сито и вынести на лед. Когда он уже хорошо застынет, опять внести его, выложить в кастрюлю, положить ароматных трав, жженого сахару для цвета. 0,5 стакана уксусу, 0,5 стакана белого вина и 3 яйца вместе с скорлупою для очищения желе. Вскипятить все это несколько раз, процедить через салфетку и заливать этим дичь или телятину. Можно также снять и изрубить все мясо с телячьей головы и залить все это. Ежели будет мало, прибавить говядины. 

\z{Заливное из поросенка}\index{Заливное ! из поросенка}

Нарезать небольшими кусками переднюю четверть поросенка, обмыть их теплой водой и поставить вариться, налив столько холодной воды, чтобы все куски были ею покрыты. Потом, когда пена будет снята, положить 6 целых луковиц, немного белого перцу зернами, соли, лаврового листу, и варить до тех пор, пока куски будут совсем мягки. Тогда распустить немного соли в холодной воде, выложить туда вилкой все куски, а с бульона, в котором они варились, снять чисто жир, процедить его, влить стакан крепкого уксусу, прокипятить вместе, очистить белком яйца и пропустить чрез фланель. Потом вынуть куски из воды, положить в форму или в кастрюлю, кожей вниз, с боков облить бульоном, поставить в холодное место, чтобы все застыло, и тогда выложить на блюдо. Для красы, можно убрать, по вкусу, низ и бока формы вареной свеклой, морковью, листьями петрушки и кусочками лимона. (5--6) 

\z{Навага заливная}\index{Заливное ! из наваги}

Счистить с наваги кожу, выпотрошить ее, вымыть, просолить солью с перцем и перетереть салфеткой; завалять в кляре и тертом хлебе и обжарить в масле; когда простынет, залить дно плафона ланспиком и, убрав лимоном, раковыми шейками, зеленью, зеленым лавровым листом, положить сверху уборки навагу, залить ланспиком, чтобы навага не была видна сверху, и застудить; а когда застынет, выложить на круглое блюдо и убрать штучками из ланспика. Особо подавать в соуснике подливку для рыбы или татарский соус. (5--6)

\z{Свиной студень по-малороссийски}\index{Студень ! по-малороссийски}

Взять свиные ножки, рыло и уши, очистить; варить 1/4 часа в воде, потом остудить в холодной воде, переложить в горшок, вылить бутылку белого вина, рюмку уксусу, бутылку воды, положить 10 луковиц с натыканной гвоздикой, поставить на легкий огонь и дать упреть. В форму положить коринку, гвоздику, по местам выложить ломтиками свежего лимона; на это положить мясо, разрезанное в мелкие куски, вылить студень сквозь сито, застудить. 

\z{Ветчина с холодным соусом}\index{Ветчина ! с холодным соусом}

В котел с сухими травами: чабром, базиликом, майораном, прокипятив окорок ветчины, до всплыва его поверх воды (что будет признаком его спелости), прибавить 10 штук вареного картофелю, протертого сквозь сито, смешать в кастрюле с 1/4 ф. прованского масла, 10 вареными протертыми сквозь сито яичными желтками, мешая до смягчения и забела; присоединить горчицы, каперсов рубленых, петрушки и укропу, соку из лимона, истереть мускатный орех. присолить, разжидить хорошим соусом и засыпать ветчину (от которой должно отделить все неудобное к варению, обрезать кожу и жир) тертыми сухариками; подавать изрезанную на круглом блюде, в средину которого выложен означенный соус. (8--10)
