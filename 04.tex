\section{О ЧИЩЕНИИ И ПРИГОТОВЛЕНИИ ДВОРОВЫХ И ДИКИХ ПТИЦ К ВАРЕНИЮ И ЖАРЕНИЮ, А ТАКЖЕ ВСЯКОГО МЯСА, РЫБЫ И КОРЕНЬЕВ}

Главное достоинство припасов заключается в том, чтоб они были свежи и хорошо приготовлены. Каждое кушанье должно быть изготовлено в пору, не перепарено, не пригорелое и не сырое. Иногда из самых лучших припасов приготовляют невкусное кушанье. Нужно, чтоб кушанье было не только вкусно, но имело бы приятный вид.

\subsection*{О чищении и приготовлении всякой птицы и вообще всех родов мяса.}\addcontentsline{toc}{subsection}{О чищении и приготовлении всякой птицы и вообще всех родов мяса.}

Надобно наблюдать при чищении дворовой и дикой птицы, чтоб она хорошо была вычищена, опалена, выпотрошена и вымыта. Индеек, кур и цыплят чистят иногда не обваривая, но обваривать лучше, ибо скорее и чище можно вычистить. Заколов птицу, должно, положить в холодную воду на полчаса или на час, если время позволяет, a потом, вынув, дать отечь; между тем налить в лоханку кипятку и обмакивать в него птицу таким образом, чтоб все перья у нее были смочены; затем попробовать, хорошо ли она чистится, и тогда вынуть ее из горячей воды; иначе кожа на ней сварится. Чистить надобно проворно. Когда вся живность перечищена, дав ей обсохнуть, опалить соломой или бумагой, наблюдая, чтоб только опалить волосы, но не ожечь кожи; а наконец, выпотрошить, остерегаясь раздавить желчь. У цыплят оставляют иногда ножки и голову, но у кур и индеек отрезают. Выпотрошив, вымыть в чистой воде и вытереть пшеничными отрубями. Гораздо лучше приготовлять птицу для употребления за сутки; этого времени довольно, чтоб живность была мягка и годна к употреблению. Зимой вынести в такое место, где бы она не могла замерзнуть, a летом на лед, завернув в мокрое полотенце. Если хотите, чтоб птица лежала несколько дней, в таком случае чистить ее должно сухую, не обваривая, потом выпотрошить и, не паливши, вынести на погреб. Когда захотите придать птице отличный вкус, то накануне ее употребление должно, вымыв в чистой воде, положить в удобную посудину, налить свежим молоком и поставить в холодное место. Молоко для этого можно употреблять снятое. Мочат в молоке индеек, кур и цыплят, также поросенка. Гусей и уток чистят всегда сухих; ощипав сначала перья, потом должно ощипать пух и, опалив, выпотрошить, вымыть в нескольких водах, вытереть отрубями и положить в холодное место. Всякая птица может лежать в холодном месте трое или четверо суток, без всякого повреждение, но не более; а потому не должно заготовлять живность задолго до употребления, особливо в летнее время.

Дичину, ощипав, опалить, выпотрошить, вымыть и, посыпав внутри и снаружи солью, облить уксусом, положить во внутренность птицы несколько лавровых листков, немного горошчатого перцу, поставить в прохладном месте на сутки, или, по крайней мере, на одну ночь. Приготовлять таким образом можно тетерек, драхву и глухих тетеревов; но рябчиков, куропаток и стрепетов обливать уксусом не нужно, а просто, ощипав, выпотрошить, вымыть и вынести в холодное место.

Когда поросенка заколют, то, дав вытечь крови, положить его на полчаса в холодную воду, а потом опустить минуты на две в горячую, наблюдая, чтоб не заварить кожи, а если заваришь, то, когда будешь чистить, шерсть станет вырываться с кожей. Чтоб лучше счищалась шерсть, надобно положить в горячую воду немного золы. Вынув из горячей воды, счистить с поросенка шерсть, и, если поросенок хорошо вычищен, опаливать его не нужно; но, когда будет поросенок опален, должно вымыть его теплой водой и вытереть пшеничными отрубями. Если хотят улучшить в поросенке вкус, то, выпотрошив и вымыв чисто, положить в холодное место часа на два, а потом налить молоком и дать стоять до следующего утра в холодном месте. Поросенок, приготовленный таким образом, отменно вкусен и мясо у него делается гораздо нежнее.

Зайцев обыкновенно приготовляют следующим образом: сняв шкуру, содрать осторожно ножом все перепонки, потом выпотрошить, вымыть в нескольких водах и вымачивать в холодной воде часов шесть; в продолжение этого времени воду переменить раза три. Кому угодно, можно на ночь положить зайца в удобную посудину, налить столько уксусу, чтоб он покрыл его, туда же положить английского и простого, горошинами, перцу и несколько лавровых листков.

Телятину, говядину и баранину намачивать с вечера никогда не должно, как это многие делают, потому что она вымокает и теряет часть своей питательности; разве в таком случае, когда зимой говядина бывает мерзлая; тогда, положив в лоханку, налить холодной водой, поставить в прохладное место; в продолжение ночи она оттает и будет как парная. Каждое мясо мочить должно отдельно; если мочить телятину вместе с говядиной, телятина от этого будет красная, а когда телятину мочат вместе с бараниной, тогда она принимает неприятный вкус и запах баранины. Свежее мясо мочить поутру не более часу, а потом вымыть в холодной воде; многие даже совсем мясо не мочат, а только моют в холодной воде.

Если случится бить дома теленка, барана или корову, то кишки и рубец приказать тотчас выполоскать в холодной воде, переменяя несколько раз воду, а потом обдать горячей водой и выскрести ножом. Когда будут вычищены, перемыв хорошенько, налить холодной водой поставить в погреб. Кишки и рубец лежать долго не могут, а потому их и должно употреблять в пищу в непродолжительном времени.

Телячью головку и ножки, ошпарив кипятком, вычистить. Точно также поступать с бараньими головками.

Бычачьи ноги сначала должно опалить, а потом обдать кипятком выскрести ножиком, вымыть и, налив холодной водой, поставить на ночь в погреб. Бычачья ноги обыкновенно употребляют для студня.

Потроха домашних птиц употребляют в разные кушанья; из них приготовляют вкусный суп, готовят соусы, кладут в паштеты и круглые пироги, делают фарш; об этом будет сказано в приготовлении кушаньев. Вынув потроха, осторожно вырезывают желчь. У пупков вырезав по бокам перепонку, разрезать их надвое, но не совсем; потом обварить кипятком и содрать кожу, которая внутри пупка. Гусиные и утиные лапки, обварив кипятком, содрать с них кожу, а когти отрубить. Лапки куриные, индеичьи и от цыплят также обдать кипятком и содрать кожу, но их не подают к столу, а кладут в бульон, для навару. Печенку и сердце вымыть в холодной воде.

Таблица, показывающая, сколько дней какое сырое мясо может без порчи сохраниться в прохладном месте, защищенное от насекомых.

\begin{table}[h]
	\begin{center}
		%\caption{Обозначения масс} \label{tab:tab1}
		\begin{tabular}{lcll} 
  		  &  & Летом & Зимой  \\ 
  		Олень, лось и дикая коза & . . . . . & 4 дня & 8 дней  \\ 
        Заяц & . . . . . & 3 дня & 6 дней\\  
        Кролик & . . . . . & 2 дня & 4 дня\\  
        Фазан & . . . . . &	4 дня & 10 дней \\  
        Тетерев & . . . . . & 6 дней & 14 дней \\  
        Куропатка & . . . . . & 2 дня & 6-8 дней \\  
        Говядина и свинина & . . . . . & 3 дня & 6 дней \\  
        Баранина & . . . . . & 3 дня & 6 дней \\  
        Телятина и барашек & . . . . . & 2 дня & 4 дня \\  
        Индейка, утка и гусь & . . . . . & 2 дня & 6 дней \\  
        Каплун или пулярдка & . . . . . & 3 дня & 6 дней \\  
        Старая курица & . . . . . & 3 дня & 6 дней \\  
        Цыпленок & . . . . . & 2 дня & 4 дня \\   
        Голуби & . . . . . & 2 дня & 4 дня \\ 
		\end{tabular}
	\end{center}
\end{table}

В дождливое или слишком жаркое время вполовину менее.

\subsection*{Искусство мясоразрезывания.}\addcontentsline{toc}{subsection}{Искусство мясоразрезывания.}

Передав все подробности о том, как ознакомиться со всеми сортами и частями разных родов мяса, — считаем целесообразным перейти теперь же к предмету, тесно связанному с этим делом; а именно к правильному разрезыванию всякого мяса, потому что искусство вполне правильно разрезывать мясо много содействует к успеху хорошо приготовленного обеда; от этого успеха зависите, чтобы жаркое имело надлежащую нежность, утрачиваемую на половину, если хозяйка или кухарка не умеют разнять правильно различные его части. Поваренное искусство имеет свои анатомические правила, своевольное отступление от которых не только делает лучшее мясо жестким, но даже придает ему довольно неприятный вкус.

Предпослав это введение, укажем на подробности самого дела, начав, разумеется, с говядины.

\subsubsection{Говядина.}
\begin{enumerate}
	\item Огузок, который подается и вареный, и жареный, требует большой осторожности в разрезывании кусков; его постоянно должно резать не по нитке, т. е. вкось и вглубь тонкими и длинноватыми кусками; если вы заметите, что нитка стала идти прямыми линиями под вашим ножом, перемените немедленно его направление. Мясо, подходящее ближе к хвосту, самое нежное, сочное и жирное.
	\item Лопатка, употребляемая преимущественно в супы, режется тонкими ломтиками и наискось.
	\item В грудине выбирают обыкновенно ту часть, которая находится ближе к сухой жиле, и разрезывают ее так, чтобы в каждом куске был хрящик, составляющей достоинство самого куска, употребляемого во щи и борщ.
	\item Филей требует особенного внимание, составляя лучший кусок из всей туши и доставляющий, кроме вкусного жаркого, драгоценную вырезку или нежный, продолговатый кусок мяса с жиром, который покупают отдельно за тройную цену, чтоб приготовлять настоящий английский бифстекс. Покупая большой филей для жаркого, вы получаете с ним вместе и эту драгоценную вырезку, которую (если вы захотите поэкономничать) прикажите повару или кухарке осторожно выделить и отложите до времени; ребра с их мясом подайте на стол; но чтоб ваша хозяйственная хитрость не была заметна и не портила бы фигуры всего жаркого, отдайте приказание, чтобы, прежде чем положить мясо в кастрюлю, повар свернул его и обвязал голландскими нитками. Филей разрезывается вкось, при чем держать нож как можно более в отлогом положении, отчего выходят куски тонкие и длинноватые.
	\item Язык бычачий должно резать всегда ломтями и наискось; самое толстое место, по близости горла, есть и самый нежный кусок во всем языке.
\end{enumerate}

\subsubsection{Телятина.}
Задняя четверть, или окорочок с почкою, составляет отличное жаркое; режут его обыкновенно сверху, тонкими, косыми ломтями; но иные хозяйки, зная, что число обедающих особ не так велико, чтобы все жаркое могло быть истреблено, и, не желая портить совершенно вида окорочка, приказывают резать куски сбоку, где самая мясистая часть; только эти куски, резанные по прямому направлению нитей, бывают обыкновенно несколько жестче и, так сказать, мочалистее; поэтому должно стараться избирать и тут более косвенное направление, наблюдая, чтобы студенистая жила, которая проходит в этом месте телячьего мяса, шла под ножом несколько вкось. Ребра, подходящие к главной кости ноги, должны быть предварительно надрублены мясником по суставчикам, и тогда очень легко разнимать их ' столовым ножом. Почка отделяется от жаркого, когда телятину подают на стол: она изрезывается наискось тонкими ломтиками, которые укладываются венчиком под прикрытием той жирной части, из которой выделена почка. Это делается для того, чтоб ломтики не успели остынуть. Около почки, в глубине, между слоями жира, лежит небольшой кусочек чрезвычайно нежного мяса, соответствующий вырезке в говяжьем филее и называемый сладким мясом. Его обыкновенно хозяйка предлагает самому почетному гостю.

Передняя часть телятины состоит из лопатки, котлет, грудины и шейки. Лопатку употребляют на жаркое следующим образом: всю мясистую часть осторожно отделяют от большой, плоской, в виде лопаты, кости и, скатав как круглую кишку, обвязывают голландскими нитками, чтоб кусок не мог развернуться в кастрюле. Подавая на стол, жаркое это разрезывают почти прямыми, но очень тонкими ломтиками. По причине сухости мяса своего, лопатка может быть употреблена для жаркого только таким образом, т. е. рулетом; в духовой же или русской печи она делается безвкусна и суха.

Грудина не идет на жаркое; ее отваривают и употребляют на белый соус, в который она опускается уже разрезанная на куски по хрящеватым суставчикам; кости, покрытый суховатою кожею, быв предварительно надрублены мясником, отделяются ножом и присоединяются к бульону.

Об разрезываньи котлет говорить нечего: это видно по отдельным косточкам, из которых каждая составляет особенную котлету.

Шейка употребляется только в бульон и иногда на фрикадельки; но по натуральной ее жиловатости лучше не прибегать к этой экономии, а только довольствоваться ее наваром для супа.

Телячья голова подается под сладким или кислым соусом, а иногда употребляется в студень, и так как в последнем случае все мясо с костей снимается, чтобы быть уложенным в форму, а мозги же разрезываются на ломти и укладываются по краям формы, то мы будем упоминать о разнимании телячьей головы на части только в том случае, если ее подают на стол в целом виде с присоединением какой-нибудь подливки. Во-первых, должно осторожно открыть череп, чтобы можно было легко доставать мозги из углубления, в котором они лежат. Самыми лучшими кусками считаются: уши, щеки, язык и глаза. Хозяйка, отделяя каждый из этих кусков, перекладывает в то время на тарелку некоторую часть мозгов, чтоб снабдить ими в равном количестве всех гостей. Язык, вынутый заранее на кухне, разрезывается на тонкие ломтики и укладывается от лба до носа. Зрачки из глаз отделяются и их заменяют черносливинами, если подливка сладкая, и сливами, если она кисловатая. Корень глаза, облитый жиром, составляет нежный и лакомый кусок, равно как и основание уха; поэтому, вырезывая эти части, должно запускать нож в глубину, а не отделять только их поверхность.

\subsubsection{Баранина.}
Она разнимается почти также, как телятина, но филейная ее часть, самая нежная, жирная и вкусная, должна быть нарезана в виде лент тонкими ломтиками, наискось.

\subsubsection{Поросенок.}
Жареный или вареный, он разнимается совершенно одинаково, т. е. голова разрезается пополам на две равные части; каждую ногу отнимают отдельно, спинку рубят вдоль хребтовой кости и потом разрезают каждую половину поперег на четыре или пять частей, смотря по величине поросенка. Лучшими кусками считаются: задние ножки, место около шеи и также близ хвостика.

\subsubsection{Живность и дичь.}
В дворовых птицах, как-то: курах, утках, гусях, цыплятах, индейках, лучшими кусками считаются: крылышки, душка, ножки, бочки, гузка и пупок; но в жарком почетное место преимущественно занимаюсь: бочки, крылышки и душка, называемые белым мясом; в супе же эти куски делаются сухи, потому что подвергаются продолжительному кипенью, которым невольно устраняется их сочность и мягкость. Вот почему в супах всего вкуснее и приятнее ножки дворовой птицы, в особенности при гусиных потрохах они составляют лакомый и любимый кусок знатоков. Полными потрохами называются: голова, шейка, две ноги, два крыла, пупок и печенка. У головы отрубают нос и слегка надкалывают ножом череп, чтобы не трудно было вынуть мозг, чрезвычайно жирный и вкусный в гусе; шейка разрезается на три или четыре куска, по весьма заметным суставчикам; ножки подают цельными, обрубив ногти; в крыльях оставляют обе продольные кости; пупок и печенку режут на тонкие ломтики, и последняя составляет лакомство истинного гастронома.

Всякая дворовая птица, большая или малая, совершенно одинаково разнимается на столе; для этого нет особенных правил, а нужна только некоторая сноровка и точное исследование перегибов и суставов, ясно показываемых самою природой.

Вот как должно поступать при этой анатомической операции: воткнув левою рукой вилку в крыло, стоить правою рукой нажать нож на суставе и наклонить в то же время левую руку: крыло отделится и останется на вилке; тогда, сделав с той же стороны удар ножом по жилке ножного сустава, отделить ножку таким же наклонением левой руки, вооруженной вилкою; далее приступают к отнятию крыла и другой ножки, затем отрезывают гузку в некотором расстоянии от ножек и, наконец, отделяют душку, а потом бочки.

Тетерьки и глухари режутся точно также; но мелкую дичь, каковы, например, серые куропатки, рябчики, дикие утки, пижоны, разрезывают обыкновенно вдоль на три равные части. Иные же хозяйки рассекают их на четыре части, накрест; но это вовсе неправильно.

\subsection*{О приготовлении рыбы.}\addcontentsline{toc}{subsection}{О приготовлении рыбы.}

Рыба, какого бы она ни была рода, лучше, если живая; можно употреблять в пищу и сонную рыбу, но только она должна быть свежая, что можно узнать по глазам и жабрам. Когда глаза светлые и не впалые, а жабры имеют свежий цвет, то это значить, что рыба свежа. Зимою, по большей части, употребляют замороженную рыбу; если рыба ловлена в заморозы и тотчас заморожена, то она очень хороша для кушанья и мало имеет разницы от рыбы живой. С замороженною рыбой поступают следующим образом: мерзлую рыбу, положив в лоханку, налить холодною водой, дать ей стоять до тех пор, пока она отойдет и сделается мягкая; тогда можно ее чистить.

Сперва с рыбы счищают чешую; но многие рыбы не имеют чешуи, как например: угри, налимы, стерляди и некоторые другие. Кожа у этой рыбы покрыта густою слизью, которую нельзя счистить ножом. Чистят ее так: положив рыбу в удобную посуду, облить горячею водой, поворачивая на все стороны, а потом счистить слизь ножом и вымыть в холодной воде. Со стерляди, при чищеньи, должно сбить мелкую чешую, которая бывает у нее на брюхе; а чешую, находящуюся на спине, не счищают. При обливании кипятком, следует наблюдать, чтобы вода была не слишком горяча, и не долго держать стерлядь в воде, иначе кожа на ней будет лупиться. Точно также чистят камбалу, налимов, угрей и другую рыбу, на которой много слизи. С другою рыбой поступать обыкновенным образом, т. е. оскрести с нее чешую. Когда будешь потрошить, надобно остерегаться раздавить желчь; если желчь раздавят, рыба принимает горький вкус. Приготовляя рыбу на холодное или под соусы вареную, иногда чешую с нее не счищают, а только выпотрошат; так приготовляют щук, окуней и другую рыбу.

Часто на больших обедах подают целого осетра, или большую стерлядь, за которую заплачено несколько сот рублей, то при чищении должно остерегаться, чтоб ее не испортить и не уменьшить ценности. Осетра и стерлядь чистить должно, как сказано выше, наблюдая, чтоб не заварить кожи; когда вычистишь, вымыть холодною водой; потом у самой головы, на спине, сделать острым ножом надрез, чтоб перерезать большую жилу или вязигу; у хвоста также сделать надрез, и вытянуть вязигу. Если осетр небольшой, и будет подаваться целый, тешку\footnote{Тешка — тонкая часть брюха.} у него не вырезывать, а только, распоров брюхо, выпотрошить. Также поступать должно со стерлядью. У больших осетров, которые не будут подаваться на стол целыми, тешку вырезывают; из нее приготовляют разные кушанья. Потроха больших рыб, как-то: молоки, печенки и пупки, также употребляются в разные кушанья; печенку и молоки вымыть в холодной, а пупки, обварив кипятком, выскрести ножом. О приготовлении осетровой и другой икры сказано будет особо в своем месте, именно в отделе <<Кладовой>>.

\subsection*{О приготовлении кореньев и трав.}\addcontentsline{toc}{subsection}{О приготовлении кореньев и трав.}

Коренья, употребляемые для приправы кушаньев, следующие: петрушка, пастернак, сельдерей, морковь, лук репчатый, порей, шарлот, молодая репа и чеснок. Приготовляют их различным образом: шинкуют полосками, вырезывают жестяною трубочкой или нарезывают звездочками. Очистив коренья, вымыть в холодной воде, нашинковать тоненькими полосками, в вершок длиною, или вырезать жестяною трубочкой и обделать на подобие маленьких морковок, репок и других кореньев; также вырезывают жестяною трубочкой звездочки. За неимением трубочек, вычистив средней величины коренья, нарезать вдоль корня полосок, а потом нарезать кружочком; кружочки эти будут иметь вид звездочек. Звездочками крошат более морковь, а петрушка и пастернак, накрошенные звездочками, по большей части развариваются, а потому их более шинкуют полосками. Из сельдерея и репы можно вырезывать разные фигурные штучки: треугольники, звездочки, маленькие корешки и репки. Приготовив коренья, перемыть их в холодной воде, положить в кастрюлю, налить горячею водою, вскипятить один раз; у поваров называется это обланшировать. т. е. отварить до половины спелости затем откинуть на решето и, когда остынут, положить в холодную воду. Коренья не всегда обланшировывают, а, приготовив и вымыв, опускают в холодную воду до употребления. Лук репчатый, смотря по тому для какого кушанья он приготовляется, режут кружками, шинкуют или крошат мелко. Порей шинкуют полосками и режут штучками в вершок длиною. Шарлот, очистив, кладут целыми луковками, а также крошат. Коренья кладут иногда при варении бульона, а когда бульон процеживают, вынимают вон; тогда крошат их крупно.

Травы, более других употребляемые: петрушка, сельдерей и укроп. Употребляют также: чабер, розмарин, кервель и майоран. О приготовлении всех трав будет сказано при кушаньях. Травы кладут иногда в бульон не рубленные, а связав их в пучок; для составления пучка берут: петрушку, чабер, укроп и сельдерей. При процеживании бульона травы вынимают. О прочих овощах будет сказано в приготовлении кушаньев.

Пряные приправы, как-то: мускатный орешек, гвоздика, кардамон, корица, кишнец, лавровый лист, английский перец, имбирь, мускатный цвет, перец горошчатый и шафран, в новой поварне употребляются в весьма небольшом, едва заметном количестве. Сладкие коренья теперь вышли совсем почти из употребления. Сардели, анчоусы и даже трюфели, как приправы к кушаньям, будут описаны в этой книге, при некоторых кухонных рецептах блюд модных, щеголеватых, которыми так богата новейшая французская кухня.
