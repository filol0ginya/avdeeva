\section{БЛЮДА ИЗ ЯИЦ, ХЛЕБА, КРУП ИЛИ МУКИ С ПОДЛИВОЙ ИЛИ БЕЗ ПОДЛИВЫ, УПОТРЕБЛЯЕМЫЕ В ОБЕДЕ В 5 БЛЮД, КАК ТРЕТЬЕ БЛЮДО, А В ДРУГИХ, БОЛЕЕ ПРОСТЫХ, ЗАМЕНЯЮЩИЕ СЛАДКОЕ БЛЮДО.} % 10 отдел 

\z{Ризотто по-итальянски}\index{Ризотто}

Приготовить обыкновенную рисовую кашу. Между тем истереть на терке швейцарского сыра и распустить в кастрюльке масло, на 15 минут до отпуска приготовить форму или металлическую кастрюлю для ризотто; наложить ряд рису в форму, посыпать тертым сыром, окропить маслом, снова ряд рису, опять посыпать сыром, окропить маслом и продолжать так, пока форма будет полна; верхний ряд посыпать сыром с частью тертого хлеба, окропить маслом и поставить в горячую печку, а когда заколеруется, подать на стол вместе с формою. (4) 

\z{Настоящий стамбульский пилав или плав}\index{Плов ! стамбульский}

Смотря по желанию, берут баранину одну, или еще курицу и голубей и варят их в горшке, не совершенно, после чего мясо и бульон выливают в чашку. Горшок выполаскивают и ставят снова на огонь, положив в него масло, которое должно не только, что распуститься, но дожидаться, чтоб оно сильно нагрелось; тогда, изрезав полусваренные мяса, которые мы выше упомянули, кладут их в горшок. Рис, вымыв в 2 или 3 водах, насыпают в горшок сверх мяса, а на это наливают бульона, оставшегося от варения мяса, столько, чтоб его было на палец выше риса. Покрывают горшок, разводят огонь, и по временам вынимают несколько зерен рису, чтоб знать, размягчился ли он и не нужно ли еще прибавить ложку, другую бульона. Нужно, чтоб рис был сварен, оставаясь цел, и чтобы каждое зерно отделялось: в этом-то и состоит настоящее достоинство восточного способа приготовлены пилава. Когда рис готов, покрывают горшок холстиною, сложенною вчетверо, а сверху крышкою, и чрез несколько времени распускают немного масла, вливая его в ямки, которые делаются в рисе ручкою ложки; после чего дают ему еще немного вариться. Готовый пилав кладут на большие блюда и убирают сверху мясом; на одном блюде оставляю т белый рис, другое обыкновенно подкрашивают малейшею частичкою шафрана, третье — каплею клюквенного или свекловичного сока. 

\z{Крокеты из риса}\index{Крокеты ! из риса}

Перебрать и вымыть 0,5 ф. рису, налить немного бульоном, положить ложку масла, соли, перцу, луковицу с тремя гвоздиками поставить на огонь и варить покрытым. Между тем сварить в бульоне 1 маленького цыпленка, вынуть на доску, изрезать мелко мягкие части, а из бульона сделать соус, процедить на сотейник и поставить на огонь; положить в выкипяченный соус один желток, прокипятить, положить изрезанного цыпленка, размешать, посолить, положить немножко перцу и выложить на тарелку. Когда рис сварится до мягкости, вынуть лук, размять рис, всыпать одну ложку тертого сыру, размешать и оставить, чтобы остыло; потом взять в ложку размятого рису, положить в средину фаршу из цыпленка, покрыть рисом, слепить так, чтобы фарш из средины не был виден, выложить на стол, обсыпать тертым хлебом и обровнять крокеты правильно. Переделав так все крокеты, обмакнуть каждую штуку в разбитое яйцо, обвалять в тертый хлеб, уложить на растопленное в сотейнике масло и обжарить с обеих сторон до колера. Приготовленные этим способом крокеты можно жарить и на фритюре.

\z{Пудинг из белого хлеба}\index{Пудинг ! из белого хлеба}

Взять 3 французские десятикопеечные хлеба, срезать с них корку, размочить в молоке, потом положить в салфетку, выжать досуха. Стереть 1/4 фунта коровьего масла добела, 8 яичных желтков смешать с маслом, положить туда же размоченный белый хлеб, стереть хорошенько; прибавить 1/4 фунта изюма, размочить его в воде и выбрать семечки, u фунта сладкого миндалю, нашинкованного полосками, 2 ложки мелкого сахару, немного толченого кардамону и чуть-чуть мускатного цвета, перемешать все вместе; 8 яичных белков сбить, выложить туда же. Вымазать салфетку маслом, выложить в нее пудинг, завязать салфетку крепко шнурком, опустить в кипяток, воду немного посолить. Во время варенья пудинг поворачивать; когда вода укипит, доливать кипятком, варить часа три. Кому угодно, можно варить пудинг вместо воды в бульоне. Между тем приготовить соус: взять свежего чухонского масла столовую ложку, тертого шоколаду 1/4 фунта, поставить на огонь, стереть с маслом, развести виноградным вином, подбить 4 желтками; вместо вина можно развести сливками. Когда пудинг будет готов, выложить в соусник и облить соусом. (8--10) 

\z{Клёцки из заварного теста}\index{Клецки ! из заварного теста}

Отмерить в кастрюльку 2 ложки кипящей воды, положить ложку масла, поставить на огонь; когда закипит, всыпать 2 ложки просеянной муки и мешать на огне. Когда загустеет, снять на стол, вбить в горячее тесто по одному 2 яйца. Вслед за каждым вбитым яйцом размешать лопаткой тесто до гладкости (употребляются яйца средней величины), остудить, выложить на стол, посыпать мукою, раскатать в рулет и нарезать клёцки какой угодно величины. Перед отпуском опустить в кипящую соленую воду, варить на легком огне 5 минут, потом выбрать дуршлоговой ложкой и выложить в бульон в суповой чашке. (Из этих количеств припасов будет 10--12 клёцок). 

\z{Настоящие сибирские пельмени}\index{пельмени ! сибирские}

Пельмени~--- одно из любимейших блюд сибиряков, которые зимою, заморозивши их, часто берут с собою в дорогу; и где нет готового кушанья, то стоит только вскипятить воды с солью, положить в нее пельменей~--- и через полчаса будет готово вкусное и сытное кушанье. Пельмени приготовляются следующим образом: берут хорошей мягкой говядины сколько нужно, которую рубят очень мелко, прибавляют в нее также мелко изрубленную луковицу или две, смотря по количеству говядины; мелко истолченного перцу, мускатного орешка и соли; потом перемешивают хорошенько. По изготовлении начинки, берут 2 яйца и небольшую чайную чашку холодной воды; смешав яйца с водою, положить немного, для вкуса, соли, и замесить на этом крутое тесто как для лапши, потом раскатать его скалкой в тонкие листы, но потолще, как раскатывается для лапши; из раскатанных листов нарезать небольших квадратов (4-х-угольников), или вырезать кружки небольшим стаканом; потом положить на каждый кружок или квадрат приготовленную начинку, защипать их и готовые класть на сито или решето; когда будет изготовлено достаточное количество пельменей, процедить бульон, прежде уже приготовленный из говядины; поставить в кастрюле на огонь, и когда закипит, то класть в него понемногу пельмени, и дав прокипеть раза два ключом, вынимать, так как пельмени уже готовы. Пельмени должно варить перед самым обедом, потому что им не нужно упревать, как другим похлебкам. Пельмени можно варить и просто в воде, немного посоленной, но тогда они не будут так вкусны. Вот еще приготовление пельменей, с другою только начинкой. Тесто и для этих пельменей готовится точно так же, как и для вышеописанных; разница только в приготовлены начинки, которая делается таким образом: берут хорошей, нежирной, мягкой, свежей свинины, бьют ее ножом до тех пор, пока она превратится в мягкое тесто, при чем, во время битья, прибавляют в свинину самых густых сливок, а равно мелко истолченного перца, мускатного орешка и соли; когда будет готово, начинают делать пельмени и варить в бульоне; на стол подают с бульоном, вместо подливки, в которую, по желанию, прибавляют уксусу и перцу. С этою начинкой пельмени еще вкуснее, нежели с говядиною. 

\z{Яичные котлеты}\index{Котлеты ! яичные}

Сварите крепко 15 яиц, отделите белки от желтков, белки изрубите мелко, а желтки протрите чрез волосяное сито. После того распустите в кастрюльке немного больше полстакана чухонского масла, всыпьте протертые на сите желтки, размешайте хорошенько и мешайте до тех пор, пока масса не остынет и не окрепнет. Тогда вбейте 5 свежих желтков и цельных 8 яйца, посолите, посыпьте перцем обыкновенным и душистыми, вложите мелко искрошенной зеленой петрушки и шнит-луку, а если зимою, то хоть обыкновенного луку, тертой булки или сухарей, по пропорции, чтоб масса была густа, перемешайте крепко и из этой массы делайте род котлет. Эти котлеты посыпайте тертою булкой, обмакивайте их в сбитые желтки и жарьте на сковороде, в масле. Этими котлетами можно обкладывать всякую зелень или есть их с растопленным маслом, или с бульонным обыкновенным соусом. (Выйдет от 8--10 котлет). 

\z{Плацинды молдавские}\index{Плацинды молдавские}

Отмерить на стол 1,5 стакана муки, положить ложку масла, 2 яйца и столько молока, сколько примет умеренной густоты тесто, вымесить до гладкости и накрыть полотенцем. Между тем приготовить творог, потом разделить тесто на ровные частицы, каждую раскатать в тоненький пласт, наложить на средину творога, размазать его кружком и, смазав края теста яйцом, загнуть так, чтобы весь творог был покрыть, а плацинда имела кругло-8-угольный вид и была толщиною в полпальца. Тогда сложить на наслоенный маслом лист и, выделав все, смазать, яйцом, проколоть по средине каждой плацинды концом ножа поставить в горячую печку, и когда начнут колероваться, приготовить блюдо и растопить масло, а когда заколеруются, складывать по штуке на блюдо, смазать маслом, положить сверху вторую штуку, смазать маслом и, сложив так все, подать на стол горячими. (4) 

\z{Вафельный паштет}\index{Паштет ! вафельный}

Надобно напечь вафлей, сварить из бульоне соус из: сладкого мяса, помидоров, раковых шеек, маленьких зеленых обваренных огурчиков и шампиньонов штучками, сделать белый и зеленый кнели и обварить их, потом заправить сливками, хорошенько осаженными; положить туда самую малую часть крепкого бульона и осьмушку пармезана; все это смешать вместе и положить в этот соус; в блюдо или в соусник надобно класть вафли на ребро и каждую обмазывать этим соусом; когда все эти вафли будут уложены бортиком, то замазать сверху этим же соусом с сливочною бешамелью и, залив чухонским маслом, поставить в печь; когда же подрумянится, то надобно залить крепким бульоном и, по надобности, подавать на стол. (6--8) 

\z{Пастетцы яичные}\index{Пастетцы яичные}

Взять несколько облупленных и в густую сваренных яиц, изрубить их с таким же количеством яблок, с которых очистить кожу и лузгу с семенами, и с фунтом говяжьего почечного сала, так мягко, чтоб все это имело вид теста; все это смешать с мякишем, в молоке вымоченного и выжатого, папушника, прибавить корицы, гвоздики, коринки, сахара и белого вина. С этим делать пастетцы из слоеного теста. 

\z{Раковый пудинг № 1}\index{Пудинг ! раковый №1}

Отварить 30 раков, мясо из них вынуть и изрубить мелко; из оставшейся лузги с делать с 1/4 фунта коровьего масла раковое масло, растереть его деревянного ложкою в пену, впустить в него 3 яйца цельных и 4 желтка, прибавить фарш из рубленной телятины, мускатного цвета, соли и тертого хлеба столько, чтоб замесить в тесто; туда же положить и изрубленное раковое мясо. Вымесив все как должно, вымазать салфетку коровьим маслом, положить в нее замешенное и, завязав ее, варить в воде поболее часа. Подавая на стол, утыкать пудинг фисташечными ядрами и раковыми шейками. (8--10) 

К этому пудингу подается следующий соус: истолочь часть раковой лузги, положить в кастрюлю с куском коровьего масла, обжарить, посыпать туда же немного муки, положить 2 ломтика ветчины и налить мясным бульоном столько, сколько надобно для соуса. Приправить по вкусу пряностями и приварить. 

\z{Раковый пудинг № 2}\index{Пудинг ! раковый №2}

Сварить полсотни раков, вычистить шейки, а клешни и верхние черепки истолочь. Положить 0,5 фунта коровьего масла в кастрюлю, вместе с толчеными раковыми клешнями, поставить на огонь, мешать, пока масло покраснеет; тогда масло слить, на гущу налить 2 стакана бульону, варить для кулиса. С 2 трехкопеечных булок обрезать корку, размочить в молоке, потом выжать сквозь салфетку досуха. Раковое масло стереть хорошенько, чтоб оно с делалось густо, как сметана. Раковые шейки изрезать мелко, смешать с маслом; туда же положить и размоченный хлеб, прибавить 8 яичных желтков, белки взбить, и смешать все вместе, посолить, положить немного сахару. Потом взять кастрюлю, обложить ее бумагой, вымазанной маслом, выложить приготовленный пудинг, поставить в печь, дать стоять час. Соус к пудингу приготовить следующий: стереть ложку ракового масла с 0,5 ложк. муки, развести раковым кулисом, вскипятить, подбить 2 желтками, смешанными с 2 ложками сливок, прибавить немного сахару и щепоть соли. Выложить пудинг в соусник, облить соусом. (8--10) 

\z{Пудинг из шпината}\index{Пудинг ! из шпината}

Взять 2 фунта шпинату, выбрать, перемыть, налить кипятком, дать вскипеть раза 2, откинуть на сито; когда стечет вода, изрубить мелко. Полфунта масла коровьего стереть добела, прибавить 8 яичных желтков, положить в масло, смешанное с желтками, шпинат, 1/4 фунта коринки, чайную чашку сдобных сухарей, мелко-истолченных, поднять белки, смешать все вместе, немного посолить, положить ложку сахару. Вымазав кастрюлю маслом, обсыпать сухарями, выложить в кастрюлю пудинг, поставить в печь. Между тем приготовить соус: стереть ложку растопленного коровьего масла с мукой, развести виноградным вином пополам с водою, положить цедру с 1 лимона, стертую на сахар, выжать из лимона сок, вскипятить, подправить сахаром, подбить 2 яичными желтками, вымешать, подержать на огне, не давая более кипеть. Отпуская на стол, выложить пудинг в соусник и облить соусом. (6) 

\z{Пудинг из гречневой каши}\index{Пудинг ! из гречневой каши}

Разварить гречневую крупу в молоке, взять на 0,5 ф. крупы 1/4 ф. масла коровьего, взбить его, подмешать 6 яичных желтков с поднятыми белками, сахаром, корицею, цедрою, солью, тертыми сухарями. Все это размесить хорошенько, положить в эту смесь сваренную на молоке кашу, мешать с 1/4 часа, переложить в салфетку и варить час. К этому пудингу подается тот молочный соус, какой находится здесь в числе соусов и подливок под № 232. (6—8) 

\z{Пудинг из трески}\index{Пудинг ! из трески}

Приготовив как следует треску, варить ее или, лучше, парить на самом слабом огне. Когда треска перестанет пениться, значить, что она поспела; тогда выбрать из нее кости и изрубить мелко-на-мелко. Потом растереть 10 яиц с полуфунтом битого масла, положить туда несколько рубленного шарлоту, сахару и соли, далее 2 тертых молочных французских белых хлеба и всю рубленную треску. Размешав все это хорошенько, положить в салфетку и варить с час. К этому пудингу подают густой соус из чухонского масла или соус из шарлота и сливок. Блюдо это в обеде в 3--4 блюда может идти вторым; но ни в каком случае не принадлежит к числу сладких блюд или пирожных, подаваемых в конце обеда. (8--10) 

\z{Пудинг из сафойной капусты}\index{Пудинг ! из сафойной капусты}

Упарив прежде всего сафойную капусту, разделите ее листья. Потом возьмите телячьей печенки, бычачьей печени, свиного мяса и говяжьего сала, по фунту каждого; все это вместе мелко срубите и прибавьте к этой смеси соли и лимонной цедры. Вымажьте кастрюльку коровьим маслом, обсыпьте тертым хлебом и укладывайте смесь в кастрюльку поочередно: слой капустных листьев и потом слой приготовленного фаршу. Укладку таким образом продолжайте до тех пор, пока наполнится кастрюлька; последний ряд должен быть из листьев. После того поставьте в вытопленную или, лучше, в духовую печку на целый час,~--- и вы получите пудинг отличного вкуса. 

\z{Овощной русский пудинг}\index{Пудинг ! русский овощной}

Разрежьте на кружки репу, брюкву, морковь, возьмите мясистые или нижние части (фонды) артишоков (отбросив листья), брунколь, обыкновенную кочанную капусту, словом, все сладкие овощи, исключая картофеля и поваренной зелени. Всего надобно взять по равной части, каждый овощ надобно особенно упарить в масле, т. е. репу в особой кастрюле, морковь в особой, и так далее, и в каждую кастрюлю должно подсыпать немножко пшеничной муки и влить несколько бульону, столько, чтоб из масла сделался густой соус. Когда каждый овощ поспеет, т. е. сделается мягким, надобно встряхнуть каждую кастрюлю, но осторожно, чтоб овощи не рассыпались. Потом вымажьте особую кастрюлю маслом, дно уложите тонкими ломтями свиного сала, и обложите как дно кастрюли, так и бока ее тонким, как лист бумаги, блином, изжаренным на сковородке (из молочного теста с яйцами), и укладывайте в эту кастрюлю вашу зелень рядами следующим образом: на дне кастрюли ряд тонких ломтей свиного сала, потом ряд моркови или чего другого, например, брюквы, потом ряд тонких ломтей вареного копченого языка и тонких ломтей вареной ветчины, а потом снова ряд овощей, артишоков или капусты, на верх снова ряд языка и ветчины, а потом снова ряд овощей, и продолжайте таким образом укладывать отдельно рядами языка и ветчины. Когда кастрюля полна, закройте ее, замажьте тестом и поставьте на легкий огонь, чтобы она прела около 0,5 часа или 20 минут. Тогда снимите крышку, опрокиньте осторожно кастрюлю над блюдом, чтоб зелень не рассыпалась, но сохранила свою форму кастрюли, т. е. цилиндра,~--- и подавайте на стол. (8--10) 

\z{Настоящий плум-пудинг}\index{Пудинг ! плум настоящий}

Вот что входит в состав этого кушанья, при приготовлении его совершенно по-английски. что На полштофа хорошего молока берется по полуфунту коринки, изюму и сахарного песку,1/4 ф. лучшей пшеничной муки и столько же свежего говяжьего мозгу из костей, 20 яиц, мелко изрубленную кожу с 1 лимона, достаточное количество толченой корицы, стакан хорошего рому. 

Сперва, влив в молоко ром, начинают одно за другим подбалтывать в него яйца, после чего, не переставая прилежно мешать, выкладывают и прочие материалы. Тогда составится довольно густоватый раствор, который можно влить в форму, предварительно вымазанную маслом и посыпанную мелко истолченными сухарями, или в салфетку, которая также должна быть вымазана маслом. Пудингу этому нужно вариться часов 7 или 8, причем выкипающую воду можно доливать кипятком, наблюдая, чтобы кипение не прекращалось. Некоторые прибавляют в этот пудинг миндалю горького и сладкого, померанцевой цедры, гвоздики, кардамону или мускатного цвета. Это совершенно зависит от вкуса.

В подливку к этому кушанью можно подавать растопленное и зарумяненное масло, или делать ее из вишен. Для этого должно их разварить в небольшом количестве воды и протереть сквозь сито; косточки же, расколов в иготи, положить в жижу, прибавить сахару, корицы, лимонной корки, гвоздики, и варить до сгущения; после чего, процедив, подавать на стол. Другая подливка: смешав стакан рому со стаканом воды, прибавить к этому полфунта сахару и десятка 1,5 густо сбитых яичных желтков, и варить, беспрестанно мешая метелкой из прутиков. Ром можно заменять столовым вином.

Иногда подают пудинг этот в огне: для этого стоит только облить его ромом, и зажегши, тотчас нести на стол. (Пропорция 6--8 персон). 

\z{Вареники настоящие хохлацкие}\index{Вареники ! настоящие}

Возьмите 2 или 3 яйца, стакан воды, немного соли, замесите тесто, рассучите тонко и нарежьте стаканом кружков. Творог с вечера положите под гнёт и, когда тесто готово, протрите сквозь решето, прибавьте 2 или 3 яйца и немного соли, наделайте вареников, налейте в кастрюлю воды, дайте закипеть ключом, и опустите вареники, а сварив, откиньте на сито, положите в какую-нибудь посудину, облейте чухонским маслом и подавайте к столу со свежею сметаною. 

\z{Хлебальники}\index{Хлебальники}

Их делают из сдобного теста, кислого или пресного натертого. Рассучив тесто толщиною в полпальца, начинку сделайте из моркови или свежей капусты, перемешав с рубленными яйцами; иногда делают с тыквой, а подают к столу с растопленным маслом. 

\z{Пампушки}\index{Пампушки}

Свежий творог протрите сквозь решето, смешайте с мукой, так, чтоб было по фунту творогу и муки, положите на каждый фунт по 3 яйца, посолите, сварите в воде и, откинув на сито, облейте сметаной. (10--12) 

\z{Плачинда}\index{Плачинда}

Сделайте тесто, как делается оно для тянутого пирога; когда положите пластов десять, намажьте их творогом в полпальца толщиною; потом еще положите пластов десять, и снова намажьте творогом; так можно наложить слоя два или три. Творог с вечера надобно положить под гнёт, a потом протереть сквозь решето, положить немного соли, сырых яиц, столько, чтоб удобно было намазывать. Каждый пласт мазать маслом как в тянутом пироге. (10) 

\z{Юражная каша}\index{Каша ! юражная}

Когда перетапливают коровье масло, то остается пахтанье, к которому еще прибавляют другого пахтанья из-под сбитого масла, называемого юрагой, и с этими-то двумя пахтаньями варят кашу, называемую юражною; при варении ее поступают следующим образом: заваривают кашу на молоке из гречневой крупы и прибавляют в нее столько этих пахтаньев, чтоб только не сделать каши жидкою, после ставят в печь и дают упреть хорошенько. К столу подают горячую с маслом. 

\z{Каша из тыквы}\index{Каша ! из тыквы}

Возьмите спелую тыкву, очистите с нее кожу и выберите семена, изрежьте кусками, поставьте вариться, и когда хорошо уварится, то откиньте ее на решето, чтобы стекла вода, и протерев сквозь него, разведите кипяченым молоком и поставьте в печь, чтоб вскипела, потом засыпать пшеном или гречневого крупою, но не очень густо, и опять поставить в печь, чтоб упрела. К столу подается с растопленным коровьим маслом. 

\z{Макароны обыкновенные}\index{Макароны ! обыкновенные}

Отвив макароны в воде с солью, за полчаса до обеда, откиньте на сито и дайте стечь воде; потом обсыпьте их тертым сыром пармезаном таким образом: сперва положите слой макарон и посыпьте пармезаном, потом положите опять слой макарон и посыпьте сыром, и все полейте хорошим бульоном; наконец растопите потребное количество чухонского масла, облейте им верхний слой макарон и подавайте на стол очень горячие. 

\z{Яичница с паюсной икрой по-грузински}\index{Яичница ! с паюсной икрой}

Взбить штук 5 яиц. У булки или сайки корку срезать, мякоть смочить кипятком и растереть с паюсной икрой. Затем нарубить мелко луку, смешанного с толченым перцем и солью. Смешать все это хорошенько и сложить на сотейник, густо обмазанный чухонским маслом. Обровнять ложкой, поставить в духовую печь, и как только поднимется, отпустить к столу. (6--8) 

\z{Пезы}\index{Пезы}

Взять 1 стакан молока вместе с дрожжами, половину назначенной муки, растворить тесто как обыкновенно, дать подняться, потом замесить остальною мукой, всыпать соли, положить 2 ложки масла, 2 яйца, чтобы тесто было не так густо, как обыкновенно делают его на сухари, выбить как можно лучше веселочкой, поставить опять в теплое место; когда тесто вторично поднимется, наделать круглых булочек величиною в грецкий орех и, обмакнув каждую булочку в растопленное масло, класть их в небольшую форму, дать подняться, потом вставить в кипяток и варить пезы на пару около получаса, или вставить их в печь. Подавая, выложить на блюдо; подавать горячими. Подаются и в кастрюле.

Две луковицы мелко изрубить, поджарить докрасна в 2--3 ложках масла, подать отдельно. (6) 

\z{Пудинг из смоленских круп}\index{Пудинг ! из смоленских круп}

Влить в кастрюльку 1,5 стакана молока, положить 8 кусочков сахару (натереть каждый немножко лимоном), поставить на огонь; когда закипит, положить 0,5 ст. л. масла, всыпать 3 ст. л. смоленской крупы и, размешав, варить, пока каша не загустеет; тогда снять с огня, положить очищенного 1/8 ф. кишмиша (изюм) и вбить 2 желтка, размешать, потом взбить в пену белки и размешать с белками окончательно. Между тем смазать маслом кастрюльку без ручки, обсыпать тертым хлебом, выложить массу в кастрюльку, поставить в горячую печку, и когда заколеруется и поднимется как следует, выложить на тарелку, отделив сперва тонким ножом от кастрюльки. (4)

\z{Макароны по-немецки}\index{Макароны ! по-немецки}

Растопить 2 ф. коровьего масла, растереть добела, потом прибавить туда 0,5 фунта муки, 2 полные ложки густых дрожжей, сахару, корицы, тертой лимонной корки, несколько зерен соли, полчашки молока, 3 или 4 яйца, и размешивать все это до тех пор, пока тесто будет отставать от ложки. В этом виде дать тесту несколько взойти.

После того густо вымазать кастрюлю коровьим маслом, на дно ее насажать ложкою небольших кучек из приготовленного тес та пересыпать каждую из них мукою, для того, чтоб они не приставали одна к другой. Когда тесто станет подниматься, поставить в теплое место дать взойти совершенно, вымазать маслом, наложить крышку, на которой находятся уголья, поставить на таган, под которым разведен умеренный угольный огонь, но так притом, чтоб сверху было более жару, чем снизу, и дать испечься. 

По прошествии 3/4 часа осведомиться, зарумянилось ли тесто; в последнем случае посыпать побольше сахаром и корицею, подлить 1/4 бутылки кипящего молока, накрыть, оставить еще на 1/4 часа в жару, осторожно вынуть ложкою и подавать с молочным соусом. (6) 

\z{Лапша из формы по-славянски}\index{Лапша ! по-славянски}

Приготовить тесто для лапши из 3 яиц и исшинковать мелко. Между тем влить в шарлотную форму немного очищенного масла посыпать ровно дно формы шинкованною лапшою и, поставив на легкий огонь, заколеровать вполовину. Потом опустить остальную лапшу в соленый кипяток, сварить и, полив сверху холодною водою, отлить на дуршлаг, а когда вода совершенно стечет, выложить на растопленное в кастрюле масло, размешать, разогреть па плите, вбить 3 желтка, снабдить по вкусу солью и мускатным орехом, наложить в форму полно, поставить в горячую печку, и когда заколеруется кругом, выложить на блюдо. (4--5) 

\z{Тесто для лапши}\index{Тесто ! для лапши}

Отмерить на стол 2 столовые ложки конфектной муки, раздвинуть средину вбить 1 яйцо, размешать вначале ножом и постепенно забирать муку, пока не образуется тесто, - тогда вымесить его руками до гладкости, покрыть и, дав несколько расстояться, раскатать, посыпая мукою, до самого тонкого состояния. Когда тесто в половину высохнет, изрезать его согласно надобности, т. е. лапшою наподобие тонкой вермишели, потолще — продолговатыми полосками наподобие макарон, или четырехугольными лазанками. (6) 

\z{Каша гречневая с пармезаном}\index{Каша ! гречневая с пармезаном}

Приготовить рассыпчатую гречневую кашу. Между тем натереть сыру пармезану и распустить в кастрюльке масла; когда каша будет готова, наслоить маслом шарлотную форму, наложить ряд каши, посыпать тертым пармезаном, окропить маслом, положить снова ряд каши, посыпать сыром, окропить маслом и продолжать так до верху формы, посыпать сверху пармезаном, окропить маслом побольше, поставить в горячую печку, и когда сверху заколеруется и прогреется, подать на стол с формою. (4) 

\z{Каша гурьевская рассыпчатая}\index{Каша ! гурьевская рассыпчатая}

Каша гурьевская приготовляется из живности и дичи, которая бывает дома в излишестве, например: куры, куропатки, баранина, телятина и т. п. Вначале приготовить и сварить рассыпчатую из смоленской, т. е. самой мельчайшей гречневой крупы, кашу. Потом назначенную дичь, напр. куропатку, сжарить, снять с костей, мягкие части изрубить мелко, а кости сложить в кастрюльку, где куропатка жарилась, налить бульоном, выварить сок до совершенной густоты и снабдить по вкусу солью и толчеными пряностями. Когда каша будет готова, выложить из кастрюли на сотейник, перемешать с рубленною дичью, растопленным маслом и соком из дичи, наложить полно в луженый горшок, поставить в горячую печку, и когда сверху заколеруется, подать на стол в горшке. (4) 

\z{Рюрэ (шведская яичница)}\index{Рюрэ}

Жестяную или серебряную кастрюльку намазать ложкою масла, осыпать дно и бока сухарями; 12—15 яиц выпускать по одному в кастрюльку, посыпая каждое яйцо солью, простым перцем и мелко изрубленным зеленым луком-сеянцем; на каждое яйцо положить по кусочку сливочного масла, поставить на горячие уголья; как только немного поджарятся, подавать на стол в той же самой кастрюльке. (6--8) 

\z{Гусиные яйца печеные}\index{Яйца ! гусиные ! печеные}

Взять 6 яиц, провертеть в них дырочки, выпустить из них всю жидкость. Распустить 2 ложки масла, вылить яйца, испечь густую яичницу, протереть сквозь сито, положить 1 ложку тертой булки, 1/4 чайной ложки мускатного цвета, 1 ложку мелко изрубленного луку-сеянца, влить 3--4 куриные яйца, размешать хорошенько, влить обратно в скорлупы тонким шприцем и испечь. 

\z{Пудинг из телячьей печенки}\index{Пудинг ! из телячьей печенки}

Телячью печенку отварить, натереть на терке. Изрубить мелко 1 луковицу, пошарить ее в 1,5 ложках масла, остудить, смешать с 3 яйцами и 3 желтками, положить 3/8 фунта белого хлеба, намоченного в молоке и выжатого, с 1/2 стакана коринки, соли, мускатного ореха, натертую печенку; растереть все это в каменной чашке как можно лучше; варить в салфетке 1,5 часа. 

Соус к нему следующий: 1 ложку муки поджарить в 1 ложке масла, развести 1,5 стаканом бульона, 1/2 стаканом вина, вскипятить, процедить, влить уксусу 1 ложку, положить немного сахару, ломтики лимона, 1/4 стакана изюму, вскипятить все это, облить пудинг. (4--5) 

\z{Строфокамилово яйцо}\index{Строфокамилово яйцо}

Возьмите 2 пузыря — один большой, другой поменьше, вымойте их тщательно, высушите и опять вымойте в несколько приемов, чтобы они были совершенно чисты и не имели бы ни малейшего запаха. Разбейте дюжину свежих яиц, выпустите желток от белка отдельно; желток положите в маленький пузырь, завяжите его и положите в кипяток; когда желток окрепнет, снимите с него пузырь; потом вылейте белок в большой пузырь и опустите туда прежде сваренный желток, большой пузырь завяжите и в таком виде варите его в воде, пока он не сварится вкрутую. В продолжении этой последней варки, полезно оборачивать пузырь то завязкой вверх, то завязкой вниз, чтобы желток пришелся в самой середине. Когда окончательно сварится, снимите большой пузырь, и вы получите огромное яйцо, которое положите на плоское блюдо, разрежьте на четыре части и кругом обложите кресс-салатом или другой, зеленью. Такое блюдо очень эффектно на столе, приготовленном для завтрака.\footnote{Из лекции доктора Пуфа. <<О кухне>> в 1844 г.}

\z{Морковь со спаржею и с яйцами}\index{Морковь ! со спаржей и яйцами}

Очистить морковь и разрезать намелко, положить в кастрюлю с куском масла, солью, бульоном, покрыть и варить. Спаржу тоже очистить, нарезать на куски, длиною в вершок, разварить в особенном горшке с примесью соли, и потом положить, приготовленную таким образом, спаржу в морковь, прибавив мелкого сахару и рубленной петрушки. Наконец положить несколько муки, размешанной с желтком от нескольких яиц, поворачивать кастрюлю и держать ее еще несколько минут на умеренном угольном огне. При подаче на стол посыпать тертым хлебом. (6) 

\z{Яичница с переменами}\index{Яичница ! с переменами}

Распустить на сковороде кусок масла, разболтать в особом горшке потребное число яиц и вылить в растопленное масло, которое же, однако, не должно нагревать дотемна; оставить так на огне на несколько минут, и между тем отделять пригорелые части до тех пор, пока снимать яичницу с огня. Ее нужно приготовлять на угольном огне; чем она нежнее, тем лучше. Если хотят положить колбасу, то изрезав ее на куски, кладут масло и тотчас же наливают яйца, чтоб она не была слишком жестка. Если колбаса очень суха, то нарезывают ее на блюдо, наливают на нее яичницу и слегка посыпают тертым мелким перцем.

Яичница с яблоками также очень вкусна; для этого нужно очистить яблоки, разрезать на куски, дать распариться в масле, облить разболтанными яйцами и хорошенько размешать. (5--6)

\z{Цветная капуста с яичницей}\index{Яичница ! с цветной капустой}

Из 2 яиц, 2 полных столовых ложек муки и 1 полной чайной чашки молока сделать яичницу, разварить цветную капусту в воде с примесью соли, откинуть на решето, дать стечь воде, опустить капусту в яичницу и испечь в коровьем масле. Приготовить соус из крепкого мясного бульона, лимонной корки, мускатного цвета, подбить 2 желтка, размешанные с одною полною ложкою муки, и облить этим соусом выложенную на блюдо цветную капусту. (4--5) 

\z{Рамекини (итальянское блюдо)}\index{Рамекини}

Кладется в кастрюлю 0,5 ф. тертого пармезана, 1/4 ф. сливочного масла, стакан воды, немного соли (смотря по сыру) и 1 протертый анчоус, или протертую же половину селедки; смешайте все хорошенько, поставьте кастрюлю на огонь, и пока кипит, прибавляйте понемногу муки столько, сколько жидкость может поглотить, и держите на огне до тех пор, пока это тесто не загустеет окончательно. Тогда выложите его в другую холодную кастрюлю, в которой выпустите яиц столько, сколько может их принять тесто, не делаясь жидким. Это тесто должно сливаться с ложки, не перерываясь и не приставая. Из такого теста наделайте род булочек, величиною с куриное яйцо, положите на железный лист и поставьте в вольный дух. Рамекини должны быть легки как пух и хорошо подрумянены.\footnote{Из лекции доктора Пуфа: <<О кухне>> в 1844 г.} (4--5) 

\z{Макароны по-неаполитански}\index{Макароны ! по-неаполитански}

Сварить в соленой кипящей воде макароны, сложить на растопленное в кастрюльке масло, размешать, наложить ряд в металлическую кастрюльку без ручки, полить пюре из томатов, посыпать тертым сыром, наложить снова макарон, полить томатами, посыпать сыром и продолжать так до конца; верхний ряд полить томатами, посыпать сыром и, окропив маслом, заколеровать в горячей печке. (4) 

\z{Лазанки с творогом}\index{Лазанки ! с творогом}

Замесить тесто для лапши из 1 стакана муки и 2 яиц, раскатать тонко, просушить и нарезать лазанки четырехугольными правильными штучками. Между тем истереть на терке досуха 1 ф. отжатого творога; опустить лазанки в кипящую соленую воду (опускать постепенно и мешать лопаткой, чтобы в лазанках не образовалось комков); когда закипит и лазанки всплывут вверх, отлить на дуршлаг, перелить немного холодною водою, сложить на растопленное в кастрюле масло, разогреть, размешать, снабдить по вкусу солью, класть постепенно творог, размешать окончательно, выложить на глубокое блюдо и полить сверху маслом. (4) 

\z{Ломанцы малороссийские с маком}\index{Ломанцы ! с маком}

Замесить тесто из 2 стаканов муки и испечь пляцки; когда будут готовы, вынуть и остудить. Между тем вымыть в холодной вод е 1/4 ф. маку, истереть в каменной чашке деревянным пестиком так, чтобы из мака показалось белое молоко; тогда выложить мак в глубокое блюдо, положить 1/4 ф. меду, размешать, прибавить 0,5 стакана воды и за 1/4 часа до подачи изломать в мелкие куски приготовленные пляцки и размешать так, чтобы все куски были обмочены, переложить на блюдо и полить остальным соком вместе с маком. Приготовляя мак с медом, нужно наблюдать, чтобы воды было столько прибавлено, сколько окажется нужно, ибо приготовляющий, подливая воду в мак, когда трет, может иногда перелить лишнее. (4) (Пост.) 

\z{Вареники малороссийские с кислою капустою}\index{Вареники ! с кислой капустой}

Перебрать шинкованную кислую капусту, сложить в обширную кастрюлю, налить водою, поставить на плиту и сварить до мягкости; когда будет готово, отлить на дуршлаг и остудить, а потом запасеровать на прованском масле мелко нашинкованного луку, сложить сваренную и отжатую досуха капусту, размешать, снабдить по вкусу солью, перцем и мускатным орехом и приготовить постное тесто, раскатать тонко, и вырезав выемкой умеренной величины кружочки, положить на каждый капусты, сложить края вместе и залепить так, чтобы изнутри фарш не был виден, и когда все будет готово, сложить на подсыпанное мукою сито и покрыть салфеткой; за 10 минут до отпуска опустить в соленый кипяток, помешать осторожно, покрыть, вскипятить, и когда вареники всплывут наверх, выбрать ложкою в назначенную для вареников посуду, полить прованским маслом, размешать и подавать. Для любителей поджаривается в прованском масле мелко нашинкованный лук. (4) (Пост.) 

\z{Кисель из белого хлеба}\index{Кисель ! из белого хлеба}

Размешать 0,5 фунта размоченного в молоке белого хлеба, 1/4 фунта очищенного мелко истолченного миндалю, 12 золотников масла, растертого добела с сахаром и корицею, положить в небольшую, вымазанную маслом форму, испечь в печи, посыпать сахаром и подавать после супу. 

\z{Фаршированный французский белый хлеб}\index{Хлеб ! фаршированный}

Взять гладкий, не растрескавшийся белый хлеб такой величины, какая потребна, смотря по числу персон. Из нижней корки вырезать кружок, отложить его в сторону, вынуть, не повреждая корки, весь мякиш из хлеба, начинить хлеб заранее приготовленным мясным или рыбным фаршем, вымазать яйцами вырезанный кружок, положить его на прежнее место и завязать хлеб крепко нитками. Приготовленный таким образом хлеб выложить на жестяной лист, вымазанный маслом, накрыть мокрою салфеткой, чтобы корка немного поразмякла, вымазать потом маслом и испечь в печи в умеренном жару. Если хлеб начинен мясным фаршем, то подавать его с соусом фрикасе, а если рыбным, то с соусом из шампиньонов или из каперсов. При подаче на стол хлеб выложить на блюдо, облить одною частью соуса, а остальную подавать отдельно в соуснике. (5--6) 

\z{Пудинг постный}\index{Пудинг ! постный}

Взять большую круглую булку, с резать с нее верх, выбрать мякиш, размочить в миндальном молоке, смешать с разваренным рисом, положить сахару, коринки и толченой корицы, начинить хлеб, налить миндальным молоком, закрыть срезанным сверху кружком, поставить в печь. Вынув из печи, загласировать следующим образом: взять ложку или 2 мелкого сахару, смотря по величине булки, смочить сахар белым виноградным вином, прибавить чайную ложку лимонного соку, стереть хорошенько деревянной ложкой и, обмазав хлеб, поставить в печь в легкий дух. Соус приготовить следующий: взять морсу вишенного, клюквенного или из других ягод, положить по вкусу сахару, развести немного водой, подправить картофельной мукой, дать раза два вскипеть. Отпуская на стол, выложить пудинг на блюдо, а соус подавать в отдельном судке. (П.) (5--6) 

\z{Пудинг рисовый}\index{Пудинг ! рисовый}

Сварите из риса густую кашу, положите в нее сахару и изюму и перемешайте хорошенько все вместе; форму вымажьте маслом, обсыпьте сухарями, выложите туда рисовую кашу, уравняйте и поставьте в печь. Соус к пудингу следующий: возьмите рисового отвару, положите сахару, виноградного вина, подправьте немного картофельной мукой и вскипятите; выложив пудинг, облейте этим соусом. Другой род соуса: возьмите соку из каких угодно ягод, положите сахару, подправьте картофельной мукой, вскипятите и облейте пудинг. 

\z{Макароны в тесте}\index{Макароны ! в тесте}

Берут 1/2 ф. лучших итальянских макарон (не мелких, а крупных и прозрачных), варят их в бульоне с полчаса, потом откидывают на решето, дают бульону хорошенько стечь и, положив снова в кастрюлю, посыпают слегка пармезаном, а к нему присоединяюсь значительный слой тертого швейцарского сыра и ставят на огонь, вложив туда же кусок чухонского, самого свежего масла, величиною с куриное яйцо. Пока эта смесь хорошенько разогревается и макароны пропитываются насквозь едкостью обоих сыров, должно приготовить следующее тесто: 1/2 ф. пшеничной муки развести 2 ложками воды, прибавить 2 ложки прованского, самого чистого масла, 2 желтка и 1/4 ф. чухонского масла. Тесто, таким образом приготовленное, раскатывают как можно тоньше и обкладываюсь им внутренний край каменной чашки, которую кроме того смазываюсь маслом. Сложив в эту чашку макароны из кастрюли, заворачивают окраины теста в виде крышки, и поставив чашку на горячую плиту, накрываюсь ее медною крышкою, на которую насыпают раскаленных угольев. Через 0,5 часа макароны достаточно пропекутся и должно подавать их очень горячими к столу в той же чашке, где они пеклись. Можно в особенном подливочном соуснике подать подрумяненное чухонское масло, для любителей жирных кушаньев. (6) 

\z{Омлет с соусом из сметаны}\index{Омлет ! с соусом из сметаны}

12--15 яиц разбить с 1 ложкою мелко изрубленной зеленой петрушки и немного укропа, посолить, влить на растопленное на сковороде масло, поджарить, свернуть в трубку, или сложить в виде пирога, переложить на блюдо, облить следующим соусом: 1 ложку растопленного масла, 1 ложку муки вскипятить раза три, потом положить 1,5 стакана сметаны, 3 желтка, ложку сахару; вскипятить, мешая; когда погустеет, облить омлет на блюде, посыпать сахаром и корицею, вставить в печь на 1/4 часа. (6) 

\z{Яйца с соусом фрикасе}\index{Яйца ! с соусом фрикасе}

Сварить яйца в густую, положить в холодную воду, облупить, разрезать вдоль и вынуть желтки. Если хотят брать раков, то сварить их, вынуть мясо, растереть масло добела, положить в него 2 яйца, смешать с рубленным мясом и тертым хлебом, сделать фарш и начинить им яйца. Сваренный желток тоже примешать в масло, но не класть в него много тертого хлеба, и положить фарш так, чтобы яйца казались как будто целыми, после чего выложить начиненные яйца на бумагу и печь на сковороде или в печи, дав им зарумяниться; наконец приготовить приличный соус и облить ими яйца, выложив их на блюдо. (Пропорция: одно яйцо на двух человек). 

\z{Яичные колбаски}\index{Яичные ! колбаски}

Разболтать 6 яиц, посолить немного, распустить на сковороде свежее сало, дать опять остынуть, вылить на сковороду разболтанные яйца и смешать с салом. Прибавить 1/4 ф. мелко искрошенного миндалю, изюму, корицы, немного сахару, мелко изрубленных цукатов, или сваренной в сахаре померанцевой корки, истертого белого хлеба, и вымешать. Потом выложить тесто на доску, посыпанную мукою, обвалять в муке, переделать в колбаски и обжарить в свежем сале. После того сварить виноградного вина с сахаром, облить им выложенные на блюдо колбаски и посыпать сахаром и корицею. (6--8) 

\z{Яичница с печенками}\index{Яичница ! с печенками}

Очистить печенки из куриц, индеек, гусей или уток, изрубить их и обжарить в чухонском масле, вместе с изрубленными петрушкою, луком, грибами и зубком чеснока. Когда они обжарятся, дать им остынуть. Выпустить на печенки дюжину яиц, прилить ложку сливок, посолить, прибавить пряностей, вымешать и обжарить в чухонском масле. (10) 

\z{Яйца по-испански}\index{Яйца ! по-испански}

Истолочь сваренного рябчика вместе с костями и положить в бульон, который и варить несколько на огне с шампиньонами, травами и ветчиною не очень жирною; разболтавши толченого рябчика в бульоне, пропустить жижу сквозь сито. Между тем взять 8 яиц, из 4-х из них белки выпустить, a прочие яйца и желтки в се выпустить в чашку и, приправив их солью и перцем, протереть сквозь сито. Смешавши бульон и яйца, сварить. Подавая на стол, облить немного говяжьим соком. (6--8) 

\z{Грибные пельмени постные и скоромные}\index{Пельмени ! грибные}

Возьмите 30 штук сушеных грибов, вымойте их в трех водах, и варите до тех пор, пока совершенно не размокнуть; тогда выложите на сито, чтоб вода стекла, протрите каждый гриб, порежьте мелко и изжарьте в прованском или маковом масле, прибавив туда третью часть противу грибов рубленной рыбы: осетрины, лососины или даже обыкновенной речной рыбы, очистив от костей. Не забудьте посолить и посыпать перцем. Несколько луковиц изжарьте особо в масле и положите туда же. Мешайте эту массу, и когда она совсем прожарилась, с делайте обыкновенные пельмени, то есть пирожки из водяного теста (без дрожжей), закипятите воду в кастрюле и вложите туда пирожки. Когда пирожки поспели, то они всплывут на верх. Если пельмени скоромные, вместо рыбы надобно взять мелко изрубленную телячью печенку или хоть просто телятину, и употреблять чухонское масло вместо прованского, и в смесь грибов с мясом прибавить 4 яичные желтка. (6) Из свежих грибов, разумеется, пельмени будут еще вкуснее. (П.) 

\z{Малороссийские галушки}\index{Галушки ! малороссийские}

Взять тарелку творогу, 3 яйца и столько муки, чтобы тес то держалось и можно было раскатать его руками в виде колбасы, и нарезать ножом кусочками в виде клёцок, который варить в соленой воде до тех пор, пока они все всплывут наверх; тогда откинуть их на решето, дать хорошенько стечь воде и уложить в гладкую форму или кастрюлю, облить свежей молодой сметаной, не кислой, и поставить в печку, чтобы запеклось. (6)

\z{Рис, приготовленный по-итальянски}\index{Рис ! по-итальянски}

Разварить совершенно фунт рису, хорошенько перемытого, стереть на терке полфунта ветчинного сала и цветной капусты, и перемешать их вместе, прибавив туда рубленной петрушки, чесноку, перцу, соли и несколько волосного укропу. (6--8) Когда капуста постоит на легком огне часа три в закрытой посуде, тогда приложить к ней рис, немного его смочивши бульоном, так чтобы он едва только покрыл рис, после чего еще варить около 1/4 часа. Это кушанье едят с сыром пармезаном.

\z{Котлеты из яиц}\index{Котлеты ! из яиц}

Сварить крепко 15 яиц, отделить белки от желтков~--- белки изрубить мелко, а желтки протереть сквозь волосяное сито. После того распустить в кастрюльке немного более полстакана чухонского масла, всыпать протертые на сите желтки, размешать хорошенько и мешать до тех пор, пока масса не остынет и не окрепнет. Тогда вбить 5 свежих желтков и цельных 3 яйца, посолить, посыпать перцем обыкновенным и душистым, вложить мелко искрошенной зеленой петрушки и шнит-луку, а если зимою, то хоть обыкновенного луку, тертой булки или сухарей, по пропорции, чтоб масса была густа, перемешать креп ко и из этой массы с делать род котлет\footnote{Из этого количества припасов выйдет до 20-ти котлет, считая по одной на персону.}. Эти котлеты посыпать тертою булкой, обмакивая их в сбитые желтки, и жарить на сковороде, в масле. Этими котлетами можно обкладывать всякую зелень или подавать их с растопленным маслом, или с бульонным обыкновенным соусом. (12--20) 

\z{Блины с ветчиной}\index{Блины ! с ветчиной}

Взять полфунта рису, кусок сырой ветчины, несколько кореньев и луковиц, все это сварить в крепком бульоне; потом вынуть лук и коренья и разрезать ветчину на тонкие, продолговатые куски, замешав их снова с рисом, несколькими ложками пармезана, приличным количеством мускатного ореха и соли. Из всей массы с делайте любой формы лепешки, обваляйте в муке и жарьте в масле, подавая на стол горячие. (6)
