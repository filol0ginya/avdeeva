\section{УСТРОЙСТВО КУХНИ И ПОВАРЕННАЯ ПОСУДА} % отдел 3

\subsection{Устройство кухни.}
Кухни строятся различными манерами. Устройство кухни должно соответствовать достатку и образу жизни хозяина. Но как в среднем, так и в высшем сословиях, необходимо иметь в кухне русскую печь и английскую или немецкую плиту. Если состояние позволяет, то хозяйская кухня должна быть отдельна от людской, для соблюдения чистоты; здесь, кроме повара или кухарки, не должны толпиться другие служители. Русская печь необходима для печение разных родов хлеба, булок, кренделей и некоторых пирожных. Плиты делаются со шкафом и без шкафа. Плиту со шкафом называют английскою. Прежде были в большом употреблении очаги, но теперь их совсем оставили; только в больших кухнях очаги употребляются и до сих пор; они нужны для жаренья на вертеле жаркого. В небольших и средних хозяйствах легко можно обойтись и без вертела, имея в кухне русскую печь и небольшую плиту. Когда кухня одна для хозяев и служителей, в таком случае печь должна быть большая и устье у нее так высоко, чтоб можно было ставить в печь корчаги с квасом или пивом.

В кухне необходимы полки для посуды и шкаф, где класть разные припасы. В поварских кухнях делается каток, занимающий одну, а иногда и две стены; это род низкого шкафа; он, вместе с тем, служить и столом; внутри ого поделаны полки. Если он велик, то бывает разделен на несколько отделений, с особыми дверцами в каждом отделении. Столь есть вещь необходимая в кухне. Также не последнее удобство имеет в сенях кухни холодный чулан. В домах хорошо устроенных бывает недалеко от кухни погреб.

В кухне надобно соблюдать чистоту; посуда также должна быть вычищена, вымыта и расставлена по своим местам. Пыль со стен в кухне надобно обметать каждую неделю; полки наружный и в шкафах раз в неделю следует мыть щелоком с песком; стол или каток смывать всякий день. Пол, если он деревянный, должно мыть каждую в неделю и усыпать песком, а каменный — раз в месяц. Также надобно стараться истреблять в кухне мух и тараканов; для этого есть много безвредных средств, о которых будет сказано впоследствии.

\subsection{Поваренная посуда вообще.}
Кухонная посуда разделяется на деревянную, медную, железную и глиняную; здесь будет говориться о посуде, собственно принадлежащей к кухне. Из деревянной посуды необходимы: квашня для хлебов, и две или три поменьше, для булок, ситного хлеба и кренделей. В некоторых местах заменяют квашенки горшками или какой-нибудь другой посудой; но деревянные квашенки лучше, удобнее и прочнее. Лотки и ночвы для сеяния муки и вываливание теста для хлебов, булок и кренделей; доски, на которых рубят говядину, шинкуют коренья, кладут свалянные и приготовленные для печенья булки и крендели; разные веселки, мутовки и скалки; корытца и чашки, для рубления разных начинок, фаршей и тому подобного; лоханки для вымачивание мяса и рыбы; сита, разных сортов и решета. Все это можно иметь в большем и меньшем количестве, смотря по тому, сколько нужно. Но, чтобы для каждой хозяйки было виднее, то почитается нужным сделать полный реестр всей кухонной посуде, из которого можно убавить, смотря по семейству и состоянию.

\subsection{Медная посуда.}
\noindentКастрюль разной величины с крышками 12.\\
Медных, вылуженных форм, разной величины, для желе, бланманже, галантиру и печенья различных пирожных, 3.\\
К ним маленьких форм сколько угодно, но по крайней мере 12.\\
Продолговатая медная кастрюлька для варения рыбы 1.\\
Форма с крышкою, для варения в воде пудингов, 1.\\
Ступка медная 1.\\
Ступка мраморная 1.\\

\subsection{Посуда из белого железа и жести.}
\noindentРешетчатая ложка или шумовка, для снимания с бульона пены, 1.\\
Тёрок различной величины 3.\\
Тазик с решетчатым дном, для откидывания зелени, макарон и прочего, 1.\\
Кружки разной величины 3.\\
Глубокое блюдо для паштета 1.\\
Блюдо плоское, для печенья воздушных и тянутых пирогов, 1.\\

Также делают блюда медные, вылуженные и оловянные.

\noindentДва или три ковша медные, вылуженные, или из белого железа, 3.\\
Одна или две чумички из белого железа, ибо они гораздо опрятнее, нежели деревянные чумички, 2.\\
Несколько разной величины воронок жестяных или медных, вылуженных, 4.

\subsection{Формы.}
\noindentТрубка для колец 1.\\
Вафельная форма 1.\\
Форма для стружек 1.\\
Форма для трубочек 1.\\
Маленькие жестяные формы, для печение разных пирожных; они делаются сердечками, звездочками, круглые и продолговатые; их можно иметь, по крайней мере, 25.\\
Резец для обрезывания пирожного 1.\\
Шпиковальные иглы разных сортов 3.\\
Жестяная трубочка для выжимания сердцевины из яблок и прочего 1.\\
Трубочка для вырезания картофеля на манер молодого 1.\\
Большой, широкий нож, для рубления мяса на котлеты и фарши, 1.\\
Ножей разной величины 4.

\subsection{Железная посуда.}
\noindentПротивни и сковороды разной величины; маленькие сковородки для печенья гречневых и разных других блинов.\\
Железные листы для печенья разных хлебов, булок, кренделей и разного пирожного.\\
Ухваты разной величины, кочерга, сковородник, щипцы и лопатка для угольев.\\
Две сечки, одна обыкновенная, другая на манер буквы S.\\
Чугуны разной величины. В зажиточных домах делают чугуны медные, вылуженные внутри; их называют медники, и они гораздо прочнее чугунных.

\subsection{Деревянная посуда.}
\noindentКвашня для хлебов, квашенки разной величины, для булок, ситного или пеклеванного хлеба и для кренделей, 3.\\
Чашки, в которые кладут сваляные хлебы, плетеные или деревянные, 6.\\
Лотки для сеяния муки и вываливание теста.
Корытца, в которых рубят мясо и разные начинки.\\
Кадка для затора пива и кваса; другая, в которой разводят квас.\\
Бочонки для пива и квасу.\\
Ушаты и ведра для воды.\\
Лоханки, продолговатые кадочки, в которых мочат мясо и рыбу.\\
Сита и решета.\\
Доска с закраинами, на которой рубят фарш и мясо для котлет.\\
Доски, на который кладут сваляные булки и сделанные крендели, приготовленные для печенья и маленькие веселки и мутовки.\\
Лопатка, которой берут тесто, когда валяют ржаные хлебы.\\
Лопатки для сажания в печь хлебов, булок и пирогов.\\
Скалки для слоения теста и другие скалки, для раскатывания теста для пирогов.\\
Венчик для сбивания сливок и выгнутая вилка для поднятия яичных белков.

\subsection{Посуда глиняная.}
В русской печи кушанья готовят в горшках. Горшки бывают простые и муравленные; их должно иметь несколько, разной величины.\\
Корчаги для кваса и пива.\\
Глиняные противни.\\
Форма для караваев и сальников.\\
Глиняные муравленные блюда и чашки.\\
Есть еще поваренная посуда чугунная, покрытая внутри эмалью, но она мало еще ввелась в употребление и потому нельзя ничего сказать об ее удобстве и прочности.

\subsection{Общие замечание о кухонной посуде.}
Вообще всю посуду кухонную должно содержать в чистоте; после каждого употребление надлежит мыть и чистить; деревянную мыть щелоком, с песком, a медную всего лучше чистить таким образом: нужно иметь кадочку с квасною гущею; когда кастрюли более не нужны, то положить их в гущу часа на три или более, а потом вымыть горячей водой. Иногда нужно кастрюли чистить снаружи мелким кирпичом с уксусом или золой\footnote{Нынче этот способ чистки посуды успешно заменяет особенный порошок, продаваемый во многих посудных и металлических лавках.}. За недостатком гущи, или, когда кастрюли пригорели, скрести ножом их не должно, а, положив в них золы, налить водою, поставить в печь или на плиту выварить, и наконец вымыть. Также не должно позволять чистить кастрюли внутри песком, ибо от этого сходить полуда, а при недосмотре это может обратиться во вред для здоровья; а потому хозяйка или экономка должны по временам осматривать полуду на медной посуде. Медную посуду, смотря по ее употреблению, должно лудить, по крайней мере, раза два в год. Глиняную посуду каждый раз после употребления вымыть и поставить в печь, чтоб она просохла. Медные формы для пирожного тотчас после употребления вымыть, вытереть сухим полотенцем, а иногда их нужно также выварить с золой. Вафельные формы и формы для трубочек после употребления мыть не нужно, а только вытереть чистым полотенцем; держать их должно в сухом месте, чтоб они не ржавели. Противни и сковороды после употребления следует чистить золой.

Все то, что составительницею этой книги, госпожою Авдеевой, сказано относительно важного значение, какое в кухонном хозяйстве имеет вообще медная посуда, весьма справедливо и правильно. Но совсем тем из всего этого явствует, что посуда эта требует крайне осторожного с собою обращение и постоянного возобновление полуды, потому что малейшее в этом случае со стороны хозяйки невнимание или забывчивость влекут за собою печальные результаты, именно проявление медной зелени, не только расстраивающей желудок, но даже производящей отравление, случаев какого бывало и, к сожалению, бывает не мало. За всем тем, там, где кухонное хозяйство в руках опытной и рачительной хозяйки, все-таки медная посуда лучше всякой другой, преимущественно по причине своей прочности, соединенной с наружною красотой, потому что нельзя не сознаться, что любо-дорого взглянуть на хорошую кухню, наполненную медною, горящею как жар посудою, покрывающею все полки по кухонным стенам.

Однако существуют в торговле и введены в некоторых хозяйствах и кое-какие другие сорта кухонной посуды, о которой именно мы считаем теперь необходимым сказать здесь несколько слов.

\begin{enumerate}
	\item Жестяные кастрюли, довольно распространенные и очень дешевые (50 к. штука самая крупная), имеют то качество, что в них все варится и жарится очень быстро; но при этом качестве имеют тот недостаток, что, при малейшей неосторожности, дно этих кастрюлей легко прогорает, сделать же новое дно стоит 50 к., так что <<дешевизна выходит дороговизною>>, да к тому же, при малейшей неосторожности как нельзя легче испортить при варке или жарении самый лучший материал для хорошего кушанья, которое, таким образом, будет подано на стол или переваренным, или пережаренным, что то и другое, как известно, никуда не годится.
	\item Кастрюли из белого железа, впервые лет за 25 появившиеся в магазине Гризара, существующем и теперь на Невском, где Милютины лавки, a ныне находимые во многих посудных лавках, — имеют за собою много весьма хороших качеств, относительно безопасности для здоровья, опрятности, удобства всех варев и жарений, даже наружной красоты, потому что ежели хорошо вычищенные медные кастрюли похожи на золото, то кастрюли из белого железа имеют вид серебряной посуды; но тут-то и камень преткновение, особенно в руках нашей русской прислуги, у которой так трудно выбить из упрямой головы, что для чистки посуды нет ничего лучше мелкого кирпича и кислот, с незапамятных времен употребляемых при чистке медной посуды, между тем как именно этот кирпич и эта кислота действуют губительно разрушительным образом на бело-железную посуду, для успешной и правильной чистки которой необходимо употреблять, так называемый, английский композиционный кирпич, продаваемый по 25 к. кусок в английском инструментальном магазине г. Чиксона, на Екатериненском канале, против Казанского собора, или, скорее, против монументов фельдмаршалов. Превращенный, в ступке, в наимельчайший порошок, в виде красноватой пыли, этот кирпич — превосходное средство для всевозможных металлических вещей, будучи вполне пригоден и для медной посуды, которую в истинно домовитых кухнях простым кирпичом не чистят.
	\item Каменная кухонная посуда, вполне безопасная и превосходная, ежели только она такова, какою была лет за 20 пред сим <<гинтеровская>>, т. е. фабрики Гинтера посуда, темного, почти черного цвета и весьма неграциозных форм. Но, к сожалению, этой настоящей первоначальной каменной кухонной посуды, в настоящее время всяких наружных усовершенствований и кажущегося прогресса с огнем, как говорится, найти трудно: ее заменила светло-красноватая, весьма изящная и красивая посуда, полезная, преимущественно, для вида, и чтобы ею любоваться; но она ломка, хрупка, огнеупорна и вообще далеко не полезна, хотя ею битком набиты посудные лавки.
\end{enumerate}

\subsection{Кухонное белье.}
В кухне должно иметь достаточное количество белья. Вот реестр самого необходимого:\\
Полотенцев толстых 12.\\
Полотенцев тонких 12.\\
Фартуков 12.\\
Квашенников, для накрывания теста, разной величины, 6.\\
Салфеток, для процеживание желе, бланманже и галантиров, 3.\\
Следует наблюдать, чтоб белье всегда было хорошо вымыто; чистота и опрятность суть основные правила поваренного искусства.
