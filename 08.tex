\section{СОУСЫ И ПОДЛИВКИ РАЗЛИЧНЫЕ К ХОЛОДНЫМ И ГОРЯЧИМ БЛЮДАМ И К ЖАРКИМ} %отдел 8

\z{Красная подливка}\index{Соус ! красная подливка}

Поджарить на сковороде в масле докрасна муки, потом нарезать кружечками луку, прибавить лаврового листа, толченой гвоздики, немного коринки, шинкованного сладкого миндалю, свежей шинкованной лимонной корки; все это вскипятить один раз и откинуть на сито. Когда стечет вода, смешать с поджаренной мукой, развести бульоном, дать повариться, беспрерывно мешая, чтоб не пригорело; наконец прибавить немного уксусу, сахару, лимонного соку и вскипятить один раз. С этой подливкой подают утку, телячью головку, языки. 

\z{Темный соус}\index{Соус ! темный}

Поджарить докрасна в масле муки, прибавить мелко изрубленного луку и еще поджарить; потом положить туда же ложку или две горчицы, лимону, нарезанного четвертинками, развести бульоном, чтобы вышел густоватый соус, прокипятить и положить по вкусу уксусу и сахару. 

\z{Желтый соус}\index{Соус ! желтый}

Взять ложку свежего чухонского масла, 2 яичные желтка, 2 столовые ложки горчицы, цедру с 1 лимона, стертую на сахар, немного уксусу; растереть все хорошенько ложкой, а потом развести бульоном, чтоб был густоватый соус. 

\z{Белый голландский соус}\index{Соус ! белый ! голландский}

Две ложки чухонского масла, 5 желтков, ложку муки, немного толченого мускатного цвету, цедру с лимона, тертую на терке, смешать в се в месте и тереть до тех пор, пока побелеет; потом развести, по пропорции, горячею водой и выдавить сок из одного лимона. 

\z{Красный соус}\index{Соус ! красный}

Взять кусок чухонского масла величиной с куриное яйцо, полную ложку муки, мелко порезанного шарлоту или простого луку, рубленной зеленой петрушки, немного тмину и все в месте хорошенько растереть. Потом, искрошив луковицу, поджарить докрасна в масле, влить стакана два бульону, дать раза два или три прокипеть, процедить, развести этим бульоном соус по пропорции и выдавить в него сок из одного лимона. Соус этот можно подавать ко всем паштетам. 

\z{Желтый лимонный соус}\index{Соус ! желтый ! лимонный}

Взять ложку чухонского масла, немного муки, 4 яичные желтка, цедру с 2 лимонов, стертую на сахар (сок из лимонов выдавить), в с е это хорошо растереть и развести не очень жидко виноградным вином пополам с водой. С этой подливкой или соусом подают начиненные и просто сваренные языки, цыплят, утку. 

\z{Лимонный соус с миндалем}\index{Соус ! лимонный с миндалем}

Полную ложку чухонского масла стереть с 4 желтками добела, прибавить щепоть муки, цедру с 2 лимонов, стертую терочкой, выдавить сок из лимонов, остерегаясь, чтоб не попали зерна, положить сахару по вкусу и еще немного растереть; потом развести по пропорции белым виноградным вином и положить сладкого шинкованного миндалю. К пудингам хлебным. 

\z{Белый соус с каперсами}\index{Соус ! белый ! с каперсами}

Кусок чухонского масла с куриное яйцо стереть с 2 желтками добела, прибавить немножко толченого мускатного цвету, цедру с 1 лимона, растереть хорошенько; потом положить горсть каперсов, развести по пропорции бульоном и прибавить по вкусу крошечку сахару. 

\z{Красный соус с каперсами}\index{Соус ! красный ! с каперсами}

Поджарить докрасна в масле муки, искрошить помельче луку, прибавить лаврового листу, рубленной зеленой петрушки, горсть каперсов положить в горячую поджаренную муку, подержать на вольном жару, мешая, чтоб не пригорело; потом развести бульоном, прибавить цельной гвоздики, лимонного соку, немного сахару и дать повариться, чтоб был густоватый соус. Соус этот подают к паштетам, зайцу, домашним и диким уткам, вареным языкам. 

\z{Белый соус с шарлотом}\index{Соус ! белый ! с шарлотом}

Взять кусок масла с куриное яйцо, ложку муки, горсти две мелко рубленного шарлоту, а за неимением его~--- простого луку, стереть все в месте хорошенько; потом развести бульоном, чтоб соус был густоват; напоследок приправить по вкусу уксусом и вскипятить один раз. Его можно подавать ко в сем жареным мясам, также к жареному каплуну. 

\z{Анчоусовый соус}\index{Соус ! анчоусовый}

Взяв ложку или две чухонского масла, смотря по тому, сколько хотите сделать соусу, ложку муки, горсть мелко изрубленного шарлоту, 5 анчоусов, мелко изрезанных, лимону, нарезанного четвертинками, лаврового листу, шампиньонного порошку; смешать все вместе, потереть, развести бульоном и прокипятить. Соус этот можно подавать ко всякой рыбе и к рыбным паштетам. 

\z{Соус из селедки}\index{Соус ! из селедки}

Вымочив в молоке или квасе селедку, снять с нее кожу, выбрать кости, изрубить мелко с куском чухонского масла и шарлотом или простым луком, прибавить рубленных шампиньонов, лимонного соку, немного перцу толченого, лаврового листу, стереть все вместе развести бульоном и вскипятить раза два. Соус из селедки по нужде может заменить анчоусовый. 

\z{Соус из устриц}\index{Соус ! из устриц}

Вынув устриц из раковин, налить водой, чтоб вода только покрыла их, дать раз вскипеть. Потом кусок чухонского масла стереть на жару с ложкой муки и прибавить ложку шампиньонного порошку и лимонного соку. Процедив сквозь сито воду, в которой варились устрицы, вылить туда же, а если воды будет недостаточно, налить виноградного вина чтоб был густоватый соус, посыпать толченого мускатного цвету и вскипятить; наконец, отняв у устриц бороды, чистое тело положить в соус. Эту подливку подают к жареному каплуну, к жареной и вареной рыбе, и особливо к щуке, равно как к паштетам и разварным цыплятам. 

\z{Соус из масла}\index{Соус ! из масла}

Взять сколько нужно хорошего чухонского масла, вымыть его хорошенько в холодной воде; потом половину масла положить в кастрюлю, с несколькими ложками воды и со щепоткой муки; варить на вольном жару, непрерывно мешая и подымая масло вверх ложкой; затем и другую половину масла класть по кусочку в кастрюлю, а когда положите последний кусочек масла, и он разойдется, то тотчас должно снять с огня: иначе масло створожится или оттопится, и гуща осядет на дно. Подают его к рыбе и рыбным паштетам, а особливо к паштету из трески. 

\z{Сливочный соус}\index{Соус ! сливочный}

Взять ложку чухонского масла, щепотку муки, 2 яичные желтка, горсть петрушки, сваренной один раз в воде и мелко изрубленной, немного толченого мускатного цвету и соли; все вместе хорошенько растереть и развести по пропорции сливками. Соус этот подается к вареным карасям и другой рыбе, также к тресковому паштету; тогда можно прибавить в него мелко изрубленного шарлоту или простого луку. 

\z{Желтый соус с коринкой}\index{Соус ! желтый ! с коринкой}

Берется кусок мытого коровьего масла величиной с куриное яйцо, 4 желтка, щепоть муки, стереть вместе, прибавить сок из одного лимона, коринки, выбранной и вымытой в холодной воде, мускатного цвету и сахару по вкусу, и развести виноградным вином. Соус этот подается к пудингам исключительно. 

\z{Соус из виноградного вина}\index{Соус ! из виноградного вина}

Ложку промытого чухонского масла, 2 желтка, щепоть муки, немного толченого мускатного цвету, стереть хорошо вместе и развести белым виноградным вином. Подается к разным сладким блюдам. 

\z{Шоколадный соус}\index{Соус ! шоколадный}

Кусок промытого чухонского масла с куриное яйцо, 4 желтка, 1/4 ф. мелко-натертого шоколаду все вместе хорошенько стереть, а потом развести виноградным вином. Если от шоколада соус не довольно будет сладок, прибавить сахару. Подают его обыкновенно к пудингам. 

\z{Подливка на рыбу камбалу}\index{Соус ! подливка на камбалу}

Несколько горстей щавелю перебрать, вымыть и раза три порубить, чтоб был крупноват, потом положить в кастрюлю и, закрыв, дать попреть в собственном соку. Курок чухонского масла с куриное яйцо, пол-ложки муки и 2 желтка стереть вместе, положить в щавель и развести бульоном, чтоб была густоватая подливка, посолить и посыпать мелким перцем кайенским. 

\z{Соус из трюфелей, сморчков и шампиньонов}\index{Соус ! из трюфелей, сморчков и шампиньонов}

Взять сморчков, трюфелей и шампиньонов, изрезать, а потом изрубить мелко. Ложку муки поджарить с ложкой масла докрасна, положить в нее мелко крошенного луку, трюфелей, шампиньонов и сморчков; все вместе перемешать, развести бульоном, приварить, чтоб была густоватая подливка, и затем приправить лимонным соком. Эту подливку подают к вареной говядине и ко всему, что варится в брезе. 

\z{Английский соус}\index{Соус ! английский}

Взять кусок чухонского масла с куриное яйцо, ложку муки и стереть с маслом. Потом взять 2 или 8 анчоуса, выбрать из них кости, полную столовую ложку каперсов, горсть рубленного шарлоту, горсть зеленой петрушки, отваренной в воде; перемешав все вместе, изрубить мелко, смешать с маслом и мукой, развести по пропорции бульоном, приварить, а когда будет готово, изрубить несколько желтков из круто сваренных яиц и положить в соус. Этим соусом обливают жареных каплунов и молодых кур; также подают его к жареной телятине. 

\z{Белый соус под жаркое}\index{Соус ! белый ! под жаркое}

Взять 5 анчоусов или сардинок, вынуть из них кости и с луком и небольшим кусочком масла мелко изрубить. Потом положить в кастрюлю кусок масла с куриное яйцо, ложку муки и рубленные анчоусы и все вместе на легком жару потереть, посыпать немного толченого перцу, развести бульоном, размешать, подержать на жару, приправить по вкусу лимонным соком, положить лимону, нарезанного четвертинками. Соус этот можно подавать ко всякому жаркому; также хорошо обливать им жареную говядину. Если нельзя будет достать ни сардинок, ни анчоусов, то вместо них взять кусок вымоченной в молоке или квасу селедки. 

\z{Пряный соус}\index{Соус ! пряный}

Взять 2 луковицы, 1 луковицу чесноку, базилики, лаврового листу, изрубить все вместе, прибавить ложку чухонского масла и дать на жару попреть. Потом влить полстакана уксусу, посыпать немного толченого перцу, имбирю и еще поварить, развести по пропорции бульоном, вскипятить один раз, процедить сквозь сито; кусочек чухонского масла с грецкий орех стереть с мукой, заправить соус и приправить по вкусу лимонным соком. Подается к рыбным блюдам. 

\z{Немецкий соус}\index{Соус ! немецкий}

Ложку чухонского масла стереть с ложкой муки, приправить крошенным шарлотом и простым луком, поджарив немного; потом положить рубленной петрушки и базилики, лаврового листу, каперсов, крупно накрошенных трюфелей, нарезанных кружочками корнишонов или нежинских огурчиков (можно класть и свежие огурцы, очистив с них кожу и нарезав кусочками), очищенных от костей оливок, парочку сарделей, нарезанных кусочками, развести виноградным вином пополам с бульоном, дать всему вместе повариться и приправить лимонным соком. 

\z{Белый соус из сарделей}\index{Соус ! белый ! из сарделей}

Кусок чухонского масла и ложку муки стереть вместе, подержать на жару, потом прибавить 5 желтков и парочку мелко изрубленных сарделей; все это с маслом, яйцами и мукой хорошо растереть, прибавить рубленной зеленой петрушки, развести бульоном пополам с виноградным вином, приправить лимонным соком и уксусом; положить круто сваренных и нарезанных кусочками яичных желтков. 

\z{Рейнский соус}\index{Соус ! рейнский}

Поджечь кусочек коровьего масла дожелта, положить в него крупно-изрубленной зеленой петрушки, прибавить рубленных сарделей, развести бульоном, приправить уксусом и немного поварить. Соус этот можно употреблять ко всякому жаркому, как то: к жареной говядине, котлетам, дичине, дворцовой птице, также к жареной рыбе. 

\z{Польский соус}\index{Соус ! польский}

Пяток сардин или анчоусов вымыть в холодной воде, потом вместе с костями изрубить с кусочком чухонского масла, положить в кастрюлю, прибавить бульону, рейнвейну, две цельные луковицы, нашпигованные гвоздикой, корешок петрушки, тертого белого хлеба; все вместе довольно поварить, процедить сквозь сито, поджарить горсть рубленного шарлоту в масле и налить процеженной жижей. Подается к зразам и ко в сем мясам. 

\z{Соус-ремулад}\index{Соус ! ремулад}

Луку, зеленой петрушки, чисто вымытых сарделей, каперсов, все вместе изрубить мелко, прибавить немного толченого перцу, смочить прованским маслом, положить столовую ложку горчицы, развести по пропорции уксусом (если от сарделей не будет довольно солон, то присолите) и смешать все хорошенько. Подают его к мясным и рыбным жарким. 

\z{Вестфальский соус}\index{Соус ! вестфальский}

Обрезать корку с 1 французского хлеба, взять хлуп (белое мясо у птицы) из жареной курицы, 2 яичные желтка, 3 печеные луковицы, истолочь все вместе домягка, развести бульоном, уварить хорошенько, прибавить толченого перцу, процедить сквозь сито. 

\z{Вишневый соус}\index{Соус ! вишневый}

Возьмите сухих вишен, истолките их в ступе, чтоб косточки разбились; потом, отварив их в воде, смешанной пополам с виноградным вином, протрите сквозь сито и приправьте сахаром, толченой корицею и мелко нашинкованной лимонного коркой. К пудингам. 

\z{Итальянский соус}\index{Соус ! итальянский}

Влив в кастрюлю полбутылки белого виноградного вина, положите кусок чухонского масла, рубленной зеленой петрушки, мелко-накрошенных шарлот и рубленных шампиньонов, поставьте на огонь; варите до тех пор, пока останется мало соусу; тогда разведите по пропорции бульоном и приправьте лимонным соком. Особенно к рыбным блюдам. 

\z{Красный петрушечный соус}\index{Соус ! красный ! петрушечный}

Растопите на сковороде кусок чухонского масла, положите в него зеленой рубленной петрушки и жарьте до тех пор, пока трава сделается почти сухой. Потом посолите, посыпьте немного толченого перцу, приправьте по вкусу уксусом. К рыбе и к мясам. 

\z{Молочный соус}\index{Соус ! молочный}

Возьмите кусок чухонского масла, величиной с куриное яйцо, сотрите в кастрюле с полной ложкой муки, прибавьте стакан сливок; потом поставьте на огонь и мешайте, чтоб смесь не пристала к стенкам кастрюли; после влейте еще два стакана сливок, прокипятите и процедите сквозь сито. Соус этот подают к рыбе, к фаршированным яйцам и к шпинатному пудингу. 

\z{Соус-пикан (Sauce piquante)}\index{Соус ! пикан}

Подожгите в кастрюле дотемна кусок коровьего масла, положите в него горсти 2 тертого белого хлеба и жарьте, пока хлеб потемнеет. Тогда разведите по пропорции бульоном, виноградным вином и уксусом, приправьте толченой гвоздикой, имбирем, лимонного цедрой, стертой терочкой, сахаром и коринкой, и уварите до надлежащей густоты. Этот соус подают с жареными карасями и с другой рыбой. 

\z{Соус из кислой капусты}\index{Соус ! из кислой капусты}

Возьмите шинкованной кислой капусты, сварите в воде домягка и откиньте на сито; когда вода стечет, положите капусту налейте сметаной, прибавьте хороший кусок коровьего масла, перцу и дайте еще попреть на вольном жару. Ко всякому мясу, как говядина, баранина, свинина и пр.

\z{Соус перечный}\index{Соус ! перечный}

Положите в кастрюлю горсть петрушки целыми листами, порею, лаврового листу, щепотку тмину, чайную чашку перцу, чайную чашку уксусу, немного коровьего масла; поставьте на огонь и дайте вариться, пока уксус почти совсем выкипит; тогда влейте 2 стакана хорошего бульону и поварите еще. Ложку муки поджарьте с ложкой масла докрасна, разведите немного бульоном, заправьте соус, прокипятите и процедите сквозь сито. Подавать к мясным и рыбным блюдам. 

\z{Соус метр-д'отельский холодный}\index{Соус ! метр-д'отельский холодный}

Возьмите 1/4 ф. чухонского масла, щепотки 2 мелко изрубленной петрушки, шарлоту мелко накрошенного, соли, крупно толченого перцу, сок из одного лимона; смешайте все вместе и разотрите хорошенько деревянной ложкой. Этот соус употребляют с мясом и рыбой. 

\z{Соус с маслом раковым}\index{Соус ! с маслом раковым}

Возьмите 2 стакана бульону, ложку муки, сотрите с ложкой масла, положите в бульон и уварите до надлежащей густоты. За 1/4 часа перед тем, как подавать на стол, положите в соус кусок ракова масла с куриное яйцо и мешайте хорошенько, чтоб масло соединилось с соусом. Преимущественно к телятине и ко всякой рыбе. 

\z{Соус грибной}\index{Соус ! грибной}

Полстакана муки поджарить в 1 ложке масла, развести 2 1/2 стакана грибного бульону, сваренного из 3--4 грибков и 2 целых луковиц, можно влить 1/2 стакана сметаны, вскипятить несколько раз, положить мелко нашинкованных сваренных грибов, соли; облить на блюде вареную утку или гуся.

Этот соус делают также на 6 человек из большой пропорции т. е. из 3/4 стакана муки, 5- 6 грибов, 2 луковиц, 1 или 1 1/2 стакана сметаны, 1/8 фунта масла; в таком случае соусу будет от 4 до 5 стаканов, облить им утку или гуся, подавать на глубокой блюде или в салатнике.

Соус грибной подается также к картофельным или рисовым котлетам; его должно быть тогда около 3 стаканов. Выдать: 1/3 стакана муки, 1/8 фунта масла, 2 луковицы, 1/4 фунта грибов, 1/4 – 1/2 стакана сметаны. 

Такой соус делается также постный, в таком случае вместо чухонского выдать 1 1/2 ложки горчишного или иного постного масла. (6) 

\z{Раковое масло}\index{Соус ! раковое масло}

Скорлупу от раков вымыть, вытереть, истолочь как можно мельче. На 2 стакана истолченной скорлупы взять 1 стакан масла, распустить его на сковороде, всыпать скорлупу, жарить на легком огне, мешая, чтобы не пригорело, до тех пор, пока масло не с делается темно-красного цвета; тогда процедить и выжать сквозь новую салфетку, слить в банку, остудить, поставить в холодное место. Если же это масло тотчас пойдет в употребление, то на 1/8 ф. масла истолочь скорлупы от 25 до 30 раков, процеживая сквозь сито; чтобы в скорлупках не осталось масла, облить их, мешая и выжимая, молоком, из которого будет вариться соус, или бульоном, если будет вариться суп, как наприм. суп из раков. (6 ) 

\z{Соус вишневый к пудингу}\index{Соус ! вишневый к пудингу}

Взять 1 ф. зрелых вишен, вынуть косточки, 12--15 косточек истолочь, смешать с вишнями, влить 1/2 стакана столового вина, 1--1 1/2 стакана воды, вскипятить раза два, протереть сквозь сито, положить немного толченой корицы и цедры с 1 1/2 лимона, 1 чайную ложку картофельной муки, размешанной с ложкой воды, 1/4 ф. сахару, чтобы было сладко, вскипятить. 

\z{Соус клюковный к пудингу}\index{Соус ! клюковный к пудингу}

2/3 стакана клюквы налить немного водой, положить 1/2 вершка корицы, вскипятить, процедить, растирая ложкой; взять этого соку 2 1/2 стакана, всыпать сахар, вскипятить, остудить, облить пудинг или положить 2 чайные ложечки картофельной муки, размешанной с 1 ложкой воды, вскипятить, мешая шибко минуты 3--4 , облить пудинг; остальное подать в соуснике. 

\z{Подливка из помидоров}\index{Соус ! подливка из помидоров}

Очистить 12 красных помидоров, размять их в кастрюле и дать увариться в своем соку. Когда будут готовы, протереть сквозь решето и пюре выложить в кастрюлю с двумя ложками белой подливки. В случае необходимости ее можно заменить простым белым соусом, разведенным бульоном. Смешав все, посыпать немного мелкого сахару и прибавить кусок масла в то время, когда снимается с огня. 

\z{Холодный соус к холодной дичи}\index{Соус ! холодный к холодной дичи}

На куске сахару истереть совершенно 2 лимона, и этот сахар мелко истолочь и смешать с несколькими мелко искрошенными шарлотками, а при недостатке их с истолченной луковицей и 4 твердыми желтками; прибавить сок из 2 лимонов, 20 мелко истолченных ягод можжевельника, влить уксусу и прованского масла, по пропорции, взбить все это вместе, и подавать в соуснике. Холодная дичь гораздо вкуснее с этим соусом, нежели с салатом. 
