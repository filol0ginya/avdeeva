Полная поваренная книга опытной русской хозяйки или руководство к уменьшению расходов в домашнем хозяйстве заключающая: описание разных кушаньев, печений, домашних запасов, водок, наливок, напитков и проч., с подробным указанием выдачи для них провизии мерою и весом; описание разных обедов: скоромных, постных, простых и праздничных; полное описание кухни и ее принадлежностей и практическое наставление, как закупать и экономнейшим образом распределять провизию.
С рисунками в тексте.

Сочинение Екатерины Алексеевны Авдеевой.
Новое издание, дополненное новейшими сведениями.

С.-Петербург.

Издание книгопродавца Д.~Ф.~Федорова

1875~г.

\newpage
\section*{ОТ ИЗДАТЕЛЯ}

Управлять домом и распоряжаться столом -- вещь трудная, в особенности для молодых, неопытных хозяек. Сколько им приходиться делать излишних расходов, вследствие неумение отличать хорошие припасы от дурных и экономно распорядиться провизией, а также вследствие недобросовестности прислуги, готовой на каждом шагу пользоваться неопытностью хозяйки. Избежать всех этих неприятностей хозяйка может лишь основательным изучением хозяйства, и потому нашею целью было издать такое руководство, которое, объясняя всю сущность хозяйства: закупку и сохранение провизии, выдачу и распределение ее мерою и весом, приготовление, как можно экономнее, как простых, так и изысканных блюд, давало бы каждой хозяйка возможность в короткое время усовершенствоваться в ведении домашнего хозяйства и поверять действия прислуги, и тем избавляло бы ее от излишних расходов. Мы будем вполне вознаграждены за наши труды, если хотя отчасти достигнем нашей цели.

С.-Петербург.

Август, 1874 года.
