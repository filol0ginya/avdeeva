\section{РАЗНЫЕ МЯСНЫЕ, РЫБНЫЕ, ОВОЩНЫЕ \\БЛЮДА ПОД СОУСАМИ} % отдел 9

\z{Индейка под соусом}\index{Индейка ! под соусом}

Разняв молодую индейку начетверо, изжарить в кастрюле с чухонским маслом, зеленым рубленным луком и петрушкой; когда индейка поспеет, разнять на части, облить следующим соусом: взять ложку муки и ложку чухонского масла, стереть в месте, поставить на огонь, и дав разойтись маслу, прибавить два куска сахару, выжать сок из лимона развести бульоном, посыпать немного мускатного цвету, подбить двумя яичными желтками. 

\z{Курица с шампиньонами}\index{Курица ! с шампиньонами}

Выпотрошив курицу, вымыть, сварить до половины готовности в воде с солью, потом разнять на части. Положить в кастрюлю 1/4 ф. коровьего масла, кипятить, пока покраснеет, прибавить ложку муки, мешать, не снимая с огня; когда мука поджарится, развести бульоном, положить курицу, разнятую на части, мелко искрошенных шампиньонов и выжать сок из одного лимона; варить до готовности. 

\z{Баранина под соусом с вином}\index{Баранина ! под соусом с вином}

Кусок баранины разрезать на части, вскипятить раза два, перемыть в холодной воде, налить процеженным тем же бульоном сварить до мягкости, положить соли, 2 луковицы, лаврового листу и английского перцу. Распустить ложку масла, поджарить в нем ложку муки, развести бульоном влить немного лимонного соку, положить баранину, вскипятить, на - конец прибавить столового вина, 2--3 куска сахару; перед отпуском взбить 2--3 желтка, подогреть. 

\z{Верещака}\index{Верещака}

Свежую свиную грудинку разрубить на куски, посолить, поджарить с обеих сторон в кастрюле, подложив свежего шпику. Переложить в другую кастрюлю, влить воды и столько свекольного рассолу, чтобы вкус был кисловатый, положить 5 зерен английского и 5 зерен простого перцу, 1 мелко изрубленную печеную луковицу, вскипятить хорошенько; наконец всыпать ложки 4 тертого хлеба, но, чтобы соус не слишком был густ, размешать; вскипятить и отпускать к столу. 

\z{Гусиные потроха в пряном соусе}\index{Гусиные потроха ! в пряном соусе}

Очистить гусиные потроха вместе с гусиной печенкой, налить водой, положить кореньев и пряностей, сварить. Около стакана гусиной крови разбить с ложкой или двумя уксусу, процедить, смешать с 3--4 стаканами бульону от потрохов, положить куска 2--8 сахару, 2 толченые гвоздики, 5--6 зер. англ. перцу, вскипятить, беспрестанно мешая, сложить потроха, подогреть, выложить на глубокое блюдо. (6--8) 

\z{Котлеты с бешамелем из сметаны или молока}\index{Котлеты ! с бешамелем}

Приготовить котлеты рубленные или битые, не намазывая яйцом и сухарями, положить их на противень, намазанный 1/2 ложкой масла, покрыть каждую котлету сверху бешамелем; можно посыпать сыром; поставить в печь. Бешамель приготовить следующим образом: 1,5 ложки масла, 0,5 стакана муки вскипятить раз 5, влить 3 /4 стакана сметаны, вскипятить, мешая несколько раз, чтобы погустело, остудить; можно сверху смазать яйцом. (Каждая котлета на одну персону). 

\z{Шарлот из говядины}\index{Шарлот ! из говядины}

Взятую от филея часть говядины уварите до спелости; между тем приготовьте соус таким образом: возьмите 2 ложки коровьего масла, ложку муки, обжарьте докрасна в масле, туда же положите 2 горсти шарлоту целиком или простых мелких луковиц, уварите их в масле до половины; потом возьмите поровну красного вина и бульону, прибавьте немного кореньев, положите в кастрюлю, уварите до спелости и облейте этим соусом говядину. Будет очень вкусное блюдо, которого хватит на 5--6 человек, смотря по числу обедающих. 

\z{Марешаль из рябчиков}\index{Марешаль ! из рябчиков}

Взяв рябчики, снять с каждого по 2 филея, оставив косточки крылышек по 1-й сустав, потом надрезать филеи вдоль сбоку, нафаршировать, зашить, обвалять в яйце, изжарить на фритюре или обвалять в яйце и сухарях и изжарить на рашпоре. 

Фарш следующий: заправить красный соус, а именно: 1/8 фунта масла, 1/2 стакана муки, развести 1,5 стаканами бульону, посолить, прокипятить раза два-три, влить рюмку мадеры, положить рубленных сырых шампиньонов, шт. 6, трюфелей 1--2 штуки, вскипятить раза 4, остудить, нафаршировать надрезанные филеи. 

Сложить на блюдо рябчики, в середину положить следующее рагу: заправить белый соус, а именно: 2/3 ст. муки, 2 ложки ракового масла, с деланного из очистков раков, развести 2 стаканами бульону, положить сырых вымытых шампиньонов шт. 12, раковых шеек шт. 25, прокипятить все это вместе раза два. (Соус на 10--12 чел.). 

\z{Курица со спаржей}\index{Курица ! со спаржей}

Изжарить курицу в кастрюле, поливая маслом; когда будет готова вынуть и разнять на части. В масло, в котором жарилась курица, положить немного муки, развести бульоном. Очистив спаржу, отрезать все твердое, положить в приготовленный соус, уварить домягка; прибавить мелко изрубленной петрушки, немного сахару, выжать сок из одного лимона, подбить двумя желтками и облить соусом курицу. (5--6) 

\z{Каплун с начинкой}\index{Каплун ! с начинкой}

С делать начинку из печенки каплуна и 12 каштанов таким образом: очистив каштаны, изрубить вместе с сырой печенкой, обжарить в чухонском масле, развести 2 сырыми лицами, приправить мускатным орешком. Потом, отделив осторожно у каплуна кожу от мяса, начинить каштанами, вымазать каплуна маслом и зажарить в печи. Между тем приготовить следующий соус: поджарить ложку муки докрасна в масле, развести бульоном пополам с виноградным вином, прибавить уксусу, немного гвоздики и лимонной корки, дать вскипеть, процедить, облить каплуна, разняв на части. В этот соус, кому угодно, можно положить коринки, изюму и миндалю, нашинкованного полосками. (6--8)

\z{Начиненная утка с каштанами}\index{Утка ! с каштанами}

Взять хорошую утку, выбрать из нее кости, сделав прорез назади, потом приготовить фарш: взять телятины от задней ноги, мозгу из костей, или почечного сала, столько же, сколько телятины, изрубить мягко, положить изрубленного луку, зеленой петрушки, шампиньонов, 2 сырых яйца, 2 ложки сметаны, посолить, посыпать перцу и мускатного орешка. Начиняя этим фаршем утку, залить, положить в кастрюлю, обложить тоненькими ломтиками шпеку и разными кореньями, и луком, нарезанным кружочками, налить столько бульону, чтоб утка могла свариться в нем. Когда утка поспеет, вынуть из бульона, разнять на части; подавать с следующим соусом: очистить горсть каштанов, положить в кастрюлю, налить полстакана виноградного вина и полстакана кулису, уварить до готовности, а когда соус будет готов, облить утку. (5--8) 

\z{Утка с кореньями}\index{Утка ! с кореньями}

Выпотрошив и вымыв утку, заправить, как обыкновенно заправляют ее для жаренья; положить в кастрюлю ложку коровьего масла и ложку муки, поджарить докрасна, влить стакан бульону, смешать хорошенько, опустить в бульон утку, прибавить петрушки, кервелю и сельдерею, посолить, посыпать перцу, положить кореньев, петрушки, пастернака и моркови, нашинковав полосками, или нарезав покрасивее иным манером; положить их в одно время с уткой, а если коренья молодые и мягкие, то в половине варенья; накрыть кастрюлю и варить на легком огне. Когда утка будет готова, вынуть, разнять, уложить на блюдо, обложить кореньями, а травы, петрушку, сельдерей и кервель отбросить; если бульон выкипел, прибавить еще полстакана, вскипятить раза два и облить утку. (6--8) 

\z{Гренад из цветной капусты}\index{Гренад ! из цветной капусты}

Отобрать круглые кочаны цветной капусты, обрезать кочерыжки, но так, однако, чтобы самые кочаны не развалились, отварить в воде с солью, однако не очень мягко, и откинуть на решето, чтоб вода стекла. Потом взять разнятых на части цыплят, молодых голубей или молодого барашка, сморчков, отварить; далее обжарить в коровьем масле с вылущенными раковыми шейками, прибавить немного отвара, в котором варилось мясо, приправить чуть-чуть мускатным цветом и подбить истертым белым хлебом. В тоже время сделать фарш из мелко изрубленного мяса с тертым белым хлебом, коровьим маслом, пряностями, и замесить на яйцах. Когда все будет приготовлено, взять глубокую кастрюлю или форму, вымазать ее холодным коровьим маслом и уложить полосками ветчинного сала, на них положить цветную капусту, цветками вниз, плотно между собой, как на дне кастрюли, так и по сторонам; промежутки наполнить фаршем и сверху покрыть им же на палец толщиной, уровнять нагретым ножом и вымазать взбитыми яичными желтками. После того положить туда же фрикасе, т. е. изрезанное мясо, грибы, раковые шейки и проч., но без соуса, намазать еще фаршем, потолще, чем в первый раз, и укрепить со всех сторон, чтобы держалось. Наконец покрыть тоненькими ломтиками ветчинного сала, поставить форму в большую кастрюлю, в которую влить немного горячей воды, накрыть кастрюлю плотно и поставить запекаться в печь. Перед тем, как подавать на стол, ветчинное сало снять, положить на форму блюдо, в котором должен подаваться гренад, опрокинуть осторожно, обтереть сало и подавать соус от приготовленного этого фрикасе. (8--10) 

\z{Компот из телячьих мозгов}\index{Компот ! из телячьих мозгов}

Вымочив мозги, вскипятите, снимите перепонку, потом с делайте масляный соус, поджарьте в масле 12 шампиньонов с мелко рубленной зеленой петрушкой, положите их в соус; сварив и очистив 20 раков, также сварив и отрезав головки у спаржи, положите в соус вместе с мозгами; все это поварив 1/4 часа, выложите на блюдо и украсьте венчиком из слоеного теста. (8--10) 

\z{Фрикасе из гуся}\index{Фрикасе ! из гуся}

Вычищенного и надлежащим образом приготовленного гуся обдайте кипятком; потом положите гуся в кастрюлю, налейте столько воды, чтобы она покрыла гуся; посолите, прибавьте кореньев, луку, нарезанного кружками лимону, зеленой петрушки и уварите гуся домягка. Между тем ложку коровьего масла сотрите с ложкой муки, положите в масло вареных сморчков или шампиньонов, спаржи мелко-нарезанной и обжарьте все вместе; потом влейте немного мясного или гусиного отвара, влейте, по вкусу, уксусу, подбейте яичными желтками, выдавите сок из одного лимона и прокипятите 1 раз. Вынув гуся, дайте немного остынуть, облейте соусом и посыпьте толчеными сухарями. (6--8) 

\z{Гусь в шампиньонном соусе}\index{Гусь ! в шампиньонном соусе}

Взять молодого гуся, обжарить в масле до половины готовности, разнять на части, положить в кастрюлю, прибавить накрошенных крупно шампиньонов и артишоковых донышек, налить немного бульону, посолить посыпать перцу, положить несколько луковиц шарлоту, уварить до спелости, приправить уксусом. Когда гусь будет готов, уложить на блюдо, облить соусом. (6--8) 

\z{Гусь с каперсами}\index{Гусь ! с каперсами}

Изжарить молодого гуся, но чтоб он не был пережарен, разнять на части, уложить на блюдо. Подавать с следующим соусом: положить в кастрюлю 2 ложки каперсов, 2 вымоченные и мелко изрубленные анчоуса немного подпальной муки, ложку масла коровьего, посыпать перцу, прибавить горсть рубленных шарлот, или 1 луковицу, искрошенную мелко, обжарить все вместе, влить 1,5 стакана бульону, поварить 0,5 часа, облить гуся. (6--8) 

\z{Говядина разварная, с гарниром из кореньев}\index{Говядина ! с гарниром из кореньев}

Чтобы можно было выбрать из бульона хороший кусок говядины без костей на второе блюдо, на 6--8 человек, надо варить бульон из 6, но не менее 5 фунтов говядины. Вымыть ее, очистить, варить с кореньями, как обыкновенно. Вынув на блюдо, нарезать ломтиками, обложить капустой, красиво нарезанной морковью, репой, картофелем, итальянскими макаронами. Все это надо сперва сварить в бульоне, потом вынув, сложить в кастрюлю, положить ложку масла, влить 2 стакана жирного бульону, соли; поставить на легкий огонь на 0,5 часа. 

\z{Рябчики с оливками}\index{Рябчики ! с оливками}

Пару или более рябчиков изжарить в кастрюле с маслом. Соус приготовить таким образом: взять 1 луковицу, изрубить мелко, прибавить горсть рубленной зеленой петрушки, положить в кастрюлю лук и петрушку с ложкой коровьего масла, обжарить немного, прибавить ложку муки, полстакана жюсу, рюмку виноградного вина, развести бульоном, чтоб был негустой соус, дать раза 2 вскипеть, процедить. Потом влить опять в кастрюльку, положить чайную чашку оливок, поставить на огонь, дать увариться. Рябчиков разнять, уложить на блюдо, облить соусом. 

\z{Фаршированные оливки}\index{Оливки ! фаршированные}

Взять 50 оливок, выбрать из них кости, начинить их фаршем из телятины, опустить в кипяток и дать раз вскипеть. Вынув из воды, положить в кастрюлю, прибавить полстакана жюсу и полстакана бульону, положить небольшой кусок ветчины, закрыть кастрюльку; варить на легком огне 0,5 часа, а когда соус будет готов, вынуть ветчину и подавать ко всякой дичине, как то: к рябчикам, куропаткам, тетеркам, к диким гусям и уткам, к зайцу и даже к дикой козе. (10) 

\z{Зразы из зайца с лапшой}\index{Зразы ! из зайца с лапшой}

Очистить зайца, как следует, снять мягкие частицы с костей, изрезать их в продолговатые полоски, разбить осторожно сечкой и, разложив на доску, посыпать солью и перцем, наложить сверх каждой штуки ряд рубленных шампиньонов, завернуть в продолговатый рулет, сложить на растопленное масло в сотейник и поставить на легкий огонь покрытыми. Когда все обжарится до колера, залить красным соусом и сварить под крышкой до мягкости, потом выложить на блюдо, перекладывая крутонами. Зразы этим способом приготовленные подаются по желанию с рисом, кашею, лапшой и зеленью. (4) 

\z{Гаши из жареной телятины}\index{Гаши ! из жареной телятины}

Разрезать жареную телятину на куски, изрубить, положить в кастрюлю и вскипятить с коровьим маслом, лимонной коркой и бульоном. Если нужно, то посолить еще, потом прибавить тертого хлеба и подавать на стол, облив лимонным соком. В место мясного бульона можно подлить стакан вина и немного воды. (Ежели четверть телятины, то смело на 10--12 персон). 

\z{Телячья печенка в виде пирожков}\index{Телячья печенка}

Взять вареную телячью печенку (которую однако же не должно долго варить) дать ей простыть, растереть на терке, смешать с 3 или 4 яйцами, очищенным изюмом и коринкой, цедрой, сахаром, несколькими полными ложками растопленного коровьего масла и размоченной в молоке булкой (последняя не так нужна); положить все это в сетку, растянутую на блюде, завернуть и зашить. Пирожки должны иметь круглую, высокую форму. Потом поставить в печь и жарить в кастрюле с крышкой на которую наложить угольев. Эти пирожки подают с голландским или с красным соусом, без изюма, а также с соусом из красного вина. (5--6) 

\z{Телячья голова}\index{Телячья голова}

Хорошенько обмыть и, если надобно, отмочить в воде, приставить к огню, снимать пену, посолить и варить 3 часа. Голову класть в горшок таким образом, чтоб она не могла пригореть. Как только она достаточно размягчится, выложить на блюдо, срезать кожу, разложить красиво на том блюде, на котором хотят подавать на стол, снять все мясо с костей, а с языка кожу, очистить ее, нарезать полосками и обложить блюдо. Из мозга также вынуть кости, обвалять его в тертом хлебе и поджарить дожелта в растопленном масле. От головы отнять челюсть, положить голову посередине блюда, а все остальное красиво уложить вокруг головы. Подавать с изюмным соусом, в который можно прибавить мелко изрезанного миндалю. 

\z{Телячий бифштекс}\index{Телячий бифштекс}

Взять 6 фунтов филею, вынуть из него кости и изрубить их намелко, потом изрубить 1,5 ф. жирной свинины, взять 10 целых яиц, 1/4 ф. растопленного коровьего масла, мелко изрубленную луковицу, несколько лимонной корки, маленький белый хлеб, смоченный в молоке, и все это хорошенько перемешать. Кости варить 0,5 ч, положить на сковороду, густо обмазанную коровьим маслом, потом выложить на них мясо и поставить в печь. Когда телятина поспеет, ее обливают соусом из сарделей или иным острым соусом. (10--12) 

\z{Телячья грудина в фрикасе à la bourgeoise}\index{Телячья грудина ! в фрикасе}

Вычистив как следует хорошую телячью грудину, изрезать ее в круглые или четырехугольные кусочки, по усмотрению. Тогда налить на них холодной воды и поставить на огонь. За сим дать минут 6 покипеть и снять с огня. Действие это на поварском жаргоне называется бланшированием. Затем 1/4 фунта сливочного масла кладется в особую кастрюльку, где оно распускается на легком огне. В это распущенное масло складываются куски грудины, при чем они слегка пообжариваются на огне. Как только они обжарились, на них высыпается 2 ст. л. муки. Перемешайте все это хорошенько, чтобы оно порядочно соединилось. На эту смесь налить немного горячего бульону, размешать все, через что получится жидковатый соус, и положить сюда крошечку соли перцу корзиночку хорошо вычищенных шампиньонов и букетик\footnote{Букетик или пучок состоит из нескольких веток петрушки с одним лавровым листком и парочки перышек зеленого луку. Все это перевязывается тонкой ниткой и погружается в кастрюлю. (Эмбер).}. Все это поставить вариться часа на 2. Прибавьте к этому 1/4 ф. очищенного луку репчатого. Когда все это вместе сварится, то снимая с плиты грудину, лук и шампиньоны, сложите их в блюдо, соус же пропустите через тонкую чистую ветошку в особую кастрюльку. Ежели соус окажется слишком жидок, то поставьте его на плиту, осаживая его до тех пор, пока сгустится до степени густоты хороших сливок. Рагу, состоящее из грудины с шампиньонами и луком, сложите туда же и перед подачею на стол закрепите все это лезоном из 4 желтков и прибавьте сок из одного хорошего лимона, (6--8) 

\z{Телячья печенка по-мещански}\index{Телячья печенка ! по-мещански}

Возьмите хорошую, большую телячью печень. Чем она светлее и розовее, а не багровее, тем лучше. Нашпигуйте ее продолговатыми кусками шпеку, которые необходимо немного посолить и вспрыснуть молотым или толченым перцем и крошечкой мускатного ореха, доведенного до тонкости пыли. Когда печень нашпигована, положите в кастрюлю несколько тоненьких ломтиков шпеку; на эти ломтики выложите печень с 4 штуками репчатого луку, одним лавровым листом, пучком свежей петрушки и крошечкой соли. Все это прикройте еще новым слоем шпеку, а поверх этого слоя шпеку положите лист белой писчей бумаги только не глассированной, т. е. не глянцистой, а матовой; прибавьте 0,5 бутылки обыкновенного столового белого вина и поставьте на огонь. Когда закипит, то надобно поставить в плитную печку с крышкой и дать вариться или печься часа 1 времени, в течение которого надобно кушанье поливать, и ежели сок чересчур загустеет, то прибавьте обыкновенного говяжьего бульону, потом снимите с огня печенку, пропустите сок через салфетку, снимите жир и половину этого соуса соедините с таким же количеством пуаврадного соуса очень горячего. Дайте этой смеси повариться несколько минут и подавайте печень с этим соусом всю им облитую. (6)

\z{Рябчики в соусе}\index{Рябчики ! в соусе}

Изжарив рябчиков, разнять; подавать с следующим соусом: влить в кастрюлю стакан бульону и стакан виноградного вина, прибавить 2 кружочка лимону, немного лаврового листу, 1 корень петрушки и 1 луковицу; варить на легком огне час, процедить сквозь сито, посолить подбить двумя круто сваренными желтками положить рубленной зеленой петрушки, вымешать, подержать на плите 1/4 часа, облить рябчиков. (По пол рябчика на каждого человека). 

\z{Фазан или глухой тетерев в специальном соусе}\index{Фазан ! в соусе}

Фазана, или тетерева, изжарив, разнять на части. Соус приготовить следующий: влить в кастрюлю 0,5 бутылки красного виноградного вина, прибавить сахару, корицы, мускатного цвету, кардамону и гвоздики; когда начнет кипеть, положить фисташек и померанцевой корки, варенной в сахаре, изрезав ее мелко, подбить немного подпальной мукой, выжать сок из 1 лимона. Потом нарезать тоненькими ломтиками белого хлеба, обжарить в масле, положить на блюдо, а на хлеб уложить разнятую на части дичину и облить соусом. (6) 

\z{Язык бычачий в кисло-сладком соусе}\index{Язык ! в соусе}

Взять свежий язык, отварить в воде с солью, с прибавкой лука и лаврового листа, а когда поспеет, сиять кожу, разрезать по толщине надвое. Потом приготовить соус таким манером: очистить 1/8 фунта сладкого миндалю, нашинковать полосками, прибавить горсть коринки, сахару по вкусу, цедру с 1 лимона, влить понемногу уксусу и кулису, прибавить стакан бульону, уварить, подбить подпальной мукой, дать еще раз вскипеть и облить язык. (Один большой язык на 10 порций). 

\z{Язык с яблоками и миндалем}\index{Язык ! с яблоками и миндалем}

Сварив язык, снять с него кожу; положить на сковороду масла и дав ему закипеть, положить в масло язык, разрезав его по толщине надвое обжарить румяно. Между тем очистить несколько кислых яблок и 1/4 фунта сладкого миндалю; миндаль и яблоки нашинковать полосками, положить в кастрюлю, влить стакан виноградного вина и стакан воды, прибавить по вкусу сахару и немного гвоздики, уварить хорошенько. Положив язык на блюдо, облить соусом. 

\z{Фаршированный язык под соусом}\index{Язык ! фаршированный под соусом}

Уварив язык домягка, снять кожу, разрезать вдоль по толщине надвое. Приготовить из телятины фарш, намазать обе половинки языка в палец толщиной, пригладить горячим ножом, чтоб каждая половинка языка походила на целый язык; положив язык на сковороду, облить маслом, прибавить немного бульону, поставить в печь или духовой шкаф, дать зарумяниться. Соус к языку приготовить следующий: ложку коровьего масла растереть добела, положить немного муки, цедру с 1 лимона, стертую на сахар, выжать сок из 1 лимона, влить стакан виноградного вина и стакан воды, поставить на огонь, дать 1 раз вскипеть, потом подбить 2 желтками, вымешать. Положив язык на блюдо, облить соусом. 

\z{Рагу из телячьего сладкого мяса}\index{Рагу ! из телятины}

Взять телячьего сладкого мяса, вылущенных раковых шеек, сморчков, чашечек артишочных или что есть подобного в запасе; все это изрезать и сварить каждое, как должно, порознь. В кастрюлю положить коровьего масла, обжарить в нем немного муки, мелко изрубленного луку, прибавить лаврового листу, положить все это, налить хорошим мясным отваром, приправить необходимыми пряностями, и уваривать, почасту встряхивая кастрюлю, чтоб ко дну не прикипело. Подавая на стол, надобно подавить в соус лимонного соку. 

\z{Телятина в гвоздичном соусе}\index{Телятина ! в гвоздичном соусе}

Отварить телятину в воде с солью, а между тем положить в кастрюлю истертого ржаного хлеба, влить несколько мясного отвару и уксусу, приправить перцем, имбирем и гвоздикой; уваривать все это достаточно и положить туда же немалый кусок коровьего масла; наконец положить в соус телятину и дать прокипеть. (10--12, смотря по куску телятины). 

\z{Цыплята под соусом со спаржей}\index{Цыплята ! под соусом ! со спаржей}

Вычистив пару цыплят, выпотрошить, вымыть, отделить осторожно кожу от мяса; начинить цыплята за кожу следующим фаршем: обрезав с белого хлеба корку, размочить в молоке, а потом выжать досуха, положить 2 или 3 яйца сырые и ложку чухонского масла, растереть все вместе, прибавить рубленной зеленой петрушки, немного мускатного орешка, посолить, начинить цыплят, заткнув сначала отверстие у зоба пупком, чтоб фарш не проходил во внутренность цыпленка; когда начините, кожу у шеи завязать ниткой, a разрез назади зашить; таким образом начиняют всех птиц. Начинив и заправив цыплят, положить на противень, вымазать маслом, изжарить в печи, или положить в кастрюлю, жарить, поливая маслом, поворачивая на обе стороны, чтоб цыплята хорошо зарумянились. Соус под цыплят приготовляют следующий: отварить спаржу в соленой воде, но не давать перевариться; обрезав у спаржи все твердое, нарезать ее кусочками, в вершок длиной. Положить в кастрюлю ложку сливочного масла, стереть с ложкой муки развести сливками, положить туда же спаржу, приправить мускатным орешком, варить на малом огне полчаса. Отпуская на стол, цыплят разнять, уложить в соусник, посыпать зеленой рубленной петрушкой, облить соусом. (4) 

\z{Цыплята под соусом с шейками раковыми}\index{Цыплята ! под соусом ! с раковыми шейками}

Вычистив и вымыв цыплят, заправить, обжарить в кастрюле, поливая маслом; когда в половину ужарятся, разнять на части обвалять в сырых яйцах, обсыпать сухарями и дожарить на глубокой сковороде, в коровьем масле. Потом приготовить соус: взять полсотни вареных раков, очистить шейки, верхние черепки и клешни истолочь, положить в кастрюлю с ложкой свежего масла, жарить, пока масло покраснеет, тогда слить масло, а на гущу налить 2 стакана бульону; поварив полчаса, процедить сквозь сито, опустить раковые шейки, заправить немного мукой с раковым маслом, посолить, выжать сок из одного лимона, облить цыплят. (Пол цыпленка на каждую персону за столом). 

\z{Цыплята с горохом}\index{Цыплята ! с горохом}

Изжарив цыплят, разнять на части. Потом налущить зеленого гороху положить в кастрюльку с ложкой чухонского масла, подержать на легком жару, чтоб масло распустилось, налить бульоном и варить на легком огне час; бульону наливать не очень много, лишь бы он покрыл горох; 2 яичных желтка смешать с 2 ложками густых сливок, подбить соус. Выложив горох в соусник, уложить на него цыплят. 

\z{Телячьи легкие с шейками раковыми}\index{Телячьи легкие ! с раковыми шейками}

Сварив легкие, изрубите мелко и примешайте изрубленную же сырую печенку. Положите в кастрюлю и жарьте с 1/4 фунта масла. Потом изжарьте в масле рубленный лук, прибавьте тертой булки, взбейте пять сырых яиц, перцу, соли, и потом смешайте с жаренным легким и печенкой, и еще жарьте, мешая крепко, прибавив немного лимонном корки. Тогда намажьте эту смесь на тонкие пластинки сырой телятины, пластинки сверните, свяжите ниткой и жарьте в печи, на противне, в масле. (6) 

\z{Сальме из куропаток}\index{Сальме ! из куропаток}

Очищенные и заправленные куропатки (пара) сложить на растопленное в кастрюльке масло, посолить, поставить на огонь и изжарить под крышкой до мягкости; тогда вынуть куропатки на стол, разрезать частями, как следует, сложить в сотейник, а косточки положить обратно в кастрюлю, налить бульоном и выварить сок. Между тем очистить 10 шампиньонов, сварить в воде с лимоном, изрезать пластами, а сок из них и очистки положить к соку из куропаток; потом распустить ложку масла, положить 1,5 ложки муки и поджарить на огне. Когда мука начнет желтеть, развести соком из куропаток, выкипятить до надлежащей густоты и вкуса, процедить сквозь салфетку, залить куропатки и, вскипятив один раз, выбрать их на блюдо, переложить крутонами, а в соус положить ложку мадеры, 0,5 ложки масла, размешать и залить куропатки. (4) 

\z{Оленина маринованная}\index{Оленина ! маринованная}

Взять филейную часть оленины, вымыть, очистить, обрезать пашину и верхнюю кожицу, сложить в чашку или соразмерной величины горшок, налить сырым уксусом так, чтобы оленина была залита, положить соли, перцу, изрезанного пластами луку, частицу чесноку и оставить в маринаде 48 часов. Обрезки сложить в кастрюлю, налить водой выварить сперва бульон, а из бульона выварить сок до совершенной густоты, слить в чашку и оставить в холодном месте до приготовления оленины. Когда оленина промаринуется, вынуть из маринада, сложить на плафон и залить маслом, поставить в горячую печку, заколеровать, подлить бульону, покрыть крышкой и дожарить в печке окончательно. Когда будет готово снять с плафона на блюдо, а на плафон положить 1,5 л. муки, размешать, развести, как быть должно соусу, положить сок приготовленный из обрезков, выварить до надлежащей густоты, процедить и залить оленину на блюде. Любителям подается к этому жаркому особо желе из ягод или варенье. (4) 

\z{Почки с красным соусом}\index{Почки ! с красным соусом}

Запасеровать на масле мелко изрубленную луковицу. Между тем очистить и снять с почек верхнюю кожицу, изрезать в тоненькие ломтики и когда лук начнет желтеть, положить в кастрюлю, размешать, покрыть крышкой и поставить на огонь; потом взять на тарелку ложку масла и 1,5 ложки муки, смять вместе, положить в кастрюлю к запасерованным почкам, размешать, прибавить столько бульону, чтобы соус в почках был умеренно густ, снабдить по вкусу солью, перцем и мускатным орехом, положить рубленной зелени, вскипятить и выложить в глубокое блюдо. (4) 

\z{Зразы из телятины}\index{Зразы ! из телятины}

Изрезать мягкую часть телятины в длинные ломтики, разбить каждый ломтик сечкой, разложить на стол и посыпать солью и перцем. Между тем взять в чашку ложку масла, разбить добела, положить соли, перцу, мускатного ореха и тертого хлеба, размешать, наложить на каждую пластинку, свернуть в рулет, сложить на растопленное в кастрюльке масло, поставить на огонь и покрыть крышкой. Когда поджарится, перевернуть на другую сторону и поджарить окончательно; из обрезков телячьих сварить отдельно сок. Взять в кастрюльку ложку муки, развести немного холодным бульоном, потом злить стакан соку, процедить, залить зразы, положить букет зелена, кипятить еще немного, а когда зразы будут мягки и соус выкипит да густоты, тогда выложить на глубокое блюдо, вынуть букет прочь, а соус размешать, снабдить по вкусу солью, положить рубленной зелени и залить уложенные на блюде зразы. (4) 

\z{Отбивные телячьи котлеты}\index{Телячьи котлеты}\index{Котлеты ! телячьи}

Взять телячьих ребер, отделить каждое ребро от позвонков, спустить все мясо к низу кости, но от костей не отделять, избить обухом ножа, посыпать немного солью, обвалять в сырых яйцах, обсыпать сухарями, оправить ножом, как обыкновенный котлеты, изжарить в масле. Потом взять кочан капусты, исшинковать мелко, обдать кипятком, дать полчаса постоять, откинуть на сито, выжать воду, положить в кастрюлю с ложкой свежего коровьего масла, поставить на огонь, поджарить, беспрерывно мешать, чтоб капуста не пригорела; налить немного бульону, прибавить ложку мелкого сахару, ложку уксусу и ложку муки, вымешать, уварить на легком огне; когда капуста будет готова, выложить в соусник, а на капусту уложить венчиком котлеты. (По одной котлете на персону). 

\z{Котлеты с сафоем}\index{Котлеты ! с сафоем}

Взяв сколько будет нужно кочанов сафоя, очистить верхние листы, разрезать каждый кочан начетверо, обдать кипятком, дать постоять, откинуть на сито; когда вода стечет, положить в кастрюлю, налить бульоном, чтоб он покрыл капусту; стереть муки с маслом, заправить соус, посолить, положить по вкусу уксусу и крошечку мускатного орешка. Котлеты приготовить как сказано выше. (Тоже). 

\z{Котлеты с горохом}\index{Котлеты ! с горохом}

Взять ребра от телятины или баранины, отделить от позвонков, снять с них все мясо, прибавить мякоти от задней ноги, вырезать жилки, изрубить мягко, посолить, наделать котлет, вложить в каждую котлету по ребрышку, обмакнув в сырые яйца, обвалять в сухарях, изжарить в масле. Потом, налущив свежего гороху, сварить в бульоне мягко, откинуть на сито, и когда бульон стечет, протереть сквозь сито, прибавить 2 горсти перебранного и сваренного шпинату, также протереть, смешать вместе с горохом, положить 2 ложки свежего чухонского масла и влить бульону, в которому варился горох, поставить на легкий огонь, прибавить мускатного орешка, выжать сок из одного лимона; дав прокипеть, выложить в соусник и обложить котлетами. С этим соусом можно подавать телячью грудинку; сварив ее, нарезать ломтиками, обвалять в сырых яйцах, обсыпать сухарями и обжарить в масле. (Тоже). 

\z{Телячья печенка с красным вином}\index{Телячья печенка ! с красным вином}

Нарезать телячью печенку продолговатыми ломтиками, нашпиговать ветчинным салом, поджарить в масле. Соус в ней приготовить следующий: ложку муки стереть с ложкой масла, поджарить докрасна, развести бульоном, прибавить рюмку виноградного вина, ложку уксусу, подцветить подожженным сахаром и облить печенку. (6) 

\z{Картофель а ля метр-д'отель}\index{Картофель ! а ля метр-д'отель}

Вымыть и сварить в соленой воде картофель; когда будет готов, очистить и изрезать тонкими ломтиками, сложить в сотейник, положить 2 л. масла, 2 л. бульону и 2 л. молока, поставить на огонь; когда закипит, помешать, положить рубленной петрушки и, снабдив по вкусу солью, выложить на глубокое блюдо. (4) 

\z{Соте из печенки по-французски}\index{Соте ! из печенки ! по-французски}

Изрезать тонкими пластами воловью печенку, разложить на столе, посыпать солью и перцем, обвалять в муке, сложить на растопленное в сковороде масло, поставить на огонь и обжарить с обеих сторон; потом переложить на сотейник, а в сковороду влить бульону, выварить сок и процедить сквозь сито. Между тем запасеровать на масле одну луковицу; когда лук начнет желтеть, положить ложку муки, развести процеженным соком, залить переложенную в сотейнике печенку, поставить на огонь, вскипятить, уложить печенку на блюдо, а соус снабдить по вкусу солью, перцем, чуть-чуть мускатным орехом и залить печенку на блюде. (4—6) 

\z{Крепинет из свиных ножек}\index{Крепинет ! из свиных ножек}

Очистить, вымыть и сварить с кореньями свиные ножки; когда будут готовы, остудить, вынуть из бульона, очистить, выбрать косточки, а в место них положить фаршу для сосисок, так чтобы ножки имели круглую форму, обвернуть крепиной\footnote{Внутреннее свиное сало наподобие сетки.}, сложить на сковороду, поджарить немного, поворотить, поставить в горячую печку, заколеровать, сложить на блюдо, слить жир в чашку, на сковороду прибавить бульону, вскипятить сок и подлить крепинет. (4) 

\z{Душеный карбонад}\index{Карбонад}

Для этого можно брать баранину, свинину или телятину. Счистить мясо с костей так, чтоб оно осталось лишь на их оконечностях, сравнять его ножом, промыть, положить на противень, посолить, налить водой, накрыть и поставить в жар. Если берут телятину, то необходимо подкладывать к ней коровьего масла; для прочих же двух сортов мяс это бывает нужно только тогда, когда мясо недостаточно жирно. Наконец, посыпать мясо тертым хлебом и поджарить дотемна. (По величине куска мяса от 6--12). 

\z{Рябчиковая каша}\index{Каша ! рябчиковая}

Возьмите несколько рябчиков и варите их с луком, перцем и солью до тех пор, пока они не разварятся. Протрите рябчики сквозь сито, процедите этот бульон и на нем заварите гречневую кашу, из мелкой, так называемой смоленской, или из простой крупы, как угодно. Гастрономы предпочитают смоленскую крупу. Каша эта чрезвычайно вкусна и питательна. Масло кладется тогда, когда подают кашу на стол. (От трех рябчиков на 6--8 порций блюда).

\z{Сыр из дичи}\index{Сыр ! из дичи}

Вымыть назначенную для этого дичь, сложить на растопленное в кастрюльке масло, посолить, покрыть крышкой, изжарить до мягкости, снять с огня, выбрать мягкие части, изрубить, истолочь в ступке и протереть сквозь частое сито. Кости сложить обратно в кастрюлю, положить пряностей и кореньев, налить бульоном (прилить, если окажется, бульон из головки и ножек телячьих) и сварить до возможной крепости; потом процедить на сотейник, прибавить ланспику, соку или глясу, какой есть под рукой, и выварить до совершенной густоты. Между тем изрубить, истолочь и протереть сквозь сито свиной шпик; когда будете готово, поступить так: вначале взять в обширную кастрюльку протертое пюре из дичи, размешать и, взбивая, прибавлять по ложке вываренного соку и протертого сквозь сито свиного шпику. Положив таким образом весь шпик и разбив сыр до гладкости, снабдить по вкусу солью и толчеными пряностями, поставить на лед пробу, и когда застынет и окажется достаточно крепкой, то наложить плотно полную паштетную чашку, покрыть наслоенной свиным салом бумагой, остудить и, закрыв крышкой, держать в холодном месте. (4) 

\z{Штуфад (stufatto) с макаронами}\index{Штуфад с макаронами}

Взять мягкую часть говядины от костреца с верхним жиром, нашинковать толстыми кусками шпику, сложить на растопленное в кастрюльке масло, поставить на огонь и заколеровать с обеих сторон. Тогда подлить бульону, положить 2 луковицы, 2 моркови, 1 сельдерей и варить на легком огне под крышкой до тех пор, пока штуфад упреет до мягкости, а сок выкипите до густоты; выложить штуфад на блюдо, обложить кореньями и залить процеженным соком без жиру. Макароны подаются особо. (4) 

\z{Телячья печенка под морковно-петрушечным соусом}\index{Телячья печенка ! под морковно-петрушечным соусом}

Нарезать телячью печенку кусочками, положить в кастрюлю, прибавить 2 ложки чухонского масла, 2 луковицы, нарезанные кружочками, по 1 корню петрушки и моркови, нашинковав их полосками, поставить на огонь, обжарить слегка, посолить, посыпать мукой, вымешать, развести бульон, положить немного крупно истолченного перцу; перед обедом подбить двумя яичными желтками. (5--6) 

\z{Спаржа}\index{Спаржа}

Очистив спаржу, сварить перед обедом в воде с солью; подавать горячую, с следующим соусом: взять 4 желтка, 2 ложки мелкого сахару, бить ложкой до тех пор, пока желтки побелеют; тогда влить стакан виноградного вина и рюмку воды, поставить на 1/4 часа на огонь; мешать беспрерывно, не давая кипеть. Другой соус к спарже приготовляют таким манером: взять 1/4 ф. сливочного масла, ложку муки, стереть вместе, поставить на огонь, развести сливками, положить, цедру с одного лимона, стертую на сахар, и прокипятить. Также подают спаржу с растопленным коровьим маслом. 

\z{Цветная капуста}\index{Цветная капуста}

Отварив цветную капусту в воде с солью, наблюдая, чтоб капуста не переварилась, откинуть на сито. Потом сварить 3 десятка раков, очистить шейки, а клешни и верхние черепки истолочь; жарить с ложкой коровьего масла до тех пор, пока масло покраснеет; тогда масло слить, а на гущу налить бульону; варить час, процедить сквозь сито, подправить мукой, стертой с раковым маслом, положить раковые шейки, дать раза два прокипеть, прибавить чуть-чуть мускатного орешка. Выложив капусту в соусник, облить соусом. Подают также цветную капусту с тем соусом из виноградного вина с желтками, который, подают к спарже. Еще приготовляют капусту с следующим соусом: взять ложку свежего чухонского масла, стереть с половиной ложки муки, развести бульоном, снять на сахар цедру с 1 лимона, положить в соус, выдавить сок из лимона, дать прокипеть, подбить 2 яичными желтками. Когда цветная капуста приготовлена с кулисом из раков или с бульоном, то ее можно подавать с жареными и разнятыми на части цыплятами. 

\z{Артишоки}\index{Артишоки}

Обрезав верхнюю зеленую часть у артишоков до половины листьев, отварить фонды (нижнюю часть) в воде с солью; когда будут готовы, очистить твердые части. Соус для артишоков приготовить следующий: взять хорошего кулису, положить в него ложку свежего чухонского масла, посолить, посыпать немного крупно истолченного перцу, чайную ложку уксусу и, уварив соус, облить артишоки. Подают также артишоки с соусом из виноградного вина, с которым подается спаржа. Иногда подают к артишокам нижеследующий соус: положить в кастрюлю горсть белого хлеба, истертого на терке, луку, нарезанного кружочками, несколько лавровых листов, немного цельной гвоздики и мускатного о решка, влить 2 стакана бульону, уварить хорошенько, про- цедить сквозь сито; потом вылить опять в кастрюлю, дать раз вскипеть, снять с огня, подбить 2 яичными желтками и ложкой сметаны, вымешать, подержать еще на плите, мешая беспрерывно ложкой; но кипеть более уже не давать. 

\z{Маседуан из разной зелени и овощей}\index{Маседуан из зелени и овощей}

Сварить в соленом кипятке очищенные и порезанные зеленые бобы, отлить на дуршлаг, перелить холодной водой и, осудив, сложить на сотейник; вылущить молодой горошек, сварить в соленом кипятке, отлить на дуршлаг, остудить и переложить в сотейник к бобам; очистить и сварить 1 ф. спаржи зеленой и белой, цветную капусту и молодого картофелю, нарезать выемкой и сварить в бульоне 10 шт. молодой моркови и 3 шт. репы\footnote{Зелень для маседуана варится в обширной кастрюле и на большом огне, чтобы зеленый цвет остался в натуральном виде, и разогревая маседуан, не должно кипятить.}. За 15 минут до отпуска поставить сотейник на огонь и, осушив немного, положить 0,5 ф. сливочного масла, ложку белого соусу или бешамеля, разогреть и, размешав, выложить на глубокое блюдо и обложить крутонами. (6) 

\z{Фонды из артишоков с горошком}\index{Фонды из артишоков с горошком}

Очистить низы и срезать листы артишоков так, чтобы остался лишь один середок, на подобие круглой чашечки, потом вытереть каждую очищенную штуку лимоном и класть в холодную воду, разведенную уксусом. Когда все фонды будут очищены, выложить в соленый кипяток, обланширить, выбрать в холодную воду, очистить середину от мякоти, сложить на растопленное в сотейнике масло, выжать сок из лимона, покрыть бумагой и за 0,5 часа до отпуска влить 0,5 чумички бульону, поставить на плиту, покрыть крышкой и варить до тех пор, пока сок выкипит и артишоки заколеруются. Тогда влить красного соуса и прокипятить. Между тем сварить вылущенный молодой горошек, сложить на сотейник, размешать со сливочным маслом, уложить артишоки на блюдо и, наложив на середину каждого артишока горошку, подлить процеженным из-под артишоков соусом. (6) 

\z{Цикорий под бешамелью}\index{Цикорий ! с бешамелем}

Перебрать листки цикория, вымыть и обланширить в 2 водах; когда закипит во второй воде. отлить на дуршлаг, перелить холодной водой и, отжав досуха, перебрать, изрубить мелко, положить на растопленное масло в кастрюлю, влить немного бульону, покрыть крышкой и поставить на час в горячую печку, чтобы совершенно упрело. Когда будет готово, протереть сквозь сито, переложить в кастрюлю, положить соответственную пропорцию густой бешамели из сливок, снабдить по вкусу солью, глясом и мускатным орехом, выложить на глубокое блюдо и обложить крутонами из хлеба. (6) 

\z{Латук с красным соусом а л'екарлат}\index{Латук ! с красным соусом}

Очистить нужное число кочанного салату, вымыть дочиста и обланширить в соленой воде, отлить на дуршлаг, перелить холодной водой, отжать до сухости, разложить на доску, посолить, завернуть в виде филеев, уложить в сотейник или шарлотную форму, покрыть тонким шпиком и, залив бульоном, сварить на легком огне до мягкости. Потом снять шпик, осушить на сите, сложить на блюдо, перекладывая крутонами, и залить красным соусом. Обланширенный латук можно фаршировать фаршем, а потом варить, как сказано выше. Подать в перекладку с языком вместо крутонов и полить сверху выкипяченным до вкуса красным соусом. (6)

\z{Спаржа с бешамелем}\index{Спаржа ! с бешамелем}

Очистить спаржу средней толщины до половины, изрезать правильно и сварить в соленом кипятке; когда будет готова, отлить на дуршлаг, перелить холодной водой, осушить на салфетке, сложить на растопленное в сотейнике масло и запасеровать немного. Пред отпуском положить бешамеля из сливок столько, чтобы спаржа была достаточно густа, размешать, снабдить по вкусу солью и самой малостью мелкого сахару, выложить на глубокое блюдо и обложить крутонами. (6) 

\z{Красная фаршированная капуста}\index{Капуста ! красная ! фаршированная}

От кочана красной капусты обрежьте кочерыжку вровень с листами, обдайте кипятком и дайте ей так постоять. Между тем приготовьте следующий фарш: возьмите телячьего ссеку, немного говяжьего сала, горсть рубленной петрушки, луку, шампиньонов, соли, перцу, изрубите все мягко, прибавьте 3 целых яйца, еще порубите, чтоб все хорошо перемешалось; потом капусту откиньте на сито, выжмите хорошенько воду, начините приготовленным фаршем, дно кастрюли устелите ломтиками ветчинного сала, положите кочан на сало вниз кочерыгой, облейте хорошим бульоном и варите на легком огне. Когда капуста поспеет, накройте кастрюлю блюдом и, придерживая левой рукой, опрокиньте капусту на блюдо и облейте соусом эспаньоль. 

\z{Зеленая фасоль}\index{Фасоль ! зеленая}

Молодые зеленые стручки очистить, нашинковать наискось, опустить в соленый кипяток, варить до мягкости, только не под крышкой, отлить на дуршлаг, перелить холодной водой; когда вода стечет, переложить в кастрюлю, обсыпать сахаром, положить 1 ложку масла, развести 1,5 ст. сливок, положить еще сахару, зеленой петрушки, мускатного ореха, вскипятить, подавать; огарнировать котлетами или чем другим. 

\z{Земляные груши}\index{Земляные груши}

Очистить их как можно лучше, бросая тотчас в холодную воду с уксусом; перемыть, сварить в соленой воде, в которую влить немного уксусу, и положить 0,5 ложки муки, размешанной с 0,5 ложкой масла; подавая, выложить на блюдо, обложенное рантом из теста; облить маслом с сухарями или сабайоном. 

\z{Свекла}\index{Свекла}

Сварить, но лучше испечь фун. 2,5--3 свеклы, очистить, изрубить очень мелко; 1 ложку мелко изрубленной луковицы поджарить в 1,5 ложках масла, всыпать 0,5 ложки муки, свеклу, размешать, влить 3/4 стакана сметаны, соли, 2--3 ложки уксусу, вскипятить; некоторые прибавляют сахар или мед. К зайцу или к тетереву подается свекла без сметаны, a вместо нее взять несколько ложек жирного бульону. 

\z{Фаршированная репная капуста}\index{Кольраби ! фаршированная}

Очистить, вымыть, срезать сверху 1 ломтик, выбрать осторожно середину, изрубить ее мелко, поджарить с ложкой масла, прибавить мелко изрубленного жаркого телятины или почек, немного говяжьего жиру или мозгов из костей, натертую и в масле поджаренную булку, 2 рубленных яйца, соли, немного перцу, ложки 2 сметаны, смешать все вместе. Нафаршировать этим кольраби, накрыть срезанным ломтиком, обвязать ниточкой, уложить в кастрюлю одну при другой, налить бульоном или водой, варить до мягкости. Перед отпуском снять нитки; подавая, облить соусом, в котором варились. (5) 

\z{Пюре из чечевицы}\index{Пюре ! из чечевицы}

Чечевицу перебрать, вымыть в теплой воде, всыпать в кастрюлю, положить сырой ветчины, 2 очищенные луковицы, 2 моркови, 2 порея; налить бульоном, вскипятить, поставить в печь на 2 часа, чтобы упрела до мягкости, потом вынуть ветчину и коренья, а чечевицу протереть сквозь сито; за 15 минут до отпуска смешать со сливочным маслом, развести вскипяченной малагой или бульоном. Подавать с гренками. (5) 

\z{Зеленый горошек с котлетами из грибов}\index{Зеленый горошек ! с котлетами из грибов}

1,5 стакана зеленого горошку перебрать, вымыть, сложить в кастрюлю, налить полнее водой; когда закипит, посолить, положить 1 луковицу, 1--2 куска сахару, варить до мягкости, откинуть на решето. 1/8 ф. белых грибов сварить в воде, мелко изрубить; 1 стакан рису разварить в воде с солью, мускатным орехом, петрушкой, смешать с грибами, сделать котлеты, обвалять их в яйце и толченых сухарях, изжарить в чухонском или маковом масле. Ложку муки растереть с 1 ложкой масла, развести 2--3 стакан, грибного бульону, смешанного с бульоном, стекшим с зеленого горошка, облить горошек, сложенный на блюдо, огарнировать рисовыми котлетами. (Скоромный или постный). (8--10) 

\z{Картофельное пюре}\index{Пюре ! картофельное}

Вымытый картофель очистить, вымыть, сварить в соленой воде, слить воду, протереть сквозь дуршлаг, сложить в кастрюлю, посолить, положить масла, неснятого, цельного молока, размешать, подогреть. Подавать с котлетами, с солониной, с ветчиной, с языком, с зразами, с жаркими из говядины, с сосисками. 

\z{Пюре из шампиньонов}\index{Пюре ! из шампиньонов}

Возьмите 4--5 корзинок шампиньонов, очень белых, отрежьте у них кончики и вымойте в двух водах. Тогда налейте немного воды в кастрюльку, выжмите туда сок из 1 лимона, положите щепотку соли и дайте в этой воде шампиньонам вариться минут 5. Спустив затем с шампиньонов воду, изрубить их мелко-намелко; потом взять 1/4 ф. сливочного масла, положить в другую кастрюлю, выжать туда сок второго лимона, сложить изрубленные шампиньоны, и поставить все на огонь на 1/4 часа, прибавив стакан или 1,5 велуте (см. описание) и столько же говяжьего хорошего бульону. Все это уваривается на довольно сильном огне до тех пор, пока пюре не достигнет надлежащей степени густоты. Ко всему этому прибавьте немного крупно смолотого перцу и перелейте пюре в другую кастрюлю. (6—8) (Эмбер). 

\z{Свежий зеленый горох под соусом}\index{Зеленый горошек ! под соусом}

Взять сколько нужно стручьев сахарного гороха, обрезать концы и содрать с боков жилки, положить в кастрюлю, налить бульоном, поставить на огонь. Когда стручья сварятся мягко, подправить чухонским маслом с мукой, вскипятить раз. Это овощное блюдо можно подавать к котлетам, сосискам и цыплятам. 

\z{Турецкие бобы под соусом}\index{Турецкие бобы ! под соусом}

Обрезав у фасоли концы, содрать боковые жилки, или тоненько срезав, нашинковать, отварить в соленой воде, откинуть на сито. Когда вода стечет, положить в кастрюлю, прибавить ложку масла, поджарить, положить немного муки, вымешать, развести по пропорции бульоном, прокипятить. Соус из турецких бобов подают с котлетами, сосисками и с поджаренным белым хлебом. Белый хлеб приготовляют для этого таким манером: взять сколько будет нужно яиц, смешать со сливками; с белого хлеба обрезать корку, размочить в приготовленных яйцах, обжарить румяно в масле. 

\z{Свежие белые грибы в соусе}\index{Белые грибы ! в соусе}

Взять нужное количество белых грибов, отнять корешки, шляпки вымыть в 2 или 3 водах, чтоб не оставалось на них земли и песку; большие грибы разрезать начетверо, a маленькие оставить целыми, поджарить в масле, посолить, положить немного муки и, смотря по количеству грибов, 2 или 3 ложки сметаны, зеленого рубленного укропу. Соус этот подают к курице, также к жареной телятине и баранине: нарезав ее ломтиками, обвалять в сырых яйцах, обсыпать сухарями и обжарить, в масле. 

\z{Шпинат с яйцами}\index{Шпинат ! с яйцами}

Три фунта шпинату перебрать, вымыть и сварить в воде с солью, откинуть на сито, а когда стечет вода, изрубить. Положить в кастрюльку 1/4 ф. масла и шпинат, поджарить на легком огне, мешая непрерывно, чтоб шпинат не пригорел и хорошо соединился с маслом; потом всыпать 2 ложки мелко истолченных сухарей из белого хлеба, влить чашку густых сливок, подавать с выпускными лицами. Яйца для этого соуса приготовляют таким манером: налить в глубокую сковороду воды, поставить на огонь; дав закипеть воде, влить немного уксусу и посыпать соли; выпустить в воду 10 яиц, разбивая осторожно, чтоб желток оставался цел и яйца соединились друг с другом белками. Спускать яйца не вдруг, а одно за другим, и когда белок довольно окрепнет, вынимать яйца решетчатой ложкой, класть на сито, чтоб стекла вода. Уложить шпинат в соусник, а на него положить яйца. (10--12). (Еще рецепт другой под № 322). 

\z{Брюссельская капуста по метр-д'отельски}\index{Капуста ! брюссельская ! по метр-д'отельски}

Выбрать штук 8 брюссельской капусты потверже, покомпактнее, снять со ствола маленькие вилочки и перемыть тщательно, отделив листки пожелтевшие или попортившиеся. На плиту поставить кастрюлю, наполовину налитую водой, прикинув в воду горсть соли. При этом не лишне заметить, что величина кастрюли должна быть сообразна с количеством капусты. Когда вода сильно закипит, то надо в кипяток опустить все перемытые вилочки капусты, оставляя кипеть на сильном огне. Таким образом, капуста не изменится в цвете и будет иметь красивый темно-зеленый колер. Когда капуста окажется довольно сваренной, в чем можно удостовериться, попробовав один вилочек,~--- должно откинуть всю капусту на решето, чтоб дать стечь воде. Тогда в сотейник или в другую кастрюлю, которая была бы поменьше, положить 0,5 ф. свежего сливочного или особенно хорошего мызного масла и маленькую щепотку, так в орех величиной, рубленной петрушки. Сюда выложить капусту из решета, перевалять ее хорошенько в масле, попробовать, достаточно ли она посолена, выжать на нее сок из одного лимона и отпускать к столу. (10--12) (Эмбер). 

\z{Духовой шпинат}\index{Шпинат ! духовой}

Он делается без воды. Очистите шпинат, вымойте тщательно холодной водой, разберите каждый листик, чтоб не было на нем и порошинки земли или песку, сбросьте на решето, чтоб вода стекла досуха, потом возьмите кастрюльку, которая плотно закрывалась бы, положите в нее шпинат целиком как есть; небольшой кусочек сливочного масла и соли крошечку (и больше ничего), закройте плотно крышкой, поставьте в печь на 1/4 ч., или на 1/2 часа даже, смотря по величине кастрюльки — и ваш шпинат готов; остается положить на него выпускных яиц, если хотите; истинные любители предпочитают шпинат без яиц — он душистее; а яйца отбивают аромат, особливо у молодого тепличного шпината\footnote{Из лекции доктора Пуфа: <<О кухонном искусстве>> 1844 г.}.

\z{Бобы свежие по-английски}\index{Бобы ! по-английски}

Оборвать стебельки от молодых зеленых бобов, изрезать вдоль, опустить в кипящую ключом соленую воду, сварить, отлить на дуршлаг, наложить ряд в разогретую чашку для зелени, покрыть пластинками сливочного масла, снова бобы и масло, и, когда все будут сложены, подать тотчас за стол. (6) 

\z{Шпинат с выпускными яйцами другим манером}\index{Шпинат ! с яйцами}

Очистив и вымыв молодой шпинат, выбрать на сито; между тем вскипятить в обширной кастрюле воды, посолить, опустить шпинат и усилить огонь, чтобы вскипело скорее, и как только шпинат окажется мягким, отлить на дуршлаг, перелить холодной водой, потом отжать досуха, выложить на доску, перебрать, изрубить мелко, сложить на масло в кастрюлю, запасеровать немного, прибавить белого соусу столько, чтобы шпинат был ни густ, ни жидок, положить по вкусу соли, глясу, и выложить на глубокое блюдо. Между тем приготовить выпускные яйца. Взять в кастрюлю 4 стакана воды, 1 стакан уксусу и ложку соли, вскипятить на плите и, надбив осторожно скорлупу у каждого яйца, выпускать из рук в кипяток по одному (до 5 штук), покрыть крышкой и варить от 2 до 3 минут; потом выбрать осторожно дуршлаговой ложкой в холодную воду, остудить и, обрезав кругом лишний висящий белок, сложить в холодную воду, а перед отпуском разогреть в той же воде и обложить шпинат на блюде. (6) 

\z{Душеные огурцы}\index{Огурцы ! тушеные}

Очистить огурцы, разрезать, вынуть семечки, посыпать солью и оставить так на несколько времени. Между тем разогреть масло, слить воду с огурцов, положить их в масло, прилить туда несколько бульону или воды и уварить домягка, потом загустить 2 яичными желтками, размешанными в уксусе с одной полной ложкой муки. Приготовив огурцы таким образом, подавать к бараньей задней ноге или к карбонаду. 

\z{Корюшка фаршированная}\index{Корюшка ! фаршированная}

Очистить корюшку, как следует, прорезать осторожно в середине, выбрать спинную кость, посолить и поставить в холодное место до времени. Взять в сотейник немного красного соусу, вскипятить до совершенной густоты, прибавить нужное количество рубленных шампиньонов и прокипятить снова до густоты. Когда будет готова, выложить на плоскую тарелку и поставить в холодное место. За 20 минут до отпуска осушить корюшку на салфетке, наложить в середину каждой понемногу приготовленного фаршу, зашить нитками, потом обвалять в муку и запанировать в яйцо и тертый хлеб; когда все будут готовы, опустить по несколько штук в горячий фритюр, обжарить до колера, выбрать на салфетку, вытащить нитки, сложить на блюдо с салфеткой и обложить обжаренной зеленой петрушкой и разрезанными лимонами. (6) 

\z{Булетки из рыбы}\index{Булетки ! из рыбы}

Прежде всего очистить рыбу от чешуи, выпотрошить, перемыть, отварить, выбрать из мяса кости, изрубить его, растопить потом в кастрюле коровье масло и пропарить в нем изрубленное мясо. 

Между тем промешать на огне хорошенько 6 золотников смоченного молочного хлеба с полной чайной чашкой молока, луковицами, лимонной коркой, солью, немножко мускатным орехом, дать остынуть, подмешать 2 или 3 яичные желтка и тертого молочного хлеба, наделать из этого теста круглых, плоских лепешечек, обвалять в яйцах и тертом молочном хлебе и испечь в коровьем масле. 

Булетки эти подают как отдельное блюдо с соусом, приготовленным из сарделей с каперсами. (Выйдет от 10--12 булеток, смотря по их величине). 

\z{Лососина под белым соусом}\index{Лососина ! под белым соусом}

Положив звено лососины в какую-нибудь посудину, налить уксусом, прибавить 2 ложки прованского масла, посолить, положить горошинами перцу, лаврового листу, и дать стоять сутки; вынув лососину из маринада, обжарить ее в коровьем масле, посыпав мукой. Соус приготовить следующий: нарезать кружочками 2 луковицы, положить в кастрюлю, прибавить 1 лимон, также нарезанный тоненькими ломтиками, ложку каперсов налить бульоном, поставить на огонь и, дав укипеть, заправить подпальной мукой. 

\z{Караси в постном соусе}\index{Караси ! в постном соусе}

Вычистив, выпотрошив и вымыв карасей, изжарить в ореховом или маковом масле; если караси большие, разрезать каждого надвое, а маленьких изжарить целыми. Потом взять кочан красной капусты, нашинковать, посолить, дать полежать, чтоб капуста дала из себя сок, обжарить в ореховом масле, положить немного муки, влить рыбного бульону, уварить домягка, выжать сок из 1 лимона. (Постное). 

\z{Ерши в раковом кулисе}\index{Ерши ! в раковом кулисе}

Очистив ершей, вымыть, изжарить в масле; облить следующим соусом: положить кусок чухонского масла в кастрюлю, развести раковым кулисом, положить крошечку перцу, еще меньше мускатного орешка и ложку каперсов, и дав вскипеть, облить ершей. 

\z{Ерши под шампиньонным соусом}\index{Ерши ! под шампиньонным соусом}

Вычистив и выпотрошив ершей, отрезать у них головы; потом вымазать кастрюлю коровьим маслом, посыпать солью и перцем, нашинковать луку и петрушки, уложить дно кастрюли, положить ершей, прикрыть сверху теми же кореньями, облить раковым маслом, обсыпать тертым белым хлебом, поставить в печь. Соус приготовить следующий: нашинковать шампиньонов или других грибов, обжарить в коровьем масле, посолить, посыпать немного перцу, влить по пропорции бульону. Когда соус хорошо укипит, подбить подпальной мукой, положить мелко изрубленного зеленого укропу; положив в соусник ершей, облить соусом. 

\z{Солянка из рыбы московская}\index{Солянка ! из рыбы ! московская}

Приготовляется из той рыбы, которая бывает в остатке, как из рыбы сваренной, так и рыбы соленой, как-то: осетрины, белужины, тёшки малосольной или свежей, лососины, судаков, сигов и проч., следующим способом: в начале взять в кастрюлю шинкованной капусты, налить кипящей водой и сварить до мягкости; потом запасеровать на прованском масле одну мелко исшинкованную луковицу; когда лук начнет желтеть, положить ложку муки, развести постным бульоном (какой есть) и выкипятить соус до густоты; потом выбрать из воды на сито сваренную капусту, отжать немного воду, положить в соус, размешать и вскипятить. Между тем приготовить 4 сваренные или маринованные гриба, 2 очищенные огурца, 1/2 ф. малосольной тешки и 1/2 ф. свежей лососины, которую изрезать в ломтики и поджарить на масле. Положить на сковородку (сковородка для солянки употребляется без ручки и луженая) ряд капусты, сверх капусты положить ломтики грибов, огурцов, тешки и лососины, покрыть снова капустой; наложить грибов и проч., и продолжать так доверху; сверху докрыть капустой, посыпать тертым хлебом, поставить в горячую печку, заколеровать, убрать сверху красиво корнишонами, оливками, раковыми шейками, маринованными вишнями, маленькими рыжиками и т. п. (1) (Пост.) 

\z{Тельное из щуки}\index{Тельное ! из щуки}

Очистить, выпотрошить и вымыть щуку, снять с костей филеи, подрезать верхнюю кожицу и оскоблить ножом филеи так, чтобы косточки и жилки не попадали к мякоти, потом изрубить мелко, посолить, положить в каменную ступку, прибавить 1 л. прованского масла, 2 л. тертого хлеба, протолочь и, когда из этого образуется одинаковая масса, выложить на стол, посыпать мукой, скатать в рулет, разделить на 4 части, из каждого куска сделать продолговатую котлетку, обвалять в муке и сложить в разогретое на сковороде масло. Пред отпуском, обжарить котлетки на легком огне с обеих сторон, сложить на блюдо и полить маслом, в котором жарились. К этому подается салат или огурцы. (4) (Пост.) 

\z{Карп с медом по-славянски}\index{Карп ! с медом по славянски}

Очистить, выпотрошить и вымыть карпа; посолить, положить в рыбный с решеткой котел, обложить очищенным цельным луком, положить очистки из шампиньонов, влить стакан мадеры, покрыть кружками нарезанного лимону и оставить так час. Между тем взять в кастрюльку 2 л. белого меду, поставить на огонь и, когда мед выварится и пожелтеет как карамель из сахара, влить 4 л. прованского масла, положить 3 л. муки, размешать, развести бульоном рыбным или из кореньев, так, чтобы соусу было 4 стакана, вылить в рыбный котел, где карп, и, закипятив на огне, поставить в горячую печку. Очистить 1/4 ф. кишмишу, 1/8 ф. миндалю сладкого пополам с горьким, 1/8 ф. маленького луку, под названием перлового, и обжарить его на прованском масле, потом очистить 4 корзинки свежих шампиньонов, сварить в воде с лимоном, положить к ним миндаль, лук и изюм и оставить на плите покрытым. Когда карп будет готов, вынуть осторожно с решеткой, очистить коренья и лук и переложить цельным на рыбное блюдо. Соус с кореньями слить на сотейник и выкипятить до густоты (постоянно мешать); тогда прибавить по вкусу чего будет не доставать, как то: лимонного соку, соли или меду, процедить сквозь салфетку влить часть в гарнир и поставить на огонь, а остальной мешать на плите. Когда гарнир вскипит, обложить с обеих сторон (можно нафаршировать и середину) карпа на блюде и залить сверху горячим соусом. (4) (Пост.) 

\z{Осетрина, запеченная с горчичным соусом}\index{Осетрина ! запеченная с горчичным соусом}

Намазать 1 ложкой масла сковороду или медный лист, положить тоненькими ломтиками нарезанную морковь, луковицу, петрушку. Кусок осетрины от самого мягкого места очистить, вымыть, посолить, положить на сковороду коренья; ложку масла поджарить с мелко изрубленной луковицею, остудить, вбить 2 яйца, облить этим рыбу, посыпать тертым ситным хлебом, скропить растопленным маслом, вставить в горячую печь. Когда будет рыба готова, переложить на блюдо, a изжарившиеся коренья смешать с 0,5 ст. столового вина, 1,5 ст. крепкого бульону, 2--3 ложками уксусу, всыпать ложку горчицы, вскипятить, процедить, облить осетрину. (6--8) 

Осетрину можно нашпиговать 1/4 фунтом шпика. 

\z{Окуни под соусом}\index{Окуни ! под соусом}

Разрезав на части больших окуней, обвалять в масле, посыпать мукой, поставить в печь, дать зарумяниться. Соус к ним приготовляется следующий: влить в кастрюлю ложку масла орехового или прованского, положить пол-ложки муки, поджарить докрасна, влить сколько нужно для соуса рыбного бульону, наложить лаврового листу, несколько гвоздиков гвоздики, ложку сахару и 1/4 фунта коринки. Когда соус укипит, влить рюмку виноградного вина, положить окуней в соусник и облить соусом. (По числу персон). (Пост.) 

\z{Осетрина в луковом соусе}\index{Осетрина ! в луковом соусе}

Нарезать ломтиками осетрину, обжарить в прованском или горчичном масле, обсыпать сухарями. Соус к осетрине приготовить таким образом: исшинковать 2 луковицы, обжарить в прованском масле, прибавить немного муки, каперсов, два очищенные от костей и мелко изрубленные анчоуса, развести рыбным бульоном, уварить. Когда соус уварится, приправить лимонным соком, положить чайную ложку приготовленной горчицы, вымешать, облить осетрину. (П.) 

\z{Судак с цветной капустой}\index{Судак ! с цветной капустой}

Положить в кастрюлю ложку муки и ложку коровьего масла, поджарить докрасна; когда поджарится мука, положить в кастрюлю судака, разнятого на части, обжарить, а потом влить мясного бульону, варить на малом огне, прибавить мелко изрубленной петрушки и укропу, влить немного ракового кулиса. Цветную капусту сварить в воде с солью отдельно. Когда рыба будет готова, положить в соусник, обложить капустой и облить соусом. (6) 

\z{Караси с яблоками}\index{Караси ! с яблоками}

Очистить яблоки, нарезать тоненькими ломтиками, вырезать семена, положить в кастрюлю, обжарить в коровьем масле. Когда яблоки уварятся влить стакан виноградного вина, прибавить коринки, по вкусу сахару, прокипятить. Карасей отварить в воде с солью, выложить в соусник и облить соусом. (По штуке рыбы на каждую персону). 

\z{Сазан в пивном соусе}\index{Сазан ! в пивном соусе}

Вычистить, выпотрошить и вымыть сазана, нарезать звеньями; посыпать солью и дать так полежать часа два; потом вытереть салфеткой, положить в кастрюлю, налить столько пива, чтоб оно покрыло рыбу, прибавить несколько шинкованных луковиц, цельной гвоздики, немного мускатного орешка; поставить на огонь. Когда рыба поспеет, вынуть, положить в соусник, а пиво уварить, чтоб осталось его немного, и облить рыбу. (6) 

\z{Щука фаршированная вареная под соусом из сметаны}\index{Щука ! фаршированная ! под соусом из сметаны}

Очистить щуку, разрезать вдоль хребта, вырезать осторожно мясо с костями, чтобы не прорезать кожицы, голову и хвост оставить. Отделить мясо от костей, посолить, мелко изрубить, смешать с луковицею, мелко изрубленной и поджаренной в 1/2 ложке масла, положить 1/2 франц. белого хлеба, намоченного и выжатого, англ. и простого толченого перцу по 4--5 зерен, мускатного ореха, прибавить еще 1/2 ложки масла, вбить 1 яйцо, всыпать зелени, размешать все хорошенько или истолочь еще в ступке, нафаршировать рыбу, зашить плотно. Сварить бульон из разных кореньев и пряностей, соли, процедить, опустить в него щуку; когда уварится, слить этот бульон в другую кастрюлю, а рыбу поставить на пар, чтобы не остыла. В рыбный же бульон положить очищенного картофелю, сварить. Распустить 1,5 ложки масла, поджарить в нем 0,5 мелко изрубленной луковицы, всыпать 2/3 ст. муки, развести 2--3 ст. рыбного процеженного бульону, 2 или 1 стаканом сметаны, так, чтобы соусу было стакана 4, не менее, вскипятить, переложить к рыбе картофель, облить этим соусом, подогреть, выложить на глубокое блюдо, посыпать зеленой петрушкой и укропом, подавать. (6--8) 

\z{Лососина в папильотах}\index{Лососина ! в папильотах}

Очищенную от чешуи лососину нарезать тонкими ломтиками, посолить посыпать перцем, мелко изрубленной луковицей, окропить прованским маслом и так оставить на несколько часов. Взять чистой бумаги, намазать ее чухонским маслом, а в пост прованским, положить на каждую четвертку бумаги кусок лососины, на нее масло из сарделек, завернуть красиво бумагу и уложить на противень, намазанный маслом. Перед отпуском вставить в горячую печь; когда бумага подрумянится, подать, не вынимая из нее лососины. (П.) 

\z{Рыба вареная по-жидовски}\index{Рыба ! вареная ! по-жидовски}

Какую-нибудь свежую рыбу (лучше всего щуку) очистить, разрезать на большие куски, посолить сухой солью, оставить так на 1 час. Эту же рыбу можно нафаршировать; в таком случае с каждого куска срезать осторожно мясо с костями, выбрать их, а мясо мелко изрубить с 1 луковицей, посолить, положить толченого англ. и простого перцу по 5--6 зерен, 1 яйцо (если не в пост), размешать. Очистить и нарезать довольно мелко 1 петрушку, 2 луковицы, 1 сельдерей, 1 порей, прибавить соли, англ. перцу зерен 15, лавр, листу 4--5, с вершок корицы, 5 штук гвоздики, 1/4 чайной ложечки шафрану, 8 зерна толченого простого перцу, налить все это водой и варить. 

В этот бульон опустить после рыбу так, чтобы едва ее покрыло, поставить на сильный огонь, влить с 1/2 стакана холодной воды, наблюдать, чтобы рыба не пригорела, варить так минут 10, снимая накипь; потом опять подлить с 1/2 стакана холодной воды и поступать так каждые 10 минут в продолжение 1 часа. Переложить рыбу на блюдо, облить процеженным соусом, в который некоторые прибавляют 1/2 ложки меду и 1/4 стакана изюму, с тем раз вскипятить, и тогда облить рыбу. (Постная). (8) 

\z{Лабардан с бешамелем}\index{Лабардан ! с бешамелем}

Сложить очищенный свежий лабардан в кастрюлю, посолить, положить частицу чесноку, налить водой, поставить на плиту, дать раз вскипеть; снять с огня и оставить так до времени. Между тем положить в кастрюлю ложку масла и 2 ложки муки, размешать, развести 1 бутылкой сливок, поставить на огонь, вскипятить и, когда загустеет, выбрать лабардан на сите, разобрать отделяющимися пластинками, наложить ряд в форму, залить бешамелем, сверх бешамеля наложить снова ряд лабардана, залить бешамелем и продолжать так до верха формы; верхний ряд залить бешамелем, посыпать тертым хлебом, окропить маслом, поставить в горячую печку, заколеровать и подать на стол с формой. Любители кладут в бешамель частицу чесноку. (4) 

\z{Котлеты пожарские из рыбы с шампиньонами}\index{Котлеты ! пожарские ! из рыбы с шампиньонами}

Очистить 2 судачка, снять филеи, подрезать кожицу, вырезать косточки и изрубить мелко. Когда будет готово, посыпать солью с перцем, положить 1/4 ф. масла и размять так, чтобы образовалась ровная масса, тогда положить 2 л. сливок, размешать окончательно, разделить на столько частей, сколько предполагается иметь котлет, сформировать котлетки (обмакивая нож в разбитое яйцо), обвалять в тертый хлеб, а потом в разбитое яйцо и тертый хлеб; обровнять, сложить на растопленное в сковороде масло, изжарить, заколеровать ровно с обеих сторон, сложить на блюдо в кружок, а в середину наложить шампиньоны. (4) 

\z{Сосиски из зайца с картофельным пюре}\index{Сосиски ! из зайца}

Приготовить сосиски, с той разницею, что свинина заменяется филеями из зайца, и когда фарш будет совершенно готов, запасеровать на масле одну луковицу, вбить 3 яйца и изжарить яичницу; потом остудить, положить в фарш и, прорубив окончательно, с делать сосиски и положить их на картофельное пюре. (4)

\z{Вольвант, гарнированный телячьими молоками}\index{Вольвант}

Приготовить слоеное тесто из 1 ф. муки, 3/4 ф. масла, раскатать в 6 раз, толщиной в палец, сложить на плафон и поставить на лед; когда немного остынет, положить на тесто крышку такой величины, какой предполагается иметь вольвант, и обрезать ножом тесто кругом крышки так, чтобы нижний слой теста был несколько шире. Потом положить на средину вольванта маленькую крышку такой величины, чтобы остался край кругом вольванта в вершок и меньше, надрезать кругом крышки тесто до половины и смазать яйцом так, чтобы смазка не потекла на край вольванта; в середине проколоть ножом, поставить в умеренно горячую печку, закрыть плотно и наблюдать, чтобы вольвант ровно колеровался; в случае колер начнете брать более снизу или сверху, то подложить под низ крышку или листе осторожно, или покрыть сверху бумагой, не вынимая из печки; когда поднимется как должно и заколеруется, вынуть, снять из средины крышку, выбрать прочь тесто, подчистить, уложить на блюдо и наполнить. Многие убирают края вольванта слоеным тестом, нарезанным бордюрными выемками, но это допускается лишь из самых тонких пластинок. Вольванты делаются и накладные из двух кусков теста слоеного; верхняя половина с вырезанной серединой приготовляется из цельного теста, а под низ может употребляться тесто, смешанное из обрезков. Подобный вольвант приготовляется лишь по необходимости. Вольванты овальные, четырех- осмиугольные выделываются одинаковым вышеозначенным способом. (4) 

\z{Солянка с тёшкой}\index{Солянка ! с тешкой}

Изрубивши капусту, кислую или свежую, обжарить ее на сковороде в постном масле, с прибавкой рубленного луку; потом обсыпать немного мукой и перцем. Когда хорошо обжарится, то смочить немного уксусом или квасом, положить малосольной тешки, изрезанной жеребейками, ужарить сгладивши ложкой, и поставить в печь зарумяниться. В эту солянку можно прибавлять, когда угодно, остатки жареных свежих рыб. (Постная)

\z{Солянка из разных разностей с капустой}\index{Солянка ! с капустой}

Солянка с капустой большею частью приготовляется из остатков жаркого, прибавив к нему: ветчины, сосисок, солонины, шампиньонов, грибков, корнишонов, — одним словом, то, что есть в доме. Поступить следующим образом: взять 0,5 ф. мягкой свинины, сложить на масло в кастрюльку, посолить, поставить на огонь и обжарить кругом. Между тем перебрать 1 ф. шинкованной капусты (хорошего вкуса, не кислой); когда свинина заколеруется, положить туда капусту, влить стакан воды и варить под крышкой на легком огне; потом очистить филейного рябчика, положить в кастрюлю к капусте и варить до тех пор, пока капуста, рябчик и свинина сварятся до мягкости; тогда распустить в сотейнике 1/2 л. масла, положить исшинкованную одну луковицу и, когда поджарится и начнет желтеть, положить ложку муки, размешать, развести бульоном и поставить на огонь. Когда капуста и прочее будет готово, выбрать рябчика и свинину на тарелку, а капусту (если в ней много соку) отлить на сито; сок же из капусты вылить в сотейник. Пока соус вываривается, разрезать свинину и рябчика кусочками и такими же кусочками изрезать соленого сваренного языка, маринованных грибов и корнишонов; а когда соус выкипит до надлежащей густоты и вкуса, наложить ряд капусты на луженую (без ручки) сковородку; сверх капусты положить ломтиками нарезанной свинины, рябчика, языка и грибков, полить соусом. покрыть рядом капусты, положить снова свинины и прочего, залить соусом и повторять так далее, пока сковородка не наполнится; сверху полить соусом, посыпать тертым хлебом, убрать узором из нарезанных корнишонов и языка, окропить маслом, поставить в горячую печку и, заколеровав, подать на стол со сковородкой. (4) 

\z{Сюпрем из цыплят с трюфелем а ля Перигор}\index{Сюпрем ! из цыплят}

Снять с крупных цыплят филеи, подрезать верхнюю кожицу, выбрать и надрезать посередине жилку, разбить слегка сечкой, обровнять, сложить на наслоенный маслом сотейник, залить растопленным маслом и покрыть сверху наслоенной бумагой; кости сложить в кастрюльку, налить белым бульоном и, сварив до вкуса, сделать из него белый соус, который процедить на сотейник, выкипятить с соком из трюфеля до надлежащей густоты, процедить сквозь салфетку, положить сверху несколько кусочков масла, покрыть крышкой и поставить в горячую воду на пар. Откупорить трюфель, выложить в кастрюльку, положить ложку масла, рюмку вина шампанского, поставить на огонь, вскипятить и поставить покрытыми в горячую воду на пар. Приготовить кашу из манных круп, вымазать маслом формочки для постамента, убрать дно нарезанным трюфелем, наложить массой, сровнять ровно с краями, сложить в сотейник, подлить немного воды и сварить на пару. Пред отпуском поставить филеи на огонь и, когда поджарятся, т. е. ровно побелеют, снять бумагу, поворотить филеи, покрыть снова бумагой и дожарить до готовности. Между тем выложить из формочек кашу на блюдо, сложить вместе как должно, уложить запасерованные филеи в кружок, средину наполнить трюфелями залить размешанным до гладкости горячим соусом. (6) 

\z{Кострец воловий натурально}\index{Кострец воловий}

Снять с костей кострец воловий, срезать жилы, вымыть, завязать голландскими нитками, сложить в овальную кастрюлю, налить холодной водой и поставить на огонь; когда закипит, бульон процедить сквозь салфетку, а кострец вымыть в теплой воде, положить обратно в ту же кастрюлю, посолить, положить кореньев: петрушки, сельдерея, порея, моркови луку, пряностей и букет зелени, и варить на легком огне до мягкости (поспевает от 3-х до 4-х часов). Пред отпуском вынуть из бульона на доску, снять нитки, изрезать порционными кусками, сложить на блюдо в цельном виде и полить белым бульоном с частью рубленной зелени. Гарнир подать особо, вот такой: картофель цельный натурально, капуста, сваренная в брезе, морковь или репа; хрен заварной подать в соуснике особо. Этим способом говядина приготовляется натурально из различных частей, как-то: филейной, котлетной и грудины. (8) 

\z{Гатчинские форельки с маслом}\index{Форели ! гатчинские}

За 3 часа до отпуска очистить\footnote{Внутренность вынимается сквозь отверстие, где жабры, не прорывая у рыбы брюшка; также не соскабливать у форелек чешую.} нужное число живых гатчинских форелек, вымыть, свернуть кольцом и заправить голландской ниткой. Потом, закипятив уксусу в кастрюле, обмакивать спинку каждой форельки и, когда получат голубой цвет, сложить на дуршлаг в рыбный котел; за 15 минут до отпуска посолить, налить осторожно водой, закипятить, вынуть с дуршлагом, снять нитки и, уложив правильно на блюдо, обложить зеленой петрушкой и сваренным в воде картофелем. Растопленное сливочное масло подать особо в соуснике. (По числу персон). 

\z{Бадиджаны под бешамелем}\index{Бадиджаны ! с бешамелем}

Очистить от верхней кожи нужное число бадиджан, разрезать пополам вдоль, вынуть семена и обланширить; потом сложить на подслоенный маслом плафон, полить бешамелем из сметаны, посыпать тертым пармезаном, поставить в горячую печку, заколеровать и повторять так, пока бадиджаны наполнятся и изжарятся до мягкости, сложить на блюдо, а на плафон прибавить бульона и белого соуса, выкипятить, процедить сквозь салфетку и полить на блюде бадиджаны. (6)

\z{Репа глясованная в вине малаге}\index{Репа ! глясованная}

Очистить от верхней кожи нужное количество молодой реп ы, разрезать на 8 частей, обточить правильно, обланширить и, когда закипит, отлить на дуршлаг и перелить холодной водой. Потом сложить на сотейник, положить 2 ложки масла, глясу и 1 ложку сахару, налить бульоном пополам с малагой и варить на большом огне под крышкой так, чтобы репа упрела, а сок выварился до соусной густоты. Когда будет готово, уложить репу правильно на блюдо и залить собственным соком. Этим же способом приготовляется и брюква. (6--8) 

\z{Артишоки по-лионски}\index{Артишоки ! по-лионски}

Очистить и с резать ровно нижнюю часть артишока, натереть лимоном, разрезать каждый пополам, вырезать из средины мякоть, обровнять и класть в холодную воду, разведенную уксусом; когда в се будут очищены, обланширить в соленой воде и, когда закипят, отлить на дуршлаг, перелить холодной водой, уложить на растопленное масло в сотейнике и выжать сок из 1 лимона. За 0,5 часа до отпуска, влить суповую ложку бульону, поставить на плиту и варить, пока бульон выкипит, а низы или фонды артишоков заколеруются; потом залить красным соусом, сварить до мягкости, а пред отпуском выбрать артишоки и уложить на глубокое блюдо; соус же откипятить до надлежащей густоты, снабдить по вкусу солью, соком из лимона, процедить сквозь сито, размешать с частью сливочного масла и залить артишоки. (6--8) 

\z{Фрикандо телячье со щавелем}\index{Фрикандо ! телячье}

Обровнять костречную часть телятины, снять сверху перепонку, нашпиковать, сложить на сотейник, обложить кореньями, положить пряностей, полить жирным бульоном, поставить на плиту, а когда закипит, переставить в горячую печку, покрыть крышкой и жарить в вольном жару; потом снять крышку, полить соком, заколеровать как следует, выложить фрикандо на блюдо, а сок выварить до надлежащей густоты и вкуса, процедить, снять жир в чашку и полить фрикандо. При этом подается особо щавель. (6--8) 

\z{Котлеты из щуки с грибами}\index{Котлеты ! из щуки ! с грибами}

Снять филеи с назначенной для котлет щуки, изрубить мелко, с делать из 3 яиц яичницу (не крутую), остудить, сложить на изрубленную щуку, размешать и рубить, пока масса будет в одинаковом виде, потом снабдить по вкусу солью, перцем и мускатным орехом, сделать умеренной величины котлетки, запанировать в яйцо и тертый хлеб, обровнять и сложить на растопленное масло в сотейник. Пред отпуском изжарить с обеих сторон до колера, сложить на блюдо, а средину наполнить грибами. (6) 

\z{Брунколь с каштанами}\index{Брунколь ! с каштанами}

Очистив брункольный кочан, бросьте его в кипяток и варите, пока не с делается совсем мягким, потом искрошите его мелко-намелко и отставьте в сторону на блюде, но не в кастрюле. Теперь сотрите, на терке, одну или две сладкие свеклы и вдвое против свеклы моркови, и положите эту мякоть в готовый уже мелко искрошенный в чухонском масле обжаренный лук и прожарьте его еще, чтоб свекла и морковь с ним изжарились. Тогда смешайте брунколь с поджаренными: луком, свеклой и морковью, посолите по пропорции, положите в кастрюлю, влейте стакан хорошего мясного бульону и, закрыв кастрюлю крышкой, дайте всему порядком вскипеть, на легком огне, перемешав несколько раз брунколь, чтоб он не приставал к краям кастрюли. Когда все готово, выложите на блюдо и обложите каштанами, приготовленными следующим образом. Каштаны с шелухой изжарьте в печи, на противне или на бляхе, очистите от шелухи, вытрите салфеткой, и каждый каштан на деревянной шпильке обмакивайте в кухонный леденец. Леденец этот делается следующим образом: положите в кастрюлю фунт сахару, влейте полбутылки воды и варите до тех пор, пока не с делается густо. Тогда пробуйте деревянного палочкой сахар. Если масса не пристает к зубам — это значит, что кухонный леденец готов. Если нет каштанов, можно заменить их жареным в масле картофелем. (6) 

\z{Говядина упаренная}\index{Говядина ! упаренная}

Часть говядины от ссека избить мягко, растопить в кастрюле масла, коровьего, положить в него говядину и с обеих сторон обжарить румяно, налить водой, прибавить соли, лаврового листа, кубебы или английского перца, лимонной корки, и продолжать ужаривать. Ежели мясо употреблено будет хорошее — соус будет довольно густой и темный; когда же не темнеет — подбить его подпаленной дотемна в коровьем масле мукой, упаривать еще полчаса и приправить ренским уксусом или лимонным соком. Ежели прибавить анчоусов, каперсов и подобного, соус будет еще вкуснее. (6—8) 

\z{Тетерка по-венски}\index{Тетерка ! по-венски}

Самые вкусные серые тетерки те, которые величиной вдвое против голубя. Очистив их, шпигуют, натираюсь внутри солью и перцем, вкладываюсь туда кусок масла или шпеку и, посолив, жарят, при беспрестанном обливании растопленным маслом. 

Если тетерки немного стары, то, по очищении, кладут их в горшок, наливают половиной бутылки уксусу, прибавляюсь разных кореньев и оставляют на 4--6 дней; потом варят на легком огне и подают под приличным соусом. (4) 

\z{Телячья грудинка с крыжовником}\index{Телячья грудинка ! с крыжовником}

Грудинку сварите до половины, и вынув из воды, доварите в бульоне, порежьте на куски и, когда обсохнут, обмочите каждый кусок в сбитые яйца, обсыпьте тертыми сухарями и изжарьте хорошенько в масле. Наберите неспелого крыжовника и, закипятив в воде, выложите на сито. В особой кастрюле сделайте сахарный сироп, кладя фунт сахару на стакан воды, или по пропорции, полфунта на полстакана воды — или 2 ф. сахару на 2 стакана воды, смотря потому, как велико блюдо. Прибавьте к сахарному сиропу несколько лимонной корки и, вскипятив, вложите в этот сироп отваренный крыжовник, и когда он лишится своей остроты, выложите на блюдо и обложите кусками жареной грудинки. Надобно наблюдать, чтоб соус и грудинка изготовлялись в одно время, потому что куски грудинки должны быть подаваемы на стол горячими. (6--8) 

\z{Ревельская печеная картофельная каша с селедкой}\index{Каша ! картофельная с селедкой}

На это блюдо нужен хороший, мучнистый картофель, который тщательно обчищают, потом разрезывают на куски и, прибавив достаточно воды и немного соли, варят таким образом, чтоб он сильно кипел, при чем должно чисто и часто снимать пену и прежде, чем картофель совсем сварится и развалится, слить всю воду дочиста и, закрыв картофель, дать ему преть в собственном пару. Потом его истирают мелко или пропускают через сито; далее смешивают с 1/4 квартой хороших сливок, 3/8 фунта (12 лотов) свежего масла, четырьмя яичными желтками, малым количеством соли и мускатного ореха и варят на огне, пока все превратится в густую кашу, которую ставят пирамидообразно на блюде, намазанном маслом и выдерживающем жар; затем чайной ложечкой делают на нем разные фигуры, наполняя отверстия разведенным маслом (лучше раковым), и 15 или 20 минут дают печься в сильном печном жару. Снизу кругом обкладывают этот картофель венчиком из кусков жареной селедки, которую приготовляют следующим образом: обмыв 3 селедки и дав им пролежать часов 12 в воде, их очищают от костей, снимают кожу, и половинки селедок на время мочат в молоке, чтобы соль как можно более вытянуло. После того их обсушивают, каждую половинку разрезывают еще раз, в то же время мешают 1/4 фунта растопленного масла с яйцом, катают в нем селедок, которых потом посыпают крошеной булкой, и наконец поджаривают на угольях с обеих сторон. (6--8) 

\z{Паштет из щуки с соусом}\index{Паштет ! из щуки ! с соусом}

Очистив рыбу, изрежьте на куски и посолите. Часа через два вытрите рыбу, чтоб она была суха и чтоб соли на ней не оставалось. Поджарьте рыбу в масле (в кастрюле), прибавив маленечко мелко искрошенной лимонной корки. Потом положите в ту же кастрюлю изрезанных в мелкие кусочки артишоков (разумеется, фонды, а не листья), сморчков, раковых шеек целиком, всыпьте немного тертой булки или сухарей, влейте с 0,5 ст. бульону и дайте всему этому раз порядочно вскипеть вместе с рыбой. После того вымажьте маслом жестяное (или серебряное) блюдо, вылейте в него весь этот соус с рыбой, прибавьте немного крошеной лимонной корки и, если любите, немного мускатного цвета, и сделайте из паштетного теста крышку, прикрепив к краям блюда, помажьте тесто яичным желтком и поставьте в печь. Когда тесто до половины спечется, вырежьте маленькое отверстие в корке, чтоб паштет не лопнул, и оставьте в печи, пока не спечется совершенно. Между тем сделайте соус: вбейте в кастрюлю три или четыре желтка, влейте немного винного уксусу, всыпьте немного пшеничной муки, перемешайте хорошенько, положите порядочный кусок (по пропорции) масла и две цельные луковицы и подлейте горячей воды, в которой выварилась петрушка, чтоб вода имела сильный запах петрушки; варите, мешая, пока соус не будет довольно густ. Этот соус влейте в паштет и подавайте на стол. Паштет можно с делать постным, употребляя вместо коровьего масла прованское, не прибавляя яичных желтков, и вместо мясного бульону подлив крепкой рыбной ухи. (6) 

\z{Утка с грибами}\index{Утка ! с грибами}

Возьмите, сколько нужно для вашего стола, грибов боровиков или, вообще, так называемых белых грибов, очистите, закипятите воду в кастрюле, бросьте грибы в кипяток и дайте вскипеть раза три ключом. Потом выньте грибы и спустите с них воду на решете. В другой малой кастрюльке изжарьте в масле несколько покрошенных луковиц и положите в кастрюлю, где находятся грибы. Влейте туда же несколько сметаны, посолите, всыпьте перцу, перемешайте, переложите грибы в глиняную кастрюлю, додайте, не жалея, коровьего масла и положите, по числу гостей, одну, две или три домашние утки, опаренные или лучше несколько поджарившиеся. Эту глиняную кастрюлю, с грибами и утками, вставьте в горячую печь и дайте грибам жариться до тех пор, пока они и утки не поспеют. Чтоб грибы не запекались, можете прибавить несколько ложек бульону. Известно, что мясо некоторых животных, птиц и рыб имеет отличный вкус с известными растениями, как будто по каким-то особенным законам симпатии. С грибами лучше всего кормленная домашняя утка. (По числу персон). 

\z{Колбасы по-испански}\index{Колбасы ! по-испански}

Обрежьте лучшее мясо с передних и задних окороков, с хребта и с ребер молодой, хорошо откормленной свиньи, изрубите его помельче, посыпая мелкой солью, по вкусу, прибавьте толченого ямайского (крупного душистого) перцу 2 золотника, истертого в порошок мускатного цвета 1/4 золотника, мелко изрубленных свежих лимонных корок 3 золотника, измельченных лавровых листьев одну щепотку, хорошего рейнвейна одну бутылку, очищенных и мелко изрубленных, соленых воловьих языков 6 штук. Хорошенько перемешайте все это, положите на вычищенную и разостланную кожу поросенка, заверните в нее всю смесь, обвяжите чистыми веревочками, заверните в салфетку, поварите около получаса в смеси двух частей воды с одной частью винного уксусу и одной же частью белого виноградного вина, и после выкоптите колбасу обыкновенными образом. (6--8) 

\z{Фрикасе из цыпленка}\index{Фрикасе ! из цыпленка}

Очищенного и вымытого цыпленка изрезать правильно, как транжируется живность, т. е. отнять крылышки, ножки, филей, грудинку и спинку на 2 части, обровнять и сложить на тарелку. Между тем распустить в кастрюльке ложку масла, положить 1,5 стол. ложки муки, развести 1,5 стаканом воды, посолить, положить изрезанного цыпленка и одну цельную луковицу, поставить на огонь и часто мешать, чтобы ко дну не пристало; когда закипит, варить на легком огне, и когда цыпленок сварится, выбрать по кусочку на тарелку, а соус выкипятить до надлежащей густоты, снять с огня, положить один сырой размешанный со сливками желток, по вкусу соку из лимона, процедить сквозь салфетку и, размешав до гладкости, полить цыпленка. (4) 

\z{Жареная цветная капуста с сыром}\index{Цветная капуста ! с сыром}

Разрезать головки цветной капусты в длину на несколько кусков, снять твердую кожицу со стеблей и положить часа на два в холодную воду. Потом варить капусту в соленой воде, пока не сделается мягкой, и тогда выложить на блюдо, вымазанное хорошо маслом, притиснуть, чтобы она улеглась плотно, полить растопленным чухонским маслом, посыпать крупно истертым сыром пармезаном (а если его нет, то хоть швейцарским), поставить блюдо в горячую печку, и печь до тех пор, пока капуста не примет, так называемого, румяного цвета, т. е. пока не поджарится. На стол должно подавать на том же блюде, на котором капуста жарилась. (По одной головке на персону). 

\z{Шпинатный каравай}\index{Каравай ! шпинатный}

Намочить булку в сливках или в цельном молоке, сбить 0,5 фунта коровьего масла и смешать с булкой, прибавить 8 штук свежих яиц, 4 лота толченого миндалю, корку одного лимона, 0,5 фунта сахару и 4 ложки шпинатного соку, впустить 8 яичных белков, сбитых прежде в пену, и, смешав все вместе, положить в форму, вымазанную маслом, и испечь в легком духе. (6--8) 

\z{Заяц по-английски}\index{Заяц ! по-английски}

Внутренность зайца начиняют мякишем белого хлеба, намочив его в молоке и сварив до состояния теста, к которому прибавляют сырых яичных желтков, соли, перцу, пряностей, одну изрезанную на части луковицу, немного масла и шалфею. Начинив этим фаршем зайца, кладут его на вертел, обкладывают шпеком и обвертывают бумагой; через полтора часа заяц готов. Перед тем, как подавать на стол, бумагу и шпек снимают, и едят зайца, приготовленного таким образом, с вареньем из красной смородины, которое подают отдельно. (8) 

\z{Куропатки с капустой}\index{Куропатки ! с капустой}

Возьмите кочан капусты средней величины, разрежьте его надвое и обварите в кипятке; затем выньте из воды, дайте ей стечь, остудите капусту, и обвяжите оба куска ниткой или веревочкой с двумя куропатками, хорошо очищенными и выпотрошенными, и даже нашпигованными; положите на дно кастрюли несколько ломтей шпеку, капусту, 2 или 4 сосиски, 2 моркови, столько же луковиц, обсыпьте все солью, перцем, покройте шпеком, и поставьте кастрюлю на слабый огонь. Когда все совершенно сварится, выложите капусту на полотно и дайте воде с нее стечь и даже несколько нажмите, чтобы весь бульон вышел. Положите на блюдо куропаток, и обложите их кругом капустой, изрезанной на ломти, на которые положите по половине сосиски и по небольшому куску шпека; разрежьте на кружки морковь и обложите ими блюдо, подогрейте остаток жижи в кастрюле и облейте ею капусту. (6) 

\z{Говяжьи мозги в тесте}\index{Мозги говяжьи в тесте}

Очищают мозги от сгустившейся крови, покрывающих их перепонок и волокон и ополаскивают в теплой воде. Потом кладут в кастрюлю, вместе с несколькими ломтями свиного сала, лавровым листом, морковью и репчатым луком, предварительно изрезанными. Затем, прибавив к этому пучок петрушки и мелкого луку, бульону и белого вина по равной части, приправляют солью и крупным перцем, и варят на слабом огне в продолжение получаса. Сваривши таким образом мозги, вынимают их и дают им остынуть. Между тем делают тес то из 4 ложек муки и достаточного количества теплой воды, в которой должно распустить прежде небольшой кусок коровьего масла, приправляют солью и прибавляют 2 яичных желтка и 2 взбитых белка; тесто это должно иметь густоту сливок. Небольшое количество французской водки, прибавленное в это тесто, делает его легким и хрупким. Разрезав мозги на куски, обваливают их в приготовленном тесте и поджаривают до тех пор, пока они получат красный цвет. После этого выкладывают их на блюдо вместе с поджаренной сверху петрушкой. (Одна штука мозгов на три персоны). 

\z{Утка в белом соусе}\index{Утка ! в белом соусе}

Сварите одну или две утки в воде или, лучше, в белом бульоне, разрежьте как следует к подаванию на стол, куски уложите на блюде и залейте следующим соусом: возьмите полфунта масла, две ложки пшеничной муки и разведите в бульоне, так чтобы соус был густ, но переливался через ложку; прибавьте 1,5 стакана сметаны, осьмушку сахару, и варите, пока соус не сгустеет. Потом процедите этот соус чрез кухонное сито, и положите сверху тонкие ломтики лимона или посыпьте мелко искрошенными зелеными укропом и петрушкой. (6) 

\z{Гусь под хреном}\index{Гусь ! под хреном}

Вымыв и разрезав гуся, разварить его в воде с несколькими луковицами, перцем и солью. Положить в маленькую кастрюлю ложку масла и несколько щепоток муки, стакан сливок и, перед тем, как подавать, 3 горсти только что натертого хрену; варить все это только несколько минут. Разложить потом гуся на блюдо, облить его немного хреном; остальное же подавать в соуснике. (8) 

\z{Пудинг из тетерки}\index{Пудинг ! из тетерки}

Снять с тетерки филеи, изрубить мелко, сложить в ступку, истолочь, положить размоченный в бульоне и потом отжатый досуха хлеб без корки, протолочь, положить 2 ложки масла, одно яйцо, протолочь окончательно и протереть сквозь сито; потом сложить в кастрюльку, размешать, прибавить столько взбитых сливок, сколько примет крепость фарша, т. е. положить в кипяток кусочек фарша, и если окажется крепкий, прибавить сливок, поставить пробу и прибавлять сливки постепенно, пока фарш станет нежен; тогда наслоить маслом форму, наложить 3/4 формы фаршем, поставить в горячую воду так, чтобы форма была в воде лишь до половины, покрыть и сварить на пару до готовности (поспевает от 15 до 30 минут). Из костей и обрезков с делать сперва сок, а потом соус, и выкипятить до надлежащей густоты и вкуса. Когда пудинг будет готов, выложить на блюдо и залить соусом. (4) 

\z{Котлеты из поросенка с красной капустой}\index{Котлеты ! из поросенка ! с красной капустой}

Взять котлетную часть поросенка (8 косточек), нарезать котлетки, очистить, как должно, разбить сечкой, посыпать солью с перцем, запанировать в яйцо и тертый хлеб, сложить на растопленное в сотейнике масло, поставить на огонь, заколеровать, перевернуть, дожарить (не пересушить), сложить на блюдо и подлить сову. Красную капусту подать особо на тарелке. (4) 

\z{Сосиски с красным соусом}\index{Сосиски ! с красным соусом}

Приготовить сосиски, сложить на растопленное в сковороде масло, поставить на огонь и изжарить до колера. Между тем исшинковать одну луковицу, сложить в кастрюльку, залить маслом, в котором жарились сосиски, поставить на огонь, и когда начнет желтеть, снять с огня; изжаренные сосиски снять со сковороды на блюдо, а на сковороду положить ложку муки, развести стаканом бульону, сварить до надлежащей густоты и вкуса, процедить в кастрюльку, где лук, вскипятить еще раз, положить по вкусу соку из лимона или 0,5 ложки уксусу, немножко рубленной зелени, залить сосиски и подать горячими. Любителям подают особо картофельное пюре (4) 

\z{Капуста жареная}\index{Капуста ! жареная}

Изрезать мелко очищенный верхний жир из филея, сложить в кастрюльку и растопить. Когда будет готов, процедить в сотейник, положить 2 исшинкованные луковицы, поставить на огонь, поджарить вполовину, положить 1,5 ф. шинкованной кислой капусты (без соку), размешать, покрыть крышкой, запасеровать, снять крышку прочь и поставить в горячую печку; когда начнет сверху колероваться, помешать и продолжать наблюдать за этим до тех пор, пока капуста изжарится до мягкости и получит красновато-желтый цвет; тогда выложить, на сито, отжать ложкой жир досуха, сложить в горячий салатник, размешать и подать к жаркому. (4) 

\z{Судак разварной с маслом}\index{Судак ! разварной с маслом}

Очищенного и вымытого судака сложить на решетку в рыбный котел, посолить, налить холодной водой, поставить на огонь, и когда закипит, сдвинуть на край плиты. Между тем сварить в соленой воде правильно очищенный картофель, выложить судака на блюдо, обложить картофелем и зеленой петрушкой, а масло подать особо в соуснике. (4) 

\z{Почки телячьи с соусом пикантным}\index{Почки телячьи ! с пикантным соусом}

Выбрать крупные и белые телячьи почки с верхним жиром, разрезать вдоль пополам, сложить на растопленное в сотейнике масло жиром вниз, посолить, наложить крышку и поставить на огонь. Когда поджарится, перевернуть, поджарить снова, а когда будут готовы, слить часть, жира (если окажется много), вынуть почки на тарелку, а на сотейник положить 1,5 ложки муки, влить стакан бульону, выварить до густоты и надлежащего вкуса и, процедив половину, залить почки, а другой половиной залить гарнир-пикант. Между тем вырезать из белого хлеба 4 крутона такой величины, чтобы на каждый можно было положить удобно половинку телячьей почки, и поджарить на масле с обеих сторон. Сперва уложить на крутоны почки телячьи, потом выложить гарнир в средину на блюдо и сверху залить соусом. (4) 

\z{Биток со сметаной}\index{Биток ! со сметаной}

Отвесить 2 ф. мягкой говядины, вырезать жилы, положить 0,5 ф. воловьего от почки жиру, выбрать жилы, сложить вместе и изрубить до совершенной мелкости, посолить, положить немного перцу, 0,5 столовой ложки масла, размешать и разделить на столько частей, сколько предполагается иметь битков, обвалять каждую часть в муке, сформировать круглые битки, сложить на растопленное в сковородке масло и изжарить на огне; когда будут готовы, снять на блюдо, а на сковороду положить 0,5 стакана сметаны, выварить сок до густоты, процедить сквозь сито и залить битки. Любителям подают к биткам жареный картофель.

\z{Рагу из птичек с ветчиной}\index{Рагу ! из птичек ! с ветчиной}

Птичек (3) очистить обыкновенным образом, и вместе с ломтями ветчины пообжарить в горячей воде; после этого растопить ветчинного сала в кастрюле, и в нем подпалив муки дотемна, положить туда же жаворонков и ломти ветчины; когда довольно прожарятся, надобно развести сок в кастрюле немного мясным отваром, приправить уксусом, солью, перцем, гвоздикой, лимонной коркой, прибавить, если угодно, сахару и уварить. (6). 

\z{Вермишель с шампиньонами}\index{Вермишель ! с шампиньонами}

Приготовить домашнюю лапшу, или 1 ф. купленной вермишели или макарон отварить в соленой воде, откинуть на решето, перелить холодной водой; вбить 2 яйца, 2 ложки сметаны, 1 ложку масла, соли, немного английского толченого и 1--2 зерна простого перцу. Полную тарелку шампиньонов очистить, мелко нарезать и жарить в кастрюле под крышкой с солью и маслом; форму намазать маслом, посыпать сухарями, класть ряд лапши, ряд поджаренных шампиньонов, вставить в горячую печь; подавая, выложить на блюдо. (6) 

\z{Лещ печеный с соусом}\index{Лещ ! печеный с соусом}

Очистить, посолить, оставить так на 0,5 часа, обтереть, обсыпать мукой. Распустить на противне ложку масла, положить на него леща, полить сверху также 2 ложками масла, вставить в печь; когда будет готово, переложить осторожно на блюдо и облить красным соусом. (6) 

\z{Штокфиш или треска}\index{Треска}

Свежепросольную треску надо мочить целые сутки, переменяя чаще воду, и потом перед самым отпуском сварить ее в двух водах. 

Штокфиш же надо положить в деревянную посуду, налить крепким щелоком, переменять его раз в день, и таким образом в продолжение 4 дней. Потом слить щелок, налить на одни сутки водой, смешанной с негашеной известкой; как только побелеет, опять сполоснуть, намочить в речной воде на один день, три раза в день переменяя воду. 

Зимой можно заготовлять ее таким образом на несколько раз, и когда нужно варить, взять фунта 4--5, налить водой, поставить на плиту на умеренный огонь; когда вода согреется, слить, налить свежую, и поступать так, пока вода не перестанет быть клейкой, тогда дать воде согреться, но не дать вскипеть, потому что штокфиш сделался бы тогда жестким, откинуть на решето, чтобы стекла вода, посолить немного, переложить на блюдо, прикрыть, чтобы не остыл; подавать. 

Другой способ скорейшего приготовления штокфиша следующий: с вечера выбить треску хорошенько деревянным обухом или топором, потом намочить на ночь в речной воде, и на другой день сварить ее в соленой воде. 

\z{Щука под желтым соусом}\index{Щука ! под желтым соусом}

Очистить щуку, нарезать кусками, на 1 час посолить, положить в кастрюлю, влить 1 стакан столового вина, рюмку уксусу, воды, так, чтобы рыбу покрыло, разных точеных и отваренных кореньев, 0,5 ст. изюму, 1/2 лимона, нарезанного ломтиками, без зерен; варить на большом огне. Как только рыба уварится, взять 2 ложки масла (1/4 ст. мелкого сахару), 1,5 ложки муки, 1/4 чайной ложечки шафрану в порошке. Все это размешать в особенной кастрюльке, развести рыбным бульоном до 3--4 стаканов и кипятить, мешая, пока не погустеет, процедить, потом облить этим щуку, обсыпать изюмом, лимоном и точеными кореньями; можно прибавить разварного картофеля. (6) 

В пост вместо сливочного масла, взять ложки две прованского. 

\z{Утки, вареные с вермишелью и грибами}\index{Утка ! с вермишелью и грибами}

Взять утку, разрезать хребет, вынуть осторожно хребтовую кость, посолить, посыпать 1/4 чайной ложкой англ. и простого перца и 2 толчеными гвоздиками. Приготовить домашнюю лапшу, отварить ее в воде с солью, откинуть на решето, вбить 2 желтка, положить несколько отваренных, мелко нашинкованных белых сушеных грибов, ложку масла, посолить, размешать, нафаршировать этим утку, зашить. Положить ее в кастрюлю, налить грибным бульоном, прибавить кореньев, варить до мягкости, разрезать на части, сложить на блюдо, облить с лед. соусом: 2,5 стакана бульону от утки, 0,5 или 1,5 ст. сметаны, муки, ложку масла, мелко нашинкованные отваренные грибы, вскипятить. (6) 

\z{Каплун с соусом из можжевеловых ягод}\index{Каплун ! с соусом из можжевеловых ягод}

0,5 ложки можжевеловых ягод, 1 стакан тертой булки, ложку масла, 2 яйца, 2 ложки простого сыру, истолочь все в ступке, протереть сквозь сито, нафаршировать каплуна. 1 ложку можжевеловых ягод истолочь, протереть сквозь сито, намазать целого каплуна и бумагу, которой обвязать его. Жарить на вертеле, поливая маслом; потом снять бумагу, обсыпать 1 толченым сухарем, облить стекшим маслом; когда подрумянится, сложить на блюдо, облить маслом, поджаренным с 2 ложками толченых сухарей. (6) 

\z{Стерлядь на виноградном вине}\index{Стерлядь ! на виноградном вине}

Очистить фунта 3--4 стерляди, разрезать на куски, перемыть в холодной воде, вытереть чистой салфеткой, уложить в один ряд в серебряную кастрюльку, а за неимением ее в сотейник, положить 1/4 фунта сливочного масла, соли, сок из 1/2 лимона, влить 2 стакана белого вина. За пять минут до отпуска поставить на плиту, чтобы сварилась, и тотчас подавать в той же самой кастрюльке. (4—6) 

\z{Картофельно-рыбные котлеты}\index{Котлеты ! картофельно-рыбные}

Отварив несколько окуней, или морских небольших сигов, очень вкусных для этого употребления, снимите все мясо с костей, изрубите его мелко и прибавьте к нему некоторое количество протертого печеного картофелю. Старайтесь выбирать самый желтый. В эту смесь кладут, по вкусу: масла, соли, перцу, петрушки и два сырых яйца; стирают все как можно лучше, даже толкут, чтобы тесто было мягче, потом делают круглые лепешки, стараясь, чтоб все они были ровной величины, и тогда, распластав их в виде котлет, валяют в сыром яйце, потом в сухарях, и жарят на сковороде точно также, как бархатные котлеты (см. № 468 в IX отделе), с той разницею, что для них необходима подливка на грибном или, еще лучше, собственном шампиньонном бульоне. 

\z{Красная капуста в соусе}\index{Красная капуста ! в соусе}

Нашинковав капусту очень мелко, кладут ее в просторную кастрюлю с большим куском чухонского масла и дают преть, мешая как можно чаще, и держа кастрюлю накрытой. Таким образом, капуста дает свой собственный сок и сохранит хороший цвет. Многие кухарки варят ее в воде, но от этого теряет капуста и вкус, и вид. В масле преть должна капуста часа полтора, а перед обедом, сцедив ее сок в небольшую кастрюльку, кладут туда ложку муки, разбивают, дают прокипеть раза два и вливают в капусту, которую ставят на горячее место, чтобы соус проник в нее хорошенько. Перед самым обедом прибавляют небольшую ложку уксусу; но тогда уже соус отнюдь не должен кипеть, чтоб капуста не могла побелеть. 

\z{Говядина-фарш}\index{Говядина-фарш}

Кусок говядины от филейной части разрезать на большие зразы, выбить их деревянным пестиком, потом порубить немного тупой стороной ножа и слегка посолить. Один зраз мелко изрубить, прибавить немного жиру, истолочь в ступке, положить 2 яйца, 3 ложки тертой булки, англ. и простого перцу, размешать хорошенько и этим фаршем переложить зразы, накладывая их один на другой; наложить легкий пресс. Дно кастрюли выложить тоненькими ломтиками шпику, сложить на них мясо, намазать сверху яйцом, посыпать хлебом, наверх положить также ломтики шпику или полить маслом, подлить стакан бульону; вставить в печь. Подавая, нарезать ломтиками, облить процеженным соусом, подлив бульону. (6) 

\z{Зразы без фарша с картофелем, кашею и проч.}\index{Зразы ! с картофелем}

2,5 фунта говядины без костей нарезать довольно большими ломтиками в палец толщиной, выбить хорошенько с одной стороны деревянным пестиком, а потом с той же стороны тупой стороной ножа, посолить, посыпать перцем, через час свернуть в трубочку, обвалять в муке; 1/4 фунта масла распустить в кастрюле; когда масло закипит, положить зразы, подрумянить их со всех сторон, положить 1 мелко изрубленную луковицу, тогда налить бульоном, чтобы их покрыло, накрыть крышкой и душить так с 1,5 часа; потом подлить 3--4 ложки сметаны, вскипятить раза 2; если соуса мало, то подлить еще бульону, так, чтобы соуса было не менее 3 стаканов, процедить. (6) 

\z{Брюква фаршированная}\index{Брюква ! фаршированная}

4--8 шт. брюквы, смотря по величине, очистить, отварить до мягкости, разрезать каждую на 2 части, выбрать осторожно середину, вынутую массу растереть, смешать с 2 ложками мелких сухарей, поджаренных в ложке масла, 2--3 ложками сметаны; если брюква не сладка, прибавить немного сахару, соли, мускатного ореха; можно подлить немного бульону; смешать, нафаршировать брюкву, уложить ее в кастрюлю, намазанную ложкой масла, вставить в печь, чтобы фарш погустел и сверху подрумянился, переложить на блюдо; 0,5 ложки масла размешать на огне с ложкой муки, влить с 0,5 ст. сметаны, стакан воды, в которой варилась брюква, сахару, вскипятить; облить на блюде брюкву. (6) 

\z{Фаршированная капуста по-литовски}\index{Капуста ! фаршированная ! по-литовски}

Два небольшие кочана капусты очистить от зеленых листьев, разрезать каждый на четыре части, опустить в кипяток соленой воды на 1/4 часа, откинуть на решето, выжать капусту осторожно в руках; 3/4 фунта говядины, 3/4 фунта почечного сала изрубить очень мелко, положить соли, простого и английского толченого перцу, 2 ложки мелко изрубленного луку, истолочь все вместе, начинить этим фаршем каждый кусок капусты, перекладывая его между листьями и перевязывая ниткой. Приготовленную таким образом капусту положить в кастрюлю, налить водой или бульоном, немного посолить и варить, пока фарш не уварится; 1/2 ложки масла распустить в кастрюле, всыпать ложку муки, мешая на плите, развести 4 стаканами бульону или воды, в которой варилась капуста, вскипятить, облить этим капусту, дать еще раз вскипеть; подавая, снять нитки. (6) 

\z{Артишоки в виноградном вине}\index{Артишоки ! в виноградном вине}

Очистить фонды или внутренняя части артишоков от листьев, раз вскипятить их в соленой воде, откинуть на решето, положить в кастрюлю на распущенное сливочное масло, влить малаги, положить сахара, сухого бульона и кипятить час или два, подливая понемногу бульона, чтобы не пригорело, выложить на блюдо; подавать. (На 6 персон, ежели не менее 6 шт. артишоков). 

\z{Аморетки}\index{Аморетки}

Мозги из воловьих костей опустить в кипяток на самое короткое время, вынув их дуршлаговой ложкой, нарезать, положить в кастрюлю, налить бульоном так, чтобы их покрыло, посолить, положить с 0,5 ложки масла и вскипятить на большом огне. (6) 

\z{Свиные котлеты}\index{Котлеты ! свиные}

Взять из свинины — котлетную часть, разрезать так, чтобы при каждой косточке было мясо, выбить его хорошенько деревянным пестиком, посолить, посыпать перцем, намазать 1 яйцом, посыпать 5-- 6 сухарями, сложить на сковороду на 2 ложки раскаленного масла, поджарить с обеих сторон на плите. Или другим манером: 1 ложку мелко изрубленной луковицы поджарить в ложке масла; когда остынет, вбить 3 яйца, размешать; намазать тем котлеты, посыпать 5--6 сухарями; положить на сковороду или противень, намазанный 1,5 ложками масла, вставить в печь. Подавать с соусом. (6) 

\z{Колдуны с фаршем из телятины с селедкой}\index{Колдуны ! с фаршем из телятины с селедкой}

1,5 фунта телятины без костей сварить, мелко изрубить, смешать с вымоченной, очищенной от костей и мелко изрубленной 1 шотландской или голландской селедкой, 2 крутыми мелко изрубленными яйцами; 1 ложку масла распустить, поджарить в нем 1 небольшую мелко изрубленную луковицу, положить туда же фарш, слегка поджарить, мешая, прибавить крошечку мускатного ореха, толченого простого и английского перцу; когда остынет, нафаршировать колдуны, и далее вообще поступить с ними как описано в другом месте. Подавая, облить маслом. (6) 

\z{Баранья грудинка с яйцами и тертым хлебом}\index{Баранья грудинка ! с яйцами}

Сварить баранью грудинку, вынуть из бульона и дать ему стечь. Потом взять одно яйцо, выпустить его на тарелку, взбить ножом, обвалять мясо со всех сторон сперва в нем, после того в тертом хлебе, перемешанным с изрубленной петрушкой. Распустить коровье масло и, когда начнет оно кипеть, положить в него мясо и дать ему пожелтеть. Приготовленное таким образом мясо кладут на молодой горошек, молодую морковь, рубленную белую капусту и пр. (6) 

\z{Грудинка в рисе}\index{Грудинка ! в рисе}

Приготовить грудинку как следует, положить в горшок, налить холодной водой и поставить на огонь. Сняв пену, положить кореньев и соли и варить домягка. На 8 фунтов мяса взять фунт рису, отварить его, выбрать, дать ему хорошенько разбухнуть в холодной воде, налить на него жирного бульона из-под мяса и, прибавив только изрезанную лимонную корку и пол щепотки мускатного цвета, варить до тех пор, пока превратится в густую массу. Между тем уложить красиво на блюде разварившуюся домягка грудинку, обрезать ее так, чтоб она имела красивый вид, вымазать яйцом и облить загустевшим рисом, который гладко выровнять ножом. Остальным рисом обложить боковые стороны грудинки, посыпать все это тертым сыром пармезаном, поставить в печь и дать зарумяниться. При подаче на стол поставить это блюдо на другое. (6—8) 

\z{Телятина со спаржею и раками}\index{Телятина ! со спаржей}

Изрубить телятину точно также, как для фрикасе, обварить до полуспела; выложить в кастрюлю с куском коровьего масла, прибавить целую луковицу, два лавровых листа, лимонной корки, мускатного цвета и разварить домягка. 

Между тем сварить, сколько нужно, раков, слупить черепки со спины, шеек и клешней, чтоб раки остались целыми; очистить сморчки и отварить их; приготовить спаржу и также отварить. Напоследок сделать еще следующий соус: замесить 2 или 3 желтка с щепотью крупитчатой муки, положить туда кусок коровьего масла, мелко искрошенной лимонной корки и целую луковицу, прибавить немного мясного бульону и сварить на угольях, беспрестанно мешая. Если соус густ, то развести его мясным отваром или, лучше, соком, вытекающим при разрезывании жареного мяса. (6--8) 

\z{Телячья печенка в виде ежа}\index{Телячья печенка ! в виде ежа}

Отварить телячью печенку, когда остынет, растереть на терке, и вместе с 4 яйцами, полной ложкой растопленного масла, небольшим количеством соли и с тертым молочным хлебом замесить сухое тесто. Из этого теста выделать на жестяном листе, который вымазать коровьим маслом и посыпать тертым хлебом, фигуру наподобие ежа, натыкать в нее очищенного и нарезанного продолговатыми ломтиками миндалю, обкапать коровьим маслом, поставить в печь, в вольный дух, и, часто обливая маслом, дать пропечься исподволь. Между тем сделать густой соус из подпаленной дотемна муки, коринки, кружков лимона, полной чашки винного уксусу и сахару с кипящею водой, выложить ежа на блюдо, облить вокруг соусом, а остальной подать на стол отдельно в соуснике. (6) 

\z{Фрикасе из цыплят с вином}\index{Фрикасе ! из цыплят с вином}

Приготовить цыплят как следует, разрезать для фрикасе на красивые куски, положить в кастрюлю с куском масла, с завязанными в тряпочку душистыми травами, несколькими кружками лимона, солью, шампиньонами, предварительно приготовленными для того телячьими молоками и полной суповой ложкой воды, так чтоб последняя не доставала до мяса, накрыть кастрюлю и варить исподволь. Потом вскипятить два стакана вина, налить его в фрикасе и варить с ним, пока останется уже немного времени до подачи на стол. Тогда прибавить туда два яичных желтка, разболтанных с полной ложкой муки в нескольких каплях воды, размешать все это, чтоб вышел густой соус; с этой минуты перестать варить и дать окончательно пропреть. Можно также прилить к разболтанным яйцам бульон и вскипятить с ним. Подавая на стол, облить цыплят приготовленным соусом. 

\z{Тимбал из риса, кур, голубей или телятины}\index{Тимбал ! из риса}

Приготовить нежное тесто, скатать его на подобие колбасы, толщиной в палец, обложить им дно кастрюли улиткообразно, намазать яйцом, раскатать из теста лист и обложить им стенки кастрюля, которую наперед вымазать маслом. На этот лист положить слой риса, толщиной в палец, на него фрикасе из курицы, голубя или телятины, опять слой риса, нажать к стенкам, сверху накрыть другим листом, раскатанным из того же теста, поставить тимбал в печь; когда поспеет, вытряхнуть на блюдо, облить частью лимонного соуса, а остальную подать в соуснике отдельно. (4—5) Впрочем и на большее число, смотря по количеству материала д ля фрикасе. 

\z{Пупетон из риса и курицы}\index{Пупетон ! из риса и курицы}

Густо сварить рис, сделать из него на блюде края, положить в середину заранее изготовленное фрикасе из курицы, разложить на нем остальной рис в виде звезды, вымазать яйцом и маслом и испечь исподволь в печи. В соус подмешать несколько бульону и загустить его яичными желтками и небольшим количеством муки. (4--6) 

\z{Говяжий язык с яблочным соусом}\index{Язык ! с яблочным соусом}

Отварить говяжий язык в воде с солью и пряностями, которые положить неразрезанные. Когда язык размякнет, слупить с него белую кожу и разрезать вдоль по середине. Можно подавать его с соусом точно в таком виде, или обвалять прежде в тертом белом хлебе и обжарить в коровьем масле. Для соуса очистить немного кисловатые яблоки, изрезать на куски, положить в кастрюлю, налить немного воды и сварить мягко. Потом протереть сквозь сито, развести красным вином, прибавить лимонной корки, сахару, корицы и уварить. (6--8) 

\z{Красивое блюдо из гороха и щавеля}\index{Щавель ! с горохом}

Из мягкого теста сделать крест, положить в глубокое, но плоское блюдо; в середине креста посадить какую-нибудь фигурку или цветок, намазать яйцом и поставить на жестяном листе в печь. (Если ставить в печь на блюде, то оно может треснуть). Когда крест зарумянится, переложить его опять на блюдо и разделить на четыре отделения, из которых два наполнить горохом, а два щавелем. Горох обложить ломтиками белого хлеба, изжаренными в масле, края блюда — кружками из того же хлеба, а щавель — маленькими сосисками. Для этого блюда нужно брать желтый горох, который предварительно отварить в воде и протереть сквозь решето. (8) 

\z{Бараньи котлеты à la polonaise}\index{Котлеты ! из баранины}

Нарезать котлеты с косточками, отбить, посолить, жарить на растопленном масле на легком огне; когда будут готовы, отставить в холодное место. Положить на сотейник красного соуса, ложку пюре из томатов и вскипятить до густоты; обмочить в этом соусе каждую котлетку и застудить. За 1/4 часа до отпуска, обмочить в кляр и жарить в горячем фритюре; положить на блюдо с обжаренной зеленой петрушкой. (По количеству котлет). 

\z{Курица или пулярдка с грибами и яйцами}\index{Курица ! с грибами и яйцами}

Положите очищенную курицу в кастрюлю, залейте бульоном (или водой), вложите понемногу моркови, петрушки, 3 головки сельдерея, 2 луковицы, 2 пригоршни грибов (свежих или сушеных, разваренных прежде), прибавьте перцу в зернах и соли, и варите, пока курица и грибы сварятся совершенно. Тогда выньте курицу, отрежьте мясо ровными кусками от костей и поджарьте в масле на сковороде, посыпав сперва хорошенько мукой и прибавив к маслу ложки четыре густой сметаны, всыпав туда же 5 растертых яичных желтков, крепко сваренных. Когда все это пережарится, смешайте с грибами, которые должно сперва мелко искрошить, а всю прочую зелень отбросьте, и когда масса сделается довольно густой, выложите ее на блюдо, сверху положите мясо, и блюдо поставьте в теплую печь, чтоб все это запеклось. Для 2 или 3 кур всего двойная и тройная пропорция. (Одна курица на 5--6 человек). 

\z{Бараньи мозги à la poulette}\index{Мозги ! бараньи}

Положить мозги в воду и очистить от крови и оболочек, потом распластать их, в другой раз положить в теплую воду на полчаса, после чего уже положить их в кипяток и держать там до тех пор, пока они получат белый цвет. Тогда положить их в кастрюлю с надлежащим количеством чухонского масла и, обсыпав мукой, прилить воды, положить луку, шампиньонов и варить на легком огне. Перед отпуском положить несколько яичных желтков, смешанных с лимонным соком. (Одна штука мозгов на двух). 

\z{Цыплята в рисе}\index{Цыплята ! в рисе}

Разварить и разрезать цыплят и сварить рисовую кашу с маслом. Потом взять форму, положить ряд риса и ряд цыплят, посыпать перцем, солью, чуть-чуть мускатным цветом; потом положить опять ряд риса и цыплят, и продолжать таким образом, пока форма наполнится; тогда насыпать на верх побольше сухарей, облить сметаной, положить несколько кусочков масла, и поставить в печь по крайней мере на целый час. (Один цыпленок на 2 персоны). 

\z{Морковная или репная каша}\index{Каша ! морковная}

Возьмите морковь или репу целиком, очистите и сварите в крепком бульоне, a после выньте, протрите чрез волосяное сито и положите снова в тот же бульон. Сбейте вместе густых сливок и хорошего чухонского масла, по пропорции, чтоб не было весьма жирно, вбейте несколько яичных желтков и все это влейте в кастрюлю, в которой преет на легком огне тертая морковь или репа. Если угодно, можете прибавить немного сахару, а когда не любите сладкого — посолите. Если вы готовили морковь, то к ней лучше всего идет телячья грудинка, изжаренная на сковороде и омоченная сперва в жидкое тесто, на молоке с яйцами. Для репы же предпочитается баранья грудинка, сваренная в соленой воде. Должно наблюдать, чтоб бульона было немного и чтоб эта каша была очень густа. (Одна морковь и одна репа на две персоны). 

\z{Утки по-немецки}\index{Утка ! по-немецки}

Положить 2 неразрезанные домашние утки, хорошо кормленные, в кастрюлю, с 0,5 ф. масла, подлить бульона, чтоб покрывал уток, положить луку, порею, сельдерей, петрушки, посолить и сварить уток, а когда они поспеют, выложить их на блюдо (без бульона, в котором они сварились) и разрезать на части, как следует. В это время должен быть готов брюквенный соус, приготовленный следующим образом: взять молодой брюквы, величиной с крымское яблоко, снять кожу, вырезать в середине отверстие (в третью часть толщины брюквы), и положить туда телячий фарш. Фаршированную брюкву уложить в неглубокую кастрюлю, налить столько бульону, чтоб покрывал только третью часть брюквы, положить по пропорции масла, и когда брюква до половины поспеет, влить туда поджаренный в масле сахар, чтоб сообщить соусу необходимую сладость и темный цвет. Когда брюква совсем поспеет, обложить ею уток, облить соусом и подавать на стол. (8--10) 

\z{Суфле из рябчиков с шампиньонами}\index{Суфле ! из рябчиков}

Снять с двух рябчиков филеи, очистить, изрубить мелко и истолочь; когда будут готовы, положить 1/4 ф. сливочного масла, соли и перцу; потом истолочь снова, протереть сквозь сито, положить в кастрюлю, размешать, и прибавить полбутылки сбитых сливок. За 1/4 часа до отпуска, выложить в подслоенную маслом форму, поставить в горячую воду на пар, и кипятить на легком огне до готовности; перед отпуском, вынуть форму, положить в средину шампиньоны и залить белым соусом. Шампиньоны приготовляются следующим образом: выжать в кастрюлю сок из 1 лимона, влить стакан воды, вымыть шампиньоны в холодной воде и положить в приготовленную с лимоном воду; когда будут готовы, положить масла, соли и вскипятить на огне. (4--5) 

\z{Сальме из перепелок}\index{Сальме ! из перепелок}

Очищенных и заправленных перепелок положить в кастрюлю на растопленное масло, изжарить на легком огне до готовности и, остудив на льду, разрезать и уложить на холодный лист; обрезки положить в кастрюлю, налить бульоном, прибавить немного мадеры, и отварить до гляса, процедить на сотейник, прибавить красного соуса, соку из шампиньонов и трюфелей, 1 ложку пюре из томатов, и вскипятить на плите; потом процедить, убрать трюфельным бордюром, залить ланспиком и застудить на льду; перед отпуском, выложить на блюдо и положить в средину салат для холодного. (Одна перепелка на двух). 

\z{Заяц с оливками и лимоном}\index{Заяц ! с оливками}

Хорошенько вычистить ж выпотрошить зайца, снять с него кожу и, разрезав на небольшие куски, налить бульоном и варить, пока мясо будет мягко, положив туда предварительно 4 изрезанные луковицы, 0,5 ф. мелко изрезанной ветчины, а подправку из ложки масла и немного муки. Потом прибавить чашку сливок, 0,5 стакана красного вина, перцу, соли и сок из 1 лимона; вскипятить все это раз вместе, и поскорее подавать на стол. (6--8) 

\z{Соте из линей с картофелем}\index{Соте ! из линей}

Снять филеи с очищенных линей, подрезать верхнюю кожу, сложить в сотейник на прованское масло; из костей сварить бульон, сделать из него на прованском масле белый соус, вскипятить до густоты, процедить в кастрюлю и поставить на пар; нарезать картофелю, сварить в соленой воде до мягкости, а пред отпуском, слив воду, положить рубленной петрушки и ложку прованского масла; за 1/4 часа до отпуска, запасеровать на огне филеи, сложить на блюдо, залить соусом и положить в средину картофель. (Пост.) 

\z{Сосиски из мозгов}\index{Сосиски ! из мозгов}

Очистить от жил двое телячьих и двое бычачьих мозгов и искрошить вместе с полфунтом мозгов из костей, смешать с мякишем из 2 (3-х-коп.) булок, моченых в молоке, посолить, прибавить перцу, по пропорции, всыпать щепотку мелко-искрошенной лимонной корки, вбить 2 цельных яйца и 2 яичных желтка, растереть хорошенько эту массу, начинить ею кишки, завязать, бросить сосиски в кипяток и дать вскипеть раза два ключом, а потом изжарить в масле. Лучше жарить на роштере. (6--8--10, смотря по величине мозгов). 

\z{Сосиски из каплуна или пулярдки}\index{Сосиски ! из каплуна}

Изжарить каплуна или пулярдку, обобрать от костей все мясо и искрошить мелко. Сварить в малом количестве бульона мякиш из 2 булок (величиной с 3-х-коп. булку), замешанных на молоке, и мешать беспрестанно этот мякиш, чтоб в кастрюле сделался род клейстера, потом всыпать в кастрюлю искрошенное мясо каплуна или пулярдки, щепотку или 2 лимонной корки, мелко изрезанной, взбить 4 желтка, всыпать, малую щепотку мускатного цвета и посолить по пропорции. Перемешать и, когда булка и лимонная корка разварятся, начинить этой массой кишки, завязать, помочить немного в воде и жарить в масле. Жарить лучше на роштере, обливая маслом. (6-8) 

\z{Битки из телятины}\index{Битки ! из телятины}

Нарезать ломтиками, в палец толщиной, жареной телятины и избить их обухом ножа, но так, чтобы они однако же остались цельными. Мелко изрубить почку и жир, смешать с 2 тертыми луковицами, сухарями, перцем, мускатным цветом, 4 рубленными сардинками, положить понемножку на каждый кусок, завернуть, прикрепить деревянной шпилечкой, положить в кастрюлю, налить бульоном и варить, пока они будут мягки. Под конец влить стакан красного вина, положить 3 рубленные сардинки, 1 ложку каперсов, проварить все хорошенько вместе, выжать пол-лимона, и украсить лимонными ломтиками. (6--8) 

\z{Телячьи печенки но-дерптски}\index{Телячья печенка ! по-дерптски}

Для этого телячьи печенки, которые, как известно, очень нежны, изрезать в ломтики, пересыпать солью и дать им таким образом с час времени полежать. Потом соль обтереть чистым полотенцем и обвалять ломти довольно густо в муке. Тогда приставить в кастрюле на огонь самого свежего сала или сливочного масла и, когда оно разгорячится и разойдется, обжарить в нем печенки до румяности, наблюдая, однако ж, чтоб они не пересохли, а остались сочны. Для этого не надобно жалеть жирного вещества, т. е. сала или масла. Когда это сделано, подпалить в другой кастрюльке сливочного или чухонского масла с столовой ложкой муки, развести бульоном (домашним), а отнюдь не покупным (который как можно реже должно допускать в хорошей кухне), пол стаканом медока и ложкой или двумя уксусу, приправить истолченной гвоздичкой, имбирем, лимонного коркой и кусочком сахару, чтобы соусу придать сладковато-кислый вкус, и варить до сгущения; тогда прибавить ложку свежего сала или масла. Этот соус вылить на печенки, а потом все выложить на блюдо и отпустить к столу горячее. Количества приправ зависят от вкуса. (Одна почка на двух). 

\z{Баранина с картофелем}\index{Баранина ! с картофелем}

Отрезать почку и ногу молодого барана, положить в приличную для этого кастрюлю или чугун и налить столько воды, чтобы баранина была ею покрыта. Потом, когда она будет кипеть, снять пену, положить 5 луковиц и несколько кореньев. И варить ее до тех пор, пока она будет почти мягка; тогда положить горсть хорошо вымытых сморчков, тарелку вычищенного картофеля, перцу простого и индейского, сделать подправку из муки и масла; варить, пока все будет совершенно мягко, и подавать на стол. (6) 

\z{Караси с ветчинным соусом}\index{Караси ! с ветчинным соусом}

Очистив и выпотрошив карасей по обыкновению, отварить в воде с солью. Между тем в другой кастрюле поджарить дотемна муку в коровьем масле, развести мясным бульоном и уксусом и приварить. Потом обжарить 1/4 ф. ветчины, нарезанной на ломтики, положить в соус, приправить перцем, имбирем и лавровым листом, и положить туда отваренных карасей. Подавая на стол, обсыпать карасей белым хлебом и ветчинными обжаренными ломтиками. (По карасю на человека). 

\z{Корюшка с кислым соусом}\index{Корюшка ! с кислым соусом}

Взять большую свежую корюшку, выбрать ее, промыть, бросить в кастрюлю и поставить на огонь с достаточным количеством соли, с луком и с водой. Когда поспеет, слить с нее воду, откинуть на сито, переложить в соус, приготовленный, как описано в соусах, вскипятить с ним еще раз и выложить на блюдо. В соус должно подливать крепкого уксуса, потому что корюшка сама по себе трудно варится в желудке. (По 2--3 шт. корюшки на человека). 

\z{Лабардан}\index{Лабардан}

Нарезать на куски, вымочить в воде в продолжение 2 дней и 3 ночей, наливая трижды в день свежую воду, потом, за час до подачи на стол, поставить на огонь, но не кипятить, а только дать хорошенько пропреть. Для соуса взять: 3 желтка, кусок масла, 2 полные столовые ложки приготовленной для стола, сарептской горчицы, 3 чашки молока и несколько муки, вскипятить; потом выложить рыбу на блюдо, слить всю воду, облить приготовленным соусом и употреблять с поджаренным дотемна маслом и горчицей. (8--10) 

\z{Стерлядь вареная на пару}\index{Стерлядь ! вареная на пару}

Убив живую стерлядь, за 2 часа до отпуска очистить, вымыть в холодной воде, разрезать, сложить в кастрюлю, посолить и поставить в холодное место. Особо очистить соленых огурцов, немного белых кореньев, маленького луку и несколько штук оливок. За час до отпуска, выложить это к стерляди, влив немного белого столового вина с частью огуречного рассола, закрыть очень плотно, закипятить на плите и поставить в печку на вольный жар на 1 час; когда упреет, отпустить с кастрюлею на стол. (Скоромное и постное). (6--8) 

\z{Разварная стерлядь}\index{Стерлядь ! разварная}

Очистить стерлядь и вымыть, а середину начинить солеными огурцами; обвернуть ее в салфетку, завязать ниткой и положить в котел, нарезав звездочками разных кореньев, влить рассолу и белого вина; прибавить воды, положить соли; за 1/4 часа до отпуска, поставить на огонь. Особо положить в бульон обланшированных зеленых огурчиков и резанной вареной кнели штучками, также белых грибов, влить 0,5 рюмки мадеры и выдавить свежий лимон. Положить стерлядь на блюдо, облить соусом из-под нее же, обложить зеленью, на головку и на хвост положить лимону по 2 и 3 штуки. Соус подавать особо в соуснике. (6--8) 

\z{Капуста с начинкой}\index{Капуста ! с начинкой}

Для этого берут большой кочан капусты, обчищают зеленые листья, вымывают его в кипятке и после, выжав с осторожности), вырезывают середину и начиняют кочан изрезанными мелко сосисками, лесными каштанами или какой-нибудь другой начинкой; закрывают отверстие капустным листком и, завернув в шпек или свиное сало, кладут в кастрюлю, а вокруг кочана обкладывают морковью, луком, ветчиной или другой говяжьего приправкой, солью и перцем; смачивают бульоном, белым вином и варят на легком огне, после чего, вынувши капусту и процедивши сквозь сито соус, подают на блюде. (Один кочан на 5--6 чел.) 

\z{Душеная баранина}\index{Баранина ! тушеная}

Взять хорошую заднюю ногу или заднюю часть баранины, которая уже после битья пролежала день, и хорошенько поколотить ее обухом. Потом обмыть ее и нашпиговать. Положив ее на противень или в кастрюлю, прибавить соли, перцу, гвоздики, луку, лимонной корки и воды почти вровень с бараниной. Замазавши кастрюлю или противень, поставить на легкий огонь и дать преть с час. Потом открыть, снять жир и, подлив пива, дать еще с полчаса пожариться, наблюдая, чтоб не пригорело; потом подавать на стол. (6--8)

\z{Утка с репой}\index{Утка ! с репой}

Ощипав и обмыв утку, поджаривают ее в коровьем масле, потом прибавляют муки, подливают бульона с рюмкой белого вина, приправляют заячиной, солью и перцем; когда утка начнет поспевать, то прибавляют поджаренную в коровьем масле репу, сняв с этого жир, и подают. (5--6) 

\z{Рагу из жареной баранины}\index{Рагу ! из жареной баранины}

Распустить в кастрюле полфунта коровьего масла, изрезать несколько луковиц ломтиками и обжарить в масле дотемна; прибавить к этому кусок ветчины, рюмки две виноградного вина, или эстрагонного уксусу, стакан хорошего мясного отвару 1/4 фунта мелко изрубленных анчоусов, выдавить сок из одного лимона, прибавить гвоздики и ку - сок сахару и приварить. Между тем, от холодной жареной бараньей ноги нарезать тонких ломтиков и положить их в другую кастрюлю, а соус, пропустивши сквозь сито в эту кастрюлю на баранину, немного приварить. (5--6) 

\z{Курица душеная}\index{Курица ! тушеная}

Очистивши курицу, как следует, положить в глубокую черепню или кастрюлю с небольшим количеством коровьего масла, крупитчатой муки, виноградного вина, бульону, уксусу и воды; покрыть кастрюлю плотно крышкой, варить полчаса. Нарезав несколько ломтиков лимонных, положить туда же с перцем и другими пряностями по произволу; поваривши немного, прибавить подливки из муки в коровьем масле; сок пропустить сквозь сито и, если можно, прибавить еще лимонного сока и коровьего масла — душеное готово. При подаче на стол надобно выложить курицу в соусник, облить соусом и обложить ломтиками лимона. (5) 

\z{Настоящий тифлисский плов (пилав)}\index{Плов ! тифлисский}

Рис отобрать и вымыть хорошенько, в большой кастрюле вскипятить воду с солью, насыпать рис, но не надо разваривать, тотчас откинуть на решето, облить холодной водой раз, и потом другой. Чтоб узнать, что рис сварился, вынуть несколько зерен и разломать: ежели есть еще внутри мучное - немного оставить в воде. 

Когда пар пройдет у риса, облить шафраном, потом взять другую кастрюлю, положить половину масла на дно, чтоб растаяло, брать рукой рис и сыпать слегка в кастрюлю, потом сверху облить согретым маслом, накрыть крышкой, поставить на небольшой жар, и как можно чаще обтирать крышку чистым полотенцем, чтоб вода с крышки не капала в рис. 

Даже можно накрывать чистой бумагой, а сверху крышкой, и на нее насыпать каленые уголья. (0,5 ф. риса на 6 персон). 

Рис с маслом поставить за час до обеда. 

Телятину отварить немного, потом нарезать кусочками, положить в кастрюлю масла 1/8 фун., лук нарезанный, миндаль и, когда ужарится, налить 2 столовых ложки бульону; когда телятина станет немного краснеть, положить коринку и кишмиша и дать всему вместе жариться. 

К столу жареную телятину положить на блюдо, а сверху посыпать рис. Если на дне кастрюли немного рис поджарится, нужно все соскоблить и сверху насыпать и вылить оставшееся масло. Шафран у нужно самую малость положить в чайную чашку, налить столовую ложку кипятку, хорошенько растереть и сквозь сито облить. 

В этот фарш можно класть и шапталу. 

Плов можно также приготовить с укропом. Когда у риса пройдет пар, хороший пучок укропу изрубить мелко, посыпать на рис и класть в кастрюлю, в масло (тогда не надо шафрана); телятину или баранину вареную нарезать кусочками и обложить кругом. 

Плов с икрой паюсной точно также приготовить; но 1/4 фун. икры нарезать кусочками, положить, когда будете класть рис в кастрюлю в масло, пересыпая с рисом. (6--8--10, смотря по количеству). 

\z{Шпинат в яичнице}\index{Шпинат ! в яичнице}

Шпинат отобрать, выполоскать в 2-х или в 3-х водах, положить вариться в горячую воду; когда станет мягок, откинуть на решето, немного отжать, потом изрубить мелко. За полчаса до обеда, положить масло на сотейник, согреть с луком и шпинатом, немного поджарить; потом влить столовую ложку бульону, положить немного соли и все смешать. Яйца выпустить в чашку, взбить и налить на шпинат, и когда яичница немного запечется, подать на стол в сотейнике; можно выложить и в соусник, но в таком случае надо осторожно ложкой брать и класть. 

\z{Зеленые турецкие бобы в белом соусе}\index{Турецкие бобы ! в белом соусе}

Очистите стручки и сварите в соленой воде, бросив стручки в воду в то однако время, когда она сильно кипит. Как сварились, выложите на сито, чтоб стекла вода, и тогда положите в кастрюлю и прибавьте, по пропорции, чухонского свежего масла, соли, перцу в порошке, крошечку мелких луковиц или шарлоток, и когда бобы прожарятся на огне порядком, снимите кастрюлю с огня, вбейте туда несколько яичных желтков, влейте немного лимонного соку или хорошего винного уксусу; приставьте кастрюлю на минуту к огню, чтоб заварилось, мешайте ложкой, чтоб яйца не сселись, и отпускайте очень горячее. 

\z{Щука с картофелем}\index{Щука ! с картофелем}

Это блюдо требует разнообразных приготовлений и манипуляций. Большую щуку, от 5 до 7 фунтов, чистят и вынимают жабры из-под жаберных крышечек. Дав пролежать рыбе несколько часов, ее варят в воде с солью и кореньями, и потом, дав остынуть, вынимают. Следующая за тем трудная задача состоит в том, чтобы снять всю шелуху и мясо так, чтоб костяк остался совершенно неповрежденным. Очистив снятое мясо от всех косточек и приставшей к нему шелухи, снимают листочками из лучшего мяса столько, чтоб составило третью часть всего мяса, остальное рубят мелко и смешивают с таким же количеством мелко истертого картофеля, отваренного для этого накануне и перед тем очищенного от шелухи и истертого на терке. Потом смешивают полфунта масла со сливками, прибавляют 6 или 7 яичных желтков, 8 лотов истертого пармезана, 8 лотов истертой булки, нужное количество соли и хорошенько мешают с картофелем и рыбой и с половиной взбитого яичного белка. 

Теперь начинается самое важное: образование рыбы, так чтобы помощью этой смеси и рыбьего костяка передать как можно сходнее первоначальный вид щуки, столь же гладко и правильно; потом чайной ложкой накладывают шелуху; эту новую щуку приготовляют на длинном блюде, выдерживающем жар, т. е. из огнеупорной глины, и намазанном предварительно слоем масла. Посыпав щуку пармезанным сыром и булкой, и облив раковым или другим маслом, за час до подавания на стол, жарят ее, причем голову и хвост закрывают широкими пластами шпека и бумаги, чтоб укрыть эти части от излишнего жара. 

Обтерев края блюда и вложив рыбе поперек в рот отдельно сваренную печенку, ее обкладывают подробно описанным ниже жареным картофелем и подают на стол с следующим соусом: смешайте пол- фунта свежего масла с тремя или четырьмя столовыми ложками муки, прибавьте немного щучьего и говяжьего бульону, немного белого вина и уксусу эстрагона, варите все это на слабом огне, из чего составится довольно густой масляный соус, прибавьте 6 или 7 яичных желтков, от 4 до 6 столовых ложек каперсов, тонко изрезанную сердцевину лимона, немного мелко изрезанной лимонной корки, немного сахару и мускатного ореха, и смешайте с мелко исщипанным щучьим мясом. Проварив это рагу еще раз на слабом огне, должно попробовать его, чтоб решить, д о- вольно ли оно кисло и солоно, и вообще вкусно ли. 

На гарнитуру выбирают продолговатый, сколько можно ровный картофель; при чистке, всем картофелинам дают величину в 2 вершка длиной и 1,25 вершка толщиной, а по концам изрезывают их, потом переваривают в воде, поджаривают в масле, со всех сторон одинаково, и жидким светлым гласиром придают блестящий вид. Этим картофелем гарнируют рыбу по бокам. (8--10) 

\z{Рубцы в темном соусе}\index{Рубцы ! в темном соусе}

Хорошо вымытые и выполосканные рубцы сварите в соленой воде с петрушкой и изрежьте на тоненькие продолговатые пластинки. Теперь поджарьте в масле 2 или 3 ложки пшеничной муки, прибавьте луку, а еще лучше шарлоток, и пожарьте с этой мукой, подлейте немного воды, в которой варились рубцы, или рубцового бульону, малое количество уксусу, мелко истолченного английского перцу, русского перцу в зернах, соли, и пусть в этом соусе рубцы поварятся, чтоб соус вскипел раза три. (1 рубец на 2--3 персоны). 

\z{Матлот из угря}\index{Матлот ! из угря}

На бульоне, из какой угодно мелкой рыбы, заправить соус некоторым количеством чухонского масла и муки, которую поджигать, чтобы дать бульону красноватый цвет; опустив в него изрезанного угря, приправить петрушкой, сельдереем, луком, солью, перцем, шарлотками, предварительно слегка отваренными, и достаточным количеством шампиньонов, отвар которых, для вкуса, вливается также в этот соус. Кроме того, налить красного вина, и таким образом варить угря на очень сильном огне, до полной спелости. Подавая на стол, к этому истинно отлично вкусному кушанью присоединяют гренки из белого хлеба, слегка зарумяненные в масле. (Скоромное). (Смотря по рыбе и ее величине). 

\z{Котлеты из селедки}\index{Котлеты ! из селедки}

Три или четыре шотландских селедки должно намочить с вечера в молоке, а утром, дав им хорошенько стечь, очистить от костей и изрубить как можно мельче. К ним прибавляют два сырых яйца, несколько ломтиков белого хлеба, намоченного в молоке, как это делают для начинки цыплят, мелко изрубленного укропу и ложку самой густой сметаны. Все это мешают как можно лучше и, разделив на ровные катышки, заправляют котлеты, которые обмазывают сырым яйцом и валяют в сухарях. Жарят их в русском, а не в чухонском масле и подают к тертому картофелю. Котлеты эти чрезвычайно нежны и присутствие селедки на вкус вовсе не заметно. (От 6--8) 

\z{Вкусная кислая жареная капуста}\index{Капуста кислая ! жареная}

Сварить весьма густо кислую капусту в рыбном пюре, т. е. в рыбной ухе, в которой вся рыба мелко истерта, прибавив луку и перцу, и потом поджарить эту капусту с грибами в прованском или другом постном масле. Сушеные грибы должны быть сперва хорошо вымочены, чтоб были мягки. Изжаренную, таким образом, капусту уложить на фаянсовое блюдо и поставить в печь, в вольный дух, чтоб она припрела и зарумянилась сверху. Потом нарезать ломтиков белого или ситного хлеба, намазать их, в палец толщиной, рыбным фаршем, т. е. вареной и мелко истертой рыбой с луком и перцем, с примесью рыбного клея, чтоб смесь держалась плотно, и изжарить эти ломтики с фаршем на сковороде в прованском масле. Далее обложить этими ломтиками капусту и подавать на стол. Можно подавать и отдельно ломтики с фаршем. Разумеется, что это кушанье гораздо вкуснее, когда оно скоромное. В таком случае, капусту должно варить в бульоне и жарить в ветчинном сале, а в фарш, вместо рыбьего клея, вбить яичные желтки. (Скоромное и постное). (На 5--6 перс., или и более, считая по две столовые ложки капусты на одного человека, остальное пропорционально). 

\z{Харисса (индейская по-грузински)}\index{Индейка ! харисса}

Вычистив и вымыв индейку, поставить вариться, налив 5 бутылками холодной воды. Когда будет закипать, снимать почаще пену. Затем индейку разрезать по частям и варить до тех пор, пока мясо совершенно отстанет от костей. На другой день бульон процедить, кости все отобрать, а мясо индейки поставить опять вариться, но уже с рисом, который, однако был слегка поджарен, истолчен в ступке и провеян. Когда рис будет соединен с курицей, то смешать с ними лук, мелко изрезанный и поджаренный на масле, потом варить до тех пор, пока мягкое мясо индейки и рис не составит одной массы, имеющей вид как бы каши-размазни. К столу блюдо это подается в кастрюле, обвернутой салфеткой. К хариссе подается согретое чухонское масло. (6--8) 

\z{Пудинг из жареного каплуна}\index{Пудинг ! из жареного каплуна}

Четверть фунта истолченного миндалю замесите на 3 целых яйцах и 3 желтках, с прибавкой 1/8 ф. сахару, и взбивайте 1/4 часа; между тем распустите 1/4 ф. коровьего масла и, когда простынет, смешайте с миндалем; потом обрежьте мясо с 1 жареного каплуна, искрошите кусочками и, перемешав с 1/4 ф. тертого белого хлеба и ложкой муки положите в миндаль. Разварив 0,5 ф. сорочинского пшена в бутылке молока, простудите, положите туда же миндаль и все хорошенько перемешайте. Эту смесь положите в салфетку, вымазанную коровьим маслом, завяжите и варите в сильном кипятке 1,5. часа. Потом, выложив на блюдо, облейте следующим соусом: возьмите 4 яичные желтка, чайную чашку белого виноградного вина, поставьте на огонь и взбивайте веничком, пока все не превратится в густую пену. (5)

\z{Капуста с потрохами}\index{Капуста ! с потрохами}

Вычистить потроха как можно чище, разрезать на куски, промыть хорошенько, сварить и посолить. Капусту выбрать, отрезать от нее увядшие листья, оторвать стебли, разломать сердцевину на мелкие кусочки, поставить бульон на огонь с коровьим маслом или ветчинным салом и дать капусте развариться в нем домягка. За час до того, как выкладывать на блюдо, разрезать потроха на кусочки, еще мельче первых, положить их в капусту, варить, дать увариться, прибавить сахару и ложку муки, досветла разболтанной в воде, и снять с огня, чтоб капуста не пригорела. Потроха должно разварить так, чтоб они сделались совершенно мягкими. (По одному потроху на 5 персон).

\z{Форель с анчоусами}\index{Форель ! с анчоусами}

Вычистив форель, выпотрошить и вымыть, положить в кастрюлю, влить 2 столовые ложки прованского масла и стакан бульону, прибавить мелко изрубленного зеленого луку и петрушки, лаврового листу, посолить и варить на легком огне, пока форель поспеет. Между тем приготовить соус: взять сколько нужно будет анчоусов, выбрать из них кости, изрубить, положить в кастрюлю, налить мясным или раковым кулисом, дать раз вскипеть, приправить лимонным соком. Когда рыба будет готова, выложить на блюдо, посыпать петрушкой и укропом, а соус подавать в отдельном судочке. 

\z{Форель по-голландски с маслом}\index{Форель ! по-голландски}

За 2 часа до обеда убить форель, выпотрошить, вырезать жабры и вычистить изнутри, находящуюся при спинной кости, кровь; связать голову голландскими нитками, сложить в рыбный котел с решеткой, налить водою и поставить котел на плиту покрытым. Когда вода вскипит, отставить на легкий огонь и дать вариться еще 1/4 часа. Потом вынуть с решеткой на стол, снять с рыбы верхнюю кожу, сложить рыбу осторожно на блюдо с салфеткой и обложить сваренным картофелем и зеленою петрушкою. Масло растопленное подается особо в соуснике. (6) 

\z{Лини}\index{Лини}

Очистив, выпотрошив и хорошо вымыв десяток линей, положить в кастрюлю, прибавить шинкованного луку, гвоздики цельной, перцу горошинами, лимону, нарезанного кружками, стакан виноградного вина и столько воды, чтоб можно было сварить линей. Накрыв кастрюлю, поставить на легкий огонь, дать вариться до спелости. Между тем приготовить соус: положить в кастрюлю 1/4 ф. чухонского масла, поджарить на огне, пока масло разойдется, потом прибавить в масло мелко изрубленную луковицу, 2 столовые ложки сметаны, перцу в мельчайшем порошке и 3 яичные желтка, стереть все хорошенько вместе, поставить на огонь, мешать непрерывно, не давая кипеть. Когда лини будут готовы, выложить на блюдо, облить соусом и отпустить на стол. (Пропорция на 15 чел.). 

\z{Семга}\index{Семга}

Вычистив и вымыв семгу, положить в кастрюлю, влить немного бульону, рыбного или мясного, стакан виноградного вина и 2 столовые ложки прованского масла, положить зеленого мелко изрубленного луку и петрушки, горошинами перцу и соли. Когда рыба поспеет, выложить на блюдо, процедить бульон, в котором она варилась, выдавить в него сок из 1 лимона, подбавить немного кулиса, уварить и облить им рыбу. (Скор. и постн.) 

\z{Стерлядь в белом вине}\index{Стерлядь ! в белом вине}

Вычистив, выпотрошив и вымыв стерлядь, положить в кастрюлю, прибавить понемногу кардамону, имбиря, мускатного цвета, 3 луковицы, натыканные гвоздикой, мелко искрошенной лимонной корки и кусок сахару. Вино виноградное, разведенное пополам с водой, вскипятить в особенной кастрюльке, вылить на стерлядь и варить до спелости. Взять 1 лимон, нарезать тоненькими кружочками, выбрать зерна, положить в кастрюлю, в которой варится стерлядь. Когда стерлядь будет готова, вынуть из кастрюли, уложить на блюдо, обложить кружками лимона. Бульон, в котором варилась стерлядь, процедить, уварить, чтоб его осталось немного, вылить под стерлядь. (Смотря по величине стерляди, от 5--8 перс.). 

\z{Клёцки из раков}\index{Клецки ! из раков}

20 раковых отваренных шеек и ножек мелко изрубить; из 3/4 или 1 стакана скорлупок сделать раковое масло, взять 1/3 стакана этого масла, растереть добела с 5 яйцами, положить 0,5 франц. белого хлеба, намоченного в молоке и выжатого, соли, перцового порошка немножко, зеленой петрушки и изрубленные раки, смешать; опускать ложкою в бульон. 

\z{Рыбный фарш в раковинах}\index{Фарш рыбный ! в раковинах}

Поджарить в 1,5 ложках масла 1 изрубленный порей и штук 10 шарлоток или, за неимением их, изрубленную луковицу, положить мелкими кусочками нарезанную и посоленную какую-нибудь рыбу, 2--3 отваренные и мелко изрубленные грибка; все это тушить под крышкой. Когда рыба будет готова, остудить, еще раз изрубить все вместе, смешать с 2 желтками, 0,5 стаканом сметаны, с 2 толчеными сухарями, английским перцем, солью, прибавить немного бульону, наложить этого фарша на раковины, сгладить сверху, посыпать сухарями, скропить маслом, вставить в печь на 1/4 часа. Эти пирожки подаются к супу из рыбы, и для экономии фарш этот можно с делать из сваренной в супе рыбы, поджарить ее с луком и так далее. 

\z{Бигос}\index{Бигос}

Это недорогое и вкусное, довольно экономическое блюдо делается польскими поварами и кухарками из остатков жаркого (говядины или телятины), разрезанных четырехугольными кусками, каждый величиною в вершок. Для этого приготовляется такой соус: 0,5 ложки муки поджарить с ложкою масла, развести бульоном или кипяченою водой, вскипятить, положить 2--3 соленые огурца или 3--4 кисловатых яблока, нарезанные четырехугольниками или кружками. Если соус слишком жидок, загустить его 2 ложками тертых сухарей, положить в него приготовленные мясные квадратики, вскипятить 2--3 раза и подать. (6) 

\z{Итальянская полента с птичками}\index{Полента ! с птичками}

Закипятите воду с солью в кастрюле и всыпьте в кипяток столько муки из кукурузы, сколько надобно, чтоб полента была густа, как крутая каша. Когда вода с мукою начнет снова кипеть, мешайте беспрестанно деревянной лопаткой, чтоб мука не пригорела и не свернулась в комья. Когда это тесто не будет более приставать к деревянной лопатке — это знак, что полента готова. Между тем, пока варится полента, возьмите маленьких птичек, дроздов и т. п., очистите, посолите, обверните каждую птичку ломтиком свиного сала и по 7 или 8 птичек наденьте на деревянную палочку, уложите, приготовленных таким образом, птичек в кастрюлю, в которой уже распущено чухонское масло, и жарьте, подливая несколько раз холодной воды, что составит соус. Поленту, т. е. тесто, выложите на блюдо, сделайте в середине место или гнездо и положите туда птичек с их соусом. (6--8) 

\z{Телячьи мозги}\index{Мозги ! телячьи}

Вынув мозги из их костяных чашек, опускают их в холодную воду, которую меняют раза 2 или 3; потом должно очень осторожно снять верхнюю перепонку, с которою отделятся и все остающиеся кровяные части. Варят до надлежащей спелости в соленой воде, и когда достаточно уварятся мозги, их откидывают на сито, чтобы стекли, а потом выносят в холодное место и дают остыть как можно лучше. За 0,5 часа до обеда, их разрезывают осторожно на ломтики, валяют в сырых яйцах, обсыпают со всех сторон мелко истолченными сухарями и жарят в русском масле на легком огне. Обжаренные таким образом мозги откидываются на решето, чтоб с них стекло лишнее масло, а в сковороду, где они жарились, насыпают щепотку пшеничной муки, дают прокипеть с маслом, подбавляют ложку бульону и обливают мозги этим подливочным соусом, в который для вкуса можно положить укропу. Просим заметить, что употреблять должно сухари, а не иссушенный хлеб, толочь их очень-очень мелко, и непременно обсыпать куски, а отнюдь не валять их в сухарях, как делают некоторые кухарки, отчего это нежное кушанье теряет весь свой вид и бывает облеплено какими-то лепешками.

\z{Разварные голуби}\index{Голуби ! разварные}

Кладут в кастрюлю кусок ветчинного сала произвольной величины, несколько мелко изрезанных луковиц, ложку масла, пару или две голубей и дают всему этому исподволь развариться. Потом вынимают их, подмешивают в соус некоторое количество муки, подливают бульону, уксусу, прибавляют лимонных корок, зернистого перцу, и затем все это варят. После того, процеживают соус сквозь решето, подливают сливок, кладут две ложки каперсов и, наконец, голубей. Дают вскипеть еще несколько раз и, в заключение, делают над голубями венчик из особо приготовленного теста. 

\z{Рагу из жареных голубей}\index{Рагу ! из жареных голубей}

Смешивают в кастрюле некоторое количество муки с маслом, прибавляют 1/2 штофа бульону, или 1 кв. воды и 1 кв. вина, 2 растертые луковицы, 4 рубленные сардели, лимонный сок, лимонные корки, перец, соль, мускатный орех и жареных голубей, разнятых надвое; затем, накрыв кастрюлю, оставляют все это хорошенько увариться.

\z{Голуби под соусом}\index{Голуби ! под соусом}

Ощипать пары три молодых голубей, вымыть в теплой воде и разрезать их начетверо. Смешать ложку муки с 1/4 фунта масла; налить бульону сколько нужно для того, чтобы покрыть им мясо; прибавить 4 мелко изрезанные луковицы, лимонного соку, перцу, соли и мускатного цвета; подложить разрезанных на части голубей вместе с печенкой и желудком, изрезанных мелко, и, закрыв кастрюлю, дать исподволь кипеть на малом огне. (10--12) 

Вот еще как можно приготовлять голубей. Для этого собирают вытекающую из голубей кровь и взбивают ее с двумя ложками уксусу. Голубей же, очищенных и разрезанных на 4 части, кладут в кастрюлю с бутылкой молока, полубутылкой виноградного вина, тертым хлебом, лимонным соком, перцем, солью, мускатным цветом, 2 гвоздиками, лавровым листом и столовой ложкой чухонского масла. Когда голуби сварятся, подливают в них кровь и немного разогревают ее: иначе она свертывается. При подаче на стол, голубей кладут на блюдо и обливают, сквозь сито, таким образом приготовленным, соусом. 

\z{Голуби с рисом}\index{Голуби ! с рисом}

Очистив голубей, разнимают их на 4 части, смешивают в кастрюле немного муки с ложкою чухонского масла, разжижают водой, кладут голубой с 4 растертыми луковицами, перцем, солью, лимонным соком, и дают исподволь кипеть. Между тем, разваривают 1 фунт рису в 1 штофе бульона, с прибавкой ложки масла, соли, мускатного ореха, 2 гвоздики, горсти натертого голландского или швейцарского сыра, или еще лучше пармезана. Обкладывают рисом края продолговатого блюда, кладут в средину разнятых на части голубей и покрывают их остальным рисом. Все это обмазывают взбитым яйцом, посыпают тертым сыром и пекут в печи, которая, заметим, не должна быть чересчур жарко натоплена. (1/4 голубя на персону). 

\z{Фаршированные голуби}\index{Голуби ! фаршированные}

Можно вот и так жарить голубей: свернув головки, общипывают их, потрошат, вымывают и приготовляют также, как кур. Делают начинку из тертого белого хлеба, 2 яйца, сливок, прованского масла, сахару, мускатного цвета и коринки. Начиняют голубей, обкладывают и х ломтиками шпека, обливают соленой водой и жарят в масле вместе с этим фаршем. 

\z{Телячий ливер под соусом}\index{Телячий ливер ! под соусом}

Отварив ливер в соленой воде до мягкости, вынимают его из кастрюли, откидывают на дуршлаг и, когда вода достаточно с него стечет, рубят очень мелко на деревянной доске, как это делается для начинки суповых пирожков; потом, дав в масле упариться репчатому луку, нарезанному ломтиками, подсыпают с ложку крупитчатой муки, кипятят, чтоб отбить мучной вкус, и, выложив в соус мясо вместе с мелко изрубленными яйцами, разбавляют бульоном до надлежащей густоты. Для вкуса кладут немного перцу и вливают ложку уксусу. С этим соусом подают жареную телячью печенку, нарезанную ломтиками и обсыпанную сухарями, или гренки из белого хлеба, только не смоченные молоком, как это делается для сахарного горошка, зеленых бобов и пр., а просто упаренные в чухонском масле, в кастрюле, и когда ломтики достаточно напитаются и, так-сказать, промокнуть в масле, должно их слегка подрумянить на сковороде. (6--8) 

\z{Окорок ветчины}\index{Окорок ! ветчины}

Ветчину варят и запекают в ржаном тесте. Если ветчина солона, то, налив ее теплою водою, дать мокнуть часа два, хорошенько вымыть, оскрести ножом, положить в удобную посудину, налить квасом и оставить мокнуть сутки, потом замесить из ржаной муки на воде крутое тесто, раскатать толщиною в палец; окорок, вынув из кваса, вымыть в холодной воде, отереть полотенцем, положив на раскатанное тесто, защипать, чтоб окорока нигде не было видно, и, уложив на противень, поставить в печь. Смотря по величине окорока и по тому, как жарко натоплена печь, надобно запекать окорок часа два или три; вынувши из печи, обломать тесто, оскрести ножом, снять кожу. Должно наблюдать, чтоб окорок поспел, но не перепарился. К ветчине обыкновенно подают горчицу, уксус или тертый хрен. Также подают ее с соусом из зеленого горошка и с протертым картофелем. 

\z{Окорок свежей свинины}\index{Окорок ! свежей ветчины}

Окорок свежей свинины вымыть, кожу надрезать в клетку, посолить и, кому угодно, нашинковать чесноком; уложив на противень, подлить воды, поставить в печь; дав зарумяниться с одной стороны, переворотить на другую. Вынув из печи, снять кожу, а некоторые не снимают кожи, а подают как она есть, с кожей. Уложив буженину на блюдо, облить соком, в котором она жарилась, сняв с него жир. К свежей свинине подают толченый чеснок, смешанный со сметаной, горчицу, уксус или тертый хрен. Можно подавать буженину под соусом, приготовленным таким манером: нашинковав несколько луковиц, обжарить в масле, прибавить ложку муки, поджаренной докрасна в масле, развести бульоном, подцветить подожженным сахаром, выжать сок из одного лимона.

\z{Язык копченый}\index{Язык ! копченый}

Сварив мягко копченый язык, ободрать с него кожу, изрезать ломтиками, уложить на блюдо. Потом приготовить следующий соус: нарезав ломтиками репы, посыпать мелким сахаром, обжарить в коровьем масле, посыпать мукой, налить хорошим бульоном, уварить. Когда соус будет готов, выложить на блюдо к языку. Копченые языки подают с соусом из зеленого горошка; подают также без всяких приправ с горчицею, уксусом и хреном. 

\z{Солонина}\index{Солонина}

Сварив солонину в воде, оскрести ножом; подавать горячую с хреном, смешанным со сметаною, или с горчицей и уксусом. 

\z{Няня}\index{Няня}

Взять баранью голову с ногами, налить в горшке немного водою и упарить. После этого мясо с костей обобрать, положить в чашку и изрубить с луком. Кашу крутую заварить особо; когда поспеет, намаслить ее, и смешав с изрубленным мясом, начинить этим бараний вычищенный и вымытый сычуг; туда же положить мозг из головы и глаз, зашить сычуг, положить в горшок и, накрыв, поставить в печь, чтоб уварилось хорошенько. (10--12) 

\z{Фрикасе из телятины}\index{Фрикасе ! из телятины}

Взять жирную грудинку, разрубить ее на куски средней величины, забланшировать, положить с куском коровьего масла в кастрюлю, завязать в чистую тряпочку луковицу, несколько лаврового листу, гвоздики, майорану, богородской травы, и положить туда вместе с солью, налить кипящей воды, накрыть и варить. Туда можно класть очищенные сморчки, а также клёцки, особенно сваренные в воде. Перед тем, как подавать на стол, прибавить несколько яичных желтков, с мешанных в воде или в уксусе с 0,5 ложкой муки, налить бульоном из-под мяса, приставить к огню, но только не слишком близко; наконец выложить мясо и облить соусом. (8 — 10) 

\z{Глассированная баранья нога с огурцами}\index{Баранья нога ! глассированная с огурцами}

Взять заднюю ногу баранины, приготовить в брезе\footnote{См. Брез.}, загласировать. Потом взять свежих огурцов, очистить, вырезать семена, разрезать каждый огурец на четыре части, смотря по величине огурцов, обделать на подобие маленьких огурцов, посолить, облить уксусом и, дав полежать часа два, откинуть на сито, чтоб стек уксус. Наконец взять кулису, стереть на сахар с лимона цедру; положить огурцы и цедру в кулис, варить до спелости; когда будут готовы, выжать в огурцы сок из одного лимона. Выложив соус на блюдо, положить на него баранину. (8-- 10) 

\z{Ветчина по обыкновенному в винном соусе}\index{Ветчина ! в винном соусе}

Окорок ветчины сварить, очистить, снять кожу, облить следующим соусом: четверть фунта коровьего масла стереть с ложкою муки, прибавить шесть яичных желтков и стакан бульону, дать вскипеть раза два ключом, беспрерывно мешая, облить этим соусом окорок, обсыпать толчеными сухарями и пармезаном, пригладить горячим ножом, поставить в печь и дать зарумяниться. Подавать на стол с следующим соусом: полфунта ветчины нарезать мелкими кусочками; исшинковать три луковицы, несколько шарлот, по одному корню моркови, петрушки, пастернака, изрезать кружочками один лимон и шесть анчоусов, выбрав из них кости, налить кулисом, уварить хорошенько, процедить, влить в рюмку виноградного вина и дать раз вскипеть. (10--12) 

\z{Запеченная ломтиками ветчина}\index{Ветчина ! запеченная ломтиками}

Нарезать тоненькими ломтиками ветчины; срезав корку с белого хлеба, также нарезать тоненькими ломтиками, обжарить в коровьем масле. Потом взять шпику, растереть его хорошенько деревянной ложкой накрошить мелко шампиньонов и трюфелей, также по горсти зеленой петрушки укропу и луку. Приготовив все вышесказанное, взять медное или фаянсовое блюдо, которое могло бы выдержать жар; шпиком, толщиною в ножевой обух, уложить ломтиками хлеба, потом посыпать рубленными шампиньонами, трюфелями и зеленью, а на это положить ломтиками ветчины; наложив таким образом ряда три верхний слой засыпать тертым белым хлебом, поставить в печь, дать стоять часа полтора. Подавать на стол горячее. (Смотря по количеству ломтиков). 

\z{Ветчина под соусом}\index{Ветчина ! под соусом}

Нарезать от сырого окорока ветчины тоненькими ломтиками, положить в кастрюлю, посыпать перцем и толчеными сухарями, смочить красным вином, уварить до готовности. Выложив на блюдо, выдавить сок из одного лимона, посыпать зеленой рубленной петрушкой.

\z{Мясное тесто, или бархатные говяжьи котлеты}\index{Мясное тесто}

Это то, что французские кухарки называют cotelettes véloutées и что, впрочем, есть ничто иное, как усовершенствованный биток. Из 4 ф. говядины (сырой) выходит подобных котлет 25 штук. Говядина берется хороший, мягкий край. Соскоблить все мясо с костей, очистив тщательно от жилок и перепонок и истолочь как можно лучше в ступке, чтоб образовался род мягкого мясного теста, которого весом будет по крайней мере, 3 фунта. Выложить его ложкой в каменную чашку и прибавить туда же чашку мелко истолченных сухарей, чашку сливок, два сырых яйца, маленько репчатого изрубленного лука, немного перцу, 0,5 чашки растопленного чухонского масла и соли по вкусу. Все это мешать и бить ложкой до совершеннейшего соединения (компактности), и тогда, разделив тесто на 25 равных по возможности кучек выкладываемых на особенную деревянную доску, катать каждую кучку руках, потом приплющивать ее, округлить ножом, обровнять в виде котлеты, которую, обваляв в сыром яйце и обсыпав сухарями, жарить на сковороде в чухонском масле; а когда изжарите, то прибавьте в подливку ложки две хорошего бульону, дайте хорошенько вскипеть и отпускайте к столу очень горячие в одно время с каким-нибудь соусом из зелени или овощей, или в особом мясном соусе. Ежели вы желали бы уложить эти бархатные котлеты в соусник с зеленью или овощами, то в подливке надо только обмакивать каждую котлету, а самой подливки не подавать, потому что она испортит вкус зелени или овощей. Подливка эта предпочтительно употребляется, когда котлеты подаются в виде жаркого и заменяют бифштекс. Котлеты эти при вполне успешном их приготовлении, отличаются, по нежности своей, действительно какою-то бархатистостью, делающую их сколько приятным на вкус, столько питательным блюдом. Советуем сковороду предварительно нагреть и жарить на посредственном огне, поворачивая и ощупывая вилкой. Кости и остатки сырого мяса, как то: перепонки и жилки, складываются в бульон стола прислуги. 

\z{Фаршированная баранья нога}\index{Баранья нога ! фаршированная}

Выбрать кости из задней бараньей ноги, начинить фаршем, приготовленным из телятины, зашить; уложить дно кастрюли шпиком и разными кореньями, положить баранину, сверху также закрыть шпиком и кореньями, влить стакан бульону, закрыв кастрюлю, поставить на огонь и варить до готовности; вынув баранину из кастрюли, загласировать. Глас приготовляется таким образом: взять жюса и кулиса, если то и другое есть, смешать вместе, уварить густо, как сироп, прибавить ракового масла. Можно глас делать и из бульона, уварив его до густоты сиропа, и прибавить ракового масла. Загласировав баранину, поставить на блюде в печь, дать зарумяниться. Потом взять репы, очистить, нашинковать крупно. 0,25 ф. свежего коровьего масла положить в кастрюльку, прибавить столовую ложку мелкого сахару, поджарить на легком огне докрасна, положить в масло репу и обжарить румяно, посыпать немного муки, размешать, налить жюсом и уварить до спелости. Подавая на стол, положить баранину на блюдо и облить соусом. (6) 

\z{Баранина по-польски}\index{Баранина ! по-польски}

Кто любит сырой лук и может есть его, тому советуем покушать его с бараниной, как едят его в Польше. Сварите кусок хорошей баранины, лучше — ребро, в соленой воде, выньте мясо, дайте стечь воде и подавайте на стол. Нарежьте мелко луку, положите в салатник и залейте крепким уксусом. Баранину кушайте с этим луком: — она получает удивительный вкус, а лук от уксуса теряет часть своей остроты и не вызывает слез. (6) 

\z{Филейная часть говядины, приготовленная с пряностями}\index{Говядина ! с пряностями}

Нашинковать филейную часть говядины шпиком, анчоусами и трюфелями, посыпать разными пряностями, как то: перцем, мускатным орешком и гвоздикой. Уложить дно кастрюли тоненькими ломтиками шпику, положить на них говядину, сверху также закрыть шпиком, влить стакан бульону, закрыть кастрюлю и варить на легком огне до готовности. Когда говядина будет готова, вынуть, положить на блюдо, а сок процедить, снять жир, прибавить немного кулиса *) и рюмку виноградного вина, вскипятить один раз и облить говядину. (На 6--8 чел. около 8 ф. мяса). *) См. Кулис. 

\z{Филей шпигованный с гарниром шиполаты}\index{Филей ! шпигованный}

Очистить от жил (не отнимая жира) вырезной филей, нашпиговать шпиком, положить в продолговатую кастрюлю, снабдить солью, пряностями и очищенными кореньями, влить 0,5 бутылки мадеры, 1 суповую ложку бульону и немного гляса. За час до отпуска закипятить на плите и поставить в горячую печку покрытым; а когда будет вполовину готов, снять крышку, чтобы филей заколеровался, а сок выварился до соусной густоты. Перед отпуском вынуть на доску, нарезать тонкими пластами, уложить правильно на одну сторону блюда и обложить гарниром. (6) 

\z{Говядина по-голландски}\index{Говядина ! по-голландски}

Возьмите хорошей, мягкой говядины 3 ф., изрубите мелко, посолите, положите в рубленное мясо 2 яйца, 2 ложки растопленного коровьего масла, 1 луковицу, истертую на терке. Все это перемешайте хорошенько. Кочан или два белой капусты разрежьте на 4 части, положите в кастрюлю, прокипятите и откиньте на сито; когда стечет, начините капусту промеж листьев говядиной, вымажьте маслом кастрюлю, положите в нее фаршированную капусту, накройте крышкой, поставьте в печь, и дайте хорошенько упреть. Дав вполовину поспеть, что узнается чрез втыкание вилки, подлейте стакан бульону. Обращать в особенности внимание, чтобы капуста не разварилась. (6--8) 

\z{Голубцы хохлацкие}\index{Голубцы ! хохлацкие}

Возьмите мягкой говядины, немного свежего шпеку, изрубите вместе мелко, посолите и разделите приготовленную говядину на части, по стольку, как на котлеты; потом заверните каждую часть отдельно в свежий капустный лист, положите в кастрюлю, налейте бульоном, накройте крышкой и поставьте вариться на небольшом огне; когда же хорошо уварится, влейте ложку уксусу, заправьте мукой с чухонским маслом, и дайте кипеть еще с 0,5 часа. Голубцы завертывают в виноградные листья, но это не везде можно делать, и капуста хорошо заменяет их. (По одной штуке голубцов на человека). 

\z{Шмур-братен}\index{Шмур-братен}

Возьмите кусок говядины филея, произвольной величины; смешайте уксус, если он слишком крепок, пополам с водой, и если не так крепок, то воды только третью часть, и положить в уксус лаврового листа и английского перца. В приготовленном таким образом уксусе намочите говядину, дайте ей полежать дня 3 или 4, а если не нужно торопиться, даже дней 10 и более. Когда захотите готовить, выньте из уксуса, обмойте, нашпигуйте, положите в кастрюлю, поставьте жариться на плиту, положив ложки 2 коровьего масла, и почаще поворачивайте, чтобы обжарилась со всех сторон, а когда зарумянится, выньте из кастрюли и поставьте в печь дожариться. Подавать как жаркое с салатом. (Из куска филея от 5--8 ф. выйдет на 6 персон жаркое весьма достаточное). 

\z{Зразы варшавские}\index{Зразы ! варшавские}

Нарежьте кусками говядину, избейте домягка, как делают для бифштекса, истолките в тесто кусок шпеку, и смешайте с ним 2 или 3 мелко изрубленных луковицы. Приготовьте черных сухарей, истолките, просейте и каждый кусок говядины намажьте шпеком с луком и обсыпьте сухарями. Кастрюлю вымажьте маслом, уложите приготовленную говядину, закройте крышкой и поставьте в печь часа на 2 чтобы говядина хорошо ужарилась. (Кусок мяса в 5--8 ф. на 6--8 персон). 

\z{Курник малороссийский}\index{Курник ! малороссийский}

Сварив курицу, разнимите ее на части; изрубите круто сваренных яиц; потом возьмите сдобного кислого теста, или пресного, натертого маслом, намните круто на муке и раскатайте толщиной в половину пальца, положите в глиняное глубокое блюдо или в сковороду насыпьте яиц, положите сверху курицу, засыпьте яйцами и защипите пирог; потом дайте ему круглую форму и уже защипите фигурно. Кладут в курник, яйца, смешанные с говядиной; но гораздо лучше делать курник с цыплятами. Также делают его из слоеного теста. 

Но вот этот же курник другим образом: изрубив мелко грибы, смешайте с яйцами, положите масла, посолите и засыпьте курицу или цыплят, а потом поступайте как сказано выше, обмажьте пирог яйцами и обсыпьте сухарями. (6) 

\z{Говяжий ссек, приготовленный по-английски}\index{Говядина ! по-английски}

Взять часть говяжьего ссека, положить в кастрюлю или горшок, прибавить шинкованных кореньев, 2 корня моркови, 1 петрушки, 3 луковицы, разных душистых трав, несколько лавровых листков, цельной гвоздики; посолить, влить бульону или воды, варить на малом огне пока бульон весь почти выкипит. В половине варенья положить начиненной капусты, приготовленной таким образом: кочан капусты варить 0,5 часа в воде, после этого положить в холодную; когда остынет выжать воду, разобрать листья, отделить их от кочерыжки. Между тем приготовить фарш из телятины или говядины; наложив на лист капусты тонкий слой фаршу, накрыть другим листом, на этот лист опять наложить фаршу, закрыть еще листом, свернуть в трубочку, обвязать нитками. Продолжать таким манером укладывать, пока начинится весь кочан; тогда положить капусту в кастрюлю к говядине; если мало сока, влить стакан бульону и рюмку виноградного вина, поварить еще 0,5 часа. Выложив говядину на блюдо, обложить капустой, сняв с нее нитки, а сок процедить сквозь сито и облить им говядину. 

\z{Говядина в брезе}\index{Говядина ! в брезе}

Влить в неглубокую кастрюлю стакан хорошего бульону, прибавить шинкованной петрушки и луку, немного каперсов, горошками перцу, анчоусов, посолить; на этот слой положить говядину, покрыть тою же приправою говядину сверху, закрыть кастрюлю и варить на слабом огне 0,5 часа. (6—8) 

\z{Говядина в миротоне}\index{Говядина ! миротоне}

Когда есть вареная говядина, нарезать ее небольшими кусочками, разрезывая поперек волокон; искрошить несколько луковиц кружочками, смотря по количеству говядины, поджарить их в коровьем масле, положить немного муки, вымешать, налить бульоном, посолить, приправить перцем и уксусом, дать немного покипеть. Потом положить в лук говядину; варить 1/2 часа на легком огне. (Пропорция: 1 ф. говядины на каждого человека\footnote{Пропорция такая вовсе не слишком значительна, так как надо иметь в виду, сколько теряется массы веса при варке и жарении мяса всякого рода. Руководствоваться этою пропорцией считаем безопасным и вполне верным.}. 

\z{Говядина в уксусном соусе}\index{Говядина ! в уксусном соусе}

Взяв хорошую часть говядины, посыпать солью и пряностями, налить уксусом, дать лежать в холодном месте дня 3, поворачивая каждый день. Потом, вынув из уксуса, вымыть холодной водой, положить в кастрюлю, обжарить в говяжьем сале до спелости. Уксус, в котором мокла говядина, поставить на огонь, положить в него понемногу горошинками перцу, гвоздики, лимон, нарезанный кружочками, и столовую ложку сахару. Когда хорошо прокипит, подправить подпальной мукой. Выложив говядину на блюдо, облить соусом. (6--8 перс. от 5--8 ф.) 

\z{Говядина, вареная в виноградном вине}\index{Говядина ! в виноградном вине}

Взять часть говядины, фунтов в 6, от ссека или филейную, посыпать солью, толченою гвоздикою, кардамоном и чуть-чуть мускатным цветом; положить в удобную посудину, прибавить лаврового листу, влить бутылку виноградного вина, поставить в холодное место дня на 2, поворачивая говядину каждый день по нескольку раз. Потом, сложив говядину в кастрюлю, вместе с пряностями и вином, прибавить бульона, уварить до готовности, изрезать тоненькими кружочками пол-лимона, опустить в кастрюлю, подправить немного подпальной\footnote{См. Подпальная мука или подпалка.} мукой. Отпуская, выложить говядину на блюдо, обложить лимонными кружочками и облить сквозь сито соком, в котором варилась говядина. (6--8)