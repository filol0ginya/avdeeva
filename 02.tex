\section{БЕЗОШИБОЧНАЯ ПОКУПКА ВСЕХ СЪЕСТНЫХ ПРИПАСОВ ДЛЯ ЭКОНОМИЧЕСКОГО И С ТЕМ ВМЕСТЕ ХОРОШЕГО СЕМЕЙНОГО СТОЛА} % отдел 2

\subsection{Говядина.}
При покупке свежее бычачье мясо не трудно отличить от испортившегося\footnote{Указания и наставления эти, впервые являющиеся в русской домоводственной книге,~--- заимствованы из столичной петербургской торговли; но они могут быть легко применяемы к покупке съестных припасов и на провинциальных рынках.}. Хорошее свежее бычачье мясо отличается особым мясным запахом и ярко-красным цветом. Отрезанный кусок такого мяса имеет настоящий красный цвет и как бы мраморный твердый вид, причем прорез указывает на достаточное количество жира, находящегося между мясными волокнами. Самые волокна говядины должны быть нежны и сочны (обилие сока в мясе составляет существенное его достоинство). Вообще лучшее бычачье мясо доставляют волы, т.~е. кладеные быки. К этому следует прибавить, что не все части быка дают мясо одинакового достоинства: лучшие куски, которые и в торговле ценятся дороже, получаются из хвостовой и бедренной частей, из переднего ребра и~т.~д., что все подробно описано у нас здесь в сортах говядины, согласно с рисунками, на которых ясно и отчетливо обозначены все части (см. Отд. I). При этом следует также иметь в виду, что мясо старых и изнуренных животных вообще бывает жестко, сухо, маложирно; а мясо животных слишком молодых тоже не жирно и бледноватого цвета. Иногда продажное бычачье мясо, обыкновенно давно лежалое, смазывается свежею кровью, с целью скрыть от покупателя синевато-красный цвет такой говядины; но зоркий глаз покупателя может легко подметить этот обман по наружному виду такого прикрашенного мяса.

Известно, что самое лучшее бычачье мясо в петербургской мясной торговле есть черкасское. Это не подлежит никакому сомнению, хотя в последнее время, в эти годы, с 1871, в мясных лавках продавцы стали продавать много простой русской говядины от собственно русского скота из северных губерний. Но это последнее на вид светлее и имеет более волокон и менее осмазонных частей, почему его следует избегать для ростбифов, бифстексов и хороших бульонов. Впрочем, в торговле есть еще род мяса, довольно распространенный, который вкусом не хуже черкасского. но цветом гораздо бледнее,~--- это мясо, известное под названием яловицы. Оно получается от заяловевших холмогорских коров, хорошо кормленных, но негодных для дачи молока. Этого сорта говядины, повторяем, все-таки не дурной, в Петербурге развелось нынче не мало. Яловичное мясо хотя и бледнее, однако несравненно нежнее воловьего. к тому же и разрез каждой отдельной его части несколько меньшего размера против воловьего, например, <<вырезка>> из коровьего <<филея>> значительно менее вырезки воловьей, и поэтому куски бифстекса, само собою разумеется, будут миниатюрнее, хотя, повторяем, нежностью, мягкостью и пухлостью они превосходят бифстекс из самой лучшей черкасской темноцветной говядины, столь высоко ценимой любителями и поварами.

Когда хозяйка или ее кухарка покупает вырезку отдельно от общего куска филея, платя за нее, по нынешним (конца 1873 г.) ценам, от 40 до 60 к. за фунт, то она должна наблюдать, чтобы мясник выделял эту вырезку от ребер как только можно чище, при чем необходимо требовать, чтоб лишний жир был хорошенько срезан, оставляя только небольшое количество этого жира собственно для жаренья ломтей бифстекса в их природном соку. Из фунта хорошей, честно выданной продавцом, вырезки должно непременно выйти три порядочных куска бифстекса: но оконечная часть вырезки не может дать такого количества ломтей, потому что в этом месте вырезка суживается приметным образом.

Вообще, при покупке мяса хозяйка или ее кухарка должна, как говорится, держать ухо востро, не полагаясь на добросовестность мясника, сколько бы он ни уверял в своей честности. Да даже чем больше этих умасливательных уверений, тем хуже. Необходимо покупательнице зорко следить за положением весовых чашек, на которых взвешивается, большею частью с отменною ловкостью и быстротой, купленный, какой бы то ни было кусок мяса, причем не должно мясному продавцу позволять снимать с этих чашек этот кусок до тех пор, пока стрелка весов находится еще в колебательном состоянии, потому что именно это-то преждевременное снимание товара производит и создает те фантастические, мнимые осьмушки и полуосьмушки, которые в общем месячном итоге, при более или менее изрядном заборе мяса в лавке, составляют фунты, так непроизводительно, накладно отражающиеся в бюджете экономной и рачительной хозяйки. Кроме того, обратите ваше внимание на то, что мясные продавцы при вырубании некоторых частей, как, например, <<ссека>>, <<лопатки>> и <<края>>, умеют мастерски пользоваться неопытностью или невнимательностью своей покупательницы, употребляя хорошо известные им приемы, крайне невыгодные для покупательницы. Так, например, при отпуске 6 или 7 фунтов <<ссека>>, они отрубают всю кость, принадлежащую к этому куску, а, между тем, эта кость отдельно весит фунта два, так что, если отделить жир, сухие жилки и с толстого конца небольшую плоскую косточку, собственно мяса останется очень мало, что, само собою разумеется, вовсе не составляет расчета хозяйки, тем более, что в этой огромной, продолговатой кости заключается чрезвычайно мало мозга, так высоко и справедливо ценимого при варке бульона. Вот почему должно требовать, чтобы продавец мяса, вырубая кусок ссека в 6 или 7 фунт., разрубал кость как можно более вкось, оставляя у закупленного куска меньшую часть ее, и даже отделял бы сухую жилу. Этим настоятельным требованием хозяйка выгадает наверно фунт чистого мяса.

Покупая для щей или борща несколько фунтов краю, должно настоятельно требовать, чтобы мясник отрубал от ребер их обнаженные тонкие оконечности, так как они составляют вес, не принося ни малейшей пользы.

Но вот лопатка,~--- это именно тот сорт говядины, на котором преимущественно выигрывают мясники и теряют неопытные хозяйки. Сквозь лопатку, составляющую саму по себе кусок фунт. в 15, проходит кость, круглая, толстая, плоская и без всякого мозга сначала, но кость такая, которая далее превращается в прямую, выдолбленную, довольно длинную трубку, наполненную сочным мозгом, составляющим драгоценность бульона. Если покупательница желаете купить фунт. 5 лопатки, мясник норовит расположить свой кусок так, чтобы наставить нож сначала, т.~е. с толстого края кости, которую он перерубает и тогда очень обязательно отделяет долевой кусок мяса, ловко поворачивает его перед глазами покупательницы тою стороною, где был наставлен нож и где начинается пустота, которая далее составляет резервуар мозга. Разумеется, что крохотная толика этого костяного мозга появляется в углублении; но это количество так мало, а остальная часть кости так бесполезно тяжела, что в ущерб для кармана покупательницы набегает непременно фунт или полтора бесполезного веса. Вот почему должно требовать, чтоб мясник, оставив толстую, круглую кость в стороне, непременно вырубал бы из середины лопатки, и тогда обозначится сквозная, прямая, трубкообразная кость, в которой с обеих сторон будете виден беловато-розовый мозг. Приступая к варке бульона, эту кость отделяют от мяса, обвертывают кисеей, завязывают ниткой, чтобы при процессе кипячения мозг этот не вывалился из своего костяного ложа. Многие любят им лакомиться перед обедом, посыпая солью и закусывая хлебом.

\subsection{Телятина.}
Лучший теленок, для употребления в пищу, бывает шестинедельный или двухмесячный. Мясо его должно быть сочно и непременно белеть во время жаренья. Оно студенисто, сытно, равно полезно в пищу для людей всех организмов, хотя физиология доказывает, что оно заключает в себе питательных частей менее против бычьего мяса, т.~е. говядины.~--- Задней четверти теленка отдается преимущество; за ней следуют: часть почечная, т е. между кострецом и котлетами, котлеты, грудинка, лопатка, головка, мозги, ножки и печенка.

Телятина бывает в продаже двух сортов, т.~е. мясо телят, откормленных исключительно молоком~--- первый сорт и телят обыкновенных, кормленных без особенного тщания, но все-таки довольно жирных и мясистых~--- второй сорт.

Вес крупного откормленного теленка простирается до 6 пудов всей туши. В торговле существует довольно обыкновенный обман, состоящий в надувании цельного теленка для умножения его веса. Обман этот, впрочем, довольно легко узнается. Стоит только слегка прижать пальцем кожу у нижней части телячьей тушки: если кожа плотно натянута на естественном мясе, она не поддастся давлению, если же палец образует впадину, то значит, что теленок искусственно надут. Но главное достоинство теленка составляет белизна его мяса и жир, который должен закрывать толстым пластом во внутренности почки и внутренний филей. Иногда, и даже довольно часто, мясо теленка бывает недостаточно бело, а красновато чему причиною обыкновенно неудачный убой этого теленка, причем не было дано достаточно вытечь крови, а допустили ее разлиться по всему мясу и как бы напитать собою это мясо. Умелый мясник режет теленка с отменною быстротою, моментально перерезывая ему острым ножом в известном месте горло, и тогда кровь не разливается по мясу, которое бывает вполне бело. Еще случается покупать телятину с мясом жестковатым. Это происходит от того, что продавец продал телятину чересчур скоро после убоя животного, когда мясо еще <<не провяло>>, как говорится, и оказывается <<парным>>. Такую слишком <<парную>> телятину опытные хозяйки или обегают, или оставляют на несколько времени, т.~е. часов на 10–12, на леднике, ежели, конечно, имеют ледник, что в Петербурге в небольших хозяйствах, к сожалению, не есть дело обыкновенное.

Лучшая телятина привозится в Петербург с ноября месяца до весны.~--- Телята привозятся в Петербург из его окрестностей, в особенности из Новоладожского уезда, где есть целые волости, занимающиеся отпаиванием телят. Привозят телят также из Новгородской губернии, преимущественно из Тихвина, и из-под Выборга; но выборгские телята мелки и тощи, вообще плохи. Телята доставляются обыкновенно прасолами, которые скупают их по деревням в розницу у крестьян, везут партиями в Петербург и продают на площадке мясникам по несколько штук за раз, но не с весу, а на глаз и на ощупь.~--- Мясники бьют телят на <<шпарнях>> (бойнях в рынках, устраиваемых преимущественно в подвальных этажах под лавками), с платою по 10 к. со штуки за убой.

\subsection{Баранина.}
Баранина бывает русская и шлёнская. Первая пользуется значительным преимуществом перед второю, потому что она не имеет того неприятного, своеобразного запаха, которым снабжена последняя. Узнать русскую довольно трудно, и поневоле хозяйки должны доверяться добросовестности знакомого мясника, почему нельзя советовать покупать это мясо на Сенном рынке, напр. увлекаясь дешевизной. Впрочем, осязание может несколько руководить выбором баранины, потому что кожа шлёнского барана хотя и мягче, но как будто слегка намазана мылом, a русский баран имеет кожу грубее, суше, но жир у него белее и плотнее. Мясники также избегают показывать барана в шкуре,~--- равно как и теленка,~--- потому что шкура шлёнского барана имеет неприятный псиный запах.

\subsection{Солонина.}
Хорошая солонина имеет черно-красный цвет и все другие наружные признаки свежего мяса. Такая солонина должна сохраняться в бочонке, наполненном рассолом до краев. Солонина грязновато-белого цвета, не сочная, слишком мягкая и издающая неприятный запах, должна быть избегаема, тем более, что бывают такие случаи, что попортившаяся солонина оказывается ядовитой.

Вообще нельзя слишком советовать употреблять покупную солонину, тем более, что изготовление ее домашним способом вовсе не затруднительно, как это можно видеть из наших, здесь в этой книге в Отделе <<Кладовой>> предложенных рецептов, а солонина скороспелка, заимствованная нами из бывшего в 1861 году домоводственного журнала: <<Русская Хозяйка>>, есть истинно гастрономическое лакомство, доступное для всякого хозяйства.

\subsection{Свинина.}
Свинина, получаемая от хорошо выкормленных, не старых свиней, отличается белым цветом, нежностью и не слишком обильным, довольно умеренным, пропорциональным жиром, равно как весьма нежною кожею. Окорок, или копченая свинина, должен иметь в отрезе ломтя ярко-красный цвет; волокна мяса, при всей своей нежности, не должны разлезаться и разрываться, a непременно иметь компактность. Жир в дурном окороке обыкновенно бывает мягок и даже как бы мажется; волокна же мяса представляются в разрезе зеленовато-желтоватыми и грубыми. Следует быть очень осторожным при покупке различных колбас и мясных фаршей, так как все эти товары приготовляются, к сожалению, крайне небрежно, из разных мясных остатков, иногда значительно попортившихся. Преимущественно надо остерегаться, так называемых, <<кровяных>> и <<ливерных>> колбас. Долго лежалые и потому испорченные колбасы, как известно, даже ядовиты. Отличить такие колбасы не трудно, так как они мягки, как каша, местами окрашены ярко-красным цветом с зелеными и белыми прожилками. Когда желают узнать, не приготовлена ли колбаса из попорченного мяса, до покупки, с согласия продавца, обливают колбасный фарш водою, кипятят его и затем прибавляют в эту смесь известковой воды (ее всегда легко достать без рецепта в аптеке); если колбаса сделана была из попорченного мяса, то, при обливании помянутой смеси известковой водой, из нее выделится особый газ, по запаху сходный с тем, который выделяется из отхожих мест.~--- Впрочем, как ни положительно верно это средство, лучше обходиться без приложения его к делу, на какой конец необходимо должно брать все свиные продукты и различные из них приготовление в хороших колбасных лавках, приобретших заслуженную репутацию и громко известных в столице, где они привлекают глаз покупателя превосходными своими эталажами в витринах. К сожалению, все это немецкие фирмы; что же касается до наших любезных соотечественников, русских колбасников, то к ним обращаться невозможно с той же доверчивостью.

\subsection{Живность.}
Торговля <<живностью>>, т.~е. дворовой (домашней) птицей, вмещает в себе: кур, петухов, цыплят, самоклёвов, каплунов, пулярдок, индеек, колкунов, гусей, уток и, в весьма ограниченном количестве, домашних голубей, впрочем, самых молоденьких, употребляемых почти исключительно иностранцами, всего больше французами. Оттого в куриной торговле, или в торговле живностью, ввелось и самое название этой птицы, переделанное с французского языка: пижоноты (pиgeonneau), т.~е. молоденький голубь.

Поименованная здесь птица всех наименований идет в продаже как живая, т.~е. покупаемая живою и убиваемая продавцом, большею частью, в присутствии покупательницы, или так, преимущественно, в битом виде, т.~е. так называемая <<битая>>, сохраняемая в ледниках. Такая птица, сохраняемая в превосходных кладовых куриного ряда Мариинского рынка (бывшая Щукина двора), в Чернышевом переулке, представляет собою удивительный консерв, похожий твердостью на мрамор, и купленную там именно такую замороженную, закристаллизированную живность можно употреблять совершенно безопасно: когда она у вас в кухне оттает, то, конечно, никто не отличит ее мяса от мяса только что сейчас заколотой птицы. Но, однако далеко не во всяком леднике, не имеющем специального применения для сохранения живности, может быть сохраняема с успешными вполне результатами битая птица. В дурном леднике мясо птицы принимает тот затхлый вкус, который обегают истинно сведущие покупательницы, не допускающие в хозяйстве своем никакой провизии с мало-мальским несвойственным ей запахом. Впрочем, вкус и запах эти только не страшны кухмистерской кухне, богатой искусственными, почти химическими, средствами для окрашивания, маскирования вида и изменения вкуса съестного товара.~--- Замороженные в течение зимы в мариинских погребах консервы всякой, разумеется, стоящей такого сохранения, живности могут вполне состязаться с самой свежей живностью.

К числу обманов в торговле живностью принадлежит <<надуванье>>, которое и составляет настоящее <<надувательство>>. Дело в том, что мелочник-продавец, преимущественно торгующий не в лавке, а в разнос или на Сенной,~--- этом обширном поле всяких фокусов съестной торговли,~--- купив тощую птицу, старается пустить ее в продажу <<хазовым концом>>, а для этого надувает эту птицу, т.~е. вводит внутрь нее, чрез заднее отверстие, воздух, и зашивает отверстие с некоторым искусством и маленьким фокусничеством. Таким образом, надув продаваемый предмет, назначенный в продажу малоопытному покупателю, <<надуваете>> и своего клиента~--- потребителя, который, по возвращены домой, при первых поваренных приемах, уразумевает, что им не очень-то дешево заплачены деньги за костяк. В таком виде купленная им птица к столу, мало-мальски порядочному, в виде фрикасе, жаркого или в паштете, не подается, а употребляется не иначе, как на бульон для супов, на кнели и разного рода фарши.

Со всем тем весьма многие хозяйки, а кухарки сплошь и рядом, покупают у недобросовестных продавцов дешевенькую надутую птицу, которая, чрез эту самую манипуляцию, может повредиться скорее и легче птицы, не подвергнутой вкачиванию в нее воздуха, способствующая, правда, ее округлению, но не защищающая мяса от зловредных последствий. Между тем те же неопытные покупательницы обегают ту продажную живность, которая, будучи тщательно выпотрошена от всех ее внутренностей и желудочных нечистот, имеет в пустом пространстве, образовавшемся от отсутствия внутренностей и всех потрохов, некоторое количество чистой пакли, весьма простосердечно и откровенно вводимой продавцом чрез гузку. Предрассудок, что всякая птица, набитая этим посторонним веществом, представляет собою снедь не только не вкусную, но вредную для здоровья, положительно не имеет никакого основания. Практика и продолжительная наблюдательность, опирающиеся на исследованиях точных и научных, убеждают в том, что без этой пакли, перед приготовлением птицы тщательно, разумеется, из нее вынимаемой, живность могла бы подвергнуться скорее порче и разложению, от которого именно этою самою паклей она защищена. Не говорим уже о том, что форма ее от этого делается красивее и представительнее. Впрочем, добросовестные и непричастные никакому шарлатанству торговцы обыкновенно именно так объясняют, даже это обстоятельство своим малоопытным покупательницам.

В летнюю пору в Петербурге парная живность, т.~е. только что заколотая, дорога и редка. И вот, в эту-то пору, ежели кто желает

\subsection{Дичина.}

ным. Между тем надо вам знать, что существуют дешевые сорта дикой птицы, которую можно с большим успехом накрывать силками и добывать другими способами, отнюдь не тратя на нее заряда, обходящегося выше той ценности, за какую можно птицу эту продать. Однако, смело можем уверить вас, ни одна из этих птиц, которые не убиты дробью,~--- не задавлена, так как, поверьте, задавливание могло бы совершенно испортить цвет и вид и нарушить вкус птичьего мяса, почему изловленных за раз, большею частью, в порядочном количестве, птиц убивают обыкновенно очень быстро, палкой, не подвергая их ни малейшим мучениям, могущим, повторяем, вредно отозваться на самое достоинство товара.

Рябчики,~--- это та дичь, которая в наибольшем употреблении повсеместна. Рябчики по цене своей, всего доступнее с осени до февраля месяца, хотя в курятных лавках Мариинского рынка, особенно, у известного поставщика Высочайшего Двора Никиты Федоровича Кузьмина, можно иметь, конечно за высшую цену, во все лето превосходно сохраненных во льду рябчиков. Однако недобросовестные продавцы выдумали фокус, как обманывать неопытных хозяюшек, продавая им за рябчиков, даже летом, птичку совершенно ощипанную, без перышков, которая есть ни что иное, как молоденькие, еще не умеющие летать, домашние голуби, плодящиеся в таких огромных количествах на чердаках наших столичных рынков, где на них, по найму лавочников, безжалостно охотятся уличные мальчишки, и преимущественно трубочистные ученики, великие мастера лазить по крышам. Цвет и характеристика мяса этих <<фальшивых>> рябчиков, ни дать, ни взять, как у настоящих. Обман обнаруживается при жареньи. Еще часто торговцы продают рябчиков слишком молодых, имеющих чересчур мало мясистых частей. Такой рябчик-цыпленок от совершеннолетнего рябчика, особенно, отличается тем, что у него ножки почти голенькие или мало опушенные мелким перышком. Между тем, чем старше птица, тем ножка больше обута, что называется в птичной торговле <<птица в сапогах>>. Лучшим рябчиком считается архангельский, появляющийся в Петербурге с ноября месяца. Хороши, впрочем, еще вологодские и сибирские. Из архангельских рябчиков советуем спрашивать у добросовестных торговцев тех, которые известны под названием колгачинских, отличающихся крупным ростом и необыкновенно сочным и вкусным мясом. Самыми плохими рябчиками считаются вельские, из Вологодской губернии: они всех других и мельче, и малосочнее. Еще превосходный сорт чердынских рябчиков из Пермской губернии, питающихся почти исключительно кедровыми орехами и потому называемых кедровиками. Эти рябчики выдерживают лежку долее всех других.

Дрозды дают жаркое посредственное, ни в каком случае не могущее идти на ряду с рябчиковым жарким; но под соусом бывают не дурны, и иным любителям очень по вкусу. Они <<парные>>, свежие, бывают в Петербурге только исключительно в октябре месяце. Во всю зиму можно дроздов иметь замороженных. Дрозд в особенности любит рябину, и этою ягодой почти одной питается. Если многоурожайна рябина, то и дрозда много; а как поест он всю рябину, то и отлетает дальше, где надеется на большее прокормление. Дрозда положительно никогда не стреляют, а накрывают силками и бьют палками.

Куропаток две особи. Это куропатка белая и куропатка серая. Вторая, т.~е. серая, достоинствами своими выше первой, т.~е. белой. Лучшая серая куропатка называется в лавках дончиха, т.~е. получаемая с Дона. Серая куропатка питается исключительно зерновым хлебом, предпочитая пшеницу всем другим сортам; белая же ест лесные ягоды и древесные почки. Куропаток продавцы продают со всею тою пищей, какая у птицы в зобу находилась в момент ее лова силками. Белая куропатка крупнее серой и показывается в Петербурге с 15 июля.

Тетерки и тетерева~--- птица полевая, питающаяся хлебом. Самая лучшая тетерка и тетерев сибирские, замороженные, появляющееся с половины ноября. Тетерев ростом побольше тетерки, но уступает ей вкусом. Весною тетерки~--- редкость, и в эту пору в торговле они называются орсезонами, как вероятно их прозвали французы-повара от своего слова hors-saison, т.~е. вне времени года.

Перепелка в Петербурге не в большом ходу и дорога. Лучший сорт называется курчанками, хотя получаются они не из одной Курской, а из всех смежных с ней губерний.

Глухарь и глухарка~--- не элегантное, но выгодное жаркое. Глухарка буро-серая, а глухарь синевато-черный; глухарь больше, даже гораздо больше глухарки, но мясо его грубее. Из одного глухаря выходите 8~--- 10 порций. Чем глухарь моложе, тем мясо его сочнее и приятнее.

Дикий гусь в петербургской торговле не в большом ходу и почете. Есть из них красноклювы и черноклювы. Последние известны под названием козарок и считаются вкуснее обыкновенных.

Дикая утка подразделяется на несколько сортов, по крайней мере на шесть, а именно: 1) кряква, 2) чирок (самые лучшие, самые крупные и мясистые), 3) шилохвост, 4) свиряга, 5) гогель и 6) сулок,~--- это уж сорта поплоше, и из них поплоше всех последний, т.~е. <<сулок>>, которого, однако, в продаже всего более. Мясо этого сорта дикой утки сильно отзывается рыбою, которою птица эта очень любит лакомиться. <<Чирков>> и <<крякв>> в Петербурге очень мало. Все эти птицы идут сюда наиболее с Ладожского озера, по осени.

Дупеля, бекасы и вальдшнепы носят общее название красной птицы, экземпляры которых в торговле показываются с конца июля месяца. Сезон для этого рода птичек с 15 августа и часть зимы.

Фазаны~--- блюдо чисто аристократическое и в домашнем быту недоступное, хотя все-таки и вам, хозяйка-читательница наша, на всякий случай не мешает знать, что фазанов несколько сортов: с ноября месяца показываются иностранные, прусские и австрийские, из которых последние преимущественнее первых, как крупностью, так вкусом мяса и красивостью пера. С января и по март являются наши астраханские и кавказские фазаны, замороженные, правда, не всегда вполне благополучно, так как местности, из которых везут эту птицу, не отличаются морозами. Астраханским фазанам, величаемым в торговле <<персианцами>>, уступают во всем иностранные.

Маленькие птички~--- общее название разных сортов лесной птички, употребляемой на фаршировку паштетов преимущественно, а отчасти она жарится особо, конечно, с величайшею осторожностью, и подается любителям этого жаркого пирамидою на блюде. С блюда птичек этих каждый на тарелку свою берет 5, 6 штук, употребляя только вилку, а отнюдь не касаясь до них ножом, который считается при этом вовсе ненужным. Птички эти, будучи изжарены, так малы и так нежны, что их можно, да и должно, по законам гастрономии, есть не иначе, как целиком с ножками, от которых в кухне отнимаются только цепки или ходочки, причем, впрочем, отнимаются еще и шейки, и головки. <<Маленькие птички>> общим этим собирательным названием покрывают различные мелкие породы, как: щуров, свиристелей, снегирей и воробьев, тех самых воробьев, которых мы в таком множестве встречаем везде и повсюду.

Покончив с дичиной в пере, т.~е. с птицей, обратимся к дичине в шерсти, т.~е. к лесным и полевым четвероногим. Из них прежде всего должно поговорить о зайцах, доставляющих на весьма многие семейные и домашние столы довольно хорошее и очень питательное, недорогое жаркое.

Зайцы делятся в торговле на два сорта, резко один от другого отличающиеся, а именно: а) беляк,~--- простая русская, весьма обыкновенная порода, совершенно как снег белая с зимы и пепельно-серого цвета летом, и б) русак, получаемый из северо-западных наших губерний и имеющий рубашку постоянно, и летом, и зимою, буро-сероватую; роста он очень хорошего, даже крупного и не в пример крупнее <<беляка>>. Стрелянные зайцы предпочитаются капканным по той причине, что стрелянный ест еще траву, а капканный питается исключительно древесного корою. К тому же капканный беляк бывает часто отвратительно изранен, что не мало вредит достоинству его мяса.

Оленина с весьма недавнего времени стала входить и окончательно вошла в съедобный репертуар петербургской кухни. Оленьи филеи~--- задние, так как передние части вовсе не привозятся из Архангельской губернии, откуда оленина идет к нам в замороженном виде. Это оленье мясо, надобно вам сказать, берется не от старых, бывших уже в езде оленей, каких множество петербургские жители видят зимою на Неве, а от годовиков, которые хотя и не телята, но все-таки еще пе старики. Эти годовики оставили уже молоко и ели траву и сено. Оленина продается цельными этими филеями от 4–5 р. штука, а фунтами можно иметь фунт за 12, 10, даже 8 коп. Торговый фунт оленьего мяса представляет собою больше мясного материала, чем такой же фунт черкасской говядины. В оленине нет ни костей, ни жил, ни хрящей, ни волокон, а питательность и сочность весьма удовлетворительные. Мясо это режется <<словно хлеб», как выражаются те, которые в этом свойстве этого мяса хотят видеть его, кажущиеся им, недостатки. Другие же, напротив, по этому-то самому и ценят высоко оленину, как мясо, именно режущееся подобно хлебу, а потому самому и особенно сочное. Точно такого же характера и мясо лося или лосина, которая нынче начинает очень распространяться в Петербурге. В одном из известнейших петербургских ресторанов, под фирмою <<Мильбрет», кормящем ежедневно до 700 и более человек потребителей, жаркое из лосины~--- говорят сделалось весьма обыкновенным блюдом.

Закончим этот длинный перечень сортов дичины несколькими словами о дикой козе или серне, которая в маринованном виде начинает являться на некоторых, впрочем, более роскошных и не чисто русских столах, польских преимущественно. Вес всей этой лесной скотинки от 1 п. 30 ф. со всем, т.~е. тут все части, а не как оленины~--- одни задние филеи, и все эти части с шкуркой и внутренностями, требующими немедленного выпотрошение. Цены дикой козы бывают разные: от 12 до 6 рублей штука; в морозы и в начале зимы она дороже, а при наступлении оттепелей все дешевле и дешевле, так как сохранение этих тушек с их внутренностями затруднительно и хлопотливо, почему торговцы рады-радёшеньки поскорее спустить этот неудобный товар.

\subsection{Рыба.}
Большое число постов значительно увеличивает в России потребление рыбного мяса даже и в тех местностях, где вовсе нет лова рыбы. Само собою разумеется, что живую рыбу, к какой бы породе она ни принадлежала, следует предпочитать при покупке рыбе сонной, мороженой, соленой и, наконец, копченой. В особенности необходимо избегать сонной рыбы и преимущественно в летнее время, так как она очень скоро портится, и, кроме того, неизвестно, что служит приманкой во время лова такой рыбы. Это тем более важно, что иногда рыбу ловят посредством шариков, скатанных из куколя и хлеба,~--- а куколь отравляет рыбу, почему ее употребление и бывает вредно для здоровья человека.

Более всего из рыбы, покупаемой не живою, распространено употребление сельдей, покупаемых в особенности для стола прислуги и рабочих. Поэтому не лишним считаем сделать здесь некоторые указание, при покупке этого рода рыбы. Хорошая сельдь отличается широкою жирною спинкой; мясо ее белое и нежное. Напротив того, следует избегать сельдь, если у ней мясо красноватое, шкурка легко отдирается, а глазное яблоко коричневое. Все это~--- отрицательные признаки достоинства рыбы. Голландские сельди или, скорее, шотландские,~--- каких наиболее в продаже и большею частью они идут в торговле за первые, которых очень и очень мало находится на петербургском рыбном рынке,~--- считаются лучшими, вследствие тщательного их очищение и солки. Норвежская сельдь отличается не совсем приятным вкусом, зависящего от содержание этих сельдей в сосновых бочках. Наша беломорская сельдь стоить в торговле гораздо ниже шотландской, хотя по белизне, жирности и вкусу и не уступает последней; но, к сожалению, она много теряет от неправильной ее укупорки и от крайне небрежной солки, подвергающей ее преждевременной порче. Другая рыба, имеющая в хозяйстве многообразное назначение и из которой приготовляются кушанья и не для стола прислуги только,~--- это треска или штокфиш. При покупке ее надо наблюдать, чтобы покупаемая треска имела хороший, ровный, сероватый или красно-бурый цвет, была бы без пятен и не имела бы затхлого запаху. Треска, ловленная в конце осени и зимою и оставляемая в соли до марта и апреля, далеко не так вкусна и нежна, хотя и белее мясом, чем треска весеннего посола. Треска, привозимая в Архангельск и идущая оттуда в Петербург, от небрежного соление ее в ямах, с употреблением слабой и нечистой соли, в пересоле бывает жгучею, а в недосоле~--- кислою и даже горькою, с весьма ощутительным неприятным запахом.

Сравнительно более легкая порча рыбы, встречающейся в продаже, зависит от большего или меньшего содержания в ней жира. Таким образом, в торговле чаще всего можно встретить испорченными: семгу, осетрину, севрюгу, сома, щуку, угря и пр. Вид такой рыбы снаружи бывает дряблый, волокна теряют связь; будучи сварена, она не так-то легко режется, сколько как-бы мажется ножом. Главный признак испорченности сырой рыбы есть то, что кости ее отстают от мяса, если, распоров живот, раздвинуть рукой обе полости. Глаз у свежей рыбы должен быть выпукл и светел. Напротив того, белое мясо рыб долее противится порче, именно вследствие малого количества содержащаяся в нем жира. К таким рыбам можно отнести: сига, судака, окуня, ерша, карпа, форель, леща, свежую треску, стерлядь и пр. Кроме того не следует забывать и то обстоятельство, что рыба, подобно вообще домашним животным, также подвержена болезням, при чем мясо, напр. у белой рыбы, получает желтоватый цвет и неприятный, затхлый запах, становится трухляво и пр.; мясо же больной красной рыбы, как, например, лососины, кроме трухлявости и неприятного запаха, отличается еще черноватым оттенком. К этим нашим замечанием необходимо прибавить еще, что рыба соленая, в особенности осетрина и белуга, вероятно от плохой посолки, бывает иногда ядовита. Надо быть также чрезвычайно осторожным при покупке копченой рыбы, так-как ее коптят не редко в то время, когда она попортилась и не находит сбыта. Вообще следует избегать заплесневевшей, трухлявой и горьковатой копченой рыбы.

Икра. Что же касается икры, то вообще надо заметить, что свежепросольная икра скорее портится, чем паюсная, а потому покупать ее следует с большею осторожностью и избегать такой свежепросольной икры, которая мало-мальски не зерниста, а напротив струиста, тягуча, сколько-нибудь кисловата и у которой икряные зернышки потеряли свою правильную сферическую форму. Хорошая же паюсная икра (салфеточная) должна быть сочна, жирна; зернышки ее не должны слипаться между собою, и вообще кусок ее должен представлять собою плотную, вполне компактную черную массу видимых и рельефных зернышек; излом же этой массы, при отрезе ножом, должен проявлять легкую потливость и иметь темно-серый отлив.

\subsection{Яйца.}
Ежели в Петербурге хотят иметь <<без обмана>> яйца высшего сорта и вполне свежие, и добротные, то необходимо обращаться за ними или к знакомым и полной вашей доверенности достойным торговцам, преимущественно специалистам яичной торговли, не торгующим положительно уже ни чем иным, кроме яиц, и торгующим ими с давних лет. Одна из этих специальных фирм, громко известная в Петербурге, есть фирма всеми уважаемого Алексея Федоровича Баранова, имя которого, как знатока этого дела, известно даже за границей. Склад его на Фонтанке, у Обухова моста. Впрочем, замечательно, что почти все лучшие яичные склады находятся по обеим берегам Фонтанки, между Аничковым и Обуховым мостами. Кроме того, яйца высших качеств можно найти еще и также брать с полною доверенностью в тех нескольких больших и пользующихся известностью в городе складах молочных скопов, которые поименованы нами ниже в наших заметках о молоке и молочных скопах. Заплатить за хорошие яйца несколько подороже, экономнее, чем, купив яйца очень дешево, испортить лежалым или тронувшимся яйцом какое-нибудь дорого стоящее и многих трудов требующее блюдо. Тут повторяется русская поговорка: <<от копеечной свечки Москва сгорела>>, потому что иногда между десятком употребленных вами в тонкое блюдо, как омлет, сабайон, слоеное тесто и пр., попадет одно плохое яйцо, хотя и не дешево заплаченное, по неумению выбрать взятое,~--- и все пропало! Вот для этого-то, не вдаваясь нисколько в теорию яичной торговли,~--- что не наше дело,~--- скажем несколько практических слов о том, как следует, по всем правилам, при покупке яиц в сомнительных лавках действовать, и, по поводу всего этого, упомянем только о том, что в Петербург яичный товар доставляется с начала лета до осени водою, а с ранней весны, с февраля месяца, начинают являться на яичном рынке яйца, привозимые по железным дорогам. Первые сохраняются в яичных кладовых всю зиму; вторые, по мере привоза, идут в потребление и бывает дороже первых. С недавнего времени показались в Петербурге яйца, так называемые, краковки, ранее всех других новопривозных появляющиеся, именно с февраля месяца, и идущие сюда из Кракова, Царства Польского и северо-западных губерний. Яйца эти соединяют в себе почти все достоинства, требуемые от хорошего безукоризненного яйца.

Само собою разумеется, что при выборе яиц, когда вы их покупаете в месте, не гарантированном правом слепой доверенности,~--- должно быть крайне осторожным. Иногда яйцо весьма нецеремонно говорит, о своей негодности, разя обоняние дурным запахом. Этот сорт яиц, легко обходимый всеми покупательницами. имеющими сколько-нибудь нормальное обоняние, называется в торговле <<тумак». Другие яйца поражают своим бурым цветом и резкою невзрачностью скорлупы, покрытой зловещими темными пятнами, признаками лежалости,~--- это также никуда негодные, сухие и несвежие яйца под наименованием пятинника. Затем есть сорт повыше этих, сорт дешевенький,~--- это присушка; оно свежо, но несколько деревянисто и уже ни под каким видом не может идти в такие блюда, в которых высокие качества яиц составляют особенную прелесть. Однако яйцо это годно в заливных, винегрете и некоторых других неделикатных блюдах. Выше этого сорта ординарка, которая сплошь-и-рядом идет во всех кухнях, хотя всмятку или вообще без соединений с другими материалами не годится. Советуем держаться еще высшего сорта, именно головки, пользующейся общею известностью и составляющей постоянный спрос всех хозяек и их кухарок, из которых далеко не все умеют, по тяжести в руке, отличить <<головку>> от <<ординарки>>. Но верх совершенства между яичными сортами~--- это клечик, а из <<клечика>>, то еще выше~--- <<козловка>>. Яйца эти, хотя и не несомые петербургскими курами и не те, о которых выражаются продавцы: <<сейчас из-под курочки>>,~--- отличаются всеми качествами яиц самого высокого достоинства и употребляются всмятку, равно как в самые изысканные блюда. Они привозятся по железным дорогам ежедневно во все времена года из низовых хлебородных губерний.

Степень свежести и лежалости яйца, не имеющего дурного запаха, как <<тумак>>, ни пятен и ржавчины как <<пятинник>>, узнается различными приемами и способами. Свежее яйцо должно быть тяжеловесно, так что при погружении в воду оно немедленно тонет и идет ко дну. Напротив того, не вполне свежее яйцо, хотя и не испортившееся еще, но давно снесенное, долго пролежавшее в подвале, бывает, сравнительно, легче по весу и при погружении в воду бульбулькает и всплывает на ее поверхность. Кроме того, если смотреть свежее яйцо на свечку или против солнца, то можно заметить, что оно слабо просвечивает в средней своей части; при разбивании такого яйца, желток явственно отделяется от белка. Вполне свежие яйца имеют совершенную полноту яичной массы и не содержать в себе никаких пустот. При некоторой опытности и сноровке, многие домовитые хозяйки, повара и знающие свое дело основательно кухарки узнают свежесть яиц следующим оригинальным приемом: лизнув яйцо сперва с острого конца, а потом с тупого, замечают, если тупой конец покажется на ощупь языка теплее острого,~--- то яйцо добротно и свежее; если же температура оказывается одинаковою как с тупой, так и с острой стороны, то яйцо сомнительно. Еще, при поднесении яиц к огню, свежее тотчас вспотеет и рука, его держащая, ощутить некоторую влажность, обнаруживающуюся и на самой скорлупе; не вполне же свежее и вообще лежалое яйцо никогда этого пота не даст. В торговле встречаются кроме того яйца, сохраняемый от гниения в известковом растворе или известковом молоке. Такие яйца неохотно покупаются домовитыми и сведущими хозяйками, потому что белок их чрезвычайно жидок, водообразен, трудно сбивается в пену и отличается посторонним неприятным вкусом. Яйца эти распознавать не трудно по чрезвычайной, неестественной белизне их скорлупы, которая в натуральном и нормально~--- неиспорченном своем состоянии должна иметь чуть-чуть искрасна-желтоватый колер. Наконец, при выборе яиц надо иметь в виду еще то, что лучшие яйца те, у которых скорлупа тонка, как-бы прозрачна и которые не столько шарообразны, сколько остроконечны с обеих сторон.

\subsection{Молоко и молочные скопы.}
Из всех съестных припасов, молоко и все из него приготовление или молочные скопы подвергаются бесчисленному множеству различных обстоятельств, вредно действующих на их качества и в особенности на их свежесть, потому что, как молоко во всех его видах, так и молочные скопы сохраняются крайне затруднительно. Так известно, что молоко уже через 23 часа, по выделении его из коровьего вымени начинает слегка изменяться, на поверхности его собирается слой сливок и тому подобное. Такая быстрота изменение молока вполне достойна сожаление, тем более, что, с одной стороны, в домашнем хозяйстве есть множество случаев, когда необходимо иметь цельное молоко, а с другой~--- вследствие неуменья торговцев обращаться с молоком и сберегать его на более продолжительный срок~--- в продаже вовсе не легко, как может казаться с первого взгляда, найти совершенно свежее цельное молоко. На сколько трудно найти в продаже хорошее цельное и даже снятое молоко, на столько же трудно судить о том, в какой степени виноват торговец, да и виноват ли еще он, если иной раз молоко, находящееся в его лавке, или синевато, или водянисто, или слизисто и пр. Вообще молоко, завися в своих качествах всецело от натуры коровы и способа ее содержание, может иметь все эти недостатки совершенно без злого умысла со стороны продавца. Это последнее обстоятельство делает чрезвычайно затруднительным наглядное различие в продаже молока неподдельного от поддельного, тем более, что последнее приобретает от подделки ту же водяность, синеватость и слизистость~--- свойства, которыми обладает молоко, взятое от больной коровы Впрочем, молоко разведенное водою (грубый обман, слишком обыкновенный в рыночной как гуртовой, так и мелочной торговле и преимущественно практикуемый петербургскими лавочниками и подвижными или ходячими продавщицами, известными под общим названием <<<<охтянок>>) может быть тотчас же узнано, так-как такое молоко, разбавленное водою с целью увеличение его объема, имеет обыкновенно светло-синий оттенок и бывает синевато-прозрачно у краев сосуда. Оно так жидко, что капля его, положенная на ноготь, не остается выпуклою, как это бывает всегда с неподдельным молоком, а расплывается; кроме того, оно мало или совсем не пенится и не пристает к чистому железному пруту если погрузить последний в молоко. Молоко, разводимое в торговле водою, бывает кроме того сгущено или мукою, или крахмалом, для того, чтобы скрыть что оно было разжижено водой. Если оно сгущено таким образом, то делается слизистым, и при расплывании на ногте и на краях сосуда оставляет осадок и крупинки, примешанных к нему, муки или крахмала. Сверх того, легко догадаться, что в молоке разболтана мука, еще и потому, что последняя имеет свойство пригорать и иногда обнаруживается на дне сосуда в виде бурого осадка, при кипячении молока. Кроме наглядного узнавание присутствия воды в молоке, введенной в него с целью обмана, можно применить еще следующее средство: следует влить определенное по весу количество (напр. 5–6 лотов) испытуемого молока в жестяную чашку, вес которой также предварительно определен. Затем прибавляют к молоку 30 лотов истолченного кварца, который всасывает в себя это последнее, образуя вместе с ним однородный сырой порошок, потом высушиваемый досуха и взвешиваемый; потеря в его весе, которая при этом обнаружится будет соответствовать количеству воды, содержавшейся в испытуемом молоке. Но при этом, однако, нужно помнить, что всякое нормальное, неподдельное молоко, как бы оно хорошо ни было, содержит в себе от 80 до 88 процентов воды. При описанном опыте, только излишек против этого количества должен быть признан за примесь, сделанную с целью обмана. Иногда, для сбережения молока, кладут в него поташ. Подмесь эту открывают, приливая в испытуемое молоко какой-нибудь кислоты, отчего, в случае содержание в нем поташа, произойдете шипение.

В хороших, ничем не подмешанных, сливках точно такой же существует недостаток, как и в хорошем, неподдельном молоке и притом сливки также, как и молоко, чрезвычайно легко скисают. Хорошие густые сливки не должны свертываться при кипячении. Пенистая, пузыристая поверхность сливок может служить наружным признаком, свидетельствующим, что они были предварительно разбавлены водой, а за тем и сгущены мукой и куриным яйцом. Для открытия в сливках присутствия муки, может быть употребляем тот же способ, который предложен выше для открытия этой примеси в молоке; примесь же яичного белка и желтка легко можно распознать по обильному образованию свернувшихся клочьев, получаемых после кипячения сливок и процеживание их через бумагу.

Творог вполне хорошего качества должен быть ярко-белого цвета, хрупок, отчасти слоистого сложение и сух. Если продажный творог сыр и студенист, то это может служить признаком, что он нарочно пропитан водою с целью увеличение его веса; творог же, имеющий зеленоватый оттенок, обнаруживает присутствие в нем сыворотки, которая, во время его приготовления, не хорошо была отжата.

Коровье масло совершенно чистое также не постоянно и не везде находится в торговле. Майское масло вкусом несравненно выше всех видов масла, приготовленных в другое время. Хорошее сметанное и сливочное масло должно быть однообразно во всей массе, нежно желтоватого, палевого цвета, на ощупь жирнисто, приятного, подобно орехам, вкуса и хорошо промыто; оно не должно, кроме того, нисколько хрустеть на зубах, от содержащейся в нем соли. Дурное масло, напротив того, бледного цвета, легко крошится, мало жирнисто и сухо, хрустит на зубах, пахнет горелым и кисло-горько на вкус; иногда же оно слизисто и тянется в нити. Масло часто бывает подкрашено нарочно, для придания ему привлекательного вида, в особенности в том случае, когда, как это с лавочным маслом сплошь и рядом бывает, к нему примешано сало, отчего оно делается слишком бледным. Разминая масло деревянною ложкою в воде, можно легко обнаружить, было ли масло окрашено или нет: если масло было окрашено, тогда и вода окрасится. Сало составляет самую обыкновенную и общеупотребительную торговую примесь в коровьем масле. Эту примесь можно открыть по вкусу и по виду такого масла: в разрезе оно имеет белые пятна и полосы. Гораздо реже встречается в масле, но все-таки встречается, примесь тертого вареного картофеля. Масло с этою примесью очень тяжеловесно, крошится, в разрезе шероховато; если растопить его, то на дне сосуда получатся мучнистые клочья, имеющие запах картофеля.

Топленое или русское масло редко встречается прогорклым и иногда бывает смешано с более или менее хорошим русским маслом и даже салом. Прогорклость русского масла, происходящая главнейшим образом от плохого топления, узнается на вкус; что же касается способа открытия смеси дурного масла с хорошим, то этот способ чрезвычайно прост: стоит только выложить такое масло из сосуда и сделать в куске разрез,~--- неоднородность массы легко обнаружит обман.

Все молочные продукты, не исключая и молока, в Петербурге продаются в различных лавках, вовсе не имеющих молочной специальности, или, в так называемых, весьма многочисленных, (более 1000) <<мелочных лавочках>>, с тою лишь разницею, что товар мелочной и, так сказать, <<энциклопедичной>> торговли в достоинствах уступает тому, который обращается в торговле более крупной и специальной. Так-то всеми молочными скопами торгуют, устроенные в домах и отчасти на рынках, сливочные лавки, где, однако, можно найти и яйца, и крупы, и макароны, и даже стеариновые свечи. С этими лавками соперничают <<зеленные лавки>>, держащие также: сливки, молоко, сметану, творог, масло и сыр простого, низкого сорта вместе с овощами, зеленью, соленьями, маринадами, копченою и соленою рыбою, живностью и дичью. Случайно и по давнишнему знакомству с хозяином лавки можно, конечно, и здесь в этих <<сливочных>> и <<зеленных>> лавках находить молочные скопы хорошего качества: но все-таки трудно, чтоб такой, например, нежный молочный товар, как сливочное масло или высшего сорта сливки могли находиться в соседстве с треской, балыком, чесночною колбасою и тому подобным без вреда для своего достоинства и качества, почему советуем хозяйкам за всеми этими припасами, ежели они желают иметь все такое, что хотя несколько и подороже, да уж затем представляет собою верх совершенства,~--- обращаться к специалистам торговли молочными скопами, которых, впрочем, очень немного, всего каких-нибудь шесть почтенных фирм, а именно: Максимов, Чернышев, Варламов, Сорокин, Татаринов (в Б. Конюшенной и в Б. Морской) и Ефимов, известный больше под именем Малиновского, по фамилии бывшего своего помещика, владевшего известною в окрестностях Петербурга мызою Суйдою, перешедшею впоследствии в управление и чуть ли не в собственность означенного г-на Ефимова. Торговля всех этих специальных лавок ведется в размерах широких, хотя, впрочем, и здесь можно иметь в самых малых количествах молочные произведение и цельное молоко самого высокого качества.~--- И так, положительный совет: стараться производить покупки по части молочных скопов или в этих больших и пользующихся весьма почтенною репутацией лавках, или в таких сливочных, добросовестность и дельность которых вам не с сегодняшнего дня известны. Впрочем, опытная и вполне знающая свое дело хозяйка, лишенная какими-нибудь обстоятельствами, например, отдаленностью своего местожительства, от центра города, где преимущественно находятся пункты торговли крупных специалистов, конечно, сумеет обойтись и без них, лично посещая <<свои>>, как говорится, <<насиженные>> места рыночной и лавочной торговли, где таких смышленых хозяек продавцы не надувают, оставляя арсенал своих торговых фортелей и фокусов для покупательниц менее сведущих и проницательных и в особенности для тех, которые учатся трудному делу домоводства и слишком доверчиво полагаются на своих кухарок.

Заведя речь о кухарках, кстати сказать, что все кухарки, особенно не немки и не шведки, a настоящие русские или отчасти и польки, и в особенности наиболее из них сведущие, необыкновенно пристрастны к употреблению русского масла, т.~е. топленого, которое, будучи копейками двумя на фунт дешевле обыкновенного чухонского кухонного масла, приготовленного из сметаны, но не топленого,~--- имеет отвратительное свойство страшнейшим образом чадить и дымить. Но на стороне достоинств этого сорта масла то, что оно не скоро портится, спорко для хранения и, как выражаются повара и кухарки, <<не капризно>>. Это выражение означает то, что при жарении с этим маслом, по свойствам его, нечего опасаться сжечь жаркое, которое можно оставлять довольно долго без присмотра на огне, а между тем уйти в лавочку, или в трактир, для различных каляканий и угощений. При употреблении же чухонского масла, этого рода вольностей дозволять себе нельзя, потому что как-раз жаркое подгорит. Впрочем, ежели вы жарите какую бы то ни было дичину, то, чтоб жаркое ваше достигало последней степени совершенства и имело золотистый цвет,~--- можно, с некоторыми предосторожностями от чада и угара, употреблять русское масло, точно так, как для жарения телятины оно нисколько не годится и непременно заменяется чухонским маслом или даже, что еще лучше, говяжьим почечным салом. И так, в маленьких, вполне опрятных хозяйствах, где и самые печи, и все очаги устроены ежели не роскошно, то удобно, комфортно, с правильною вентиляцией, и, главное, где глаз самой хозяйки за всем блюдет, там положительно и решительно можно сказать, что для всякого жарение другого кроме кухонного чухонского масла (которое под этим названием так и известно в торговле) употреблять не следует. Оно стоит, правда, от 30–35 к. ф., но не фарлатировано водою и даже салом, как дешевенькое в 20–25 к. ф., покупаемое у уличных разносчиков или у недобросовестных продавцов. И опять выходит дешевое на дорогое, потому что при промывке этого <<дешевенького>> масла, собственно масла в купленном фунте остается много-много две трети. Иногда, впрочем, масса воды в масле происходит от дурной его выпрессовки, чрез что оно рыхлится и увеличивается в объеме, что выгодно продавцу и убыточно покупателю. К тому же, дурно выпрессованное масло скорее портится и подвергается горьклости.~--- Лучшее кухонное масло коровье есть действительно, в полном значении слова, чухонское, потому что в громаднейших количествах доставляется из Финляндии, где на всех почти тамошних мызах выделка молочных скопов и в особенности масла <<кухонного>>, т.~е. низшего сорта, доведена до последней степени совершенства.

Независимо от кухонного масла, в торговле имеются другие сорта, а именно: мызное масло и сливочное масло. Первое очень хорошее, делаемое из сметаны и необыкновенно приятное на вкус, с легкой посолкой, идет в кухонном деле уже не на жаренье, а на более тонкие блюда, как яичные, овощные, хлебные (пюре, омлеты, пудинги, печенья и пр. и пр.). Этого же сорта масло особенно хорошо идет на домашние бутерброды. Но ежели хотят маслом лакомиться и употреблять его за обедом, завтраком, чаем, как это, по заграничному, ввелось у нас почти повсеместно, то уж тут необходимо иметь сливочное масло, доводимое ныне действительно до последней степени совершенства. Из этого сорта в Петербурге славится наиболее так называемое шварцовское масло, именуемое так по фамилии бывшего (в 1872 году умершего) владельца огромной мызы на границах Финляндии, где, как слышно, до 2000 дойных коров. Удобства сообщение дают возможность иметь в Петербурге все эти скопы и даже только-что подоенное молоко, по истечении лишь нескольких часов.

От масла перейдем к творогу. Не говоря ничего здесь о тех крайне неудовлетворительных и невзрачных сортах этого молочного скопа, какие обращаются в рыночной и низко-мелочной торговле, скажем, что те твороги, какие продаются в <<сливочных>> лавках, привозятся сюда с мыз, преимущественно ямбургских, гдовских и отчасти царскосельских. Они сохраняются на льду в погребах и могут быть получаемы покупателями во всякое время. Из этих же окрестных ферм доставляется и сметана. Узнавать достоинство сметаны и творога на вкус и на глаз так не трудно, что об этом и говорить нечего. А все-таки всего вернее покупать и то и другое в хороших лавках.

За сим скажем, что собственно молоко разделяется на два сорта: молоко снятое и молоко цельное. То и другое можно найти везде, но опять повторим, что ежели хотите иметь молоко самое чистое без всякого участия в нем воды, то все-таки надобно обращаться к фирмам почтенным, а не щеголяющим рыночными зазывами и всяческими шарлатанскими вывесками. В лучших сливочных лавках молоко и сливки получаются из ферм, расположенных по железным дорогам не более 100-верстного расстояние от столицы. Сливки подразделяются на три сорта: 1) низший сорт, т.~е. жидковатые, 2) средний сорт, т.~е. гуще, но еще не самые густые, и 3) высший сорт, т.~е. вполне густые, так называемые на сбивку. Все эти сорта сливок с порядочных ферм доставляются ежедневно в запломбированных бутылках. Владельцы лавок, так сказать, <<нанимают>> молочное хозяйство той или другой фермы, т.~е. по уговору с владельцем или арендатором фермы имеют определенное количество снятого или цельного молока, или сливок всех сортов. Запасаются они этим продуктом, большею частью, под наблюдением сведущего и добросовестного приказчика, постоянно находящегося на ферме от хозяина-нанимателя (т.~е. владельца лавки, производящей торговлю молочными скопами). Некоторые фермы, на ближайших дачах около Петербурга, имеют в столице своих агентов, которые в больших металлических, преимущественно цинковых, сосудах развозят молоко по домам своим месячным абонентам. При этом, они в этих же домах находят и других желающих снабжаться от них сливками и молоком. Сколько нам известно, этот способ продажи и покупки сливок и молока из первых почти рук не представляет собою больших неудобств для покупателя, при том, что продавцы, не платя ничего за содержание лавки, могут продавать молочный свой товар несколько дешевле сливочных лавок и магазинов; не часто бывают слышны жалобы на не добротность продукта, приобретенного таким способом. Однако, ежели кому нужны сливки действительно самого высокого качества, то знатоки дела советуют обращаться все-таки в магазины одной из шести выше названных нами фирм.

Парное молоко с некоторого времени в Петербурге можно иметь из тех городских ферм, которые в количестве, кажется, 50–60 рассеяны по Петербургу и содержатся большею частью отставными гвардейскими унтер-офицерами или мещанами, а больше всего разными вдовами~--- солдатками, мещанками, крестьянками, чиновницами, дворянками и пр. Фермы эти ни что иное, как 5~--- 6, много 10–12 коров молочных, содержимых, вопреки всем правилам молочного хозяйства, в темных и душных денниках и содержимых то и кормимых не всегда и не у всех хозяев с надлежащею тщательностью и, главное, опрятностью. Иногда удается получать от этих молочниц и молочников продукты недурные и порядочно приготовленные, как-то: молоко, сливки, сметану, творог, простоквашу, варенцы и пр.; но это случается не часто. Фермы эти или молочни с коровниками рассеяны чуть ли не по всему городу, и вы можете видеть множество на домах больших и малых (большею частью малых, в виде ярлыков об отдающихся квартирах) вывесок с надписями: <<Парное молоко. Цельное молоко. Снятое молоко. Сливки>> и пр. и пр.~--- Остается сожалеть, что это дело у нас хотя и развивается, но плохо как-то прививается. При этом, мы пройдем молчанием те крайне неудовлетворительные молочные заведение, какие устроены в наших общественных садах и скверах каким-то тирольцем, потому что в этих молочных с коровниками именно нет и в помине того, что составляет непременное условие всяких молочных скопов,~--- самой изысканной чистоплотности: здесь же, к сожалению,~--- отсутствие самой простой и необходимейшей опрятности, без которой никакое молочное хозяйство немыслимо.

Но пожелаем городским фермам успеха и развития несколько совершеннее того, какими они отличаются ныне,~--- и обратимся к, так называемому, русскому сыру, который в нынешнее время в весьма значительном количестве изготовляется в различных губерниях России и появляется большими массами в Петербурге, где не так давно еще славился и в общем употреблении был сыр, известный под названием мещерского, под каким именем, однако и поныне, по старой памяти, в продаже идут очень многие другие сыры. Этот <<русский>>, или туземного приготовление сыр, подражающий швейцарскому, голландскому, английскому и итальянскому сырам, преимущественно требуется для кухни и, отчасти, его сорта попроще во многих хозяйствах составляют хороший завтрак для прислуги. В настоящее время считаются хорошими сырами из сыроварен: артельных Тверской губернии (г. Верещагина), тверских: гг. Татищева и Кисловского; Вологодской, г. Денежкина, а также из многих мест Северо-Западного края. Между прочим, очень замечательна фирма братьев Кублей из Пошехонского уезда, Ярославской губернии. Фирма эта нынче делается очень и очень замечательною.~--- Кроме всех этих простых русских сыров из коровьего молока, употребляемых на кухне в различные кушанья, а в домашестве на завтраки запросто по домашнему, и особенно на стол прислуги, есть еще, так называемый, польский сыр. Сыр этот получается из Виленской губернии и, в своем роде, очень недурен. Он продается, смотря по остроте и спелости, от 15 до 60 к. за фунт. Попадается в торговле еще сыр из овечьего молока, получаемый также из Северо-Западного края и замечательный тем, что он имеет много сходства с пармезаном, почему довольно требуется знающими его такое свойство для макарон и вообще для блюд, заимствованных из итальянской кухни. Главное депо <<овечьего сыра>> в сливочной лавке в д. Семянникова, на Фонтанке, у самого Аничкова моста. Лавка эта содержится польскою уроженкой и отличается, приятною для глаз и обоняние, опрятностью.

\subsection{Овощи и зелень.}
В особенности весною все почти овощи, сохраняемые зимою с лета или с осени в подвалах, успевают прорости, при чем некоторые коренья, как напр. петрушка, сельдерей, морковь, свекла и другие овощи, каковы: картофель, брюква, репа и т. п., обращают свои соки на развитие и рост стеблей, листьев, корешков и корней, и в этом виде становятся совершенно негодными для употребления в пищу. Картофель же оказывается даже вредным, вследствие образования в нем в это время особая вещества~--- <<соланина>>, имеющего ядовитые свойства. Картофель, кроме того, бывает поражен этою болезнью, в последние годы очень сильно развивающеюся. В этом болезненном состоянии он отличается красно-бурыми пятнами, загнивает и в пищу оказывается негоден. Картофель с ростками бывает мягок, a после варки получает неприятный вкус; то же можно сказать и о тех родах овощей, которые к весне прорастают. К весне в особенности сильно изменяется капуста, сохраняемая в подвалах вилками в песке: наружные листья ее гниют и весь кочан издает очень резкий и неприятный запах. Подобная сильная порча овощей в весне происходить, большею частью, вследствие неуменья огородников сберегать их в зимнее время и от неимения на огородах хороших овощных амбаров. Во всяком случае, при покупке зимних овощей весною следует быть крайне осторожным и отнюдь не приобретать тех из них, которые обнаруживают признаки прорастание.

Все это нами сказано об овощах в натуральном виде, а не сушеных, квашеных, соленых и пр., почему надобно замолвить слово и обо всех заготовляемых в прок. Так, например, сушеный горошек, как простой, так и сахарный, имеющий в торговле несколько сортов, бывает очень часто перемешан с черным и серым горохом и другим сором; поэтому, при покупке его, необходимо обращать внимание на это обстоятельство, происходящее от плохой сортировки означенного товара. Вообще хороший горох должен быть чистого зеленая цвета (хорошо высушенный), без червоточины, и должен легко разбухать и хорошо развариваться в воде; закаленный горох трудно разбухает и разваривается. Все сказанное здесь о горохе может быть отнесено и к сушеной чечевице, при покупке которой необходимо, сверх того, обращать внимание на то, чтобы чечевичные зернышки не были сплюснуты, а имели бы свою натуральную форму и не были бы перемешаны с зернами куколя. При закупке кореньев в летнюю пору, надобно обращать внимание на их целость и на степень их сочности: чем сочнее корень редьки, моркови и пр., тем лучше; тогда нечего опасаться, что попадется корень стволовый, который, впрочем, можно очень легко распознавать по его сучковатости. При выборе квашеной капусты, необходимо обращать внимание на цвет, который должен быть светло-желтоватый; самая капуста не должна быть слизиста, горька и издавать того сильно неприятного, почти вонючая запаху, который так обыкновенен в наших торговых местах, где продается и сберегается этот продукт. Хорошие свежие огурцы, которые расходуются в таком множестве в Петербурге, не должны быть дряблы, желты и слишком крупны. Слишком крупные огурцы бывают или с большими пустотами внутри, или с сильно развившимися семенами. Хорошие же соленые огурцы не должны быть дряблы, желтоваты, едки на вкус и мягковаты; они должны сохранять в известной степени свой нормальный зеленый цвет, который иногда сообщается им искусственно посредством зеленого купороса, что очень и очень вредно. Эту фарлатацию легко заметить по зеленоватому оттенку огуречная рассола; нормальный же цвет этого рассола подобен цвету обыкновенного раствора поваренной соли в воде, с тем только различием, что огуречный рассол всегда бывает несколько мутный.

Впрочем, и относительно покупки овощей не могу не повторить вам того же, что уже говорено мною было, т.~е. что слишком дешевое все выходит всегда на очень изъянистое. А потому рекомендуем вам, петербургские хозяйки, как постоянные, так и временные, те две, три овощно-крупяные лавки, какие находятся на Екатерининской канаве, у Каменного моста, или, еще лучше, так называемое Коровинское овощное депо, принадлежащее первому по части огородных овощей специалисту A. Ф. Коровину, на Гороховой улице, у Красного моста, в доме Таля, на дворе. Здесь вы найдете все в совершенстве, в изобилии и, главное, здесь можно все брать с полнейшею доверенностью и уверенностью, что вас не обманут.

\subsection{Мука.}
Определение торгового достоинства муки затруднительно по той причине, что многие свойства и качества ее зависят сколько от честности торговца, столько же и от способа помола и условий роста того рода зернового хлеба, из которая она выделана. Так, например, несправедливо бы было обвинять торговца, если в муке, купленной из его лабаза, будет открыто присутствие, так называемых, рожков,~--- того самого ядовитого вещества, которое составляет болезнь ржи и пшеницы и смалывается вместе с зернами этих хлебов в тонкую муку, а будучи испечено в хлебе и потреблено человеком в пищу, производить в нем болезненные припадки, известные под именем <<злой корчи>>. Со всем тем, не смотря на чрезвычайную неустойчивость качеств муки, все-таки очень важно знать главные признаки хорошей муки. Таким образом, дознано, что хорошая пшеничная мука, почти единственная кроме гречневой для масляничных блинов и овсяной для некоторых похлебок, употребляемая в сколько-нибудь порядочных кухнях, должна быть однородна, т.~е. не различных сортов и помолов, на ощупь суха, мягка, пушиста, бархатиста и нежна. Тесто, из нее сделанное, должно иметь среднюю тягучесть: если тесто будет менее тягуче, то это будет служить признаком дурного качества этой муки. Плохая же мука, содержащая в себе много влажности (более 15 \% воды), обыкновенно легко приходить в брожение, съёживается в комья, имеет затхлый запах и дает худое, трудно поднимающееся тесто,~--- такая мука подверглась уже порче. Так-как сырость есть главнейшая причина порчи муки, то при покупке муки необходимо прежде всего обращать внимание на эту сырость, присутствие которой в муке обыкновенно узнается чрез то, что мука (все равно, пшеничная или ржаная) всегда имеет синеватый отлив, тогда как сухая мука имеет отлив желтый. Мука, попортившаяся от нагрева, что случается тогда, когда ее держат в помещении со спертым воздухом, при чем нижние мешки придавливаются верхними, отличается обыкновенно неприятным запахом и горьким вкусом. В пшеничной муке чаще всего можно найти примесь ржаной муки и картофельного крахмала. Для открытия этого обмана всего лучше взять пробу муки в лабазе, подвергнуть это количество муки брожению и испечь из нее булку: уже при 10 \% содержание в испытуемой муке означенного крахмала, вместо рыхлого и ноздреватого хлеба, получится твердая, неудобоваримая для желудка, лепешка. Что касается примеси ржаной муки, встречаемой в муке пшеничной, то она легко узнается по запаху и вкусу пшеничного хлеба, из пробы ее испеченного. Хорошая ржаная мука не так ярко-бела, как пшеничная, которая, лучшая, имеет положительно цвет первого снега. Самые высшие сорта ржаной муки, по белизне своей, подходят лишь к низшим сортам пшеничной муки; при растирании этой муки руками, она легко сваливается в комья, если даже и совершенно суха. Торговцы иногда обманывают покупателей, сбывая им низший сорт муки за высший; но этому горю можно будет помочь только тогда, когда покупатель лично ознакомится с различными торговыми сортами муки.

Впрочем, для всех печений в кухне и для всех хлебных блюд, в какие идет пшеничная мука, никакой другой этого рода муки употреблять не следует, как только ту, которая в торговле известна под названием конфетной. Она идет на пирожные, пироги, кулебяки, паштеты, клёцки, блинцы и прочие поваренные приготовление, имеющие мучное основание.

\subsection{Крупы.}
Крупы для кухни все должны быть самых лучших сортов. В избежание опасности купить сорт поплоше за сорт получше, можно хозяйкам, живущим в Петербурге, ежели они не имеют верной и вполне надежной знакомой <<зеленной>> лавки, посоветовать брать все сорта круп в тех немногих овощных лавках, которые находятся на Екатерининском канале, у Каменного моста. Здесь всего две, три фирмы, но весьма старинных и почтенных, действующих с полным уважением к своему делу, хотя, правда, у них всегда все немножко подороже; но мы уже остаемся при том мнении, основанном на опытности, что <<хорошее дорогое всегда выгоднее дешевой дряни>>. Однако, не все же могут для каждого фунта отправляться к Каменному мосту, да и к тому же, не у всех в квартирах есть порядочные кладовые (редкость в Петербурге) для делания запасов, почему волею-неволею, а приходится также покупать по мелочам и в разных, не всегда знакомых и не всегда вполне добросовестных лавках. И вот тут-то покупательница хозяйка или ее кухарка должна, например, при покупке перловой крупы предпочитать мелкую крупу более крупной, наблюдая, чтоб крупа эта была бела, без содержание крупяной пыли и мучнистых частей, зернышко к зернышку, как на подбор. Крупу пожелтевшую, или получившую серый цвет и затхлый запах, следует считать товаром более или менее испорченным. Эта крупа называется голландскою; она очень спорка и содержит в себе весьма мало мучной пыли, но все-таки ее следует перемывать. Простая, более крупная, сероватая и несколько продолговатая крупа, быв дешевле, употребляется в кушанья прислуги. Что же касается гречневой крупы, столь употребительной в русском хозяйстве преимущественно в виде каши, начинок (для пирогов и зразов), некоторых пудингов и для многого тому подобного, то в торговле существуют три главных ее сорта: а) ядрица представляет полное зерно не расколотой гречихи: лучшая ядрица должна быть тяжеловесна и в особенности суха; в ней не должно быть много черных или зеленых зерен или каких либо иных и сторонних примесей, например зерен сорных трав, что величайший ее недостаток. b) Так называемая продельная гречневая крупа есть второй сорт гречневой крупы. Зерна ее расколоты на две или на три части, следовательно, она мельче ядрицы. Этот сорт крупы должен быть особенно бел, сух и тяжеловесен. с) Третий сорт или ординарная гречневая крупа несколько крупнее предыдущего сорта и не столь бела, чиста и веска. Она известна также под названием казарменной гречневой крупы и, странное дело, любителями гречневой каши, сочной, упаристой, рассыпчатой, мягкой,~--- словом, настоящей <<русской каши>>, она предпочитается всем другим. Самый же высший сорт гречневой крупы из раздробленных круп называется в торговле вельегоркой и приготовляется исключительно в городе Моршанске; она гораздо мельче предыдущих сортов, совершенно чиста и не содержит в себе нисколько ни черных, ни зеленых зерен. Есть еще так называемая обварная гречневая крупа, продаваемая предпочтительно в зеленных лавках Каменного моста. Эта та же ядрица самого высокого достоинства, обданная самым сильным кипятком, просушенная и пущенная в продажу по 20 к. за фунт. Она красноватого цвета и изготовляется в горшке вдвое скорее сырой ядрицы. Наконец, так называемая смоленская, или манная крупа, из которой приготовляются все роды нежной молочной каши и, между прочим, знаменитая <<гурьевская каша>>, описанная в этой книге довольно подробно. Крупа эта очень мелкая; ежели она вполне хорошего качества, то не должна быть желтоватого цвета и содержать много мучнистых частиц. Прокислая манная крупа вовсе не годится в пищу.~--- Окончим сведение о крупах одною весьма употребительною в кухнях наших иноземною крупою, известною под названием сорочинского пшена, или риса. Хороший рис должен быть сух, бел, тверд, просвечивает, без обломков, не должен иметь неприятного запаха и кислого вкуса. Бразильский рис, известный в торговле под именем вест-индского риса и имеющий большие, белые, просвечивающие зерна, снабженные мелкими красными полосками, лучше итальянского риса. Но самый веский сорт товара в этом роде представляет каролинский рис: у него чистые, белые просвечивающие зерна, снабженные нежными полосками; он длиннее и более риса итальянского. Что касается русского риса (с Кавказа и из Крыма), то он, к сожалению, имеет посторонний неприятный вкус и запах и не редко содержит в себе примесь поваренной соли, которую примешивают к нему с целью сохранение его от червоточины. С некоторых пор в петербургской торговле появился новый сорт итальянского риса по 15 к. за фунт: зерна мелкие, голубовато-прозрачной белизны; он необыкновенно спорок и не легко превращается в кашу. Сорт этот превосходен для пилава.

\subsection{Фрукты.}
Из числа фруктов, по большому употреблению в кухонном хозяйстве, особенно замечательны лимоны. Тонкокожие лимоны самые сочные и дороже ценятся, но труднее сберегаются, чем толстокожие. Хороший лимон должен быть ярко-желтого цвета, без пятен; внутренность его должна быть с избытком наполнена соком свежим, приятно кислым, слегка липким и ароматным. Лимоны, у которых кожа покрыта плесенью, совершенно негодны. То же самое вообще можно сказать и об апельсинах, при выборе которых важную роль играет, как и при выборе лимонов, их тяжеловесность.~--- Яблоки и даже груши очень часто встречаются в совершенно гнилом состоянии на рынках, в особенности у самых мелких торговцев, к сожалению находящих себе покупателей, между тем как употребление таких плодов чрезвычайно вредно. Следует кроме того избегать плодов с повреждениями на коже и с пятнами. Хорошие финики, доходящие к нам уже в сушеном виде, должны быть полны, блестящи, без морщин и червоточины. Достоинство хороших винных ягод определяется их свежестью и сладостью; кроме того, они должны быть хорошо просушены, мясисты и без белого налета.~--- Изюм самый лучший считается тот, который называется пакетным. Но нельзя не предупредить доверчивых покупательниц, что эти <<пакеты>> заключают в себе много обманчивости, продаваясь от 1р. 50 к. до 2 р. за пакет в самых блестящих наших фруктовых лавках. Дело в том, что в пакете, в том виде, как он получается из Испании, вложена огромная изюмная ветка, восхищающая вас своею объемистостью; между тем такая ветка, при более тщательном ее осмотре, представляет собою связку из нескольких веток, которые будучи просто взяты из ящика, без помещения в замысловатый пакет, стоили бы не дороже 35–40 к. за фунт. Эту шутку играют испанские продавцы уже не с сегодняшнего дня, а мы все как-то ее не замечаем и идем добровольно на эту удочку.

Из сушеных фруктов в кухне большую роль играет чернослив. Лучшим черносливом считается французский и из этого французского выше других есть сорт, называемый <<империал>>. В фунте этого чернослива от 40–35 ягод, тогда как в других более простых сортах помещается в фунте больше ягод, а именно: в <<сюр-шуа>> (sur choix),~--- от 45–50 ягод; в <<шуа>> (choix) от 55–60 ягод; в <<рам>> (rame supérieure), самом обыкновенном, до 65 ягод. Ежели кто хочет купить в фруктовой лавке дорогой чернослив, не подвергая себя никакому обману или недоразумению, тот непременно должен обратить внимание на сообщенные нами здесь количества ягод в фунте, чтобы правильно и безошибочно судить о том, какой, наверное, сорт он покупает. Настоящий высшего качества бордосский или турский чернослив имеет прелестнейший аромат, долго не проходящий, когда откупоривается запаянная жестянка, в которой помещается эта изящная бакалея. В жестянках и банках сохраняемый чернослив остается без вреда для своей свежести неопределенное число лет.

\subsection{Грибы и шампиньоны.}
Главное условие при сборе грибов~--- уменье отличить съедобные грибы от ядовитых. Этим качеством не всегда бывают наделены лица, промышляющие грибами. Вот почему не редки случаи продажи на рынках ядовитых грибов вместе и с съедобными, так что покупатель должен быть непременно знатоком в деле выбора хороших грибов при их покупке. Можно посоветовать прежде всего покупать грибы не иначе, как на открытых, а не на потаенных или случайных местах; следует также избегать грибов крошеных и с очищенными корешками. Сверх того, необходимо иметь в виду и следующие соображение. Грибы, у которых мякоть не совсем свежа, или изрыта тонкими канальцами, доказывающими присутствие в них червей и т. п., не годятся для употребления. Иногда только одна нижняя часть гриба бывает источена червями, тогда как верхняя (шапочка) осталась неповрежденного. В таком случае верхнюю часть можно употреблять в пищу, а нижнюю следует отбрасывать. Наконец, хотя и взрослые грибы бывают годны в пищу,~--- также, как и молодые, однако все-таки лучше избегать употребление слишком уже старых грибов, отличающихся обыкновенно большими верхушками. К этому следует прибавить, что грибы, обрызганные сверху улитками, хороши и не опасны только в таком случае, если мякоть их не повреждена. При покупке соленых грибов, следует обращать внимание на рассол, в котором они сохраняются: если рассол тянется при переливании, а самые грибы покрыты слизью и издают плесневелый запах, или же горьки на вкус, то они не годятся для употребления.

Практическая заметка о шампиньонах.

Нынче употребление шампиньонов сделалось почти повсеместным; но грибы эти иногда и даже довольно часто оказываются ядовитыми. В сыром состоянии никак нельзя узнать ядовитости шампиньонов; но для того, чтобы удостовериться в их безвредности, необходимо брать их на пробу из знакомой лавки, или в лавке же, ежели согласится продавец, подвергнуть их следующему испытанию. Чтобы употреблять шампиньоны без опасения, режьте их пополам металлическим ножом и отнимите у них хвостики и лиственичные части. Ежели в этом состоянии шампиньоны сохранять свою белизну без изменения, по крайней мере, час,~--- вы их варите, оставив в кастрюле серебряную ложку. Ежели ложка или шампиньоны почернеют,~--- знак, что эти шампиньоны никуда не годятся и их надобно выбросить. Ежели же они цвета не переменяют, то их смело можно употреблять. Заметка эта, основанная на опыте, применима и к другим родам грибов\footnote{Заметка эта извлечена из статей бывшего придворного метр-д'отеля Эдмона Эмбера, печатанных в домоводственном журнале <<Эконом>> 1845 года.}.

\subsection{Сахар.}
В очищенных сахарных головах умышленных примесей не встречается; за то в сахаре, встречающемся в торговле, например, в виде сахарного песку, много бывает примеси в ущерб потребителям. Хороший рафинированный сахар в головах должен быть бел, отдельные кристаллы его должны быть правильны, однородны и блестящи. Такой сахар не должен быть мягок и в воде растворяться без остатка, без мути. Хороший колониальный сахар сырец бывает сух, цвета слабо-жолтого и состоит из крупных зерен. Иногда высший сорт сахара рафинад~--- содержит в себе, так называемый, некристаллический сахар, называемый виноградным, который притягивает влажность и таким образом нарушает связь между частичками сахара, отчего сахарные головы иногда распадаются в порошок. Для открытия присутствия означенного рода сахара, растирают немного испытуемого сахара с равным количеством гашеной извести; смесь эту растворяют в воде и раствор процеживают и нагревают: если он примет бурый цвет, то это будет служить признаком присутствия в испытуемом товаре некристаллического сахара.

Ко всему этому, относительно сахара, обращающегося в петербургской торговле, мы считаем не лишним прибавить, что сахар рафинированный на петербургских сахарных заводах в торговле петербургской называется <<петербургским заводским>>, в отличие от того исключительно свекловичного сахара, какой во множестве получается из Киевской и других юго-западных и отчасти Царства Польского губерний, который слывет под общим названием <<польского>>. Сахар петербургских заводов отличается замечательною крепостью и особенно сахар с завода фабриканта Кёнига, известный в столице, да и в провинциях отчасти, под названием <<кениговского>> или <<гранитного>>. Однако этот великолепный сахар уступает другим в сладости. К тому же, нельзя не заметить, что головы юго-западного или неправильно именуемого <<польского>> сахара крупнее голов петербургского заводского сахара. Не можем не обратить внимание на то, что в торговле существует, так называемый, <<постный>> сахар, при рафинировании которого употребляются на заводах точно те же далеко не постного, a вполне скоромного свойства вещества, какие употребляются и для рафинирования и непостного сахара; только та разница, что это сахар не сильно рафинированный, т.~е. не дошедший до степени кристаллизации, хотя нередко выходит из одной же с ним вари. Другой <<постный>> сахар, имеющий вид какой-то конфеты и иногда окрашиваемый в желтый или розовый цвета, продаваемый в фруктовых, зеленных и мелочных лавках, еще менее постного свойства, потому что просто-на-просто приготовляется пряничными мастерами на яичном белке.

Патока. Хорошо приготовленная крахмальная патока должна быть почти белого цвета, очень сладка на вкус, и погруженная в нее, для опыта, лакмусовая бумажка (ее легко достать во всякой аптеке) никогда не краснеет.

\subsection{Мед.}
Мед сотовый бывает слабо-желтого цвета, в свежем состоянии совершенно прозрачен, имеет приятную, цветочную душистость и сильно сладкий вкус. Плохой мед, получаемый чрез прессование сотов, при несколько возвышенной температуре, бывает бурого или красноватого цвета, мутен, не скоро превращается в зернистую массу, горьковат, кисел, содержит много воска и остатки пчел, легко приходить в брожение и пенится. Капля меду, в который подмешана вода, будучи опущена на тарелку, расплывается и тем обнаруживает обман. Мед, подмешанный крахмалом, при нагревании дает густую мутную массу, тогда как, будучи в чистом состоянии, он превращается при этом в совершенно прозрачную жидкость. Кроме того, вода не вполне растворяет такой мед.

\subsection{Уксус.}
Хороший, неподдельный уксус должен отличаться чистотой и прозрачностью; вкус его должен быть кислый, но не горьковатый, и тонкий запах его должен быть освежающий и приятный. Оставленный в покое, в хорошо закупоренном сосуде, он не должен даже по истечении нескольких недель давать осадка или делаться мутным. Самый лучший уксус есть винный (ренсковый). Он приготовляется из окисленного, чистого, прозрачного вина, чрез смешение его с уксусом и окисление такой смеси в больших бочках. Вкусом и ароматом он несколько напоминает то вино, которое было употреблено на его приготовление. Чем слабее уксус. т.~е. чем менее содержится в нем, так называемой, уксусной кислоты, тем легче появляются в нем серые студенистые клочья и развиваются в большом числе, так называемые, уксусные червячки. Вообще же признаки нечистого и поддельного уксуса обнаруживаются в малом содержании в нем уксусной кислоты, в темном его цвете, в остром вкусе, в значительном остатке при выпаривании его и пр. Для придание уксусу обманчивой крепости, иногда прибавляют к нему отвара или настоя каких-нибудь растительных веществ, имеющих остро-горький вкус, в особенности стручкового перца. Такой поддельный уксус производить в гортани и на языке жгучее ощущение и уже по одному этому признаку может быть легко отличен от действительно хорошего уксуса.

\subsection{Поваренная соль.}
Чистая поваренная соль должна быть белого цвета, тверда, чиста на взгляд и состоять из отдельных блестящих кубиков (тогда она не изменяется на воздухе), без запаха, солено-прохладительного вкуса, ни малейше не сера. Брошенная на раскаленные уголья, трещит от растрескивания кубиков и вследствие выделение находящейся в ней воды. Хорошая поваренная соль должна растворяться без малейшего остатка в трех частях холодной и горячей воды. Нечистота этой соли обнаруживается ее серым цветом, плохою кристаллизацией, расплыванием на воздухе, горьковатым вкусом, мутностью раствора, остатком, получаемым после выпаривания ее раствора на огне, и~т.~д. Между русскими сортами поваренной соли особенною чистотой отличается соликамская и дедюхинская, между немецкими~--- люнебургская. Большею частью недостатки нашей поваренной соли зависят у нас от небрежной ее промывки и просушки.

Заведя речь о поваренной соли, не лишне сказать, что в петербургском английском магазине и вообще во всех лучших наших фруктовых лавках продается столовая соль самого высшего качества, под названием ливерпульской соли, по 25 к. пакет (немного побольше фунта). Товар этот в продаже идет в синих бумажных пакетах с настоящим английским ярлыком. Никто не подозревает, что, покупая эту необычайно белую и нежную столовую соль, покупает такую соль, которая никогда в Англии не бывала, а есть настоящая русская соль, изготовляемая только особенным образом способами усовершенствованного рафинирование, известными некоему г-ну Соколовскому, владельцу той шоколадной фабрики на Лиговке, которая снабжает лотки с простонародными лакомствами тем шоколадом самого низкого сорта, который в таком огромном употреблении между нашими простолюдинами не только в Петербурге и Москве, но во всей России, хотя, правду сказать, шоколад Соколовского далеко не шоколад, a какие-то сахарные, и то из самого низкого сорта сахара, лепешки, чуть-чуть помазанные раствором жидкого шоколада.~--- Что касается до столовой ливерпульской соли, петербургского изготовление Соколовского, то остается сожалеть, что г. Соколовский, достигнув до такого высокого совершенства в приготовлении этого продукта, не заменит, наконец, заимствованный им иностранный ярлык русским, и не прекратит эту столь давнишнюю мистификацию публики, к сожалению, все еще падкой ко всему иностранному.

\subsection{Горчица.}
Ежели в сухом виде, то преимущественно покупается всеми, по справедливости, знаменитая <<сарептская>>. Ее можно иметь во всех бакалейных и фруктовых лавках, но в особенности с полною безопасностью можно ее приобретать в большой сарептской лавке в доме Сарептского общества, подле здание Почтамта, по Конногвардейскому переулку. В больших фруктовых лавках, равно как в магазинах Фохтса, можно с безопасностью покупать приготовленную жидкую горчицу, английскую и французскую; но надобно остерегаться от горчицы, так называемой, <<московской>>, состоящей из какой-то невозможной и непонятной амальгамы сомнительного свойства ингредиентов, продаваемой во многих <<зеленных>> лавках за французский товар, потому только, что для продажи этой дряни употребляются французские сосуды, украшенные нарочно выписанными из Парижа ярлыками знаменитой тамошней горчичной фабрики Maille.

\subsection{Оливковое масло.}
Оливковое масло, получаемое нами из южной Италии, уступает в доброте тому маслу, которое получается из Марсели и известно под названием прованского. В последние годы, благодаря удобству железных дорог, мы чрез Триест и Вену имеем этот товар и в зимнюю пору. К сожалению, однако, это оливковое масло, проходящее чрез руки триестских местных торговцев, всегда ими смешивается с семянным маслом, известным под названием <<сезамского>>. Чрез такую смесь или фарлатацию оливковое масло приобретает некоторую горечь, что далеко не принадлежит к числу совершенств оливкового масла, которое должно быть на вкус без всякого постороннего вкусового свойства, подобно чистой и хорошей воде, и таким-то образом оно узнается в своем достоинстве покупателем. Чистейшее прованское масло можно получать в фруктовом магазине г. Поэнта, в Апраксином дворе, именно потому, что владелец этого магазина находится в непосредственно прямых отношениех с марсельскими маслопроизводителями.

\subsection{Ваниль и корица.}
Ваниль в торговле двух отличительных сортов: мексиканская и бурбонская, т.~е. с острова Бурбона. Мексиканская ваниль имеет преимущество перед бордосской или бурбонской. В свежем виде оба сорта в качествах не различаются; но французская не выдерживает продолжительной лежки. Распределение стоимости ванили чрезвычайно многосложно: она зависит, во-первых, от степени длины палочки, доходящей от 4–8 дюймов; от цвета отдельных стручков, которые идут от темновато-рыжего до черного, как смоль; от степени зрелости, какая обнаруживается тем маленьким инеем (givre), род седины, который есть признак высшего качества ванили в глазах знатоков, принимаемый, однако за плесень невеждами, не смыслящими дела. Иней этот держится на ванили кратковременно, и исчезновение его умаляет ароматичность ванили, почему палочка ванили влажная, как-бы с потом, бывает достойнее той, которая потеряла свой иней или свою седину.~--- У нас в Петербурге в иных лавках, как русских, так и немецких, в особенности немецких, можно встретить ваниль из Гамбурга, замечательную своею дешевизною, выходящею на очень дорогое, потому что эта ваниль из Гамбурга есть ничто иное, как ванильная древесина, продаваемая услужливыми немцами-евреями петербургским доверчивым барыням-экономкам и оставившая свои соки в Германии на тамошних фабриках, изготовляющих эссенции и ароматические масла.

\subsection{Шоколад.}
Шоколад, как известно, делается из какао. В петербургской торговле 4 сорта какао, а именно: 1) Гуаяквиль, 2) Караказ, 3) Багие и 4) Тринитат. Самый распространенный сорт какао~--- это первый, Гуаяквиль. Это самый превосходный сорт, который советуем требовать в лавках, при покупке шоколада для особенно тонких кремов, пирожных, желеев и пр.; что же касается Караказа, который выше всех поименованных остальных трех сортов, то он светлее и нежнее их цветом, гуще и жирнее на вкус,~--- главное, имеет больше всех их питательных свойств, рекомендуется усиленно врачами выздоравливающим и вообще натурам слабым; но он страшно дорог и потому в шоколадную фабрикацию не употребляется. Ежели вы хотите купить для вашего стола или вообще для вашего употребление шоколада в чашках,~--- советуем спрашивать шоколад с этикетами известных фирм, каковы: в Москве Сиу, а в Петербурге: Ландрин, Виоле, Борман, Конради и Линуар.

\subsection{Миндаль.}
Миндальные орехи. под названием <<принцесские>> (amendes princesse) № 1 тонкокожие, a № 2 в соломке (à la dame). Орехи эти № 1 привозятся в бочках, a орехи № 2~--- в ящиках. Лучшие орехи французские, № 1, а похуже~--- португальские, № 2. Последние отличаются от французских миндальных орехов не столько тем, что крупны, сколько грубостью корки. Покупатель, увлекаясь величиною плода, не замечает толстой корки и покупает товар не высокого качества по довольно порядочной цене. Совет~--- не увлекаться величиною плода. Впрочем, самым лучшим привозным миндалем считается мессинский миндаль.

\subsection{Кофе.}
В петербургской торговле встречаются 6 сортов кофе, которые разделяются по чистоте запаха, крупности зерна и приятности вкуса таким образом: 1) Куба или кубский, 2) Цейлонский; 3) Порторико, 4) Ява или явский, 5) Лaгyaйpa и 6) Бразильский.~--- Истинные знатоки и любители кофе во всей Европе отдают преимущество <<явскому>> кофе; но в Петербурге, а потому и во всей России, он не столько, в ходу и в употреблении, как другие сорта, в особенности <<кубского>> и <<цейлонского>> кофеев. При этом, нельзя не заметить, что покупательницы кофе в особенности хлопочут иметь кофе очень круглый и очень мелкий, не подозревая, что слишком круглые кофеины~--- результат искусственного приема, потому что в фунте, например, плоского кофе непременно найдется несколько круглых зерен, оставшихся в нем по недосмотру того, кто занимается просейкою кофе на грохоте или сите. Это до крайности наивный фортель продавцов кофе, подметивших страстишку многих гоняться за круглым кофе, который в сущности ничем не лучше продолговатого.~--- Еще совет: не соблазняться, так называемыми мокским кофе или мока, какой, по более или менее доступной, но все-таки очень еще высокой цене, продается в фруктовых лавках в шикозных и красивых 20-ти-фунтовых плетушках. Истинного аравийского неподдельного <<мока>> в Петербурге бывает в год не более 250 пудов, стоимостью от 16–17 р. фунт, а весь этот аравийский и левантский кофе, каким мозолят глаза так беззастенчиво доверчивым покупателям наши коммерсанты-шарлатаны, есть ни что иное, как гамбургское изделие. В Гамбурге непочатый край всевозможных торговых фокусов и переделок, соперничающих с нашими шкловскими и бердичевскими метаморфозами. Такими образом, Гамбург дает Петербургу ежегодно до 1000 пудов своего <<мока>>, обморачивая петербуржцев, восхищающихся теми 20-ти-фунтовыми плетушками, очень, правда, щеголевато сделанными, в каких помещается гамбургский мока, состоящий из смеси подбора различных кофейных сортов. В этих сортах всегда можно найти некоторое количество зерен, схожих по наружному виду с настоящим мокским кофе, который, главное дело, отличается в неподдельном, как бы <<чистокровном>> своем виде чрезвычайною нечистотою укладки и отсутствием сортировки, тогда как этот исправляющий его должность <<мока>>, и продаваемый рублей за двенадцать 20-ти-фунтовая плетушка, отличается изящностью подбора зерна к зерну и аккуратнейшею, самою тщательною, чисто немецкою укладкой. Не нужно большой проницательности, чтобы видеть и понять обман.

\subsection{Саго.}
Особое крахмальное вещество, добываемое в Индии, Африке и Южной Америке из сердцевины некоторых видов тропических пальм, называемых саговыми пальмами, и известное в торговле под названием саго, бывает в чистом и натуральном своем виде белого, бурого и красного цвета. Зернышки его должны быть очень невелики, полупрозрачны, трудно раздавливаются между пальцами и трудно растираются в порошок без запаха и сладковатого слегка вкуса. Но этот товар довольно редок в торговле, а потому и дорог, не дешевле 25 коп. фунт.

Из русских фруктовых лавок можно найти это лекарственное снадобье, впрочем, употребляемое и в кухне на супы, пюре, пудинги и разные сладкие, весьма тонкие блюда, только у г. Елисеева, да еще у И. А. Поэнта, в Апраксином дворе, подле Коммисаровской школы; а то надо обращаться или к известной фирме бакалейно-аптекарских товаров Штоля и Шмидта, или в самые большие аптеки~--- Невскую и Аничковскую.

Обыкновенное же продажное саго, сбываемое в большинстве случаев за саго настоящее, состоит просто из крупинок картофельного крахмала, нарочно для этого приготовляемых по особому способу. Это картофельное саго, изготовляемое в огромных количествах в России на картофельно-мучных заводах, легко отличить от саго настоящего тем даже,~--- как бы ни было оно искусно подделано,~--- что оно удобно раздавливается между пальцами, что, будучи облито холодною водою, тотчас дает жидкость, немедленно окрашивающуюся от действия на нее иодной тинктуры (которую всегда беспрепятственно можно достать в аптеке) в синий цвет~--- признак, что исследуемый товар действительно содержит в себе картофельный крахмал,~--- и, наконец, что оно расплывается в горячей воде.
