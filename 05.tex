\section{НАВАРЫ ИЛИ БУЛЬОНЫ: МЯСНЫЕ, РЫБНЫЕ, ОВОЩНЫЕ, ГРИБНЫЕ И ПР. И ИЗ НИХ: СУПЫ, ПОХЛЕБКИ, ЩИ, БОРЩ, УХА И ПР. (СКОРОМНЫЕ И ПОСТНЫЕ)} % отдел 5

\subsection*{Общие замечание о приготовлении настоящего, хорошего мясного бульона с примерным рецептом самого лучшего и правильно приготовленного супового бульона (№ 1).}\addcontentsline{toc}{subsection}{Общие замечание о приготовлении настоящего, хорошего мясного бульона}

Предмет этот, — бульон или мясной навар, — верно трактован <<Петербургскою хозяйкою>> в бывшем журнале <<Русская Хозяйка>> (№ 1, 1861 г.). Она говорит справедливо, что всякий самый плохой повар и всякая кухарка вполне твердо уверены, что умеют варить бульон, от того только, что положат в воду кусок говядины, накрошат несколько кореньев, дадут кипеть как попало, и, накрыв кастрюлю крышкой, не заботятся о том, что в ней делается, до тех пор, пока наступит минута приготовить суп, который они бессовестно подкрашивают жареными кореньями, подправляют вкус в избытке спаржей или цветной капустой и подают на столь нечто довольно приятное на вкус, во вовсе не имеющее основой того здорового, питательного навара, который подкрепляет и поддерживает силы. Таковы бульоны общественных столов. Мы еще указываем здесь на поварих, имеющих уже достаточно опытности, чтобы замаскировать водянистость бульона разными известными в поваренном искусстве средствами; но есть кухарки, нанимающиеся в порядочный дом с бесстыдною самоуверенностью и потчующие злосчастных господ вместо бульона какой-то сальной, мутной водицей, в которой плавают коренья и крупа, не придавая ей никакого вкуса. А между тем, в расходной своей книге барыня несколько раз в неделю читает: бульонной говядины 6 или 8 ф., что в переводе на денежный язык означает около рубля сер. на одно мясо, в особенности, если кухарка объявляет, что берет всегда первый сорт. Иные барыни не имеют расходных книг, а закупают провизию сами; но в этом случае они только пользуются тем преимуществом перед первыми хозяйками, что знают наверно, сколько они купили мяса, и уверенные, что его весьма достаточно для хорошего бульона, с понятным огорчением видят на столе своем весьма невкусную сальную водицу. Кухарке делается выговор она обижается, говорит, что генеральша Т., уехавшая за границу, и покойная баронесса Р., у которой она жила чуть ли не десять лет, всегда благодарили ее за вкусный бульон, а что здесь такие уж капризные господа, и пр. и пр. Затем следуют слезы оскорбленной невинности, или грубости вспыльчивой природы; и все это кончается разлукой хозяйки с кухаркой. Но это все-таки нисколько не исправляет бульона: является новая прислуга, превосходящая самоуверенностью свою предшественницу; подают суп: он имеет довольно аппетитную наружность, сальности в нем нет; но пар, выходящий из миски, не разносит по комнате мясного аромата, а ощутителен только от него сильный вкус и запах моркови, репы, лука; питательного же здорового осмазона и в помине нет. <<У тебя бульон был очень слаб,>> говорит на другое утро барыня своей кухарке, <<а ведь я купила кусок в 10 ф. лучшего огузка!>> — <<Слаб!?>> восклицает кухарка, всплеснув с ужасом руками. <<Помилуйте, сударыня, он застыл словно желе, а говядина разварилась в мочалку!>> Действительно хозяйка убеждается собственными глазами, что слитый в горшок бульон имеет вид студня, прекрасный кусок огузка, ею самой купленный, превратился в какую-то кашу, а между тем застывшая студенистая масса, быв распущена на огне, имеет водянистый вкус, который, однако только служит к убеждению новой кухарки, что ей говорили правду и что барыня ужасно капризна. Новые неудовольствия с обеих сторон, а подаваемые на стол супы все-таки делаются каждый день хуже. Что же тут за тайна и как горю пособить? Очень нетрудно, отвечаем мы, если хозяйка согласится взять на себя труд один раз только наблюсти лично за варкой бульона и во всем следовать наставлением нашей далеко не вчерашней опытности. Словесные указание кухарке, просьбы и надежда на ее совестливость — ни к чему никогда не послужат.

Бульон можно варить из разных частей говядины: ссека, бедра, огузка, филея, грудины и даже лопатки\footnote{Малознакомый со всею этою мясною терминологией найти могут ее в предыдущем описании всех сортов мяса, помещенном в I Отделе этой книги и иллюстрированном рисунками в тексте.}. Последняя считается вторым сортом и не должна быть подана особенным кушаньем на стол в виде отварной говядины, известной у французов под названием bouillie, потому что мясо ее сухо и жиловато; но зато, имея мозговую кость, лопатка делает бульон наварнее; только находим не излишним предупредить молоденьких хозяюшек, чтобы, покупая эту часть мяса для бульона, они избегали той половины, где кость имеет вид кругловатой шишки, а приказывали бы отвешивать от того места, где она проходит сквозь мясо в роде сквозной трубки, наполненной мозгом. Мясники предпочитают отрубать первую половину, потому что в ней много веса и мало мяса, мозгов же самое незначительное количество. Ссек и бедро дают мало крепости бульону, грудина хрящевата и жирна, и годна в особенности для щей и борща; но самая лучшая и выгодная часть есть огузок, который мясист, сочен, мягок и употребляется весьма разнообразно. И от него советуем брать ту часть, которая ближе к хвосту, а не самую середину, заключающую весьма толстую кость. Также должно заметить, что брать малое количество огузка не выгодно, во-первых, потому, что тогда необходимо его разваривать до <<мочалы>>, как выражаются кухарки, а во-вторых, — небольшой кусок огузка, подходя слишком близко к хвосту, имеет более жира, чем мяса; остальное же занято костью. И, таким образом, говядина 1-го сорта, заплаченная дорого, не доставляет собою ни хорошего бульона, ни вкусного блюда. И так, покупая огузок для бульона, не берите никогда менее 10 или 12 ф., которые образуют весьма порядочный кусок, суживающийся к жирному концу и имеющий плотное, но самое сочное мясо.

Жир на нем должен лежать довольно толстым слоем и иметь цвет чухонского масла, но отнюдь не сероватый, что доказываете малоценность туши. Мясо хорошо откормленного быка бывает темно-алого цвета с белою прорезью в роде жилок, и когда варится, то не разделяется слоями, а остается твердой массой, только сочно пропитанной бульоном. Многие хозяйки держатся того мнение, что должно в кастрюлю наливать две бутылки воды на каждый фунт мяса; но многократный собственный опыт доказал нам, что этого нельзя брать за общее, без всякого исключение, правило, и подобная пропорция может быть соблюдаема только в том случае, когда мясо берется собственно для одного бульона, как делают с лопаткой, и когда оно назначено к развариванию вполне. При употреблении же большого куска огузка, стараются только выбрать для него очень просторную посудину и наливают воду так, чтобы она с излишком покрыла мясо, имея в виду убавление ее во время кипения.

Держать говядину в воде отнюдь не позволяйте кухарке или повару. Многие нз них имеют пагубную привычку опустить в лохань с холодною водою мясо, принесенное с рынка, и, занявшись другим делом, дают мясу мокнуть, что отнимает у него естественную питательность, изменяет цвет и покрывает его вредною слизью. Купленную говядину велите облить раза два холодною водою, обчистить слегка рукою, еще раз окатить, потом обтереть сухим полотенцем и тотчас опустить в кастрюлю, которую немедленно налить холодною водою. Плита должна быть предварительно уже нагрета, чтобы бульон закипел как можно скорее. Иные поварихи имеют привычку нагревать плиту в ту минуту, как ставят кастрюлю на огонь, но это весьма неправильно: чем скорее мясо пустит свой сок в воду, тем вкуснее будет бульон; в этом случае под плитой должен быть сильный жар, и кастрюли накрывают крышкой.

Многие хозяйки определяют 4 часа на полную варку бульона; но и с этим мы не согласны, потому что изготовление его зависит от толщины и качества куска. Главное правило этого важного процесса, составляющего основу кулинарного искусства, заключается в том, чтобы ваш бульон кипел белым ключом, не перемежаясь, до тех пор, пока не набежит на нем пена, которую должно снимать очень часто и тщательно. От этого первого приема зависит почти вполне достоинство бульона. Крышка должна оставаться на кастрюле недолго, и ее снимают, когда кипенье делается через-чур сильно, оттого что скопляющийся на крышке пар, стекая в бульон, придает ему водянистость. Когда вся пена снята и бульон уже наполнился янтарными струйками, снимите кастрюлю с пыла и, отставив на более прохладное место, предоставьте бульону кипеть исподволь, но отнюдь не переставая. В это уже время должно опустить в кастрюлю приготовленные коренья, соль и поджаренную луковицу. Если вы замечаете, что кусок говядины высунулся из воды, то поверните его другой стороной, но никак не дозволяйте кухарке доливать кастрюлю кипятком, как иногда они себе это позволяют. В этом виде бульон должен кипеть почти до самого обеда, и за полчаса до него можно отлить некоторое количество бульона в особую кастрюлю для того, чтобы сварить в нем коренья, изрезанные фигурно, и спаржу, припасенную собственно для супа, изготовление и разнообразная заправка которого составляют всегда особую статью, весьма немаловажную в поваренном или кухонном хозяйстве.

Если, испробовав мясо вилкой, вы найдете, что оно достаточно сварилось и имеет везде надлежащую мягкость, то не оставляйте его в бульоне, где оно можете совсем развариться, а выньте на чистое блюдо и опустите снова в кастрюлю только за минуту до того времени, как эту говядину приходится подавать с гарниром, подливками или хреном. Солить бульон должно осторожно, тем более, если он изготовляется не на один день, как это обыкновенно делается в семейных домах, где заготовляют его на несколько суток; потому что варить каждый день свежий бульон очень невыгодно, во-первых от того, что изготовление его требует большего огня и, следовательно, много горит дров под плитой, а также и потому, что, приготовив его дня на два или на три, вы берете гораздо больший кусок мяса, a чем он больше, тем может быть лучше качеством, и тем разнообразнее его польза в хозяйстве. Должно заметить, что, имея в виду подавать на стол отдельно кусок говядины, варившейся в бульоне, кость из нее вырезывают, пока она еще сыра, и если она мозговая, то ее обвязывают кисейкой, чтобы мозг не вываливался: его многие любят кушать горячим, перед тем, чтоб подали на стол суп. Самое же мясо свертывают круглообразно, в роде большой колбасы, и, перевязав тонкими бечевками, опускают в кастрюлю. Таким образом, оно сохраняет приличную форму и не представляет на блюде безобразно-отвратительной массы с клочьями и высунувшеюся костью, что, конечно, далеко не способно возбуждать аппетит.

В начале этих сведений мы упомянули о некоторых проделках кухарок с бульоном, но не объяснили, в чем именно состоят их ошибки и от чего происходят недостатки изготовляемого ими бульона. Не дав кипеть воде белым ключом, они отнимают у мяса всякую возможность выпустить свой натуральный сок; его жир только распускается постепенно, отчего бульон салится, делается мутным и получает пренеприятный вкус. Заметив это, кухарка заставляет вдруг сильно кипеть навар, — но говядина уже утратила свою сочность, она преет, распадается, а вкуса не дает.

Другие кухарки, напротив, без меры и толку дают бульону выкипать, и чтоб барыня не заметила и не сделала выговора за малое количество вышедшего бульона из известного веса говядины, они прибавляюсь кипятку, что окончательно портит навар, давая ему водянистый вкус, потому что даже полчашки воды, влитой в бульон после того, как он уже кипел, наносит ему неисправимый вред, и уже как ни станут его подправлять спаржею, цветною капустою, петрушкою, да укропом, — крепость и целебная сила его утрачены безвозвратно, а дорого заплаченный, хороший кусок мяса пропал даром, превращенный в какую-то непитательную кашу невеждой-кухаркой. Иные из них кладут в бульон кости телячьи, что вовсе недурно, и пользуясь их естественною клейкостью, заставляющею, действительно, навар застыть в роде студня, они стараются доказать хозяйке доброкачественность сваренного ими бульона; но клейкость эта не дает мясному навару крепости и, при незнании всех вышеописанных правил, бульон ваш все-таки будет безвкусен, водянист, солон, а кусок говядины, напр. в 10 ф., заплаченный по нынешним ценам от 1р. 50 до 1р. 70 к., пропадет ни за грош без всякого употребление и без малейшей пользы для хозяйки, тогда как он же, сваренный, как выше описано, выйдет из кастрюли сочный, вкусный, окаймленный своим янтарным жиром. Вы подадите его на круглом блюде, окруженный гарниром, т. е. вареными луковицами, картофелем, репою, морковью и пр.; на другой день, нарезав красивыми ломтями, вы его же разогреете в луковом соусе на красном вине, которого рецепта находится в этой же книге, в отделе, посвященном всевозможным соусам и подливам, или приготовите к нему вкусную подливку на бульоне из шампиньонов; если же после всего этого еще от этого куска осталось несколько говядины, вы превратите ее в винегрет. И таким образом, мясо, заплаченное дорого, доставило вам: 1) хорошего бульона тарелок 10 или 12; 2) вкусное мясное блюдо (бульи, bouillie) не уступающее французским entrées, не смотря на свою явную простоту, и 3) холодное блюдо, весьма красивое и вовсе недорогое, при порядочном устройстве кухонного хозяйства.

Кроме того, в небольшом, средней руки хозяйстве, где прислуги немного и большею частью женская, как это нынче вводится во многих весьма порядочных петербургских и даже провинциальных домах, из остающейся от той же говядины кости, с некоторым количеством мяса и жира, можно изготовить для прислуги прекрасную похлебку, подправив ее салом, картофелем и какою-нибудь крупою.

Всякая, истинно расположенная к экономии хозяйка согласится, что описанный нами бульон, при своих несомненных достоинствах, есть выгодное и крайне полезное поваренное приобретение, которое, составляя прочную основу хорошего стола, имеет при том и целебные свойства в медицинском отношении. Сознавая всю необходимость правильной варки бульона, мы с намерением и поместили статью эту здесь, как рецепта (№ I), от познание которого зависит большею частью успех последующих блюд и который в то же время есть основание едва-ли не всего кулинарного искусства.

А как <<ум хорошо, два же еще лучше>>, то вовсе не лишнее после этих замечаний о <<бульоне>> выслушать о том же предмете и г-жу Авдееву, при чем вы убедитесь, что как петербургская, так и провинциальная хозяйки, рассуждая об одном и том же предмете, — не повторяются и дают вам наставление своеобразный, а с тем вместе приводящие все-таки к одной цели, именно к тому, чтобы основа хорошего, здорового и вполне полезного стола — бульон, т. е. мясной отвар, отличался настоящим совершенством.

Г-жа Авдеева также говорит, что для бульона лучше брать такие части мяса, где более мозговых костей\footnote{Всего лучше брать кости отъ ссека, тонкого и толстого края и т. д., как было уже выше сказано.}.

\z{Бульон № 1.}
Бульон вообще всего выгоднее варить на два дня. Для семейства из 3 человек выбрать кастрюльку вместимостью около 12 глубоких тарелок, и уже варить в ней постоянно суп. Положить в нее от 3 — 5 фунт. (можно, конечно, и больше) говядины, налить от 9 — 10 глубоких тарелок холодной воды\footnote{Так как тарелками мерить неудобно и затруднительно, то советуемъ поступать следующим образом: налив от 9—10 глубоких тарелок воды, смерить высоту их в кастрюле чистою палочкою, сделать на ней знак и долить еще глубокую тарелку воды. Затемъ нужно только наблюдать, чтобы бульон укипелъ до знака. Такимъ образом, впоследствии не нужно наливать тарелками, а просто налить до знака и еще тарелку.}, не доливая кастрюльку до краев пальца на три, поставить на плиту, варить на небольшом огне, снимая дочиста пену. Когда бульон начнет кипеть, положить соли и кореньев: 1 целую порейку, 1 корень петрушки, пол сельдерея, 1 очищенную луковку средней величины, достаточное количество соли (2 столовые ложки без верху). Дав хорошо увариться на легком огне, процедить сквозь салфетку или сито. Одну половину его сливают и сохраняют в прохладном месте до следующего дня, в другую же кладут мелко-изрезанный или точеный картофель, морковь и прочие приправы, ставят снова на огонь и варят до тех пор, пока не будут готовы эти последние. Варить суп на 2 дня весьма выгодно, потому что таким образом соблюдается экономия в мясе и вместе с тем и в дровах.

Как мы уже сказали, для обыкновенного бульона достаточно взять на два дня от 3–5 фунт. мяса, преимущественно костей; конечно, смотря по желанию, можно взять 6 и 10 фунт.; но это, по нашему мнению, лишнее. Для щей и пюре требуется, как увидите ниже, еще менее мяса (от 1.5—2 фунт. на 1 день).

Если же бульон варится на 1 день, то нужно брать более половины показанной пропорции (от 3–4 фунт.), из чего видна выгода варить суп на 2 дня.

Как мы уже выше сказали, пропорция для хорошего бульона — на фунт говядины 2 бутылки или 3 глубокие тарелки воды. На суп или щи для 4 особ достаточно 2.5 фунта. Говядину для супа должно брать от грудины или с мозговыми костями: ссека, бедра, огузка, костистых частей края, полубедерка, лопатки, грудинки и проч. Кости от голяшек годятся для бульона, особенно с прибавкой какого-либо другого мяса. Если суп варится из телятины или баранины, тогда воды надобно наливать меньше. Для приготовления супа из курицы, лучше брать старую: — из нее будет наварнее бульон; но если кто хочет иметь вкусный суп, то для этого лучше молодые куры. Также, кто желает соблюсти экономию, можно наливать воды больше вышесказанной пропорции и дать мясу совершенно вывариться; тогда суп будет уже не так вкусен, хотя и наварен. Настоящую пропорцию варение супа можно узнать по тому, когда мясо или птица совершенно поспеют. Где большое семейство, там не нужно для бульона, употребляемого в другие кушанья, варить особенно говядину; часто бывают остатки говядины, телятины и птицы, когда делают фарш для начинки пирогов, паштетов или вырезывают мякоть для других кушаньев, то кости и обрезки можно употребить на бульон. Остатки жареной телятины, баранины, говядины и птицы хорошо класть в бульон, обрезав с них лишнее мясо; от этого бульон бывает очень вкусен и наварен. Такой бульон можно засыпать лапшей, вермишелью, макаронами или приготовить с клёцками; также, положив в бульон кореньев, засыпать крупами: перловыми, манными или рисом; отпуская, посыпать рубленой петрушки и укропу. Подают этот бульон с поджаренным белым хлебом и гренками из гречневой каши.

Суп будет чист и вкусен только тогда, когда его варить на легком огне так, чтоб он кипел только с одного боку, снимать как можно чаще накипь и процедить его чрез частое сито или салфетку. Если же он не чист, то следует процедить, дать ему слегка остыть, положить в него 2–3 белка, размешанные с столовою ложкою воды, и дать слегка кипеть; когда бульон очистится и белки поднимутся — процедить; если же нет, то положить кусок льду и вскипятить снова.

Если же, например, варятся щи или обыкновенный борщ и в них хотят подать говядину, то хорошо брать грудинку.

Если на второе блюдо хотят подать хороший кусок разварной говядины, то нужно взять от 3–5 фунт. ссека или бедра с костью и сварить суп на два дня; мякоть потом отрезать и подать под каким-нибудь соусом или гарниром. Если же вовсе не нужна разварная говядина, то всего выгоднее варить суп из голяшек или булдышек: так называется часть воловьей ноги, от колена до ступни. Булдышки весьма дешевы (копеек 20–30 пара). В каждой из них от 3–5 фунт, и из одной булдышки можно сварить хороший суп на 2 дня для семейства из 4 человек.

Если к обеду хотят приготовить рубленые котлеты, фрикадельки и проч., то нужно взять фунтов 6 ссека, огузка, толстого края и т. д., вырезать фунта 1.5—2 мякоти, а из костей сварить суп.

Если на второе блюдо хотят подать жареную говядину — говядина жаркое обыкновенное, шморфлейш, зразы и т. п., то нужно взять мякотный кусок от ссека или огузка в 6–7 фунт., кости употребить на суп, а мякоть на жаркое.

Вареную говядину из супа можно употреблять на фарш для пирожков, начиненных блинов и на форшмак.

Суп можно также варить из обрезков и костей или остатков от жаркого, если таких обрезков наберется до 3 фунтов.

О чищении и приготовлении овощей, кореньев и проч. было уже сказано выше. Пропорция для супов назначена на 3 человека; от 6–8 человек — увеличить ее вдвое, от 10–12 человек — втрое и т. д. Лук, перец, лавровый лист и масло можно класть или не класть в супы, смотря по вкусу\footnote{Чем бульон выше достоинством, тем меньше надо всех этих приправ, характеризующих кухмистерскую стряпню.}.

При назначении кореньев, мы имели в виду небольшие коренья, а потому большую петрушку или морковку следует разделить на два или на три.

Если к обеду готовятся телятина, курица и проч., то кстати и обрезки от них следует прибавить к бульону.

\z{Бульон № 2. (Не из одной только говядины).}

(Пропорция на 8 человек).

Когда нужно будет иметь хороший бульон для супа и для разных других приготовлений, взять 6 фунт. говядины, заднюю ножку телятины, мякоть с нее должно обрезать, но чтоб мяса еще оставалось на костях, старую курицу и рябчика. Говядину изрезать, телячьи кости разрубить, курицу разнять на четверо, рябчика пополам, перемыть все хорошенько, положить в большую кастрюлю, налить 8-ю штофами воды, поставить вариться на небольшом огне. Когда начнет навар кипеть, пену снимать дочиста, посолить по пропорции, и, дав мясу увариться почти до спелости, положить по 2 корня петрушки, моркови и 1 корень сельдерея. Варить бульон на легком огне до тех пор, пока третья часть его укипит; тогда, процедив сквозь салфетку, употреблять для кушанья.

\z{Бульон № 3. (Не из одной только говядины).}

(Пропорция па 6 человек).

Взять 8 фунт. говядины, 4 фунта телятины, 1 или 2 старые курицы, вымыть хорошенько, положить в кастрюлю, налить водой, поставить на огонь. Когда бульон закипит, и пена будет дочиста снята, положить кореньев: пастернака, петрушки, моркови, сельдерея, порея, репчатого луку; варить бульон на слабом огне 6 часов. Если из телятины и из кур захотите сделать какое-нибудь употребление, то не давайте им перевариться, а вынув из бульона, откиньте на сито. Бульон же с мясом поварите подольше. Когда бульон уварится, процедить сквозь сито и снять жир.

\z{Бульон экономический.}

(Пропорция на 6 человек).

Если нет готового бульона для обыкновенного домашнего стола, взять часть говядины, фунта в 4, от ссека или толстого филея, вымыть, положить в кастрюлю, налить 5 бутылок воды, немного посолить и варить до спелости. Когда говядина поспеет, вынуть ее из бульона, облить каким-нибудь соусом, или приготовить панированную, а бульон употребить для супа; можно засыпать его вермишелью, лапшой, или, положив кореньев, подавать с белым хлебом, поджаренным в масле. Приготовляя в тот же день кушанье из курицы, ее можно отварить в этом же бульоне; от этого будет крепче бульон.

\z{Бульон из костей и разных остатков.}

Приготовляя фарш для начинки паштетов и пирогов, обрезав мясо с костей, разбить кости обухом топора, а жилы и перепонки, вырезанные из мяса, приготовляемого для фарша, вместе с костями, положить в кастрюлю, и если есть остатки жаркого, дичины, телятины или говядины, то их положить туда же в кастрюлю, налить по пропорции водою, варить на малом огне, чтоб бульон ровно кипел. Когда бульон довольно уварится, процедить сквозь сито, дать отстояться, потом чистый бульон слить, и если он не нужен в тот день для употребления, вынести на погреб.

\z{Бульон необыкновенной крепости в герметической посудине.}

Этот бульон получается чрез варку усиленнейшим образом мяса в герметических бульонных кастрюлях\footnote{В Петербурге эти герметические кастрюли получать можно во всех хороших жестяных немецких лавках, преимущественно по Казанской улице (бывш. В. Мещанской). На 1 ф. цена кастрюле 3 р., а на большее количество мяса дороже, рублей до 10 штука.}.

Бульон этот приготовляется так: фунт хорошей суповой бескостной говядины, изрезанной сырою довольно мелко, складывается в эту герметическую жестяную кастрюлю, которая, будучи завинчена, за ушко вешается на поперечную палочку над котелком или большою кастрюлею, которая наполняется водою и ставится на сильный огонь плиты часов на 5. По истечении этого времени, от говядины остаются одни волокна, а самого наикрепчайшего бульона получается большая глубокая тарелка.

\z{Примечание, относительно всех родов мясных бульонов для приготовления из них всяких супов и похлебок.}

Все вышеописанные бульоны можно разнообразить следующим образом:
\begin{enumerate}
	\item Чистый с пирожками.
	\item С различными фрикадельками.
	\item С клецками.
	\item С вермишелью.
	\item С макаронами.
	\item С рисом.
	\item С рисовой кашей.
	\item С рисовыми пирожками.
	\item С точеными кореньями и пирожками, и без последних.
	\item С капустой и кореньями.
	\item С манной крупой.
	\item С перловой крупой.
	\item С саго.
	\item С кашей из смоленских круп.
	\item С морковью и надвое разрезанными листьями шпината (1/8 фунт. шпината и 1 морковь).
	\item С греночками.
	\item С лазанками.
	\item С ушками.
\end{enumerate}
 
\z{Рыбный бульон.}

Взять луку, моркови, пастернака и петрушки, по 2 корня, вычистить, нарезать кружечками, вымыть, положить в кастрюлю, прибавить кусок коровьего масла, величиною с куриное яйцо, поджарить коренья; потом положить в кастрюлю к кореньям мелкой рыбы, какая случится: карасей, окуней или ершей, налить воды столько, чтоб она покрыла рыбу. Когда начнет кипеть, снимать пену, посолить, положить понемногу (гвоздики)\footnote{Все приправы, которые можно класть или не класть по желанию, напечатаны в скобках.} лаврового листа, горошинами перцу, несколько кружков лимона, дать кипеть ключом час; затем процедить сквозь салфетку.

\z{Постный бульон}.

Взяв фунт сушеного зеленого гороху, вымыть, налить холодной водой, поставить на плиту, закрыв кастрюлю крышкою; когда горох хорошо разварится, жидкость слить, горох истолочь, положить опять в кастрюлю, налить отваром, в котором он варился, прибавить еще 2 бутылки воды, положить по 1 корню моркови и петрушки, 2 луковицы, натыканные гвоздикою, зеленого укропу, дать кипеть час, а потом процедить сквозь частое сито.

\z{Исправление окисшего бульона.}

Часто, при неимении хорошего ледника, бульон не может сохраняться — и портится. Вот верное средство исправить весьма хорошо окислость бульона, происходящую от возвышенной температуры того места, где иногда, по обстоятельствам, принуждены сохранять свою провизию хозяйки. На каждый фунт бульона положить щепотку углекислой соды, которая захватывает все кислые частицы, и тогда, дав бульону устояться, вскипятите его хорошенько: на нем набежит белая пена, производимая содовою кислотою, которую должно снять как можно лучше, — и тогда, до другого дня, бульон сохранится совершенно в свежем виде. Если почувствуете легкий вкус уксуса, положите опять немного соды, прокипятите снова — и всякая посторонняя кислота исчезнет из бульона совершенно.

\z{Голландский суп с пшеном.}

Взять телятины, от голяшки задней ноги, изрезать в куски; также взять фунт говядины, каплуна или курицу, налить водою, посолить, поставить на плиту; когда закипит, снять пену. Потом положить по 1 корню петрушки, моркови и 2 луковицы, варить на небольшом огне; дав вполовину увариться, процедить и, сложив опять говядину в кастрюлю и вылив туда же бульон, дать кипеть исподволь. Между тем взять чайную чашку рису, вымыть хорошенько, всыпать в суп, прибавить кусок чухонского масла, величиною с куриное яйцо. Отпуская на стол, телятину и говядину вынуть, а курицу разнять на части, положить в миску, вылить на нее суп и посыпать мелко изрубленным укропом или петрушкою. Иногда подают курицу отдельно, на блюде, или приготовляют из нее другое блюдо\footnote{Все рецепты блюд этой книги, где нет в скобках цифры, означающей пропорцию, на сколько особ готовится описываемое блюдо, — принадлежат г-же Авдеевой и назначены ею, как уже она упомянула в I отделе, на 3 персоны. В других же местах поставлены цифры в скобках: (5), (8), (4), (10) и пр., смотря на сколько персон назначено блюдо.}.

\z{Суп из рябчиков с пармезаном.}

Взять 2 жареных рябчиков, снять мясо с костей, истолочь кости в ступке, налить бульоном, поставить вариться, посолить, положить кореньев. Снятое с костей мясо изрубить мелко. Потом взять круглый белый хлеб или булку, срезать дно, выбрать мякиш, размочить в бульоне, смешать с рубленым мясом, обжарить в масле, порубить еще немного, начинить этим фаршем хлеб, посадить в печь, в вольный дух. Между тем нарезать тоненьких ломтиков белого хлеба, подсушить в печи, усыпав дно суповой чашки сыром пармезаном, уложить на него ломтики хлеба, сверху опять посыпать сыром и, положив начиненный хлеб, налить сквозь сито бульона, в котором варились кости; закрыть, дать постоять 1/4 часа и вылить в миску остальной бульон.

\z{Белый суп.}

Взять хлуп от курицы, шейную часть теленка, несколько сваренных яичных желтков, миндалю, корок белого хлеба, размоченных в мясном отваре, обрать мясо с кости, и все это истолочь в иготи очень мягко; положить в кастрюлю и, налив хорошим мясным отваром, варить на небольшом огне. Когда все это уварится, протереть и пропустить все сквозь сито и подавать на стол, прибавив кусочек коровьего масла.

\z{Салатный суп.}

Из 5 или 6 кочанов салата кочанного надобно вынуть сердцевины, изрубить и отварить в воде. Между тем как салат, вынутый и положенный на решето, осякает, приставьте на огонь 1/4 фунта коровьего масла, обжарьте в нем несколько золотников муки крупичатой (хотя по 1.5 на кочан) и изрубленный мелко-на-мелко салат. На это налить мясной отвар и варить с добрые полчаса. Напоследок подбейте четырьмя яичными желтками, и подавайте с ломтиками белого хлеба, обжаренными в коровьем масле.

\z{Суп шпинатный.}

Отобрав шпинат и перемыв его, должно изрубить мелко, положить два яйца, сваренных в густую, и истертого белого хлеба. Распустите потом в кастрюле коровьего масла, положите в него шпинат, ложку муки, и, перемешав, обжарьте. Наконец разведите это мясным отваром, приварите, и пропустите сквозь сито в чашу, положив несколько гренков белого хлеба.

\z{Суп из яичных желтков.}

На каждый штоф мясного отвара разболтать ложкою 3 яичных желтка, и разведите мясным отваром еще. В кастрюле растопите кусок коровьего масла и подпалите в нем дотемна столовую ложку муки. Когда все это остынет, влейте туда же отвар или воду, подбитую желтками; при постоянном мешании вскипятите, приправьте приправами, ежели любите эти приправы, и наконец еще положите кусок коровьего масла. Суп этот подается с обжаренными в коровьем масле ломтиками белого хлеба.

\z{Суп с пудингом.}

Надобно вырезать часть мякоти от телячьей задней ноги, искрошить мелко с прибавкой жирной свинины и почечного сала; последних должно взять меньше, нежели телятины; изрубите их, и еще одну луковицу очень-очень мелко, приправьте солью и петрушкою; на каждый фунт этого крошева выпустите по 2 яйца, и вмешайте несколько истертого белого хлеба. Смесь эту положите в салфетку, вымазанную коровьим маслом; завязать и варить этот пудинг 3 часа в воде. Между тем приготовьте крепкий мясной отвар, подбейте его или яйцами, или в масле подпаленною мукою, положите сморчков, и если можно иметь, то и чиненых раков, а по крайней мере, вместо обыкновенного коровьего, употребите раковое масло. Пудинг этот положите в чашу, а на него вылейте суп.

\z{Суп с каштанами.}

Взять двух рябчиков, телячьих костей и фунт говядины, налить 4 бутылки воды, посолить, поставить вариться и снимать пену. Когда хорошо уварится, процедить и опустить рябчиков, разняв на части, в бульон. Потом очистить 2 горсти каштанов, положить в суп, варить еще полчаса. Отпуская на стол, подбить двумя яичными желтками.

\z{Суп с анчоусами.}

Выбрать из анчоусов кости, налить бульоном, пополам с виноградным вином. Когда довольно уварится, положить кусок, с куриное яйцо, свежего чухонского масла, дать раз вскипеть, процедить, выжать в бульон сок из одного лимона, посыпать немного (имбирю) перца. В этот суп можно положить фарш, или подавать с поджаренным белым хлебом.

\z{Суп кислый.}

Взять 2 стакана сметаны, смешать с 2-мя сырыми яйцами, влить 2 бутылки бульона, положить столовую ложку свежего чухонского масла, дать вариться полчаса. Положить в суповую чашку сухариков из белого хлеба и вылить на них суп.

\z{Суп из спаржи, с цыплятами и раковыми шейками.}

Фунт спаржи отварить до половины в воде с солью, потом, отрезав твердые нижние части, мякоть изрезать кусочками, в палец толщиною, положить в кастрюлю с ложкою коровьего масла, посыпать пряностями и обжарить. Двух цыплят разнять, каждого на четверо, посыпать солью и пряностями, обвалять в муке и обжарить в масле, налить бульоном, поставить вариться; положить по корню моркови, петрушки и сельдерея, нашинковать полосками. Когда бульон хорошо уварится, положить в него спаржу. Между тем очистить 2 десятка раковых шеек, опустить также в бульон и дать вскипеть один раз. Отпуская на стол, посыпать мелкорубленого укропу.

\z{Суп королевский.}

Взять фунт ветчины без жира, нарезать мелкими кусочками, искрошить по 2 корня сельдерея, петрушки и 2 луковицы; положив коренья и ветчину в кастрюлю, прибавить 1/4 фунта свежего чухонского масла, ложку муки, вымешать, обжарить, налить бульоном и поставить вариться. Потом разнять двух кур на части, опустить в суп. Осьмушку миндалю сладкого и 10 миндалин горького очистить, истолочь мягко, положить также в суп. Когда все хорошо уварится, растереть 2 круто сваренные желтка, заправить суп, посыпать крошечку мельчайшего перца, дать вскипеть один раз, снять жир, процедить, поставить опять на плиту, чтоб был горяч, но более уже не кипел, мешать, поднимая ложкою вверх. Отпуская на стол, положить в суповую чашку белого хлеба, поджаренного в масле, и белое мясо от курицы, нарезав его полосками.

\z{Черный суп.}

Изотрите ломоть ситного хлеба, и обжарьте дотемна в свежем сале; потом нарежьте от этого же хлеба тоненьких ломтиков, положите в суповую чашу, вместе с обжаренным тертым хлебом, а потом мелко изрубленных луковиц и травы петрушечной. На этот слой положите еще несколько ломтиков ситного хлеба, обжаренных крошек и той же зелени. Так должно продолжать, пока чашу довольно наполните. Тогда налейте хорошим мясным отваром, и дайте полчаса стоять на горячей золе. Можно в этот суп прибавлять заварные яйца и маленьких жареных птичек. (5)

\z{Суп сырный.}

Мягко разварить, мелко-на-мелко искрошенной, серой капусты, которая бы была перед тем отварена в мясном отваре, и если бульон окажется не довольно жирен, то прибавьте в него кусок коровьего масла. Положите в суповую чашу несколько ломтей белого хлеба и поставьте ее на жар, а на ломти положите слой сафо-капусты, посыпьте ее густо пармезаном или другим каким сыром, и так продолжайте, пока чаша будет полна. Дав всему этому несколько на жару попреть, выложите на это суп, и подавайте. (5)

\z{Суп-кoнcoмe с рисом по-итальянски.}

Приготовить консоме\footnote{В отделе VI описание консоме, как основу хорошего французского супа.}. Между тем вымыть в кастрюльке 0.5 ф. риса, палить водою, поставить на огонь и, когда закипит, слить воду прочь, залить немного бульоном, посолить, положить 2 лота масла, поставить на огонь и сварить под крышкою до мягкости. Потом сварить в соленом кипятке 2 л. мелкого зеленого горошку, столько же изрезанных молодых зеленых бобов и молодой моркови. Когда все будет готово, положить в рис 2 л. тертого сыру, размешать лопаткой, потом положить сваренную зелень; перемешать осторожно, выделать ложками рис в виде кнели или клёцек, сложить правильно на блюдо и, подлив немного бульону, подать при супе-консоме. Для любителей подается еще на тарелке тертый сыр. (6)

\z{Суп-потофё (pot au feu) в горшке.}

Взять кусок говядины (тонкий край без пашины), завязать голландскими нитками, вымыть в теплой воде, сложить в каменный горшок, который уже был употребляем, налить полно холодною водою и поставить на огонь. Как начнет закипать, снимать накипь сверху, и когда совершенно очистится, снять с огня, бульон процедить, а говядину вымыть в теплой воде, положить обратно в тот же очищенный горшок, налить собственным бульоном, поставить на огонь и, когда снова закипит, посолить и поставить на вольный огонь, чтобы бульон варился тихо. Час спустя положить обланширенные в воде 2 штуки моркови, 1 репу, 1 сельдерей, 1 порей, 1 петрушку, 1 штучку сафой-капусты, 5 шт. картофеля и 1 спеченую до колера луковицу с 3 гвоздиками; покрыв крышкою, варить еще 1.5 часа, смотря по мягкости говядины, которая поспевает между 3 и 4 часами. Когда будет готово, говядину вынуть на блюдо, а с бульона снять жир, процедить сквозь салфетку в суповую чашку и опустить коренья цельными. Говядину подать натурально или огарнированную, как сказано будет ниже. (4)

\z{Суп раковый.}

Сварить в воде 60 раков, снять верхние черепки, очистить шейки и клешни, оборвать ножки, а брюшки отбросить. Оставить 40 черепков; остальные черепки, с клешнями, ножками и шелухой с шеек, истолочь мягко, положить в кастрюлю с полфунтом коровьего масла,' жарить, пока масло покраснеет; тогда процедить сквозь сито масло, а гущу опять положить в кастрюлю, прибавить ложку муки, вымешать, налить бульоном или раковым отваром, поставить вариться. Нарезать мелко полфунта ветчины, нашинковать корня по два петрушки, пастернака и моркови, обжарить вместе с ветчиной в раковом масле; положить в мясной бульон. Потом приготовить из рыбы фарш, прибавить в него половину раковых шеек, изрубить мягко, начинить раковые черепки, вымазать раковым маслом, поставить в духовой шкаф, или в печь, в вольный дух. Когда раки зажарятся, процедить бульон, опустить в него оставшиеся раковые шейки и начиненные черепки; поставить на плиту, чтоб суп был горяч, но более уже не кипел. Из фарша можно сделать род колбасы, вымазав маслом, зажарить в печи, в вольном духу и нарезав кружочками, опустить в суп. (6–8)

\z{Суп с белой репой.}

Нашинковав полосками репу, пересыпать сахаром, поджарить, чтоб зарумянилась, налить бульоном, прибавить немного жюса; когда репа довольно уварится, слить бульон в другую кастрюлю, прибавить свежего бульона столько, сколько должно быть супу. Потом взять шесть яичных белков, взбить их в пену, положить в бульон, поставить на плиту, мешать беспрерывно поваренною ложкою, поднимая вверх и сливая, пока бульон вскипит; тогда процедить сквозь салфетку, дать отстояться, слить в кастрюлю, положить в него репу и разнятую на части утку, сваренную в бульоне. В этот суп можно класть фарш или подавать с гренками из белого хлеба. (5)

\z{Суп голстинский.}

Взять 3 горсти свежей рубленной капусты, налить бульоном, поставить вариться. 2 горсти овсяных круп, вымыв хорошенько, также положить в суп; варить, пока капуста будет мягка. Потом разнять на части пару рябчиков или куропаток, нашинкованных и обжаренных положить в суп и уварить до спелости. (5)

\z{Черепаховый суп. (Soupe à la tortue; iortu soupe).}

Англичане и Французы высоко ценят черепаховый суп. У нас черепах нет, так изобрели суп другого рода, названный этим именем. Вот как он делается. Ошпарив две телячьи головки и вырезав ножичком мягкие части, варить в брезе около 2 часов. Положив потом их на лист и накрыв другим, придавите камнем да сгнетение и поставьте на лед. Когда застынет, режьте на четвероугольные части, пальца в 2 ширины и длины, отделяя прочь все негодное.

За этим надобно взять 2 фунта воды, сладкого мяса и гребешков. Вычистив гребешки солью, надобно вскипятить сладкое мясо раза 2 в воде, очистить от перепонок и жилок, и варить то и другое в брезе. По истечении срока, вынув сладкое мясо с гребешками из бреза, присоедините его к изрезанным телячьим головам. Потом приступить к кнелю из судаков, то есть истолочь в ступке мякоть большего судака с пятикопеечной булкой, яйцом, мелким перцем и ложкою чухонского масла; этот кнель переделать на чайных ложечках и варить в соленой воде. Вынутый из воды кнель присоединяется к гребешкам, сладкому мясу и телячьим головкам. Вслед за этим надобно сделать, так называемый, маринад. Положив в кастрюлю с чухонским маслом по 2 корня: петрушки, пастернака, сельдерея, порея и моркови, изрезанных в тоненькие ломтики, по 2 золотника сухих трав: чабру, майорану, штук 5 гвоздики, щепоть перечных зерен; поджаривайте, мешая лопаткой; присыпав столовую ложку муки, поджарить с нею и налить белым бульоном. Когда вскипит, сняв пену, прибавить красного бульону для зарумянки. Подогрев стакан хорошей французской водки, воспламените и в минуту вспышки, потушив пламя накрышкою, вылейте в маринад. Через полчаса, процедив сквозь ситку, в другую чистую кастрюлю, сложите гребешки, кнель, сладкое мясо, телячьи головки, и дайте вскипеть. Сняв пену, изрежьте в кружки очищенный лимон и кладите в суп со стаканом мадеры и рубленным укропом. (10–12)

\z{Суп-крем из риса.}

Приготовить бульон из 3-х фунтов говядины. Когда будет готов, процедить в кастрюльку и поставить на огонь. Взять в чашку 3 ложки рисовой муки и развести холодным бульоном, а когда бульон закипит, влить разведенную муку, постоянно мешать и оставить кипеть несколько минут. Пред отпуском отбить в чашку 2 желтка, влить 0.5 стакана сливок и размешать; потом снять на стол, влить 2 ложки супу в лейзон, размешать, вылить лейзон в суп и, размешав окончательно, процедить в суповую чашку сквозь самое редкое сито. (5)

\emph{Способ приготовлять муку и крупу из риса}\footnote{Способ этот заимствован сюда из <<Хозяйки>> г. Радецкого.}.

Взять в кастрюлю назначенный для этого рис (свежего привоза, чтобы не имел какого-либо постороннего запаха), вымыть в теплой воде, слить воду прочь, влить холодной воды, промыть снова и продолжать промывать до тех пор, пока вода останется совершенно чиста; тогда выбрать рис из воды, выложить на полотенце и, осушив немного, истолочь в каменной ступке. Между тем приготовить тамбур из 5-ти сит, 1-е сито шелковое, 2-е сито частое, 3-е сито редкое, 4-е сито самое редкое. 5-е сито крупное, на подобие решета или сита проволочного. Когда рис будет истолчен, выложить в тамбур и, покрыв его, просеять обыкновенным способом, ударяя тамбуром на руках то в ту, то в другую сторону.

\begin{itemize}
	\item Мука-крем из риса. Выходит из-под шелкового сита и остается в тамбуре.
	\item Мука обыкновенная для блинов. Остается на шелковом сите.
	\item  Крупа из риса мелкая. Остается на частом сите.
	\item Крупа из риса средняя. Остается на редком сите.
	\item Крупа из риса крупная. Остается на самом редком сите.
\end{itemize}

Приготовленную этим способом муку и крупу высушить и держать покрытыми в банках в сухом месте.

\z{Суп с поджаренным белым хлебом.}

Взять по две штуки лука и порея, по одному корню петрушки, сельдерея и моркови, исшинковать, поджарить в масле, но чтоб масла было немного и коренья не слишком напитались маслом; потом налить бульоном и поставить вариться. Между тем, нарезав белого хлеба ломтиками, поджарить в масле. Отпуская на стол, положить в миску поджаренный хлеб, вылить на него суп и посыпать рубленного зеленого укропу и петрушки. (4)

\z{Суп испанский с пармезаном.}

Взять по равной части тертого белого хлеба и сыра пармезана; стереть с 2 или 3 яичными желтками, смотря по количеству сыра и хлеба, развести бульоном, чтоб суп был не густ и не жидок, варить на легком огне, непрерывно мешая. Когда есть готовый бульон, то суп этот приготовляют перед самым обедом, потому что ему вариться надобно не более часу. (4)

\z{Суп из курицы и мозговых костей.}

Изрубить старую курицу на мелкие куски, прибавить говяжьих мозговых костей, разбив их обухом топора, налить по пропорции воды, поставить на плиту или в печь; когда бульон уварится и будет крепок, процедить, поставить опять на плиту, и, дав закипеть, опустить в этот бульон клёцки, посыпать немного мускатного цвету и рубленного зеленого укропу. Также можно положить в суп курицу, или приготовить из нее соус. (4)

\z{Суп со сморчками}\index{Суп ! со сморчками}

Взять 4 фунта говядины, изрезать кусками, а курицу разнять на части; перемыть, посолить, поставить вариться. Потом отобрать 30 сморчков, которые покрупнее, снять с корешков, вымыть хорошенько; оставшиеся сморчки также вымыть, изрубить мелко, положить в кастрюлю, прибавить 1/4 фунта чухонского масла, фунт ветчины, мелко нарезанной, и по одному корню петрушки и моркови, исшинковав их полосками; все это, хорошо поджарив, выложить в суп. Отложенные 30 сморчков начинить следующим фаршем: размочив в молоке и выжав досуха белый хлеб, смешать с 3 круто сваренными и мелко изрубленными яйцами, прибавить 1/4 фунта свежего чухонского масла и 2 сырые яйца, перемешать все вместе хорошенько; начинить этим фаршем сморчковые шляпки, обвалять в сырых яйцах и толченых сухарях, обжарить в масле, опустить в суп, а когда будет готов, заправить ложкою свежей сметаны. Отпуская на стол, говядину и курицу выбрать и, вылив суп в миску, посыпать зеленым рубленным укропом. (6 — 8)

\z{Суп С.-петербургский с навагой}\footnote{Этот превосходный суп, названный метр-д'отелем И. М. Радецким, <<петербургским>>, им самим изобретенный, по справедливости, может быть назван супом метр-д'отеля Радецкого. Приготовление этого необыкновенно вкусного и оригинального блюда может выполнить лишь дельный грамотный повар или кухарка, вполне отвечающая за повара, каких впрочем не мало в Петербурге.}\index{Суп ! с навагой}

Приготовить консоме из кур следующим способом: снять с 2-х кур филеи, покрыть шпиком и поставить в холодное место для другого употребление; с гвисов и крыльев снять мягкие части, изрубить мелко, истолочь в ступке, прибавить 3 белка, размешать, выложить в обширную кастрюлю, положить все куриные кости, развести процеженным бульоном, поставить на огонь и мешать. Когда закипит, прибавить тех кореньев, которые более нужны для вкуса консоме, и варить на легком огне пока бульон очистится и получит надлежащий вкус (бульон доливается часто по ложке холодною водою, чтобы мера бульона осталась одинакова). Когда бульон будет готов, процедить сквозь салфетку в суповую кастрюлю.

Между тем приготовить фарш-кнель из кур. Очистить навагу, снять филеи, запасировать на масле и остудить; потом пустить в кастрюльку ложку сливочного масла, положить ложку муки, размешать, развести сливками поставить на огонь, выварить бешамель до совершенной густоты, обмакивать филей из наваги в бешамель и складывать на другой плафон. Когда филеи остынут, сложить так, чтобы филей мог удобно поместиться в средину столовой ложки; тогда выделать из куриного фарша кнели так, чтобы в средине каждой кнели был замазан филей из наваги, сложить на подмазанный сотейник, а пред отпуском налить соленым кипятком и сварить на легком огне под крышкою.

Вскипятить на плите нужное количество консоме; между тем развести холодным бульоном 6 чайных ложек картофельной муки, т. е. сколько предполагается персон, влить в кипящее консоме и, проварив немного, прибавить 4 столовые ложки пюре из томатов и 0.5 бутылки сотерно, вскипяченного с ароматными травами, размешать и процедить в суповую чашку. В серебряной или мельхиоровой кастрюле особо подаются кнели, фаршированные навагою\footnote{Ежели же в хозяйстве не имеется ни той, ни другой кастрюли, то такая может быть заменена превосходно вычищенной кастрюлей из белого железа или даже медною, но окутанной в самую чистую камчатную салфетку.}. Любители прибавляют к кнелям запасерованные филеи из наваги, печенки из налимов и кайенский перец. (6)

\z{Бульон с выпускными яйцами}\index{Бульон ! с выпускными яйцами}

Приготовить бульон из 3-х ф. говядины. Между тем сварить яйца мягко следующим способом: вскипятить в кастрюле воды, опустить 4 яйца, покрыть крышкою и варить ровно 5 минут, потом вынуть в холодную воду и, когда остынут, очистить осторожно и держать в холодной воде до употребления. Пред отпуском разогреть, опустить в суп или употребить на гарнир. (4)

\z{Суп из корюшки}\index{Суп ! из корюшки}

Поставить на огонь воды в кастрюльке, положить по кусочку очищенных и изрезанных мелко кореньев: петрушки, сельдерея, порея, 1 луковицу и 2 шт. картофеля; когда закипит, сдвинуть на легкий огонь, положить по вкусу соли, 1 четверть лаврового листу, 4 шт. английского перцу и крошечный кусочек мускатного цвету, и варить до мягкости; потом процедить и поставить снова на огонь. Между тем очистить и вымыть 10 шт. корюшки, снять филеи так, чтобы головка и спинная кость остались, тщательно очистить филеи, вытереть, опустить в кипящий мясной или рыбный бульон и, дав один раз вскипеть, разлить на тарелки. Блюдо и постное, и скоромное. (2)

\z{Суп-гарбюр по-итальянски}\index{Суп ! гарбюр по-итальянски}

В начале поставить бульон из 4 ф. говядины; когда закипит, посолить, положить коренья, пряности, вымытую и заправленную молодую курицу или связанную нитками телятину, прибавить столько воды, чтобы было бульона не менее 6 бутылок, и варить до тех пор, пока курица и говядина сварятся до мягкости, а бульон получит надлежащий вкус; тогда вынуть курицу и говядину на тарелки, а бульон процедить сквозь салфетку вместе с жиром и отставить отдельно.

Между тем приготовить белую капусту, как следует, потом истереть 3/8 ф. сыра пармезана (вместе с швейцарским), очистить и сварить в бульоне 20 шт. моркови, картофеля и нарезать из белого хлеба без корки круглые гренки, которые заколеровать на роште. Когда все будет готово, процедить бульон (без жиру) в суповую кастрюлю, поставить на огонь, закипятить, капусту отлить на сито и, отпрессировав немного, класть рядами на гренки, каждый ряд пересыпать тертым сыром, сложить на блюдо в кружок, поставить в печку и, когда сыр сверху затянется, вынуть, наложить в средину морковь и подать на стол вместе с бульоном. (6–8)

\z{Суп из барашка на красном вине}\index{Суп ! из барашка}

В начале приготовить бульон красный, т. e. взять в обширную кастрюлю 2 ф. бульонной телятины с косточками и 2 ф. говядины, вымыть, поставить на огонь и покрыть крышкою. Когда сок выкипит, и говядина на дне кастрюли начнет колероваться, залить кипятком, снабдить солью, кореньями и пряностями и сварить на легком огне до готовности. (6–8)

\z{Борщ с уткой, ушками и сосисками}\index{Борщ ! с уткой, ушками и сосисками}

Приготовить борщ, как указано было уже выше. Между тем обжарить в печке до половины молодую утку, положить в борщ и уварить окончательно; на сотейник, где утка жарилась, положить сосиски, обжарить с обеих сторон, вынуть на доску, изрезать порционными кусками и положить в суповую чашку. На сотейник влить бульону, выварить сок и процедить в борщ. Когда утка будет готова, вынуть, разрезать порционными кусками и положить также в суповую чашку; потом снять борщ на стол, собрать дочиста жир, положить рубленного укропу, по вкусу соли и перцу, вылить в суповую чашку и выложить сваренные ушки. Сметану подать особо в соуснике. (6)

\z{Суп желтый с яичными желтками}\index{Суп ! желтый с яичными желтками}

На каждый штоф (2 бут.) мясного бульону разболтать ложкою по 3 яичных желтка и развести мясным отваром. Растопить в кастрюле коровьего масла и подпалить в нем дотемна столовую ложку муки. Когда остынет, влить туда же бульон, подбитый желтками; при беспрестанном мешании вскипятить, приправить чуть-чуть толченым имбирем, и наконец положить еще кусок коровьего масла. Подавать с обжаренными в сливочном или хорошем мызном масле ломтиками белого хлеба. (6)

\z{Суп капустный}\index{Суп ! капустный}

Взять кочанка 3-–4 сафо-капусты, обварить их в кипятке, потом выложить в холодную воду и выжать, оправить кочанки, обвязать нитками, положить в кастрюлю с куском коровьего масла, морковью, пастернаком, кореньями петрушки, сельдереем, базиликом, тимьяном, лавровым листом и натыканною гвоздикою луковицею, и упаривать до тех пор, пока все сделается мягко; после того счерпать жир, развести мясным отваром, накрыть кастрюлю и варить на слабом огне 1/4 часа. Подавать с размоченными ломтями белого хлеба. (6)

\z{Суп-рассольник по-эмберовски}\index{Суп ! рассольник по-эмберовски}

Сварите обыкновенный бульон, в который вы вложите пару <<самоклёвов>>. Когда они сварились, то вынуть их и положить в холодную воду. Затем взять 4 шт. моркови, 2 шт. репы, 2 шт. сельдерея. Все это точно также искрошить, как делается это для свежих щей. Когда это сделано, берется десяток огурцов соленых, которые разрезать на четвертушки в длину. Каждую такую четвертушку разделить пополам, дав ей кругловатую форму. Как скоро бульон готов, то из полфунта свежего масла сделать пассир (поджаренное слегка масло), чтоб загустить бульон. К этому прибавить 2 стакана огуречного рассолу. Минут 20 это должно покипеть на плите. Тогда эта горячая смесь из бульона, пассира и рассола наливается на овощи и огурцы, в жеребейки, изрезанные и сложенные в кастрюлю. Эта кастрюля ставится на огонь, и все варится до тех пор, пока морковь и репа совершенно уварятся. В эту пору в особую посудину кладут 1 ф. щавеля с общипанными листьями, вливают полбутылки белого виноградного вина, сыплют немножко рубленого укропа. К этому прибавляется штук 40 кнели (протертое мясо в виде теста) из курицы или вообще из живности; эта кнель должна быть кругла, в виде орехов, и сварена на бульоне. Сюда же прибавляется и пара вышеупомянутых молодых откормленных петушков, т. е. <<самоклёвов>>, разрезанных хорошенько. На это наливается кипящий отвар, — и суп-рассольник готовь. (6-–8)

\z{Суп английский <<Виндзор>>}\index{Суп ! английский}

Приготовить бульон из 3-х ф. говядины. Между тем снять с костей очищенные телячьи ножки, положить в кастрюльку, налить холодною водою и поставить на огонь. Когда закипит, слить первую воду, налить холодной воды, вымыть и изрезать ножки правильными продолговатыми кусочками, налить процеженным бульоном, положить правильно обточенных кореньев: петрушки, сельдерея, порея, маленького луку, и поставить на огонь; когда закипит, положить ложку перловой крупы, 0.5 ложки масла, и варить на легком огне под крышкою до тех пор, пока ножки, коренья и крупа укипят до мягкости. Между тем процедить бульон, поставить на огонь и, когда закипит, положить 1/8 ф. макарон, правильно изломанных кусочками, и развести в чашке холодным бульоном ложку свежей картофельной муки. Когда макароны сварятся, вылить разведенную муку в кипящий с макаронами бульон, проварить, потом положить вместе сваренные с кореньями ножки и снабдить по вкусу солью и красным перцем. Пред отпуском вбить в чашку 1 желток, положить 1/4 стакана сливок и процедить в суповую чашку; потом вылить постепенно суп, размешать и подать с рубленною зеленью. (4)

\z{Суп с сладким мясом}\index{Суп ! со сладким мясом}

Взять сладкого мяса 5 фунтов, нарезать кусками, положить в кастрюлю, прибавить кусок ветчины и ложку масла, поджарить, налить бульоном, а если нет запасного бульона, налить водой, поставить вариться; когда вполовину уварится, процедить. Потом говядину сложить опять в кастрюлю, налить процеженным бульоном, а ветчину более уже в бульон не класть. Положить в суп фаршированных сморчков, трюфелей, зеленой петрушки, укропу и заправить раковым маслом. Если случатся раковые шейки, их также можно положить в суп. Фарш для начинки сморчков приготовить из телятины с белым хлебом. (6)

\z{Суп из шпината}\index{Суп ! из шпината}

Выбрав и перемыв 0.5 ф. шпината, изрубить, обдать кипятком, накрыть и дать постоять немного; потом откинуть на сито, выжать, положить в кастрюлю с ложкою масла, обжарить, прибавить ложку муки, вымешать и налить бульоном. Когда уварятся суп, положить выпускных или круто сваренных в разрезанных пополам яиц, немного мускатного орешка, перцу и мелко изрубленного укропу. (5)

\z{Суп из перловой крупы}\index{Суп ! из перловой крупы}

Взять 2 ф. баранины и курицу, баранину нарезать кусками, а курицу разнять на четверо, налить водою, посолить, поставить вариться; когда начнет кипеть, пену снять дочиста. Потом нашинковать по одному корню петрушки, сельдерея и моркови, опустить в суп. На 2 фунта баранины и 1 курицу положить чайную чашку хорошо вымытых перловых круп. Отпуская на стол, посыпать мелко изрубленного зеленого укропу и петрушки. (5)

\z{Суп из потрохов}\index{Суп ! из потрохов}

Взять гусиный или из другой птицы потрох, а если хотят иметь наварный суп, прибавить немного грудины. Дав хорошо увариться бульону, потрох и говядину выбрать, бульон процедить. Потом, слив бульон в кастрюльку, опустить в него потрох, положить по 1 корню петрушки и моркови, две штуки порею, нашинковать коренья полосками, а порей нарезав продолговатыми штучками. Когда суп будет готов, заправить мукой, поджаренной докрасна в масле. (5)

\z{Суп из курицы с трюфелями}\index{Суп ! из курицы ! с трюфелями}

Очистив, выпотрошив и вымыв курицу, разнять на части, обдать кипятком, дать стоять полчаса, а потом, вынув, дать стечь воде. Растопив 1/4 фунта коровьего масла, кипятить, пока оно покраснеет; тогда положить ложку муки, вымешать хорошенько, опустить в масло курицу, налить бульоном. Дав увариться, положить мелко нарезанных трюфелей и шампиньонов, выжать сок из 1 лимона. (5)

\z{Суп-пюре из ершей}\index{Суп-пюре ! из ершей}

Отмерить в кастрюльку стакан перловой крупы, вымыть, влить 3 ст. воды, поставить на огонь, сварить, протереть сквозь частое сито, сложить в суповую кастрюльку и оставить покрытым до времени. Очистить ерши и другую назначенную рыбу, снять филеи, сложить на сотейник, посолить, подлить 2 л. воды, поставить на огонь, покрыть крышкою и, когда вскипит и филеи побелеют, снять с огня, остудить, истолочь в ступке, протереть сквозь частое сито и сложить к протертой перловой крупе. Из головок и костей сварить бульон; когда будет готов, процедить сквозь салфетку, развести им пюре, поставить на огонь и, мешая, разогреть (не заварить), вылить в горячую суповую чашку и подать. Любителям подают гренки. (4)(Пост.)

\z{Суп из телятины а ля Нессельрод}\index{Суп ! из телятины}

Выбрать 6 ровных лядвей от маленьких белых жирных телят, вымыть тщательно, распилить кость пополам, сложить в кастрюлю, налить водою и поставить на огонь. Когда закипит, выбрать в холодную воду, вымыть и сложить обратно в кастрюлю, налить процеженным бульоном, положить кореньев и соли, и сварить до мягкости; потом отставить в холодное место, а когда остынет, выбрать из бульона, очистить косточку и обровнять лядвеи так, чтобы каждая из них имела правильность, и все были одинаковы. Затем сложить в кастрюлю, залить собственным соком и за 0.5 часа до отпуска разогреть на легком огне. Между тем приготовить бульон; когда будет готов, взять в кастрюлю 0.5 ф. перловой крупы, вымыть, налить бульоном без жиру, положить кусочек масла и сварить под крышкою до мягкости; таким же способом сварить отдельно 0.5 ф. риса, сложить в 1 кастрюлю, положить 1/4 ф. сливочного масла, разбить лопаткою, а когда побелеет, развести бульоном, процедить сквозь частое сито или протереть сквозь салфетку, потом слить в кастрюлю, разогреть, неотступно мешая, вылить в горячую суповую чашку, а остальным супом залить уложенные в глубокое блюдо телячьи лядвеи и подавать горячими. Для любителей в средину блюда кладут сваренную свежую зелень. (6)

\z{Суп-пюре из чечевицы}\index{Суп-пюре ! из чечевицы}

В начале приготовить бульон из 2 ф. говядины, потом вымыть в теплой воде нужное количество чечевицы, положить в кастрюлю, прибавить кусочек сырой ветчины, по 2 штуки очищенного луку, моркови и порея, налить, закипятить на плите и поставить в горячую печку на 2 часа, чтобы упрело до мягкости, потом. выбрать прочь коренья и ветчину, а чечевицу протереть сквозь частое сито, развести процеженным бульоном, поставить на огонь, вскипятить, снять дочиста сверху накипь, снабдить по вкусу солью и перцем и процедить в чашку. (4)

\z{Суп из курицы в горшке}\index{Суп ! из курицы в горшке}

Положить в луженый каменный горшок очищенную, вымытую и заправленную курицу, налить холодною водой и поставить на огонь. Когда закипит, вымыть курицу в холодной воде, разрезать порционными кусками, положить в горшок, налить процеженным сквозь салфетку собственным бульоном, снабдить по вкусу солью, положить кореньев: петрушки, порея, моркови, сельдерея, крошечку пряностей, и варить на легком огне до мягкости. Пред отпуском засыпать смоленскими крупами, затертыми яйцом, заварить на огне, выбрать пряности с кореньями и подавать. (4)

\z{Бульон с бишкоктом}\index{Бульон ! с бишкоктом}

Приготовить бульон из 3 ф. говядины и сделать бишкокт. Отмерить в кастрюльку ложку нетопленого масла, разбить, чтобы побелело, положить (по одному) 3 желтка и мешать до тех пор, пока масса поднимется. Между тем взбить в пену белки, положить 1.5 ложки муки просеянной, класть белки, постепенно мешая, чтобы масса была гладка. Смазать маслом и обсыпать тертым хлебом маленький сотейник. За 20 минуть до отпуска выложить массу, поставить в горячую печку, заколеровать сверху немного, покрыть крышкою, и спечь до готовности. Пред отпуском выложить на стол, разрезать порционными кусочками бишкокт и, уложив правильно на тарелке, подать при супе. (4)

\z{Суп зеленый с выпускными яйцами}\index{Суп ! зеленый с яйцами}

Взять курицу, а если нужно, чтобы супу достаточно было для шести особ или более, прибавить 2 ф. говядины, с мозговыми костями; налить 4 бутылки воды, положить немного соли, поставить вариться, и дать кипеть небольшим ровным ключом, снимать пену; когда курица уварится, бульон процедить. Взять по горсти шпината, портулака и щавелю, перебрать, вымыть, изрубить мелко, обдать кипятком и дать стоять полчаса, а потом откинуть на сито. Между тем очистить по одному корню петрушки, пастернака и моркови, нашинковать полосками; налущить горсть зеленого гороха, нарезать молодой спаржи, кусочками величиною в полвершка. Слив бульон вь кастрюлю, положить приготовленную зелень и коренья. Уварив все хорошо, выпустить одно за другим 10 яиц; дав яйцам завариться, вылить супь в миску и подавать. Можно также сварить яйца в жидкую смятку, или, как называют, <<в мешочке>> положить в суповую чашку и вылить на них суп. Говядину и курицу в суп не класть; из курицы можно приготовить какое-нибудь блюдо, под каким-нибудь соусом. (6)

\z{Суп с пармезаном}\index{Суп ! с пармезаном}

Натереть пармезана, разварить в бульоне, процедить сквозь сито, поставить опять на плиту, подправить двумя яичными желтками, положить кусок свежего чухонского масла, с грецкий орех, посыпать немножечко имбирю, вскипятить один раз. Подавая на стол, положить ломтиков белого хлеба, поджаренного в масле. (5)

\z{Суп с улитками}\index{Суп ! с улитками}

Отварив улиток в воде, до половины спелости, разделить на две части; одну отложить, а другую вычистить из черепков, изрубить, положить в приготовленный бульон, приправить свежим чухонским маслом, мускатным цветом и перцем. Другую половину улиток, вычистив из черепков, положить в миску, с сухариками из белого хлеба, обжаренными в масле, вылить на них суп, положить корки с свежего лимона, нашинкованной тоненькими полосками. (5)

\z{Суп с устрицами}\index{Суп ! с устрицами}

Вынув из раковин 20 устриц, изрубить мелко, положить в кастрюлю, налить 6 стаканов бульону; приправить кардамоном и перцем, прибавить стакан виноградного вина, немного лимонного соку и кусок чухонского масла; дав прокипеть ключом, подбить двумя яичными желтками. В суповую чашку положить несколько вычищенных устриц и сухариков из белого хлеба, вылить на них сквозь сито суп. (5)

\z{Белый суп из репы}\index{Суп ! белый из репы}

Взять потребное количество репы, смотря по пропорции супа, обчистить хорошенько, промыть, положить в горшок, который наполнить доверху, облить кипящим бульоном и варить до тех пор, пока репа размякнет. Если бульон жирен, то в него не нужно более подбавлять масла; если же жидок, положить кусочек масла. При подаче на стол, репу кладут в суповую миску с поджаренными ломтиками размоченного в молоке хлеба, обливают бульоном, процеженным сквозь сито. (5)

\z{Красный суп}\index{Суп ! красный}

Взять от задней части 2 или 3 фунта говядины, разрезать на ломтики, избить, обложить ею дно и стенки кастрюли, обмазанный предварительно коровьим маслом, и жарить исподволь на невысоком тагане с изрезанными кореньями, луковицами порея, тмином, базиликою и крошечкой имбирю. Когда мясо получит приятный темный цвет (при чем надо стараться, чтоб оно не пригорело), то, смотря по количеству супа влить в кастрюлю потребное количество мясного бульону, вскипятить, чтоб бульон также принял темный цвет, процедить на-светло сквозь сито или салфетку, и подложить саго или мясных клёцек. (5)

\z{Суп-крем из спаржи}\index{Суп-пюре ! из спаржи}

Очистить 4 ф. белой молодой спаржи, сложить в кастрюлю, налить водою и поставить на огонь. Когда закипит, слить воду прочь, положить 1/4 ф. сливочного масла, 4 л. муки, размешать, залить белым бульоном и сварить на легком огне до мягкости; потом протереть сквозь частое сито, сложить протертое в кастрюлю, развести бульоном как быть должно супу и поставить на огонь. Между тем отбить в кастрюлю 3 желтка, развести сливками и, когда суп начнет закипать, снять на стол, сперва влить немного супа в лейзон, а потом вылить лейзон в суп, размешать, процедить сквозь сито в суповую чашку и опустить в суп зеленые кнели. (6)

\z{Суп валахский белый с уткой}\index{Суп ! валахский}

Сварить бульон из 2 ф. говядины, 1 маленькой утки, кореньев разных, 3 — 4 шт. сушеных грибов; процедить. Одну ложку масла поджарить с 1.5 л. муки, развести сметаною, потом бульоном, вскипятить, мешая, процедить сквозь сито; перед отпуском к столу опустить в бульон отдельно отваренного рассыпчатого риса и также в бульоне отваренные точеные коренья, раз вскипятить; подавая, всыпать укроп и зеленую петрушку. (6)

\z{Суп с фаршированными булочками}\index{Суп ! с фаршированными булочками}

Взять 8 или сколько потребно будет небольших сдобных булочек, срезать дно, выбрать весь мякиш. Приготовить следующий фарш: взять куриного мяса или телятины, из говяжьих костей мозгу, а если мозгу не случится, почечного сала, и немного шпика; изрубив все мелко, посолить, посыпать перцем, положить в фарш 3 сырые яйца, смешать все вместе хорошенько. Этим фаршем начинить булочки, вымазать дно сбитым сырым яйцом, уложить на глубокую сковороду, накрыть и поставить в духовой шкаф, или в печь, в вольный дух, чтоб фарш запекся. Две горсти портулака перебрать, вымыть, изрубить крупно, положить в бульон, приготовленный для супа; когда портулак уварится, прибавить две столовые ложки жюсу. Положив в миску фаршированные булочки, вылить на них суп. (6)

\z{Щи со свежей капустой}\index{Щи ! со свежей капустой}

Щи лучше варить в горшке, чем в кастрюльке. Взяв 4 фунта грудины, разрезать на несколько кусков, вымыть, положить в горшок такой величины, чтобы говядины было половина горшка, приставить горшок к огню; дать кипеть ключом полчаса, а потом, выбрав говядину, бульон процедить. Кочан капусты, разрезав на четыре или на шесть частей, обдать кипятком, накрыть и дать постоять полчаса, откинуть на сито, выжать из него воду, положить в горшок вместе с говядиной, налить процеженным бульоном, поставить вариться; прибавить 2 луковицы и 2 корня моркови, нашинковать крупно. Когда щи уварятся, заправить ложкою муки и двумя ложками сметаны. Щи должно хорошо уварить: чем больше они упреют, тем бывают вкуснее.

\z{Борщ на манер настоящего малороссийского}\index{Борщ ! малороссийский}

Борщ этот варят из говядины, баранины, свинины, утки и гуся, или берут разных мяс и живности, варят все вместе; также кладут ветчину и сосиски. Взять 2 фунта говядины, фунт ветчины и половину гуся, вымыть, положить в горшок, чтоб мяса было полгоршка, налить водой, поставить вариться и снимать пену. Свежую капусту нашинковать крупно, полосками в палец шириною, а свеклу обыкновенным манером; положить в кастрюлю, обжарить в масле, прибавить ложку муки, вымешать, развести бульоном, положить в горшок, вместе с двумя нашинкованными луковицами. Кому угодно, можно положить перед обедом фунт сосисок, обжарив их сначала в масле, подправить двумя ложками сметаны и приквасить по вкусу уксусом. После подправки, дать борщу прокипеть ключом. Борщ варят иногда из кислых бураков, с кислою шинкованною капустою. (6)

\z{Борщ со сметаною или постный с селедкой}\index{Борщ ! со сметаной}

Сварить бульон из кореньев и сушеных боровиков, процедить. Испечь 1 фун. свеклы, потом очистить ее, мелко нашинковать, сложить в кастрюлю, налить бульоном из кореньев, влить свекольного, отдельно отваренного рассола, сметаны, подогрев до горячего состояние, положить соли, простого перца, зелени и мелко нашинкованных грибов, подавать с жареной кашей из гречневых круп.
В постный день, вместо сметаны, положить селедку, а именно: 3 селедки шотландские вымочить, очистить, обвалять в муке, поджарить в прованском масле, опустить в борщ, раз вскипятить. (5)

\z{Рассольник, или огуречная похлебка}\index{Рассольник}

Сварить бульон, как обыкновенно, из 2.5 — 3 ф. говядины, прибавить, кто хочет, 1 воловью почку, положить кореньев и пряностей, также 2–3 сушеные грибка, варить, процедить; 6 маленьких соленых огурцов очистить, нарезать ломтиками, сварить, прибавить, если окажется нужным, огуречного рассолу, так чтобы была приятная кислота. Почки нарезать довольно мелко, опустить в суповую миску, также зелени, налить бульоном, подавать.

В суп этот можно иногда влить 0.5 — 1.5 стакана сметаны и еще раз вскипятить, сварить в том же бульоне немного картофелю или подправить суп мукою. (5)

\z{Суп-жульенн (Jullienne)}\index{Суп ! жульенн}

Сварить обыкновенный бульон из 3–4 фунтов говядины и кореньев, процедить, как сказано в № 1; 0.5 фунта ржаного хлеба высушить докрасна, налить бульоном так, чтобы покрыло хлеб, накрыть крышкою, дать постоять так час или полтора, слить и процедить. Между тем нарезать как вермишель 1 большую морковь, 1 галарепу или молодую репу, сельдерей, листьев 50 шпинату, также нарезать 6–8 штук спаржи, 1 ложку сушеного зеленого горошка, все это сполоснуть, варить целый час в процеженном бульоне; перед самым отпуском влить в него вышеупомянутый хлебный бульон и тотчас подавать. Нарезанные коренья можно сперва немного поджарить в 0.5 ложке масла, а потом положить их в бульон и варить полчаса. (5)

\z{Суп португальский острова Мадеры}\footnote{Достойно виимания, что суп этот был записан с натуры И. М. Радецким, сопровождавшим его покойного герцога Максимилиана Максимилиановича на остров Мадеру, в качестве метр-д'отеля двора его высочества.}.\index{Суп ! португальский}

Очистить 3 штуки крупного португальского луку, положить на растопленное масло в кастрюлю, поджарить с обеих сторон до колера и испечь в горячей печке под крышкою до мягкости; потом положить 3 штуки очищенных и мелко нарезанных яблок, 5 штук очищенных помидоров и один фунт винограду, налить красным вином и, разварив, протереть сквозь частое сито. За полчаса до отпуска закипятить в кастрюле консоме, и разведя холодною водою две ложки картофельной муки, влить в кипящее консоме и, когда загустеет, положить пюре, размешать, развести бульоном, как быть должно супу, протереть сквозь салфетку, закипятить, снабдить по вкусу солью и красным перцем, опустить в суп горячую телячью головку, влить в суповую чашку и прибавить стакан вскипяченной хорошей мадеры. (6)

\z{Суп-пюре из зеленого горошка}\index{Суп-пюре ! из зеленого горошка}

Приготовить бульон. Между тем исшинковать мелко 2 луковицы, сложить в обширную кастрюлю, влить 2 ст. воды, посолить, положить ложку масла, вылущенный зеленый горох, поставить на огонь, покрыть крышкою и сварить до мягкости. Как только горох сварится, снять с огня, протереть поспешно горячий сквозь частое сито, сложить в суповую кастрюлю, развести кипящим бульоном, снабдить по вкусу солью и мелким сахаром и, разогрев, вылить в суповую чашку. Гренки подать при супе особо на тарелке. (6)

\z{Бураки по-хохлацки}\index{Бураки по-хохлацки}

Взяв 4 фунта говядины от грудины, разрезать на несколько кусков, вымыть, положить в горшок, поставить в печь; когда закипит, пену снимать дочиста. Свеклу и две луковицы, исшинковав, обжарить в масле, положить ложку муки, вымешать. Выбрав говядину, бульон процедить; потом слить опять в горшок, опустить в бульон говядину и бураки, приквасить уксусом, заправить сметаной. Подавая на стол, посыпать рубленного укропа. Приготовляют бураки также из квашеной свеклы, но тогда их не поджаривают и не кладут уксусу, а только заправляют сметаной. (5)

\z{Щи с рыбой}\index{Щи ! с рыбой}

Взять кислой рубленной капусты, налить по пропорции водою, искрошить одну или две луковицы, положить в капусту, поставить вариться. Окуней, карасей или другую такую рыбу очистить, выпотрошить, вымыть, обвалять в муке, обжарить в маковом или ореховом масле и опустить во щи. Когда щи хорошо уварятся, стереть макового масла с мукой, заправить щи. (Постное). (5)

\z{Суп из курицы}\index{Суп ! из курицы}

Взять хорошую курицу, разнять на части; накрошить мелко зеленого луку и петрушки. Уложив курицу в кастрюлю, положить туда же лук и петрушку, две ложки чухонского масла, половину лимона, изрезанного кружками, стакан виноградного вина и немного соли; закрыв кастрюлю, варить на легком огне. Когда курица поспеет, налить хорошим бульоном, дать раза три вскипеть; посыпать зеленой, мелко изрубленной петрушки и укропу. (5)

\z{Суп из портулака}\index{Суп ! из портулака}

Перебрав портулак, выполоскать; изрезать мелкими кусочками ветчины, положить вместе с портулаком в кастрюлю, налить бульоном, варить на легком огне около часу. Потом накрошить из белого хлеба сухариков, поджарить в масле, положить в миску и вылить на них суп. (5)

\z{Зеленый суп}\index{Суп ! зеленый}

Взять сныти, щавелю и шпинату, перебрать, вымыть, изрубить крупно, поджарить в масле, налить бульоном, поставить вариться; положить шинкованных кореньев: петрушки, сельдерея и моркови. Когда суп уварится, заправить мукой, прибавить ложку свежей сметаны. (5)

\z{Бульон из кореньев с вермишелью}\index{Бульон ! из кореньев с вермишелью}

Исшинковать мелко 2 луковицы, сложить в кастрюлю, влить одну ложку прованского масла и поджарить на огне до колера. Потом влить кипящей воды столько, сколько нужно иметь бульона, закипятить, положить по одному корню очищенной петрушки, сельдерея, порея, моркови, 0.5 кочана небольшой свежей капусты, ложку желтого гороху, ложку белых бобов, 2 шт. грибов, 5 шт. очищенного картофеля, снабдить пряностями, и варить на легком огне пока коренья будут совершенно мягки, а бульон получить надлежащий вкус. Пред цежением снять бульон с огня, дать устояться, слить осторожно сверху и процедить сквозь салфетку в чашку. Сварить в соленом кипятке до мягкости вермишель, отлить на дуршлаг и опустить в процеженный в чашке бульон. (4) (Пост.)

Суп этот, принадлежащий к постному столу, можно заливать хорошим мясным отваром (бульоном), и тогда из постного вы будете иметь хороший скоромный суп. Этому, впрочем, превращению легко подвергаются все постные похлебки без рыбы и без постного масла, а овощно-мучные, в роде сейчас нами описанной здесь.

\z{Суп-пюре из помидоров с мясом или без мяса}\index{Суп-пюре ! из помидоров}

Сварить бульон из 2 фунтов говядины, 1 фунта телятины, 0.5 курицы и кореньев; процедить.

Взять самых зрелых помидоров, зернышки и сок прочь, а остальное сложить в кастрюлю, положить 1.5 ложки масла, поджарить, всыпать 1 ложку муки, размешать, влить сметаны, вскипятить, развести бульоном, вскипятить, протереть сквозь сито. (5)

\z{Суп-пюре из разных кореньев}\index{Суп-пюре ! из разных кореньев}

Сварить бульон из 3 фунт, говядины с кореньями; процедить.

1 фунт молодой моркови налить бульоном, положить ложку масла, накрыть крышкою, сварить до мягкости, протереть сквозь сито. Распустить в кастрюле масла, всыпать ложку муки, потом пюре из моркови слегка поджарить, мешая ложкою; прибавить, если понадобится, сахару, положить 1 желток, 0.5 стакана сливок, развести бульоном, подогреть, всыпать зелени; подавать. (5)

\z{Суп-пюре из сушеного гороха с ветчиной, или из свежего гороха (и голубей)}\index{Суп-пюре ! из гороха с ветчиной}

Сварить бульон из 2.5 фунтов говядины и 1 фунта жирной копченой ветчины; процедить.

2 стакана гороху налить водою, разварить хорошенько, а если долго не будет развариваться, то прибавлять понемногу холодной воды, развести бульоном, процедить сквозь сито, подогреть, но не кипятить; можно прибавить поджаренный в масле лук. (5)

\z{Суп-пюре из лука с саго}\index{Суп-пюре ! из лука с саго}

Сварить бульон из 4 фунтов телятины и кореньев процедить. очистить фунт луковиц, вскипятить, откинуть на решето, перелить холодною водою, потом положить в кастрюлю с ложкою масла, поджарить, но не подрумянить, а за тем заправить следующим соусом: ложку масла, ложку муки размешать, влить 1 стакан густых сырых сливок, мешать на плите; когда вскипит, залить этим соусом лук и кипятить, пока не сделается густо; развести, как следует, бульоном, протереть сквозь сито, положить соли, подогреть, опустить отдельно сваренное в бульоне саго. (6)

\z{Суп-пюре из тыквы}\index{Суп-пюре ! из тыквы}

Разрезать на части зрелую тыкву, вычистить середину, срезать толстую верхнюю корку, сложить на масло в кастрюлю, сварить на легком огне под крышкою до мягкости и протереть сквозь частое сито. Между тем распустить в кастрюле ложку масла, положить 0.5 л. муки, размешать, развести кипящим бульоном, положить 3 ст. протертого пюре, разогреть и когда суп начнет густеть, положить по вкусу соли, процедить в суповую чашку сквозь редкое сито или решето. (4)

\z{Бомбаш (грузинский суп)}\index{Суп ! бомбаш}

Нарезать жеребейками телячью грудину, сложить в кастрюлю, налить холодной водою и поставить вариться. Когда пена будет набегать, снимать ее почаще. Как только очистится бульон, положить в него соли и головки 3 луку репчатого, крупно изрезанного. Когда мясо довольно уварится, вынуть кусочки из бульона, срезать жир, а самое мясо положить обратно. Снятый жир нарезать мелкими кусочками, и от легкого отрезать 1/8 долю, изрубить ее мелко и положить также в бульон. Положить особо в чайную чашку самую малость шафрану, так 0.5 чайной ложки, ежели шафран высокого качества, настоящий персидский. Кто любит, может чуть-чуть присыпать и мускатного цвету. Перед отпуском к столу, слить бомбаш в суповую чашу, процедив сквозь сито для отделения шафрана и специй, и присыпать свежею зеленью. (5)

\z{Суп-пюре из картофеля}\index{Суп-пюре ! из картофеля}

Очистить, вымыть и сложить в кастрюльку назначенный картофель, посолить, положить 2 луковицы, налить водою и сварить до мягкости, потом слить жидкий бульон в кастрюлю, вынуть лук и прибавить ложку муки, разведенной водою, вскипятить, потом протереть сквозь сито, а пред отпуском, сложить пюре в кастрюлю, развести собственным бульоном и, разогрев до горячего состояние, вылить в суповую чашку. (4)

\z{Суп-пюре из щавеля с вермишелью}\index{Суп-пюре ! из щавеля с вермишелью}

Сварить бульон из 3 фунтов говядины и кореньев; процедить.
1 фунт перебранного щавелю вымыть, изрубить, сложить в кастрюлю, положить 1 ложку масла, поджарить слегка, мешая, положить ложку муки и 0.5 фунта ломтиками нарезанной сваренной ветчины, подлить бульону, кипятить на легком огне 1 час, потом протереть щавель сквозь сито, развести бульоном, как следует; перед самым отпуском вскипятить в нем 1 стакан вермишели, положить соли и перцу. (4)

\z{Суп-пюре из сушеного гороха мясной или постный}\index{Суп-пюре ! из гороха}

Два стакана гороху всыпать в кипяток, посолить, разварить, подливая понемногу холодной воды, протереть сквозь дуршлаг. 1 петрушку, 1 порей, 1 луковицу поджарить в ложке масла до мягкости, всыпать ложку муки, опять поджарить, смешать с протертым горохом, развести бульоном, всыпать простого, не очень мелко истолченного перцу; перед отпуском, можно положить 1 фун. копченой семги, или 3 копченые селедки, или поджаренные копченые 2 камбалы, или 1 стакан вскипяченной сметаны; раз вскипятить. (4)

\z{Суп из грибов}\index{Суп ! из грибов}

1/4 фунта белых грибов вымыть, налить тремя бутылками воды; поставить в печь. Когда грибы будут мягки, искрошить мелко, положив в них две ложки орехового масла и ложку муки; стереть хорошенько, опустить в грибной бульон. Осьмушку сладкого миндалю очистить, истолочь мягко, положить в грибы. (4) (Пост.)

\z{Суп из белых грибов с пшеном}\index{Суп ! из белых грибов с пшеном}

Сухие белые грибы, размочив в горячей воде, нашинковать, налить водой, поставить вариться. Когда довольно уварятся, положить чайную чашку рису; заправить двумя ложками макового масла, стертого с мукой, положить несколько кружков лимону. (4) (Пост.)

\z{Щи с осетриной}\index{Щи ! с осетриной}

Взять свежей капусты; нарезать крупно, обдать кипятком, дать постоять, откинуть на сито, выжать воду. Искрошить две луковицы, положить капусту и лук в кастрюлю, влить две ложки макового масла, обжарить, прибавить ложку муки, размешать, налить водой, поставить вариться. Когда капуста будет мягка, положить свежепросольной осетрины, нарезав кусочками. Если осетрина солона, то намочить ее с вечера в квасу. Варят щи с свежей и кислой капустой, из осетровой головы. Засольную головизну также надобно вымачивать в квасу. Головизну класть в щи раньше: она требует больше варенья. Можно сначала сварить ее отдельно, обобрать хрящи и мясо, нарезать кусочками и опустить во щи. (П.) (4)

\z{Уха из налимов}\index{Уха ! из налимов}

Очистив и выпотрошив налимов, вымыть, нарезать звеньями. Потом нашинковать по одному корню: петрушки, пастернаку и одну луковицу, налить в кастрюлю сколько нужно воды, положить коренья, а когда хорошо уварятся, опустить рыбу, посолить и варить до спелости. Молоки и печенки, изрезав небольшими кусочками, положить также в уху. Подавая на стол, посыпать мелко изрубленной петрушки и укропу. Уха из всякой другой рыбы варится таким же образом. Кроме кореньев, можно приправлять уху лавровым листом, горошчатым перцем, гвоздикою, лимоном, нарезанным кружками; кладут также маслины. (5) (Пост.)

\z{Щи зеленые из крапивы}\index{Щи ! зеленые из крапивы}

Сварить бульон из 2.5 фунтов говядины и 1 фунта ветчины, с кореньями и пряностями; процедить.

Когда бульон будет готов, взять 1.5 ф. молодой крапивы, перебрать, вымыть, опустить в кипяток, сварить до мягкости, но не под крышкою, откинуть на дуршлаг, перелить холодною водою, выжать, изрубить мелко, протереть сквозь сито. Положить в кастрюлю масла, 1 ложку муки, пожарить, положить крапиву, развести бульоном. вскипятить, мешая; в суповую миску всыпать зеленой петрушки и горсть укропу. Можно прибавить сметаны.

Подать к ним ломтиками нарезанную ветчину, или крутые, надвое разрезанные яйца, или пирожки, или свиные сосиски. (4)

\z{Суп из свежих огурцов}\index{Суп ! из свежих огурцов}

Сварить бульон, как обыкновенно, из 3 фунтов телятины с кореньями; процедить.

10 огурцов средней величины очистить от верхней кожицы, разрезать каждый на 4 части, вырезав самую середину, т. е. зернышки. Половину этих огурцов нарезать ломтиками, сварить в соленом кипятке, отлить на дуршлаг, перелить холодною водою, положить в суповую миску. Другую половину огурцов сложить в кастрюлю, положить 1/4 фунта вареной ветчины, 1 луковицу, 1–2 гвоздики, налить жирным бульоном, варить на легком огне до мягкости, положить туда же 0.5 ложки масла, размешанного с 1 ложкою муки; вскипятить. Перед отпуском, вынуть ветчину, а все остальное протереть сквозь сито, сложить в кастрюлю, развести бульоном, отставить на край плиты, дать устояться; снять жир, влить 0.5 стакана густых сливок с двумя желтками, подогреть до горячего состояние, беспрестанно мешая, чтобы не закипело; положить по вкусу соли, перцу, зеленого укропу. (5)

К этому супу подаются греночки из белого хлеба.

\z{Хашо (армянский суп)}\index{Суп ! хашо}

Суп этот следует варить за час до обеда. Крупно нарезанный лук (в головках) поджарить с маслом, потом налить горячей воды, посыпать соли и дать вскипеть раза два. Яичный желток взбить в чайной чашке и, взбивши, выпустить потихоньку в варево. Потом наломать кусками булку, положить и ее туда же и накрыть кастрюлю крышкой, чтоб булка разбухла. Подавая к столу, посыпать рубленной петрушкой. (4)
Впрочем, вместо кипятка предпочтительнее употреблять обыкновенный суповой навар или бульон.

\z{Щи постные}\index{Щи ! постные}

Взять кислой капусты и нашинкованную луковицу, стереть с двумя ложками макового масла и ложкою муки, поставить вариться. Потом, размочив в горячей воде белых грибов, нашинковать полосками, опустить во щи. Бураки постные варят так же, как и с рыбой; нашинковав свеклы, прибавить крошенного луку, обжарить в масле, положить ложку муки, размешать, налить водой, положить шинкованных белых грибов и приквасить по вкусу уксусом. (5)

\z{Суп из свежих грибов}\index{Суп ! из свежих грибов}

Изрезав грибы, посолить, налить водой, поставить вариться; положить луку и укропу, стереть макового масла с мукой, заправить суп. Иногда засыпают грибной суп гречневыми крупами и заправляют немного уксусом. (5) (Пост.)

\z{Суп на раковом кулисе}\footnote{См. раковый кулис.}.\index{Суп ! на раковом кулисе}

Полфунта рису вымыть хорошенько, разварить в воде; потом, слив воду, налить бульоном, положить небольшой кусок ветчины и разных трав: укропу, петрушки и кервелю. Когда суп довольно уварится, травы и ветчину вынуть, влить 2 стакана ракового кулису, положить зеленой мелко изрубленной петрушки и укропу. (5)

\z{Суп из судаков}\index{Суп ! из судаков}

Вычистив, выпотрошив и вымыв судака, нарезать звеньями, нашпиговать семгой или осетровой тешкой, посолить, обвалять в муке, обжарить в масле, положить в кастрюлю и, налив бульоном, поставить на огонь. Когда начнет кипеть, положить шинкованных кореньев и шариков из рыбного фарша\footnote{См. фарш из рыбы.}; отпуская, посыпать зеленого рубленного укропу или петрушки. (5) (Пост. и скоромный).

\z{Суп из карасей}\index{Суп ! из карасей}

Вычистив и вымыв средней величины карасей, начинить рыбным фаршем, обжарить в маковом или ореховом масле; потом положить в рыбный бульон, прибавить горошинами перцу, лаврового листу и лимону, нарезанного кружками; варить до спелости. Отпуская, посыпать зелени. (5) (Пост.)

\z{Чихитма (грузинский суп)}\index{Суп ! чихитма}

Сварить бульон из курицы. Когда курица размякнет, ее вынуть, изрезать по суставам и бульон процедить. Затем крупно изрезать 3–4 головки репчатого луку и поджарить их с маслом в кастрюле, но до такой степени, чтобы он отнюдь не закраснелся. Потом на этот лук положить курицу и немного все поджарить, поливая слегка куриным бульоном и перевернув и мясо, и лук раза два. Тогда влить остальной бульон в кастрюлю и дать сильно кипеть. Перед отпуском к столу, выпустите штуки 3 яиц в стакан и взбейте их хорошенько, прибавьте к ним немного лимонного соку из одного лимона, — не больше. Отодвинув кастрюлю с огня, бережно вливать яйца, не переставая мешать, но не давая закипать, и тогда уже готовую чихитму влить в миску, посыпав ее рубленною зеленью. (5)

\z{Суп из окуней}\index{Суп ! из окуней}

Очистив и вымыв окуней, обжарить в масле; если есть в них икра, изрезать небольшими кусочками и также обжарить в масле. Положить обжаренных окуней и икру в кастрюлю, прибавить 2 луковицы, натыканные гвоздикою, каперсов и оливок, налить рыбным бульоном и варить до спелости. (4) (Пост.)

\z{Суп из белорыбицы}\index{Суп ! из белорыбицы}

Взять 10 небольших окуней, обвалять в муке, обжарить в масле; белорыбицу, нарезав небольшими звеньями, обжарить также в ореховом или маковом масле. Положить в кастрюлю белорыбицу и окуней, прибавить шинкованных кореньев, петрушки и моркови, целую луковицу, налить рыбным бульоном, уварить до спелости. (4) (Пост.)

\z{Калья}\index{Калья}

Нарезав паюсной икры небольшими кусочками, прибавить из карасей или окуней икры, также нарезав кусочками, накрошив огурцов и луку, налить рыбным бульоном, уварить до спелости, приправить перцем и мускатным орешком. (4) (Пост.)

\z{Уха из стерляди}\index{Уха ! из стерляди}

Вычистив стерлядь, нарезать звеньями. Потом приготовить из мелкой рыбы бульон, процедить. Влив опять в кастрюлю бульон, поставить на огонь; когда бульон закипит, положить шинкованных кореньев: петрушки, моркови и две луковицы, один лимон, нарезанный кружками; варить, пока коренья будут мягки; тогда опустить в бульон стерлядь и влить стакан мадеры. (8 — 10) (Пост.)

\z{Уха из стерляди с фаршем}\index{Уха ! из стерляди ! с фаршем}

Две или более небольших стерлядки очистить, вымыть и нарезать звеньями. Потом взять щуку или судака, снять кожу, выбрать кости, мясо истолочь мягко, прибавить ложку прованского или макового масла, 2 ломтя белого хлеба, размоченного в воде и выжатого досуха, истолочь еще фарш вместе с хлебом, посолить, приправить мускатным орешком. Кости, вынутые из рыбы, истолочь в ступке и вместе с головой рыбы, из которой сделан фарш, положить в кастрюлю, налить водой, посолить, положить 2 луковицы, лаврового листу, поставить вариться. Когда бульон довольно уварится, процедить, поставить опять на огонь и, дав закипеть, опустить стерлядь. Из фарша накатать шариков, положить в уху, прибавить горсть или более маслин, смотря по количеству ухи; отпуская, посыпать зеленым рубленным укропом. (6) (Пост.)

\z{Суп с клёцками}\index{Суп ! с клецками}

Изготовив хороший бульон, взять 1/4 фунта чухонского масла и растирать его деревянной ложкой до тех пор, пока оно будет как пена, положить потом 2 целых яйца и 2 желтка, ложку сливок и столько муки, чтобы тесто не распадалось; потом, перед самым обедом, класть это тесто серебряной ложкой маленькими клёцками в кипящий бульон и посолить. (5)

\z{Суп питательный с луковыми кружками}\index{Суп ! с луковыми кружками}

Облупить 10 маленьких целых луковиц, нарезать тоненькими кружками, обвалять слегка в муке и поджарить в свежем русском масле, потом выложить на чистое полотенце, чтобы они не оставались жирными, и положить их в суповую чашку. Перед обедом налить на них горячий, кипящий бульон и подавать к этому кругленькие, поджаренные сухарики. (5)

\z{Похлебка a la minute}\index{Похлебка a la minute}

Взять фунт говядины, изрезать ее в кусочки величиною с наперсток; изрезать также 4 хорошие моркови, 4 репы и 2 луковицы, положить все вместе в кастрюльку, в которой распустить заранее немного чухонского масла, и оставить вариться минут на 20, на довольно сильном огне; потом наполнить кастрюлю горячею водою, дать кипеть 1/4 часа, прибавить 2 жареные луковицы, перцу, соли и пучок пряных трав; наконец процедить сквозь сито, положить в бульон хлеба или вермишели, сваренной обыкновенным образом. (2—-3)

\z{Суп-пюре из белой фасоли}\index{Суп-пюре ! из белой фасоли}

Сварить бульон из мяса и кореньев, процедить. Два стакана фасоли налить кипятком, вскипятить раза 2, процедить, налить бульоном, положить 1.5 ложки масла, варить, пока фасоль не разварится; протереть сквозь сито, развести бульоном, разбить 2 желтка, развести горячим бульоном, шибко мешая, положить соли, мускатного ореха, зелени; подавать. (6)

\z{Суп немецкий со сливками или с желтками}\index{Суп ! немецкий со сливками}

Сварить бульон с кореньями, процедить; 1.5 ложки муки и 1 ложку масла слегка поджарить, развести 0.5 стаканом бульона, вскипятить; когда немного остынет, вбить 4 желтка, размешать, развести горячим бульоном, шибко мешая, подогреть до горячего состояние, но не кипятить. Вместо желтков можно влить 1.5 стакана сливок; вскипятить.

Опустить в суповую миску отдельно в бульоне отваренные точеные коренья: морковь, брюкву, а также сваренного в бульоне и ломтиками нарезанного мясного фарша. (6)

\z{Суп-пюре из артишоков}\index{Суп-пюре ! из артишоков}

Сварить бульон из 3 фунтов говядины с кореньями, процедить. 20 артишоков очистить, вымыть, залить 3 стаканами бульона, положить 1/8 фунта масла, варить до готовности, протереть сквозь сито. Ложку масла, 2 ложки муки поджарить, развести 1 стаканом сливок, вскипятить, смешать с пюре из артишоков, развести бульоном, посолить; подавать. (6)

\z{Суп с бараньим пловом}\index{Суп ! с бараньим пловом}

Взять 1 фунт молодой баранины, разрезать на порционные куски, вымыть, сложить в кастрюльку, посолить, покрыть крышкою и поставить на огонь; когда сок из баранины выкипит и сама баранина получить желтый колер, налить кипятком, положить 1 луковицу, морковь, порей и петрушку, вымыть дочиста 2 стол. ложки риса и опустить в суп; когда баранина и рис окажутся мягкими, вынуть коренья, а суп вылить в суповую чашку. (4)

\z{Борщ из сельдерея}\index{Борщ ! из сельдерея}

Сварить бульон из 3 фунтов говядины с кореньями (прибавить, кто хочет, 1 фун. ветчины); процедить.

Взять 1 фун. сельдерея, выбрать самые листья, опустить в кипяток, сварить до мягкости, откинуть на решето. мелко изрубить. Масло и 0.5 стакана муки слегка поджарить, положить туда же сельдерей, развести сметаною, потом бульоном, вскипятить, мешая; подавая, всыпать 2 листочка сельдерея, мелко изрубленного, и укропу. (5)

\z{Щи грибные}\index{Щи ! грибные}

(Постные или скоромные)

Изрубить мелко 2 луковицы, поджарить в 2 ложках прованского масла, положить 2 стакана кислой капусты, еще поджарить, развести сваренными грибным бульоном с кореньями, варить на легком огне.
Перед самым обедом ложку муки поджарить в ложке прованского масла, развести грибным бульоном, влить во щи, положить туда же несколько мелко нашинкованных отваренных грибов, вскипятить, всыпать соли и крупного простого перцу. (5)

\z{Щи ленивые}\index{Щи ! ленивые}

Сварить бульон из 3 фун. говядины с верхним жиром, с кореньями и пряностями, процедить. 0.5 большего кочна свежей капусты очистить, разрезать кусков на 20, опустить в процеженный бульон, вварить на легком огне до мягкости; 0.5 ложки масла смешать с 1 ложкою муки, положить в бульон, вскипятить раза два; подавая, всыпать зеленой петрушки и немного мелкого простого перцу. (6)

\z{Уха оллапотрида из разной рыбы}\index{Уха ! оллапотрида из разной рыбы}

Взять 5 живых ершей, одного окуня и самого маленького сига, очистить, как следует, выпотрошить (не прорвать желчь) и вымыть; икру, если окажется, отобрать особо в чашечку. Между тем поставить в кастрюльке 4 стакана воды, положить изрезанных кореньев, т. е. петрушку, порей и лук, полштуки лаврового листу, 2 штуки английского перцу, частицу мускатного цвету и по вкусу соли (не пересаливать), поставить на огонь и варить несколько минут, потом положить окунь и ерши и сварить до готовности; пока рыба варится, разрезать сижка вдоль пополам (не разрезывая головки), с верхней половинки сига срезать лишь одни брюшные косточки, вторую половинку подрезать так, чтобы осталась вся спинная кость при головке, срезать тщательно и обрезать перышки так, чтобы не осталось при филеях ни одной косточки; тогда вынуть на тарелку из ухи ерши и окунь цельными и употребить на что угодно, а в уху положить косточки и обрезки с сига, и вбить в икру один белок, размять лопаткой, прибавить ложку холодной воды и вылить в уху, когда закипит; варить на легком огне до тех пор, пока уха очистится и получит надлежащий вкус (чтобы придать ухе желтый колер, нужно поджарить на плите половинку сырой луковицы и положить в уху пред очисткой). Перед отпуском, процедить уху в кастрюльку, сливая осторожно, чтобы бульон не помутился, поставить на огонь, и вытерев филеи из сига полотенцем, опустить в бульон; когда вскипит и рыбные филеи всплывут, то уха готова. (4) (Пост.)

\z{Габер-суп}\footnote{Как ни странным может показаться помещение здесь, между рецептами мясных и рыбных бульонов, описания этого сладкого немецкого супа, но оно обусловливается обычаем многих, преимущественно петербургских домов, — раз в неделю подавать габер-суп, завоевавший с давнего времени право гражданства во многих даже вовсе не немецких, а совершенно русских хозяйствах.}.\index{Суп ! габер}

Вымыть хорошенько овсяную крупу, налить на нее воды (считая на чашку 2 бутылки), вскипятить и, когда крупа разварится, протереть сквозь сито; положить осьмушку черносливу, небольшой кусок чухонского масла, и дать прокипеть, чтоб чернослив разварился. (4)

\z{Бульон с заварным кремом}\index{Бульон ! с заварным кремом}

Приготовить бульон из 3-х ф. говядины и крем заварной. Отбить в кастрюльку столько желтков, сколько назначено иметь формочек крема (прибавляя на 4 желтка 1 белок), размешать лопаткою, посолить, положить немножко мускатного ореха, развести бульоном (мерить формочкою, которая назначена для крема) и процедить сквозь сито. Между тем поставить в глубоком сотейнике столько кипящей воды, чтобы формочки стояли только до половины в воде; подмазать маслом формочки, налить в каждую массы полно, поставить в кипящую воду и варить до тех пор, пока крем загустеет; тогда снять с огня, вынуть формочки, остудить на льду, а пред отпуском отделить осторожно тоненьким ножичком крем от формочки, выложить на тарелку, подлить кипящего бульону и подать особо при супе. (4)

\z{Бульон с гречневыми клёцками}\index{Бульон ! с гречневыми клецками}

Приготовить бульон из 3-х ф. говядины, потом сделать гречневые клёцки: влить в кастрюлю стакан воды, — положить 0.5 ложки масла и посолить. Когда вскипит, всыпать 0.5 стакана мелкой гречневой крупы (смоленской), размешать и, когда загустеет, поставить покрытым в печку на 1/4 часа, наблюдая, чтобы каша не заколеровалась; потом вынуть из печки размешать лопаткою до гладкости, выделать столовою или чайною ложкою клецки из горячей каши и опустить в бульон, в суповую чашку. Подается с рубленною зеленью. (4)

\z{Бульон с клёцками славянскими}\index{Бульон ! с клецками славянскими}

Взять от тонкого филея кусок говядины с верхним жиром (без пашины), свернуть, как должно, и завязать голландскими нитками, вымыть, сложить в кастрюлю, налить холодною водою, поставить на огонь, снять накипь сверху и, когда закипит, процедить бульон сквозь салфетку, а говядину вымыть в теплой воде, сложить в чистую кастрюлю, налить собственным бульоном, поставить снова на огонь и, когда вторично закипит, положить соли, очищенных кореньев, т. е. моркови, сельдерея, порея, петрушку, репу, 4 шт. картофеля, одну поджаренную на плите луковицу, и варить на плите до тех пор, пока говядина будет мягка, тогда процедить бульон в кастрюльку, поставить на плиту, а говядину (с частью бульона) покрыть крышкою, оставить на горячем месте и поступать как сказано будет ниже. Между тем разбить одно яйцо в кастрюльку, посолить, положить 1 ложку муки и размешать до гладкости; когда бульон закипит, выливать из кастрюльки в кипящий бульон клёцки так, чтобы они были тоненькие, немного толще вермишели, и, прокипятив, вылить в суповую чашку. (4)

\z{Суп из молодого свекольника}\index{Суп ! из свекольника}

Сварить бульон, как обыкновенно, из 3 фунтов говядины, кореньев, пряностей, сушеных грибов; можно прибавить 1 фунт свиной грудинки; процедить.

Очистить, вымыть, нарезать мелко молодого свекольника и несколько самых маленьких корешков свеклы, положить в бульон, сварить, долить по вкусу свекольного отваренного рассолу или хлебного квасу. Перед самым отпуском влить сметаны, положить масла, размешанного с 1 ложкою муки, вскипятить. Подавая, всыпать зеленой петрушки и укропу. (6)

\z{Суп из поросенка}\index{Суп ! из поросенка}

Сварить бульон, как обыкновенно, из 3 фунт. поросенка, кореньев и пряностей; процедить; 0.5 стакана перловых круп разварить отдельно с 1 ложкою масла, выбить добела, прибавить сметаны, развести бульоном, вскипятить; подавая, всыпать зеленой петрушки и укропу. Можно подавать суп этот с фрикадельками из почки поросенка, а поросенка подать отдельно с хреном и со сметаной. (6)

\z{Суп из дичи прозрачный}\index{Суп ! из дичи прозрачный}

Взять какую-нибудь дичь, изжарить ее в масле, изрубить, истолочь в ступке с костями, налить бульоном, сваренным, как обыкновенно, из 3 фунтов говядины, кореньев и пряностей, варить, процедить, очистить белками, процедить сквозь салфетку, влить вина, вскипятить. Подавать с греночками. (6)

\z{Раковый суп-пюре с рисом}\index{Суп-пюре ! раковый с рисом}

Раки вымыть, сложить в кастрюлю, налить водою, положить немного соли, 1 мелко нашинкованную луковицу, пучок зеленого укропу, сварить на легком огне, процедить. Раки же очистить, т. е. вынуть шейки, сложить их в кастрюлю, налить бульоном, в котором варились раки. Спинки с ножками истолочь в каменной ступке, протереть сквозь частое сито, протереть пюре, оставить в кастрюле, a оставшиеся на сите скорлупки поджарить с 1/4 стак. прованского масла до темного цвета, всыпать ложку муки, опять поджарить, развести раковым бульоном, вскипятить, процедить сквозь дуршлаг, а потом сквозь салфетку. Перед отпуском, развести им пюре из раков, опустить раковые шейки, отдельно отваренный рис, подогреть. (6) (Пост.)

\z{Суп-пюре из тетерева или глухаря}\index{Суп-пюре ! из тетерева или глухаря}

Сварить бульон из 3 фун. говядины и кореньев, процедить. Взять одного большего тетерева; с половины его снять мясо (а кости положить в бульон), изрубить его, прибавить половину французской булки, 1/8 фун. масла, истолочь, протереть сквозь сито, положить в кастрюльку с 1/8 фун. масла, подогреть до горячего состояние, беспрестанно мешая, развести это пюре немного бульоном, протереть опять сквозь сито; тогда уже развести бульоном как следует. Взять другую половину тетерева, снять мясо с костей (кости положить в бульон), изрубить, истолочь, прибавив 1/4 французской булки, 1 яйцо, 1/8 фун. масла, соли, протереть сквозь сито, сделать из этой массы кнель, т. е. маленькие клецки, следующим образом: взять 2 чайные ложечки, обмоченные отнюдь не в теплой, а в холодной воде; одною взять кусочек фарша, сгладить ножом ровно с краями ложечки, а другою снять эту кнель и опустить в соленую воду, и так поступать до конца, и затем вскипятить кнели. Когда будут готовы, откинуть на решето. Отдельно отваренные штук 7-–8 шампиньонов, 1/4 ф. сладкого мяса и 6 петушьих гребешков нарезать кусочками и опустить в суп. Перед отпуском, влить полстакана малаги. В этот суп можно положить также листья щавеля и шпината, которые прежде надо вскипятить в бульоне. (6)

\z{Суп-пюре из шампиньонов и ершей}\index{Суп-пюре ! из шампиньонов и ершей}

Сварить бульон из 3 фун. говядины, 1 фун. телятины и кореньев, процедить. Взять 20 шт. ершей, снять с них филеи, а кости, перемыв, положить в бульон. 1/4 фун. риса, 10 шампиньонов разварить в бульоне, протереть сквозь сито, развести немного бульоном, протереть сквозь салфетку. развести тогда всем бульоном, поставить на воду, т. е на сотейник с кипящею водою, мешая ложкою, как можно чаще, чтобы пюре не осело. Снять с филеев кожицу, сварить их в бульоне, опустить в приготовленный суп 1/8 фун. сливочного масла, 1 желток, 0.5 стакана густых сливок положить в миску, влить суп, мешая; посыпать зеленой петрушки и укропу. (6)

\z{Суп-пюре из спаржи}\index{Суп-пюре ! из спаржи}

Сварить бульон, как обыкновенно, из 3 фунт. говядины, 0.5 фунта телятины и кореньев; процедить. Взять 1 фунт спаржи, головки отрезать, а остальное нарезать кусочками и варить с 1/4 фунтом перловой крупы или риса, подливая бульону; разварить до мягкости, положить 0.5 ложки масла, протереть сквозь сито, развести немного бульоном, протереть сквозь салфетку, развести остальным бульоном, поставить на пар, т. е. на сотейник с кипящею водой. Головки спаржи очистить, сварить в соленой воде до готовности, опустить в суп. В миску положить 1/8 фунта сливочного масла, 1 желток, 0.5 стакана сливок, размешать, влить суп, мешая. (6)

\z{Русский сморчковый суп}\index{Суп ! русский сморчковый}

Берут 4 фунта хорошей суповой говядины и курицу, кладут в кастрюлю, наливают их водою и ставят на плиту кипеть. Как только мясо вполовину поспеет, то кладут в другую кастрюлю 30 сморчков, предварительно перемытых, 1/4 фунта коровьего масла и 1 фунт изрезанной в куски ветчины и разных кореньев, мелко искрошенных. Все это хорошенько поджаривают, прибавляют к смеси несколько муки и доливают бульоном из первой кастрюли. Затем разбирают говядину и курицу штуками и кладут в суп, который в кастрюле ставят на плиту; когда жидкость закипит, то снимают пену и покрывают крышкой. Из вареных яиц, немного масла, белого хлеба, соли, петрушки и укропу, мелко изрубленных, делают фарш, а для того, чтоб он хорошо связался, прибавляют несколько сырых яиц и начиняют этим сморчки, которые кладут в суп. Когда он будет совершенно готов, то вливают в него сметаны и посыпают зеленью. (6)

\z{Борщ с рыбой}\index{Борщ ! с рыбой}

Налить в кастрюлю воды, положить туда несколько кусков соленой или свежей осетрины, или белужины и поставить на огонь. Между тем, пока рыба будет вариться, искрошить свеклы и обвалять ее крошенную в муке; взять также самой свежей капусты, разрезать на несколько частей и, обжарив масле, положить в кастрюлю, в которой варится рыба. Если отвар не довольно кисел, то прибавить в него немного квасу или же слабого уксусу. Потом взять свежей рыбы, как то: карасей или другой, очистить ее от чешуи, выпотрошить, обвалять в муке, обжарить в масле и положить в кастрюлю, к которой прибавить также несколько моркови, изрезанной в небольшие куски, и луку, четвертинками или даже цельными луковицами. Всему этому дать хорошенько увариться, и затем приправить перцем и солью. (5) (Пост.)
