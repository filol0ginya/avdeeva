%%% русский язык
\usepackage[english,russian]{babel}
\usepackage{cmap}					% поиск в PDF
\usepackage[T2A]{fontenc}			% кодировка
\usepackage[utf8]{inputenc}			% кодировка исходного текста
\usepackage{indentfirst}            % отступ первой красной строки раздела
\frenchspacing


 % глоссарий

\usepackage{makeidx} % предметный указатель

\usepackage{titlesec}				% настройка заголовков
	\titleformat*{\section}{\large\bf}

%\usepackage{lastpage} %показывает последнюю страницу

\usepackage{fancyhdr} %загрузим пакет колонтитулов
	\pagestyle{fancy} %применим колонтитул
	\fancyhead{} %очистим хидер на всякий случай
	\fancyhead[LE,RO]{\thepage} %номер страницы слева сверху на четных и справа на нечетных
    \fancyhead[LO,RE]{ОТДЕЛ \thesection}
	%\fancyhead[O]{текст-слева-нечетные} 
	%\fancyhead[CE]{текст-центр-четные} 
	\fancyfoot[LE,RO]{Екатерина Авдеева} %футер 
    \fancyfoot[LO,RE]{Полная поваренная книга} %футер

\usepackage{lastpage} % Узнать, сколько всего страниц в документе.

\usepackage{soul} % Модификаторы начертания

\usepackage{hyperref}
\usepackage[usenames,dvipsnames,svgnames,table,rgb]{xcolor}
\hypersetup{				% Гиперссылки
    unicode=true,           % русские буквы в раздела PDF
    pdftitle={Заголовок},   % Заголовок
    pdfauthor={Автор},      % Автор
    pdfsubject={Тема},      % Тема
    pdfcreator={Создатель}, % Создатель
    pdfproducer={Производитель}, % Производитель
    pdfkeywords={keyword1} {key2} {key3}, % Ключевые слова
    colorlinks=true,       	% false: ссылки в рамках; true: цветные ссылки
    linkcolor=black,          % внутренние ссылки
    citecolor=yellow,       % на библиографию
    filecolor=magenta,      % на файлы
    urlcolor=purple         % на URL
}

\usepackage{csquotes} % Инструменты для ссылок

\setcounter{secnumdepth}{3} % нумеруем subsubsections


\newcounter{nz}%[subsection] % счётчик рецептов

\newcommand{\z}[1]{%        % оформление счётчика рецептов

\addtocounter{nz}{1}         
\textbf{\arabic{nz}. #1%
}}


%%% Работа с картинками
\usepackage{graphicx}  % Для вставки рисунков
\graphicspath{{images/}{images2/}}  % папки с картинками
\setlength\fboxsep{3pt} % Отступ рамки \fbox{} от рисунка
\setlength\fboxrule{1pt} % Толщина линий рамки \fbox{}
\usepackage{wrapfig} % Обтекание рисунков текстом

%%% Работа с таблицами
%\usepackage{array,tabularx,tabulary,booktabs} % Дополнительная работа с таблицами
%\usepackage{longtable}  % Длинные таблицы
%\usepackage{multirow} % Слияние строк в таблице


%\usepackage{footmisc} % для сносок (тест)


%\usepackage{cite} % Работа с библиографией
%\usepackage[superscript]{cite} % Ссылки в верхних индексах
%\usepackage[nocompress]{cite} % 